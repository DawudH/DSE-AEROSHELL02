\section{Risk assessment} 
\label{ch:risk}
This section will cover the initial risk assessment that was carried out during the conceptual design phase. First a risk map was constructed, followed by an explanation of the contingency margins that will be used during the various design phases. These are based on the outcomes of the risk assessment. Section \ref{sec:riskmap} will cover the risk map, after which section \ref{sec:tca} will discuss the technical contingency allocation.

\subsection{Risk map}
\label{sec:riskmap}
 A risk map has been constructed in order to identify which mission and design elements pose the biggest risk. From the risk map it can be seen which elements require the most attention in order to mitigate the risks they pose. The risk map is shown in table \ref{tab:riskmap}. The colors correspond to the amount of risk each table cell represents. The numbers in table \ref{tab:riskmap} correspond to the elements shown in table \ref{tab:riskelements}.

\begin{table}[h]
\centering
\caption{Risk map}
\label{tab:riskmap}
    \begin{tabular}{|c|c|c|c|c|}
    \hline
    \textbf{Feasible in theory} & & & & \\
    \textbf{Working laboratory model} & & & & \\
    \textbf{Used in-flight on Earth} & & & & \\
    \textbf{Derived from used technology on Mars} & & & & \\
    \textbf{Feasible in theory} & & & & \\
     & \textbf{Negligible} & \textbf{Marginal} & \textbf{Critical} & \textbf{Catastrophic} \\ \hline
%     \bf{Static earth} & \cellcolor{green!25}35g - 5 & \cellcolor{blue!25}0.3W - 3 & \cellcolor{red!25}1$^\circ$ - 1 & \cellcolor{red!25}39\\ \hline
    \end{tabular}
\end{table}

\begin{table}[h]
\centering
\caption{Risk map elements}
\label{tab:riskelements}
\begin{tabular}{|c|c|}
\hline
\textbf{Number} & \textbf{Element} \\
\hline
1 & Control system \\
2 & \\
3 & \\
4 & \\
5 & \\
6 & \\
7 & \\
8 & Long space exposure \\
\hline
\end{tabular}
\end{table}

\subsection{Technical contingency allocation}
\label{sec:tca}
From the risk map of the preceding section one can see that there are many