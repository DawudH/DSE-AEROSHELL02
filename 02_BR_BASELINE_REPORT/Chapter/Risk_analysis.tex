\section{Risk assessment} 
\label{ch:risk}
This section will cover the initial risk assessment that was carried out during the conceptual design phase. First a risk map was constructed, followed by an explanation of the contingency margins that will be used during the various design phases. These are based on the outcomes of the risk assessment. Section \ref{sec:riskmap} will cover the risk map, after which section \ref{sec:tca} will discuss the technical contingency allocation.

\subsection{Risk map}
\label{sec:riskmap}
 A risk map has been constructed in order to identify which mission and design elements pose the biggest risk. From the risk map it can be seen which elements require the most attention in order to mitigate the risks they pose. The risk map is shown in table \ref{tab:riskmap}. The colors correspond to the amount of risk each table cell represents. The numbers in table \ref{tab:riskmap} correspond to the elements shown in table \ref{tab:riskelements}.

\begin{table}[h]
\centering
\caption{Risk map}
\label{tab:riskmap}
    \begin{tabular}{|p{6.5cm}|c|c|c|c|}
    \hline
    \textbf{Feasible in theory} & \cellcolor{green} & \cellcolor{yellow} & \cellcolor{red} & \cellcolor{red} 1,4,6,7 \\ \hline
    \textbf{Working laboratory model} & \cellcolor{green} & \cellcolor{yellow} & \cellcolor{red} & \cellcolor{red} \\ \hline
    \textbf{Demonstrated in-flight on Earth} & \cellcolor{green} & \cellcolor{yellow} & \cellcolor{yellow} & \cellcolor{yellow} 3,5 \\ \hline
    \textbf{Derived from used technology on Mars} & \cellcolor{green} & \cellcolor{yellow} & \cellcolor{yellow} & \cellcolor{yellow} 2 \\ \hline
    \textbf{Demonstrated in-flight on Mars} & \cellcolor{green} & \cellcolor{green} & \cellcolor{green} & \cellcolor{green} \\ \hline
     & \textbf{Negligible} & \textbf{Marginal} & \textbf{Critical} & \textbf{Catastrophic} \\ \hline
%     \bf{Static earth} & \cellcolor{green!25}35g - 5 & \cellcolor{blue!25}0.3W - 3 & \cellcolor{red!25}1$^\circ$ - 1 & \cellcolor{red!25}39\\ \hline
    \end{tabular}
\end{table}

\begin{table}[h]
\centering
\caption{Risk map elements}
\label{tab:riskelements}
\begin{tabular}{|c|c|}
\hline
\textbf{Number} & \textbf{Element} \\
\hline
1 & Flight control system\\
2 & Deployment system\\
3 & Impact of launch vibrations\\
4 & Operational mission duration\\
5 & Structural integrity under mission loads\\
6 & Heat resistance\\
7 & Long space exposure\\
\hline
\end{tabular}
\end{table}

\subsection{Technical contingency management}
\label{sec:tca}
From the risk map of the preceding section one can see that there are many risks involved with the development of a hypersonic inflatable aeroshell. Several \gls{tpm}s  will be used to evaluate the performance of the system at different stages of the design process. These \gls{tpm}s follow from the top-level requirements discussed further in chapter \ref{ch:req}. The \gls{tpm}s that will be used are:
\begin{itemize}
	\item Hypersonic deceleration system mass fraction
	\item Aerobraking duration
	\item Control system reliability
\end{itemize}
Table \ref{tab:tpm} shows the \acrfull{tpm} factors that will be used during the various stages of the design process. The lower limits for the \gls{rca}s come from NASA \cite{GoddardSpaceFlightCenter2013}. These \gls{rca}s will be used to account for the uncertainty of the analysis methods that will be used to evaluate all the concepts.
<<<<<<< HEAD

=======
>>>>>>> bf8ee5eb6f5b81467d979c3d4a2f840e416888cb

\begin{table}[h]
	\centering
	\caption{RCA factors that will be used}
	\label{tab:tpm}
	\begin{tabular}{|p{0.47\textwidth}|p{0.24\textwidth}|p{0.24\textwidth}|}
		\hline
		\textbf{Technical Performance Measurement} & \textbf{Conceptual design contingency [\%]} & \textbf{Preliminary design contingency [\%]} \\ \hline
		Hypersonic deceleration system mass fraction & 30 & 20 \\ \hline
		Aerobraking duration & 30 & 20 \\ \hline
		Control system reliability & 30 & 20 \\ \hline
	\end{tabular}
\end{table}