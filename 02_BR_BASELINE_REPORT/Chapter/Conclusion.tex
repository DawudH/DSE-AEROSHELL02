\section{Conclusion}\label{cha:conclusion}
The goal of this technical project is to design a hypersonic decelerator that meets the requirements set by the customer. Several steps have been taken to ensure that the final product will meet said requirements.

First a literature review was conducted to identify the computational methods that will be used to analyze the different product concepts. This was done for the fields of structural engineering, aerodynamics, thermal engineering, control systems and astrodynamics. In addition to this the current states of (re-)entry vehicle technologies in general and inflatable aeroshell technologies in particular were determined. Following this a mass budget breakdown was made of the crew capsule and hypersonic decelerator. Thirdly the functional flow of the design mission was analyzed. This was done by producing a \acrlong{ffd} and \acrlong{fbs}. Following this a requirement analysis was executed by making a \acrlong{rdt}. The \gls{rdt} consists of a requirement flowdown, documenting how system and subsystem requirements follow from the top-level requirements set by the customer.

After the requirement analysis a market investigation was conducted. This consisted of listing the potential customers and product functions, succeeded by a SWOT analysis. Following this an assessment was made of the potential risk sources encountered during the product design. In addition to this a procedure was set up to ensure that the \acrlong{tpm}s of the final product will meet the demands of the customer. This will be done with the use of \acrlong{rca}s that decrease as the product design matures. Furthermore the project approach with respect to sustainable development was described, followed by an exploration of all different design options in a \acrfull{dot}. This \gls{dot} contains all concepts that were envisioned, including those that will be eliminated later on in several design phases.

Following this \acrfull{br} the \gls{dot} will be used to conceive several conceptual designs that will be analysed and evaluated against each other during a trade-off process. The result of this trade-off will be a ranking of concepts, with the best concept being analyzed in further detail after the \acrfull{mtr}.