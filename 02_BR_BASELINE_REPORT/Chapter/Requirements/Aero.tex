\subsection{Aerodynamic Requirements Discovery} 
\label{sec:aero}
A number of aerodynamic requirements can be seen in Figure \ref{fig:RBS}. These will all be discussed here and have been summarized in Table \ref{tab:aeroreqs}. 


\begin{table}[h]
	\caption{Overview of aerodynamic requirements}
	\label{tab:aeroreqs}
	\begin{tabular}{|p{0.2\textwidth}|p{0.7\textwidth}|}
		\hline
		ID & Description \\
		\hline \hline
		CIA-Func-B01-Aero-01 & The system shall have a \gls{sym:CD}\gls{sym:A} of at least TBD $m^{2}$ \\ \hline
		CIA-Func-B02-Aero-02 & The system shall have a \gls{sym:CD}\gls{sym:A} of at most TBD $m^{2}$ \\ \hline
		CIA-Func-B03-Aero-03 & The system shall produce a maximum heat flux of no more than TBD [$\frac{W}{cm^{2}}$] \\ \hline
	\end{tabular}
\end{table}
Requirement CIA-Func-B01-Aero-01 complies to the top-level requirement of a maximum mission duration. If the deceleration takes too long this will add significant time to the total mission length, which is unwanted.
On the other hand, requirement CIA-Func-B02-Aero-02 is caused by the need for a limitation of the maximum allowable decelerations that occur. This can have a significant impact on the mission and mission duration, since this limits the maximum allowable drag force that can be achieved.
Requirement CIA-Func-B03-Aero-03 follows from the entry velocity of 7 [$\frac{km}{s}$]. Since the heat flux subjected upon a body is proportional to $\gls{sym:V}^{3}$ the highest values for the heat flux will be found during the first orbit around Mars \cite{Tauber1986}.