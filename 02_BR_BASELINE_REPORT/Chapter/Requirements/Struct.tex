\subsection{Structural Requirements Discovery} \label{sec:struct}
The vehicle structure faces a number of requirements, functional and operational. The functional requirements are stated in Table \ref{tab:strucfuncrequirements}, the operational requirements in Table \ref{tab:strucoprequirements}. These requirements are briefly discussed hereafter.
\begin{table}[H]
	\caption{Overview of functional requirements on structures subsystem}
	\begin{tabular}{|p{0.20\textwidth}|p{0.70\textwidth}|}
    \hline
    ID          & Description                                                                                                      \\ \hline \hline
    CIA-B01-Struc-01 & The structure shall operate at least within a temperature range of [\gls{tbd},\gls{tbd}] degrees Celsius           \\ \hline
    CIA-B02-Struc-02 & The structure shall support deployment \\ \hline
    CIA-B07-Struc-03 & The structure shall withstand loads of at least 3g in each axis without failure                           \\ \hline
    CIA-B09-Struc-04 & The structure shall connect payload and deceleration mechanism \\ \hline
    \end{tabular}
    \label{tab:strucfuncrequirements}
\end{table}
Requirement CIA-B01-Struc-01 follows from the aerodynamic heating as a consequence of the dissipation of kinetic energy corresponding to a velocity of 7 [km/s], as stated in requirement CIA-A01. The \gls{tps} reduces the temperature to within acceptable limits, which are translated to a range of temperature in which the structures subsystem should operate. These temperatures therefore follow from the \gls{tps}. This requirement is essential because temperature can have a substantial effect on the mechanical properties of materials as well as thermal expansion \cite{Callister2007}. These mechanical properties are essential to meet requirement CIA-B07-Struc-03, namely to handle the structural loads induced during aerocapture and (re-)entry. These have been limited to 3g in each axis in requirement CIA-A03. In addition, if a deployment functionality  (for example an inflation system) is a concept feature, deployment should be performed by the subsystem; in case such a functionality is not present, there is no deployment, hence nothing to support and the requirement is logically satisfied. This is stated in requirement CIA-B02-Struc-02. Lastly, payload and deceleration mechanism should be connected to prevent separation during re-entry and satisfy the payload constraint. This is stated in requirement CIA-B09-Struc-04.

\begin{table}[H]
	\caption{Overview of operational requirements on structures subsystem}
	\begin{tabular}{|p{0.20\textwidth}|p{0.70\textwidth}|}
    \hline
    ID          & Description                                                                                                      \\ \hline \hline
    CIA-B02-Struc-05 & The structure shall have a maximum diameter not exceeding 12 [m] in deployed configuration     \\ \hline
    CIA-B03-Struc-06 &  The structure shall have a maximum diameter not exceeding 5 [m] in stowed configuration                              \\ \hline
    CIA-B04-Struc-07 & The structure shall have a mass not exceeding \gls{tbd} [kg]\\ \hline
    \end{tabular}
    \label{tab:strucoprequirements}
\end{table}
The operational requirements CIA-B02-Struc-05 and CIA-B03-Struc-06 follow from the geometric constraints (by launcher considerations). Requirement CIA-B04 states that the structural subsystem should respect the mass budget.