\section{Introduction}\label{cha:introduction}
There is a need for a feasible and cost-effective vehicle that can transport human payload to surfaces of extraterrestrial locations such as Mars. One solution for this is an inflatable aeroshell, stowable within conventional launcher configurations in undeployed condition. The design of a controllable inflatable aeroshell can be considered as complex, featuring interaction between many different disciplines. To reduce the complexity of the design problem steps have been made in the \gls{pp}. As part of the \gls{br}, this baseline report reviews the functions, requirements and initial concepts.

The purpose of this report is to show the current progress of the group in the design process. After the \gls{pp}, where the specific approach of the group to the technical and management aspects of the design project has been defined, in this report the functional and requirement analysis are used to come up with feasible design options. Again Systems Engineering tools are used to help with this phase. For the functional analysis a \acrfull{ffd} and \acrfull{fbs} have been generated. The \acrfull{rdt} forms the basis of the requirements analysis and several Design Option Trees (DOTs) have been made to find feasible design options.

Tthe results of a literature review are presented in Chapter \ref{cha:litreview}. This includes an overview of past (re-)entry missions and a overview of all the currently available and used technologies for the relevant departments as defined in the \gls{obs} \cite{Balasooriyan2015}. Preliminary tool are developed for a trajectory which is discussed in Chapter \ref{ch:trajectory}. Chapter \ref{ch:func} details the functional analysis and Chapter \ref{ch:req} discusses the requirements analysis. A budget breakdown is discussed in chapter \ref{ch:budget}. After that a market analysis and a risk assessment are described in Chapter \ref{ch:market} and \ref{ch:risk} respectively. The approach with respect to sustainable development is stipulated in Chapter \ref{ch:sustain}. Finally the design options are discussed and given in the form of Design Option Trees (DOTs) in Chapter \ref{ch:design}. 



