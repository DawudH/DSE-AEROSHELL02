\section{Market Analysis} \label{ch:market}
This chapter gives the results of a market analysis for a controllable inflatable aeroshell. To define the market for the product three dimensions are used: function, technology and customer. The purpose of the market analysis is a minimization of risk of selecting the wrong combination of function and technology for a selected set of customers. A \gls{swot} analysis provides a brief overview of the product characteristics. As such, the chapter commences with subsequent sections on function, technology and customer. It follows upon this with a section on the outcome of the \gls{swot} analysis.

\subsection{Customer}
Prospective customers are scientific or governmental agencies on one hand and private ventures on the other hand. The former includes \gls{nasa} as a leading contributor to and investigator of the use of inflatable aeroshells for (re-)entry, as explained in Chapter \ref{cha:litreview}. \gls{nasa} operates by order of the \gls{us} government, currently in pursuit of human exploration of Mars in the 2030s \footnote{URL: \url{https://www.nasa.gov/content/nasas-journey-to-mars}.  Accessed 28 April 2015}. This has been formulated as goals in the \gls{nasa} Authorization Act of 2014\footnote{URL: \url{http://science.house.gov/sites/republicans.science.house.gov/files/documents/HR\%204412.pdf}. Accessed 28 April 2015} and the \gls{us} National Space Policy, issued in 2010\footnote{URL: \url{https://www.whitehouse.gov/sites/default/files/national\_space\_policy\_6-28-10.pdf}. Accessed 28 April 2015}.

The interest expressed by the \gls{us} in human exploration of Mars is shared by a number of private ventures, most notably Mars One, the Inspiration Mars Foundation and SpaceX. The former two are non-profit organisations, while SpaceX is a commercial venture. Mars One has expressed its goal as the permanent human settlement on Mars with planned departure of the first non-human payload in 2020 and the first human payload in 2026\footnote{URL: \url{http://www.mars-one.com/}. Accessed 28 April 2015}. The Inspiration Mars Foundation, in cooperation with \gls{nasa}, seeks to transport two humans, a male and female, to Mars for planned launch in 2021\footnote{URL: \url{http://spacenews.com/39714inspiration-mars-sets-sights-on-venusmars-flyby-in-2021/}. Accessed 28 April 2015}. SpaceX is a privately funded venture currently working in close cooperation with \gls{nasa} to provide launchers for manned missions to Mars\footnote{URL: \url{http://www.spacex.com/falcon9}. Accessed 28 April 2015}.

These planned missions illustrate the commercial interest in human spaceflight to Mars. Commercial interest in controllable inflatable aeroshells is directly coupled to this by the fact that these provide a cost-effective means of entry and re-entry (see Chapter \ref{cha:litreview}) primarily by reduced launch costs. Along with this commercial interest, ongoing investigations by \gls{nasa} provide an indication of scientific interest in this field of study. In the end, all interest is fueled by human curiosity and desire to explore and habitate extraterrestrial environments. These environments are expected to expand beyond Mars and therefore interest in (re-)entry vehicles is expected to remain.

Direct customers are thus governmental agencies on one hand, primarily \gls{nasa}, and commercial providers on the other hand. 

\subsection{Function}
Primary prospects for the use of a controllable inflatable aeroshell are the following:
\begin{itemize}
\item Perform entry for manned spaceflight on Mars;
\item Serve as a basis for design extrapolation to perform manned (re-)entry at other sites, for example Earth;
\item Serve as a basis for design extrapolation to perform (re-)entry of unmanned spaceflight;
\item Further the technology development and application of inflatable technologies in spaceflight.
\end{itemize}
A direct function or use is the first item: the controllable inflatable aeroshell provides aerodynamic deceleration (for (safe) transportation of) human payload in a cost-effective manner. The latter is effected primarily by requiring a small launcher volume and a lower weight than conventional and undeployable solutions \cite{Hughes2005, Cianciolo2010}. Therefore its main function can be described as performing manned entry. While the aeroshell is designed for entry on Mars, the design can be extrapolated to perform entry or re-entry on a number of sites, for example Earth. The second and third items are therefore secondary functions fulfilled, distinguished from the primary function by their indirect relation to the product. In addition, the fourth item is a secondary function.

\subsection{Technology}
The functionalities provided by the aeroshell, thus the deceleration provided during (re-)entry, are effected by an inflatable aeroshell as primary technology used. This may be further subdivided into an aerodynamic shape, a control system, a \gls{tps} and a supporting structure (including deployment mechanisms). 

\subsection{SWOT Analysis}
Identification of the primary characteristics, in terms of a \gls{swot} analysis\footnote{URL: \url{http://www.usfca.edu/fac_staff/weihrichh/docs/tows.pdf}. Accessed 28 April 2015}, of the proposed controllable inflatable aeroshell yields Table \ref{tab:swotanalysis}.

\begin{table}[h]
\caption{SWOT Analysis}
\begin{tabular}{|l|l|}
\hline
\multicolumn{1}{|c|}{\textbf{Strengths (S)}}                                                   & \multicolumn{1}{c|}{\textbf{Weaknesses (W)}}                                                                      \\ \hline
\begin{tabular}[c]{@{}l@{}}+ Lightweight solution\\ + Compact solution\end{tabular}            & - Development risk                                                                                                \\ \hline
\multicolumn{1}{|c|}{\textbf{Opportunities (O)}}                                               & \multicolumn{1}{c|}{\textbf{Threats (T)}}                                                                         \\ \hline
\begin{tabular}[c]{@{}l@{}}+ Growing demand\\ + Breakthrough technology/materials\end{tabular} & \begin{tabular}[c]{@{}l@{}}- Catastrophic failure manned mission\\ - Competing concepts (e.g. Orion)\end{tabular} \\ \hline
\end{tabular}
\label{tab:swotanalysis}
\end{table}

Strengths and weaknesses are internal, while opportunities and threats are external factors. While the design retains a development risk, being a relatively new concept, this weakness can be mitigated by proper verification activities. Such activities do, however, incur additional time and costs to the design process. As such, risk remains inherent to the design. 







