\section{Requirement analysis} \label{ch:req}
This section will analyze the requirements of the re-entry mission. It will start by providing a requirement breakdown structure, which visualizes the way the different requirements on the re-entry vehicle flow down from the top level mission requirements and constraints. This will be followed by an analysis of the subsystem requirements. This analysis will follow from the flown down presented in the requirements breakdown structure. Finally, the requirements will be precisely defined and documented. The output of this final step will be the requirements used during the product design. %Guido: Dit moet nog beter geformuleerd worden, maar kon even niets beters bedenken

\subsection{Requirement Breakdown Structure}
The overarching mission requires the system to perform a manned re-entry on Mars. Several performance requirements and mission constraints are imposed on the system. These overall requirements and constraints were detailed in the project plan \cite{Balasooriyan2015}. From these overall requirements and constraints, subsystem requirements can be derived. Figure \ref{fig:RBS} graphically displays this requirement breakdown, and provides a sample parameter which has a requirement imposed on it due to the top level requirements. Each of the subsystems' requirements will be elaborated and expanded upon in the remainder op this section. 

\begin{figure}[h]
\centering
\includegraphics[width=0.95\textwidth]{Figure/RBS.pdf}
\caption{Requirements Breakdown Structure} \label{fig:RBS}
\end{figure}

\subsection{Aerodynamic Requirements Discovery} 
\label{sec:aero}
%A number of aerodynamical requirements can be seen in figure \ref{fig:RBS}. These will all be discussed here, together with their origins and impacts.

%\textbf{Decelerate within 10 days}\\
%From the limited amount of time available for the deceleration a minimum 

%\textbf{Decelerate without excessive heating}\\
%In order to guarantee the safety of the astronauts onboard the spacecraft the spacecraft cannot be heated excessively. Since this heat is caused by the drag exerted on the spacecraft this imposes a constraint on the maximum allowable heat flux produced by said drag. 

%\textbf{Decelerate in a controlled manner}\\
%To make sure that the spacecraft is controllable during the deceleration procedure 

\begin{table}[h]
	\centering
	\caption{Overview of functional requirements on aerodynamical characteristics}
	\label{tab:aeroreqs}
	\begin{tabular}{|c|c|}
		\hline
		\textbf{ID} & \textbf{Description} \\
		CIA-B07-Aero-01 & The system shall produce no more drag than that which causes a deceleration of 3 Earth g's\\
		CIA-B08-Aero-01 & The system shall produce drag sufficient to complete the deceleration within 10 Earth days \\
		\hline
	\end{tabular}
\end{table}
\subsection{Structural Requirements Discovery} \label{sec:struct}
The vehicle structure faces a number of requirements, functional and operational. The following functional requirements apply to the structures subsystem

\begin{table}[H]
	\caption{Overview of functional requirements on structures subsystem}
	\begin{tabular}{|p{0.20\textwidth}|p{0.70\textwidth}|}
    \hline
    ID          & Description                                                                                                      \\ \hline \hline
    CIA-B01-Struc-01 & The structure shall operate at least within a temperature range of [\gls{tbd},\gls{tbd}] degrees Celsius           \\ \hline
    CIA-B02-Struc-02 & The structure shall support deployment functionality \\ \hline
    CIA-B07-Struc-03 & The structure shall withstand loads of at least 3g in each axis without failure                           \\ \hline
    CIA-B09-Struc-04 & The structure shall connect payload and deceleration mechanism \\ \hline
    \end{tabular}
    \label{tab:strucfuncrequirements}
\end{table}

\begin{table}[H]
	\caption{Overview of operational requirements on structures subsystem}
	\begin{tabular}{|p{0.20\textwidth}|p{0.70\textwidth}|}
    \hline
    ID          & Description                                                                                                      \\ \hline \hline
    CIA-B02-Struc-05 & The structure shall have a maximum diameter not exceeding 12 [m] in deployed configuration     \\ \hline
    CIA-B03-Struc-06 &  The structure shall have a maximum diameter not exceeding 5 [m] in stowed configuration                              \\ \hline
    CIA-B04-Struc-07 & The structure shall have a mass not exceeding \gls{tbd} [kg]\\ \hline
    \end{tabular}
    \label{tab:strucoprequirements}
\end{table}
\subsection{Thermal Protection Requirements Discovery} \label{sec:therm}
The requirements for the thermal subsystem flow down from the \gls{rdt} and are listed in Table \ref{tab:thermalreq}. The requirements are split up in two parts, the \gls{tps} and the \gls{tcs}. The \gls{tps} mainly distributes the heat load and flux generated by decelerating the re-entry vehicle, whereas the \gls{tcs} controls the temperature of the payload and other subsystems.


\begin{table}[H]
	\caption{Overview of thermal requirements}
	\begin{tabular}{|p{0.25\textwidth}|p{0.70\textwidth}|}
    \hline
    ID          & Description                                                                                                      \\ \hline \hline
    CIA-B01-TPS-01 & The TPS shall be able to withstand the maximum heat flux of \gls{tbd} $ \left[\frac{W}{cm^2}\right] $.               
\\ \hline
    CIA-B01-TPS-02 &  The TPS shall be able to withstand the maximum heat load of \gls{tbd} $ \left[\frac{J}{cm^2}\right] $.                
\\ \hline
    CIA-B01-TCS-01 & The TCS shall keep the payload and other subsystems within a specified temperature range in $\left[^{\circ}C\right]$.                                            
\\ \hline
    \end{tabular}
    \label{tab:thermalreq}
\end{table}

The first requirements 

The second requirement

The payload and subsystems are only able to withstand a certain temperature range. To keep the payload and subsystems within this range the TCS follows requirement CIA-B01-TCS-01.
In this section the methods used for determining appropriate control systems are explained. These systems should be able to keep the spacecraft on the trajectory as defined by the astrodynamics tool in section \ref{subsec:orbittool}. First the assumptions used and their effects on the accuracy of the analysis are explained. Than a point is determined

\paragraph{Assumptions}

**Primary and Secondary assumptions**

\paragraph{Trim point}

**Moment equilibrium figures/equations**
**CG-location plot(s)  with conclusion on CG for AoA~20 and sideslip angle=0**

\paragraph{Stability}

**From E.Mooij**

\paragraph{Available control systems}

**Intro**

\subparagraph{\acrfull{cg} offset}

**Not nessesary for AoA, not feasible for sideslip**

\subparagraph{Thrusters}

**Sebstiaan**

\subparagraph{Aerodynamic surfaces}

**Guido**




\subsection{List of Requirements} \label{sec:list}

\begin{table}[H]
	\caption{Overview of mission top-level requirements}
	\begin{tabular}{|p{0.10\textwidth}|p{0.85\textwidth}|}
    \hline
    ID          & Description                                                                                                      \\ \hline \hline
    CIA-A01 & The re-entry vehicle shall be able to cope with an entry velocity of seven kilometers per second.                \\ \hline
    CIA-A02 & The inflated aeroshell shall have a maximum diameter of 12 meters.                                               \\ \hline
    CIA-A03 & The system shall have a diameter not exceeding 5 meters in stowed condition                                       \\ \hline
    CIA-A04 & The maximum entry mass of the re-entry vehicle shall be 10,000 kilograms at the start of the mission.				\\ \hline
    CIA-A05 & The hypersonic deceleration system mass shall not be heavier than ten percent of the total re-entry vehicle mass. \\ \hline
    CIA-A06 & The control system shall have a maximum failure probability of 5.0e-4.                                           \\ \hline
    CIA-A07 & The maximum allowable loads on the re-entry vehicle shall be 3 Earth g's in each axis.                            \\ \hline
    CIA-A08 & The re-entry vehicle shall have a maximal aerobraking duration of ten Earth days.                                      \\ \hline
    CIA-A09 & The re-entry vehicle shall support $<$\gls{tbd}$>$ humans as payload.                         				            \\ \hline
    \end{tabular}
    \label{tab:toplevelreq}
\end{table}




