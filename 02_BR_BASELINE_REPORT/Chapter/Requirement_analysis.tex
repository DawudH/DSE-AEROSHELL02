\section{Requirement analysis}\label{ch:req}
After the functional analysis a requirement analysis can be performed. The requirements analysis process can be done in three steps. First the Mission \textcolor{red}{(Need?)} Statement is formulated. Then the system requirements are discovered, followed by a flow down of the subsystem requirements. A Systems Engineering tool to help with the representation of the list of requirements is the \gls{rdt}.

The mission need statement is formulated as follows:  Design a system to perform an entry on Mars while keeping loads within the limits of what the human body can endure while adhering to launch constraints in the form of launcher fairing and entry mass. \cite{Balasooriyan2015}

The basic guideline of the \gls{rdt} is to split the requirements up in two parts. Here the distinction is made between functional requirements and constraints. This is shown on the first sublevel. On the second sublevel the system requirements are stated. On the following sublevels a flow down is made for the susbsystem requirements.

\textcolor{red}{Insert RDT}