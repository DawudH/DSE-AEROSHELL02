\subsection{Review of aeroshell technology}\label{sec:aeroshells}

\subsubsection{Advantages of inflatable aeroshells}
Inflatable aeroshell systems provide the following advantages with respect to traditional rigid aeroshells: \cite{cassapakis, IRVEoverview} 
\begin{enumerate}
\item A lower weight is typically achieved, as investigated by Ref.\cite{Cruz} and to some extent Ref.\cite{EDLASphase1};
\item An unconstrained inflatable diameter, by the launch vehicle fairing, allows use of larger aerodynamic decelerators. As a result a lower \gls{bc} is achievable;
\item A smaller aeroshell volume fraction is required by a lack of need to use a backshell to protect the payload from aft side heating (in contrast to rigid aeroshells);
\item Effective cocooning of the payload by a rigid aeroshell diminishes accessibility. Adding access ports requires the use of additional verification and validation of thermal control design \cite{johnsonstardust}.
\item Heat is trapped by a rigid aeroshell cocooning the payload, causing potential interference with on-board payload thermal requirements.
\end{enumerate}
Of these reasons, the first two prevail for the current mission in view of constraints by the launch vehicle, limiting entry vehicle mass and diameter. A way of handling the diameter constraints, one in the launcher fairing shroud and a more relaxed one after decoupling of launcher and entry vehicle, is utilizing deployment mechanisms. These may be either mechanical or inflatable. Comparison of these two concepts yields the following characteristics in favour of inflatable structures \cite{cassapakis}:
\begin{enumerate}
\item Inflatables have a high reliability of deployment due to a self-correcting system;
\item Use of thin materials obtaining strength from inflation gas pressure reduces the weight required for inflatable systems compared to mechanical structures;
\item Packaging efficiency is higher for inflatable structures than for mechanical erectable structures;
\item Loads are absorbed over a large surface area for inflatable structures, as opposed to typical load concentrations in mechanical systems. Load concentrations require local addition of weight, typically resulting in a heavier structure;
\item Inflatables have a typically lower production cost;
\item Easily adapted to concave shapes with symmetric and curved surfaces;
\item Favorable dynamics due to the nearly constant inflation pressure induced force restoring encountered surface distortions;
\item A favorable thermal response by radiation exchange over a large area.
\end{enumerate}

\subsubsection{Investigation of Aeroshell Technology}
The aforementioned reasons, primarily lower weight and high packaging efficiency, have been key drivers in past and ongoing research in the use of inflatable structures for use in aerodynamic deceleration during (re-)entry. Primary contributor is \gls{nasa}, specifically NASA Langley Research Center, responsible for a series of tests on the feasibility and use of \gls{hiad} concepts for entry and re-entry \cite{irve1, irve2, irve3, thor}. A brief discussion on these tests follows.

\gls{irve} II successfully met its objectives, namely: "to demonstrate inflation and re-entry survivability, assess the thermal and drag performance of the re-entry vehicle and to collect flight data for comparison with analysis and design techniques used in vehicle development" \cite[p.1]{flightperfirve2}. IRVE-II consisted of a rigid, cylindrical centerbody with a deployable, conical inflatable aeroshell of a so-called stacked toroid configuration \cite{histoverviewtech,vehicleflexibilityirve2}.

IRVE-III proceeded with its primary aim to demonstrate the offset of the vehicle center of gravity on the lift-to-drag ratio of the vehicle, while subsystems were altered: support straps were added to inter- and intraconnect toroids and centerbody, the \gls{tps} was upgraded by a choice
