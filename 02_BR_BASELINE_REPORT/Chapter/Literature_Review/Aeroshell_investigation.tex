\subsection{Review of inflatable aeroshell technology}\label{sec:aeroshells}
This section gives an overview of the bistorical and present application of (controllable) inflatable aeroshell technology. This serves to provide an overview of the current state-of-the-art of inflatable aeroshell technology and applications, its potential advantages over traditional rigid aeroshells and a collection of literature to use for further investigation.

\subsubsection{Advantages of inflatable aeroshells}
Inflatable aeroshell systems provide the following advantages with respect to traditional rigid aeroshells \cite{Cassapakis1995, Hughes2005}:
\begin{enumerate}
\item A lower weight is typically achieved, as investigated by Cianciolo et al \cite{Cianciolo2010};
\item An (in principle) unconstrained inflatable diameter, by the launch vehicle fairing, allows use of larger aerodynamic decelerators. As a result a lower \gls{bc} is achievable;
\item A smaller aeroshell volume fraction is required as a consequence of a lack of need to use a backshell to protect the payload from aft side heating (in contrast to rigid aeroshells);
\item Effective cocooning of the payload by a rigid aeroshell diminishes accessibility. Adding access ports requires the use of additional verification and validation of thermal control design. This is not required for an inflatable aeroshell \cite{Johnson1997};
\item Heat is trapped by a rigid aeroshell cocooning the payload, causing potential interference with on-board payload thermal requirements. For an inflatable aeroshell without backshield, this is not the case.
\end{enumerate}
Of these reasons, the first two prevail for the current mission in view of constraints by the launch vehicle, limiting entry vehicle mass and diameter. A way of handling the diameter constraints, one in the launcher fairing shroud and a more relaxed one after decoupling of launcher and entry vehicle, is utilizing deployment mechanisms. These may be either mechanical or inflatable. Comparison of these two concepts yields the following characteristics in favour of inflatable structures \cite{Cassapakis1995}:
\begin{enumerate}
\item Inflatables have a high reliability of deployment due to a self-correcting system;
\item Use of thin materials obtaining strength from inflation gas pressure reduces the weight required for inflatable systems compared to mechanical structures;
\item Packaging efficiency is higher for inflatable structures than for mechanical erectable structures;
\item Loads are absorbed over a large surface area for inflatable structures, as opposed to typical load concentrations in mechanical systems. Load concentrations require local addition of weight, typically resulting in a heavier structure;
\item Inflatables typicall have a lower production cost;
\item Easily adapted to concave shapes with symmetric and curved surfaces;
\item The nearly constant inflation pressure induced force restoring encountered surface distortions is favourable for vehicle dynamics;
\item A favorable thermal response by radiation exchange over a large area.
\end{enumerate}

\subsubsection{Investigation of inflatable aeroshell technology}
The aforementioned reasons, primarily lower weight and higher packaging efficiency, have been key drivers in past and ongoing research in the use of inflatable structures for use in aerodynamic deceleration during (re-)entry. Primary contributor is \gls{nasa}, specifically the NASA Langley Research Center, responsible for a series of tests on the feasibility and use of \gls{hiad} concepts for entry and re-entry \cite{Hughes2005, Dillman2010, Dillman2012, Dillman2014}. A brief discussion on these tests follows. Little information on gls{irve}-I \cite{Hughes2005} is present, hence it is not included in the following discussion.  

\gls{irve}-II, launched in 2009, successfully met its objectives, namely: "to demonstrate inflation and re-entry survivability, assess the thermal and drag performance of the re-entry vehicle and to collect flight data for comparison with analysis and design techniques used in vehicle development" \cite[p.1]{Dillman2010}. IRVE-II consisted of a rigid, cylindrical centerbody with a deployable, conical inflatable aeroshell of a so-called stacked toroid configuration \cite{Smith2010,Bose2009}.

IRVE-III, launched in 2012, had as primary aim to demonstrate the offset of the vehicle \gls{cg} on the lift-to-drag ratio of the vehicle and demonstrate survability with flight-relevant heating \cite{Dillman2012a}. The configuration was similar to IRVE-II and subsystems were altered predominantly in the following  manners \cite{Dillman2012a}:
\begin{itemize}
\item Support straps were added to inter- and intraconnect toroids and centerbody;
\item The \gls{tps} was upgraded by use of a multi-layer system, consisting of two layers of Nextel fabric, Pyrogel insulation and a Kapton/Kevlar thin film gas barrier, in place of the Nextel fabric used in IRVE-II \cite{Dillman2012}; 
\item A heater was added to the pressure tank system used to inflate the stracked toroid structure;
\item The addition of a \gls{cg} offset mechanism to alter the lifting behavior of the vehicle and thereby control it.
\end{itemize}
IRVE-III succeeded in its goals, demonstrating the feasibility of a \gls{cg} offset for vehicle control \cite{Dillman2012}.

\gls{thor}, planned for launch in 2016, features a more blunt aeroshell with a half-cone angle of 70 instead of 60 degrees, to analyze stability and drag differences with previous configurations \cite{Hughes2005, Dillman2010, Dillman2012, Dillman2014}. In addition, it features Zylon instead of the Kevlar fibres used for IRVE-III, allowing a thinner \gls{tps} layup of a different composition. In terms of \gls{tps}, silicon carbide fabric over carbon felt insulation is used instead of Nextel fabric over Pyrogel insulation \cite{Dillman2014}.

In addition to these flight tests, ground testing has been pursued in parallel to further technology developments for \gls{hiad} application \cite{Smith2010}.

Some of the most important characteristics of the re-entry vehicles during these missions are displayed in Table \ref{tab:hiadcomparison}.

\begin{table}[h!]
	\caption{Comparison of recent HIAD missions}% CAPTION HERE !
		\begin{tabular}{|p{0.28\textwidth}|p{0.08\textwidth}|p{0.15\textwidth}|p{0.15\textwidth}|p{0.21\textwidth}|} % MAKE SURE THAT THE TOTAL WIDTH IS 0.95\textwidth!! (that way its exactly the textwidth.... haha) 
			\hline

       Mission parameter   &       Unit &     IRVE-II \cite{Dillman2010} &     IRVE-III \citep{Dillman2012,Dillman2014} & THOR (predicted) \citep{Dillman2014} \\
			\hline \hline

Launch date &          [-] & 17-08-2009 & 23-07-2012 &       2016 \\
			\hline

      Mass &         [kg] &    124.6 &        280 &        315 \\
			\hline

Shell diameter &          [m] &       2.93 &       2.93 &        3.7 \\
			\hline

Shell angle &     [deg] &         60 &         60 &         70 \\
			\hline

    Apogee &         [km] &        218 &        469 &    200-250 \\
			\hline

Peak dynamic pressure &         [Pa] &       1180 &   Unknown         &   Unknown         \\
			\hline

Peak stagnation heating &     $ [\frac{W}{cm^{2}}]$ &        2.2 &       14.4 &         65 \\
			\hline

Peak temperature &          [C] &        100 &        378 &      Unknown      \\
			\hline

Peak Mach Number &          [-] &        6.2 &  Unknown          &   Unknown         \\
			\hline

Maximum deceleration &          [g] &        8.5 &       20.2 &       8-10 \\
			\hline

		\end{tabular}
    \label{tab:hiadcomparison}% LABEL HERE
\end{table}

