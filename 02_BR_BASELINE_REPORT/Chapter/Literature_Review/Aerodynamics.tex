\subsection{Aerodynamics for (re-)entry vehicles} \label{sec:aero}
Ground testing of vehicle aerodynamics is difficult, as the high temperatures experienced during hypersonic flight may damage testing facilities \cite{AndersonJr.2006, Bertin1994}. For the solution of the full flow field around a body in hypersonic flow, \gls{cfd} is required. This solution requires significant knowledge of the problem to implement, is computationally expensive and difficult to validate \cite{AndersonJr.2006, Bertin1994}. A useful engineering method for determining the aerodynamic characteristics of an arbitrary body in hypersonic flow is the (modified) Newtonian theory. It provides an acceptable approximation of the pressure distribution on the side of the body exposed to the free stream. From this pressure distribution aerodynamic coefficients can be determined. As long as the body drag is dominated by the pressure drag, the modified Newtonian method provides a good approximation of the body drag coefficient \cite{AndersonJr.2006, Bertin1994, Bertin2006}. 

Bodies with a high ballistic coefficient experience higher peak heat loads. This has led to the development of blunt (re-)entry vehicles \cite{Bertin1994,Theisinger2009}. A method for determining an optimal shape for a (re-)entry vehicle is presented in Ref.\cite{Theisinger2009}. A nonzero $\frac{\gls{sym:L}}{\gls{sym:D}}$ ratio allows downrange and cross range control of a reentry vehicle \cite{Theisinger2009}. Hypersonic flows show an interesting characteristic known as Mach number independence; the aerodynamic coefficients of a body in hypersonic flow (\gls{sym:M}$>$5) are independent of Mach number \cite{Bertin1994,AndersonJr.2007,Hollis}. No analytical engineering method exists for supersonic flow around a blunt body. An implementation of a time marching finite difference scheme is proposed in references \cite{AndersonJr.2007} and \cite{AndersonJr.2006} to describe the flow field around an arbitrary body in supersonic flow. 

The aerothermodynamics of hypersonic flight are complex. In contrast with subsonic and supersonic aerothermodynamics during hypersonic flight internal chemical reactions and gas composition changes also occur because of the high temperatures encountered \cite{AndersonJr.2006}. This significantly increases the computational cost of \gls{cfd} methods. In order to estimate the heat flux into the vehicle body therefore the approximate method outlined in \cite{Tauber1986, AndersonJr.2006} is proposed. This method uses the same input parameters as the modified Newtonian method that will be used for the drag estimation and is adaptable to both laminar and turbulent flow around arbitrary bodies in hypersonic and supersonic flows. From this method it follows that the local heat flux into the arbitrary body is dependent on the freestream density and velocity as well as on local body angle with respect to the undisturbed freestream flow, distance measured along the body surface from the stagnation point and local entropy ratio between the wall and total conditions \cite{Tauber1986, AndersonJr.2006}.