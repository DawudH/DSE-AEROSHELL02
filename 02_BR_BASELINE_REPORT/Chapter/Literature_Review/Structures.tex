\subsection{Structures in (re-)entry vehicles}\label{sec:struc}
Applications and technologies, within the field of structures and materials, as applied in (re-)entry vehicles are investigated, with an emphasis on inflatable aeroshells. These are investigated firstly for non-inflatable structures and secondly for inflatable structures.

\subsubsection{Non-inflatable structures}
Non-inflatable structures can be either deployable or non-deployable. Conventional solutions, such as the Apollo or Soyuz capsule, employ a non-deployable heat shield. An advantage of a deployable heat shield is effecting a larger surface area for aerodynamic deceleration. In most cases, this is supplemented by retropropulsive means.



\subsubsection{Inflatable structures}
The inflatable aeroshell system implemented in the \gls{irve} satellites mainly consists of four sub-elements: an inflatable bladder, containing a pressurised medium, a structural restraint, gas barrier and a thermal protection layer \cite{Hughes2005}. In addition, this bladder may be further subdivided into isolated volumes to provide avoid \gls{spfs}. After flow initiation with pyrotechnic valves the gas flows from the storage tank to the inflatable bladder. This flow is protected by gas valves to prevent backflow from the bladders. \cite{Hughes2005} 

In terms of the inflation process, the pressure and gas used for inflation are variable. The IRVE satellites featured nitrogen gas (and subliming powders), with an operating pressure of 3000 psi for IRVE-4 \cite{Litton2011}. An alternative to the use of nitrogen gas is hydrazine, typically capable of delivering lower weight and volume, at the expense of handling, safety and cost \cite{Freeland1998}. Estimating the required minimum pressure can be done using references \cite{Samareh2011, Brown2009}.

In addition to pure inflation, rigidization may be applied. Rigidization stiffens the structure after inflation, a process that may be performed by multiple techniques. These techniques are described in Ref. \cite{Freeland1998,Jenkins2001}, for example using fibres impregnated with a resin that cures at a certain temperature. 






Inflatable structures may be preferable to conventional non-inflatable structures for a number of structural reasons. Most importantly inflatable structures typically have a lower weight. A weight estimation method for multiple types of inflatable deceleration mechanisms is provided by \cite{Samareh2011}.






