\subsection{Control} \label{sec:control}
The control of a vehicle encompasses the dynamics model of the spacecraft, implementing a controller in that model and designing the sensors and actuators on the spacecraft that perform the actual control. This subsection focuses first on dynamics and controller design, then on the implementation in hardware.
The orbit to allow an aerocapture requires a simulation of the spacecraft through both space and atmosphere. A comprehensive book on this is Computational Orbital Mechanics, \cite{Weiland2004}, while a thesis on an implementation in MATLAB is found in \cite{Leszczynski1998}.
The book \cite{Mulder2013} gives an introduction to flight dynamics in general, giving an introduction to stability and controllability of aerospace vehicles. However, it is focused on aircraft. A more in-depth study of stability and control of spacecraft can be found in \cite{Steketee1967} and \cite{Ito2002}. One of the problems of inflatable structures is the fact that they deform more under loads than rigid structures. The effect of this on the dynamics of the vehicle is given in \cite{Axdahl2009} as well as in \cite{Bose2009}, where particularly this effect on \gls{irve2} is detailed. For a general introduction into control systems engineering, the book \cite{Nise2011} is available.

An overview of state measurement components can be found in \cite{Wertz2011a}. If no gyrometer can be implemented, the angular rate can be estimated based on vector measurements \cite{Azor1998}.

Finally, the actuators will allow the spacecraft to actually follow the prescribed trajectory and orient it such that the appropriate side of the spacecraft is pointed towards the flow. Several typical ways of actuating the orientation vehicle are given in \cite{Wertz2011a}. For hypersonic flows with ionized boundary layers, \gls{mhd} can be used to control the flow and produce a moment. An overview of current state of technology can be found in \cite{Braun2009}. An overview of retropropulsion for Mars entry can be found in \cite{Korzun2009}, which may prove useful if 