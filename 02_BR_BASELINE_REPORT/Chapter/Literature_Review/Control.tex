\subsection{Control of (re-)entry vehicles} \label{sec:control}
The control of a vehicle encompasses the dynamics model of the spacecraft, implementing a controller in that model and designing the sensors and actuators on the spacecraft that perform the actual control. This subsection focuses first on dynamics and controller design, then on the implementation in hardware.

\subsubsection{Orbit}
The orbit to allow an aerocapture requires a simulation of the spacecraft through both space and atmosphere. A comprehensive book on this is \textit{Computational Orbital Mechanics} \cite{Weiland2004}, while a MATLAB implementation can be found in Ref.\cite{Leszczynski1998},. In determining the orbit of the satellite, atmospheric properties are very important. A program developed by the NASA named \gls{marsgram} is used to determine the atmospheric properties at all longitudes, latitudes and heights.\cite{Justus2001} Lift may be used to control the trajectory, a control method to incorporate this can be found in Ref.\cite{Esmaelzadeh2010}.

\subsubsection{Flight Dynamics}
The book by Mulder et al. gives an introduction to flight dynamics in general, introducing stability and controllability of aerospace vehicles. However, it is focused on aircraft. \cite{Mulder2013} A more in-depth study of stability and control of spacecraft and re-entry vehicles can be found in Ref.\cite{Steketee1967, Ito2002}. One of the problems of inflatable structures is the fact that they deform more under loads than rigid structures.\cite{Axdahl2009} Particularly this effect on \gls{irve}-II is detailed in Ref.\cite{Bose2009}. The book Control Systems Engineering is available for a general introduction in control engineering.\cite{Nise2011}

\subsubsection{Sensing}
An overview of state measurement components can be found in Ref.\cite{Wertz2011}. If no gyrometer can be feasibly implemented, the angular rate can be estimated based on vector measurements.\cite{Azor1998}

\subsubsection{Actuation}
The actuators will allow the spacecraft to actually follow the prescribed trajectory and orient it such that the appropriate side of the spacecraft is pointed towards the flow. Several typical ways of actuating the orientation of the vehicle vehicle are given in Ref.\cite{Wertz2011}. An executive overview of the thruster system of the Orion capsule, which is comparable to the present spacecraft, can be found in Ref.\cite{Jones2012}. For hypersonic flows with ionized boundary layers, \gls{mhd} can be used to control the flow and produce a moment. An overview of current state \gls{mhd} technology can be found in Ref.\cite{Braun2009}, prediction of forces can be found in Ref.\cite{Kawamura2013}. Use of retropropulsion for Mars entry has been researched in Ref.\cite{Nise2011}, which shows a beneficial interaction between retropropulsion and hypersonic deceleration in the atmosphere. Center of gravity offset can be used to control orientation as well: it can be used to provide roll control on lift-generating vehicles\cite{Petsopoulos1996}, while \gls{irve3} used it to introduce an angle of attack to generate lift.\cite{Dillman2012a} At lower Mach numbers (partial) control can also be performed by body flaps as was for example done in the \gls{nasa} SpaceShuttle missions.