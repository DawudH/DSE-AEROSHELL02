\subsection{Overview of present and past (re-)entry vehicles}\label{cha:past missions}
This section gives an overview of (re-)entry vehicles, primarily to obtain a set of reference vehicles to aid design and sizing of the payload capsule in later stages and additionally to review solutions used in the past to perform (re-)entry for human spaceflight. At this point no structure supporting the deceleration has been chosen yet, as such only payload capsule size parameters are considered. It can be argued, based on human payload constraints, that the attached payload capsule for this mission will have similar characteristics. Table \ref{tab:refmis} displays some characteristics related to the payload capsule which can be used as indicative values \footnote{Principal values from: \\
URL: \url{http://www.nasa.gov/sites/default/files/167718main\_early\_years.pdf},  Accessed: 28 April 2015 \\ URL: \url{http://www.braeunig.us/space/} Accessed: 28 April 2015 \\ \url{URL: http://wsn.spaceflight.esa.int/docs/Factsheets/35\%20Soyuz\%20LR.pdf} Accessed: 28 April 2015 \\
URL: \url{http://www.lpi.usra.edu/lunar/constellation/orion/factsheet.pdf} Accessed: 28 April 2015}. 

\begin{table}[H]
	\caption[Reference missions for payload module sizing]{Reference missions for payload module sizing.}
		\begin{tabular}{|p{0.18\textwidth}|p{0.11\textwidth}|p{0.11\textwidth}|p{0.11\textwidth}|p{0.11\textwidth}|p{0.11\textwidth}|p{0.11\textwidth}|} % MAKE SURE THAT THE TOTAL WIDTH IS 0.95\textwidth!! (that way its exactly the textwidth.... haha) 
			\hline
			Mission 						& Apollo & 	Soyuz TMA &	Shenzhou & Gemini & Mercury & Orion \\ \hline \hline
			Years [$yr$]					&	1964-1975	& 	2010-2014&	1999- &   1959-1963  & 1959-1963 & Future \\ \hline
			Reentry module mass [$kg$]  	&	5806& 	2900 &	3240 & 3402 & 1118 & 8777 \\ \hline
			Habitable volume [$m^3$]		&	6.17& 	3.5  &	6.0  & 2.55 & 1.9 & 11   \\ \hline
			Diameter [$m$]			 		&	3.9 & 	2.17  &	2.52 & 2.3 & 1.9 & 5   \\ \hline
			Length  [$m$]			 		&	3.5 & 	2.24  &	2.5  & 3.4 &  5.2 & Unknown  \\ \hline
			Crew size (max) [$persons$]		&	3   & 	3     &	3    & 2   &  1   & 6   \\ \hline
		\end{tabular}
    \label{tab:refmis}
\end{table}

It must be noted that table \ref{tab:refmis} displays typical values only to be used as first indicative values. For example the diameter is typically a maximum value since no single value can be supplied due to the cone like shape of most reentry vehicles. Moreover these designs include the size and mass of the deceleration mission of which the latter typically includes a heavy duty heat shield. Most re-entry vehicle base designs were used multiple times with minor design changes and a single externally communicated design name. As such the values in the table above should be used with proper care as indicative values only. Habitable volume estimation also depends on the mission duration and may be considerable. A study on the estimation of these parameters is given by \cite{Rudisill2008}. Although this study focuses on a lunar mission it still underlines many of the important aspects with respect to payload module sizing which are applicable for Mars missions as well. It may as such prove a proper foundation for payload module sizing. 

From the above mentioned reference missions in table \ref{tab:refmis} especially the future Orion mission, currently being designed, is of great interest. The orion crew exploration vehicle is planned to go to the moon, mars and further in the solar system. Being designed for similar missions distances using present day technologies. The orion mission can as such be considered as the primary reference payload appended with the other reference payloads.


