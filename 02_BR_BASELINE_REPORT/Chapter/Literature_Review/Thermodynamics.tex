\subsection{Thermodynamics in (re-)entry vehicles}\label{sec:thermo}
Thermodynamics is used to make sure that components of the re-entry vehicle stay within a certain temperature range. These temperature ranges are a function of the intented use of the components and material selection. This literature review is broken down in three major parts. The first briefly explains the principles needed to describe the transfer of heat within structures. The second concerns the heatshield needed for the re-entry phase, which is heavily linked to the aerodynamics. The third deals with the thermal control of the capsule itself. For example, the temperature inside of the capsule should be suitable for human payloads.

\subsubsection{Thermodynamic principles}
At the start of the mission the re-entry vehicle can be seen as a closed system that has a total energy (E) in the form of internal energy (U), kinetic energy (KE) and potential energy (PE). For successful re-entry the kinetic and potential must be reduced. This can be done by transferring the energy across the boundary of the system. There are three methods to transfer the energy. Energy transfer by heat, by work and by mass flow. The latter requires the system to be open such that mass is allowed to leave the system. An example of energy transfer by heat is the heating of gas near the wall of the heatshield and heating of the heatshield itself. Energy transfer by work could for example be the work done by the skin friction drag. The use of thrusters is an example of energy transfer due to mass flow as the propellant mass flows out of the open system. %Cite thermo-cengel

Energy transfer by heat, or simply heat transfer has three modes: conduction, convection and radiation. Conduction is the transfer of heat between particles of a material due to interactions between these particles. Convection is the transfer of heat between a solid surface and a moving fluid. Radiation is transfer of heat due to the emission of electromagnetic energy from a surface to its surroundings. %Cite thermo-cengel and karam
Each of these modes can be described by governing equations as described in ??. Radiation is given by the Stefan-Boltzmann law and conduction by Fourier's law.%Cite holman and karam



\subsubsection{Heatshield for non-inflatable structures}
Blabla

\subsubsection{Heatshield for inflatable structures}
Blabla

\subsubsection{Thermal control of the capsule}
Blabla




