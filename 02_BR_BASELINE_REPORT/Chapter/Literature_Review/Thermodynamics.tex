\subsection{Thermodynamics in (re-)entry vehicles}\label{sec:thermo}
Thermodynamics is used to make sure that components of the re-entry vehicle stay within a certain temperature range. These temperature ranges are a function of the intented use of the components and material selection. This literature review is broken down in three major parts. The first briefly explains the principles needed to describe the transfer of heat within structures. The second concerns the heatshield needed for the re-entry phase, which is heavily linked to the aerodynamics. The third deals with the thermal control of the capsule itself. For example, the temperature inside of the capsule should be suitable for human payloads.

\subsubsection{Thermodynamic principles}

\subsubsection{Heatshield for non-inflatable structures}
Blabla

\subsubsection{Heatshield for inflatable structures}
Blabla

\subsubsection{Thermal control of the capsule}
Blabla




