\subsection{Thermodynamics in (re-)entry vehicles}\label{sec:thermo}
Thermodynamics is used to make sure that components of the re-entry vehicle stay within a certain temperature range. These temperature ranges are a function of the intented use of the components and material selection. This literature review is broken down in three major parts. The first briefly explains the principles needed to describe the transfer of heat within structures. The second concerns the heatshield needed for the re-entry phase, which is heavily linked to the aerodynamics. This is split up in non-infaltible and inflatible heat shields. The third deals with the thermal control of the capsule itself. For example, the temperature inside of the capsule should be suitable for human payloads.

\subsubsection{Thermodynamic principles}
At the start of the mission the re-entry vehicle can be seen as a closed system that has a total energy (E) in the form of internal energy (U), kinetic energy (KE) and potential energy (PE). For successful re-entry the kinetic and potential must be reduced. This can be done by transferring the energy across the boundary of the system. There are three methods to transfer the energy. Energy transfer by heat, by work and by mass flow. The latter requires the system to be open such that mass is allowed to leave the system. \cite{Cengel2010} An example of energy transfer by heat is the heating of gas near the wall of the heatshield and heating of the heatshield itself. Energy transfer by work could for example be the work done by the skin friction drag. The use of thrusters is an example of energy transfer due to mass flow as the propellant mass flows out of the open system. 

Energy transfer by heat, or simply heat transfer has three modes: conduction, convection and radiation. Conduction is the transfer of heat between particles of a material due to interactions between these particles. Convection is the transfer of heat between a solid surface and a moving fluid. Radiation is transfer of heat due to the emission of electromagnetic energy from a surface to its surroundings. \cite{Cengel2010}\cite{Karam1998} Each of these modes can be described by governing equations as described in \cite{Holman2002}. Radiation is given by the Stefan-Boltzmann law and conduction by Fourier's law. \cite{Cengel2010}\cite{Holman2002}


\subsubsection{Thermal protection system}
The thermodynamic principles can be used to design a heatshield. The design of such a heat shield, also called a Thermal Protection System (\gls{tps}) depends on several parameters. In general it all comes down to the trajectory the re-entry vehicle will follow. The steeper the re-entry trajectory, the larges the descelleration and the more g-loads will be subjected to the (human) payload. The re-entry vehicle will have a nominal trajectory with deviations. An overshoot results in a longer descend and can result in skipping of the plannet, whereas an undershoot will result in a fast descend rate. The larger the overshoot angle, the longer the re-entry vehicle will be in atmoshere an thus creating a higher heat loading on the structure. On the other hand, for a larger undershoot angle, the heat developement over time will be faster inducing a larger heat flux on the system. Also, the undershoot angle determines the trajectory with the heighest dynamic pressure, which is important for structural sizing. 

Not only the trajectory is important for the design, but also the shape, its blundness. The bigger the heat shield and the larger the blundness, the more the  thermal loads can be distributed over the capsule. This will result in a thinner TPS such that less mass is needed. However, the compsule needs aerodynamic stability. A blund body has a forward airodynamic center [REFERENCE NEEDED], causing 

The temperature developement throughout the structure can be determined when the heat loading and heat flux is know. This can be done with the heat equation 

\paragraph{Heatshield for non-inflatable structures}

Non-inflatible heat shields have been used in many missions. Some examples are 

\paragraph{Heatshield for inflatable structures}

Blabla

\subsubsection{Thermal control of the capsule}
For the thermal control of the crewcapsule the book on Thermal Control by R. Karam is leading. \cite{Karam1998}




