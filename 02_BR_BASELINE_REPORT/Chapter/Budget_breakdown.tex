\section{Budget breakdown} \label{ch:budget}
This section will describe the mass breakdown of the re-entry vehicle. It will be split in two parts; the first part will give a mass breakdown of the crew compartment, the second part will provide an initial mass breakdown of the hypersonic decelerator. These breakdowns are extrapolated from existing re-entry vehicles and design studies.

\subsection{Crew Module}
From requirements CIA-A04 and CIA-A05, it can be determined that the crew module mass is 9000kg at re-entry. This is roughly 1.6 times the mass of the Apollo Command Module \footnote{URL: \url{http://braeunig.us/space/specs/apollo.htm}, Accessed 30 April 2015}, and is roughly equal to the Orion Multipurpose Crew Vehicle \footnote{URL: \url{http://www.spaceflight101.com/orion-spacecraft-overview.html}, Accessed 30 April 2015}. The Apollo missions housed three crew members, while the Orion will house 4 to 5 crew members. Realizing that the number of crew members carried appears to scale with the vehicle mass, and taking into account the mass of the heat shield of the Apollo and the Orion vehicles, the 9000 kg of re-entry mass is likely to be able to support a crew of six. It is assumed that each crew member weighs 85 kg, and that a mission payload of an additional 1000kg is carried within the crew module. 

The remaining subsystem masses of the crew module are estimated based on either the Orion or the Apollo vehicles. Subsystem masses are scaled linearly based on the number of crew members they were designed to accommodate for. The complete mass breakdown for the crew module can be found in table \ref{tab:CVMB}.

\begin{table}[H]
	\caption{Crew Module Mass Breakdown}
	\begin{tabular}{|c|c|c|}
    \hline
    Subsystem        					& Mass[kg] 	& Fraction [\%] \\ \hline \hline
    Vehicle Structure 				& 3000			& 33.3 					\\ \hline 
		Subsonic Re-entry System	& 1000			& 11.1					\\ \hline
		Martian Mission Payload 	& 1000			& 11.1 					\\ \hline
		Furnishing								& 600				& 6.7 					\\ \hline
		Crew											& 540				& 6.0						\\ \hline
		Navigation Equipment			& 500				& 5.6						\\ \hline
		Electronic Equipment 			& 500				& 5.6						\\ \hline
		Environmental Control			& 450				& 5.0						\\ \hline
		Batteries									&	450				&	5.0						\\ \hline
		Communication systems			& 300				& 3.3						\\ \hline
		Telemetry									& 200				& 2.2						\\ \hline
		Mass contingency					&	460				& 5.1						\\ \hline
		Total											&	9000			& 100						\\ \hline
    \end{tabular}
    \label{tab:CVMB}
\end{table}

\subsection{Hypersonic Decelerator}
The mass breakdown of the hypersonic decelerator is based completely on literature, until more detailed information becomes available later in the design. The primary reference for the initial mass breakdown is a NASA design study of Martian Aerocapture missions \cite{Cianciolo2010}. The allocated mass fractions for the subsystems were kept similar. A $\Delta V$ budget of $150 ms^{-1}$ for reaction control was suggested in the design study. This corresponds to a propellant mass of roughly 50kg per the Tsiolkovsky equation. The complete initial mass budget can be found in table \ref{tab:MassBudget}

\begin{table}[H]
	\caption{Hypersonic Decelerator Mass Budget}
	\begin{tabular}{|c|c|c|}
    \hline
    Subsystem        					& Mass[kg] 	& Fraction [\%] \\ \hline \hline
    Thermal Protection system	& 500				& 50 						\\ \hline 
		Spacecraft structure			& 350				& 35						\\ \hline
		\textit{Aeroshell}				& 200				& 20 						\\ \hline
		\textit{Connection}				& 150				& 15						\\ \hline
		Reaction Control System		& 150				& 15						\\ \hline
		\textit{System Dry Mass}	& 100				& 10						\\ \hline
		\textit{Propellant Mass}	& 50				& 5							\\ \hline
		
    \end{tabular}
    \label{tab:MassBudget}
\end{table}



