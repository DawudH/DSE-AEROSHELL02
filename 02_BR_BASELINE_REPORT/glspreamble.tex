% Generate the glossary
	% create a new glossary style for the list of symbols
	% copied (but eddited) from http://www.latex-community.org/forum/viewtopic.php?f=5&t=20797
	\newglossarystyle{listos}{%
	  \glossarystyle{altlongragged4col}
	  \setlength{\glsdescwidth}{0.8\textwidth}
	  % allow line wrap in the description column
	  \renewenvironment{theglossary}%
		    {\begin{longtable}{llp{\glsdescwidth}}}%
		    {\end{longtable}}%
		  \renewcommand{\glsgroupskip}{}% make nothing happen between groups
		  \renewcommand*{\glossaryheader}{%
		  \bfseries Symbol & \bfseries Unit & \bfseries Description \\\endhead}%
		  % No heading between groups:
		  \renewcommand*{\glsgroupheading}[1]{}%
		  % Main (level 0) entries displayed in a row optionally numbered:
		  \renewcommand*{\glossentry}[2]{%
		  \glsentryitem{##1}% Entry number if required
		  \glstarget{##1}{\glossentryname{##1}}% Name
			& \glossentrysymbol{##2}% Unit
			& \glossentrydesc{##2}% Description
			\tabularnewline % end of row
		  }%
		  % Similarly for sub-entries (no sub-entry numbers):
		  \renewcommand*{\subglossentry}[3]{%
		  % ignoring first argument (sub-level)
		  \glstarget{##2}{\glossentryname{##2}}% Name
			& \glossentrysymbol{##2}% Unit
			& \glossentrydesc{##2}% Description
			\tabularnewline % end of row
			}%
			% Nothing between groups:
			\renewcommand*{\glsgroupskip}{}%
			}
	
	% create a new glossary style for the list of constants
		% copied (but eddited) from http://www.latex-community.org/forum/viewtopic.php?f=5&t=20797
		\newglossarystyle{listoc}{%
		  \glossarystyle{altlongragged4col}
		  \setlength{\glsdescwidth}{0.8\textwidth}
		  % allow line wrap in the description column
		  \renewenvironment{theglossary}%
		    {\begin{longtable}{lllp{\glsdescwidth}}}%
		    {\end{longtable}}%
		  \renewcommand{\glsgroupskip}{}% make nothing happen between groups
		  \renewcommand*{\glossaryheader}{%
		  \bfseries Symbol & \bfseries Value & \bfseries Unit & \bfseries Description \\\endhead}%
		  % No heading between groups:
		  \renewcommand*{\glsgroupheading}[1]{}%
		  % Main (level 0) entries displayed in a row optionally numbered:
		  \renewcommand*{\glossentry}[2]{%
		  \glsentryitem{##1}% Entry number if required
		  \glstarget{##1}{\glossentryname{##1}}% Name
			& \glsentryuseri{##2}% Value
			& \glossentrysymbol{##2}% Unit
			& \glossentrydesc{##2}% Description
			\tabularnewline % end of row
		  }%
		  % Similarly for sub-entries (no sub-entry numbers):
		  \renewcommand*{\subglossentry}[3]{%
		  % ignoring first argument (sub-level)
		  \glstarget{##2}{\glossentryname{##2}}% Name
			& \glsentryuseri{##2}% Value
			& \glossentrysymbol{##2}% Unit
			& \glossentrydesc{##2}% Description
			\tabularnewline % end of row
			}%
			% Nothing between groups:
			\renewcommand*{\glsgroupskip}{}%
			}

	
\newglossary*{symbol}{List of Symbols}
\newglossary*{constants}{List of Constants}
\makenoidxglossaries
\setacronymstyle{long-short}
\loadglsentries{./Acronyms}
\newglossaryentry{romanletter}{type=symbol,name={},description={\nopostdesc},sort=a}
\newglossaryentry{greekletter}{type=symbol,name={},description={\nopostdesc},sort=b}
\newglossaryentry{romanletterc}{type=constants,name={},description={\nopostdesc},sort=a}
\newglossaryentry{greekletterc}{type=constants,name={},description={\nopostdesc},sort=b}
\loadglsentries{./Symbols}
\loadglsentries{./Constants}