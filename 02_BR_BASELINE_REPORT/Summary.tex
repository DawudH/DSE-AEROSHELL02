\section*{Summary}\label{cha:summary}

To fulfill the need to carry human payload to Mars one solution is a \acrfull{cia}. This project has to demonstrate the feasibility of such a Controllable Inflatable Aeroshell. The first steps in organizing the design process have been summarized in the \acrfull{pp}. The next step which is taken within this report is a literature study to familiarize with the subject. This literature study consists of an analysis of past solutions alongside with an investigation into the primary disciplines involved in the design of a re-entry mission. This is followed upon by a functional and requirements analysis is performed to generate design options for the different aspects of the mission. This baseline report also comprises a market analysis, budget breakdown, risk assessment and an approach with respect to sustainable development. A tool is developed and presented to make first order predictions for the astrodynamic characteristics. 

To arrive at mission concepts, the first step is a functional analysis. In this analysis a \acrfull{ffd} is made to show the logical order of the functions the re-entry vehicle must perform. The functions are in addition categorized in a \acrfull{fbs}. Then the requirements are analyzed. This is done using a \gls{rdt} which shows how different subsystem requirements flow down from the top level requirements and constraints. The list of requirements is stated in Appendix \ref{app:req}.

A market analysis is performed which aims to minimize the risk of selecting a wrong combination of function and technology for the customer. A \acrfull{swot} analysis provides an overview of the primary characteristics of the proposed \gls{cia}. The design process is exposed to several risks, primarily schedule overruns and insufficient technical performance. In order to identify, analyze and manage risks a risk mitigation plan is made, which uses risk mapping as a qualitative method and \gls{trb} as a quantitative method. Within this mission, sustainability will not have the highest priority in the design process since the total impact of a single interplanetary mission on the environment is relatively small. It will, however, play a role when faced with design choices. If budgets and technical performance allow, a more sustainable design is preferred.

After this the initial design concepts are generated. For this a \gls{dot} is used. First a \gls{dot} for each of the following aspects is made: the trajectory, the shape and the control. In the separate trees the early infeasible options are eliminated. Then feasible combinations of the remaining design options in the three trees are sought.

Future work includes narrowing down the design options to five concepts. In the \acrfull{mtr} these concepts will be reviewed and a trade-off process will help with the selection of the final concept that shall be designed with more detail for the \acrfull{fr}.
