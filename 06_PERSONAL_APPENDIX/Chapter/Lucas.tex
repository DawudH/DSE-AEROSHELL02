\section{Lucas Mathijssen}
During the span of the ongoing DSE there are several tasks that I have performed. Those can be split up in two main directions. My main (technical) task was being part of the thermodynamics department, whereas the secondary task was being the secretary. This last task one spanned over the first half of the DSE.

In the thermodynamics department I first obtained general knowledge by reading relevant papers, websites, books, etc. Afterwards I focussed more on the design of the Thermal Protection System (TPS). There I obtained knowledge on the aeroshell thermal protection materials, their properties and thermal interactions between materials, when exposed to high temperatures. It also included updating knowledge on the corresponding equations and technical terms. Afterwards, Suthes and I started working on a tool that could calculate the temperature difference over input lay-ups. Unfortunately at first there was an error in this program that could not be solved. Therefore, Suthes and I made a new plan, where the heat flux was integrated to a heat load. I worked on the programming of this and also on generating the relevant plots. In the second half of the project the previously described error was fixed and the temperature distribution through the material could be calculated. Afterwards, the data from the tool was used to optimise the TPS thickness for the previously selected materials. 

As a secretary I made minutes of the status meetings with the group and tutors. This task required writing during the meetings and processing the text afterwards. To do so in a consistent way, I set up a minute format. Furthermore, I tried to keep track of the action items. Moreover, I kept track of the tasks of all the group members after each ‘end of day’ meeting. I put all previously mentioned items together in one file, the logbook. For this book, I another made a format. As a last secretary tasks, I was responsible for all internal and external communication. This task included keeping the mailbox ordered and being the person through which all mail communication flows. It also included reserving meeting rooms and communicating this information to team members and tutors.

I think I could have performed better as a secretary by keeping the logbook more up-to-date.  Over all I am satisfied with my performance, especially in the second half of the DSE, where the work was more technically detailed. I noted that in my technical work that a large part of the technical work is to tell others about your work and convince them it is the right thing to do. communicational skills cannot be underestimated.

My impression of the team is good. The team is working hard to deliver assignments on time and we have obtained a lot of knowledge on a short time scale. What is very typical, however, is that the team is having a lot of discussions about all sorts of technical details. The efficiency of the group can be improved by keeping some of these discussions in sub groups. On the other hand in conceptual work the team needs to be aware of a lot of discussions.

Help from the tutors has been helpful. There is a clear distinction between the role of the customer and the tutor. Also, the tips and knowledge shared by the tutors is given only if needed and it usually is a push in the right direction. Guidance was given to think also about practical aspects of the design, like for instance the connection between TPS layers.
