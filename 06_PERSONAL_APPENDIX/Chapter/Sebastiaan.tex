\section{Sebastiaan van Schie}
For the Project Plan (PP) I made the initial risk assessment. This included determining which elements would be featured in the risk map, deciding on how to structure the labels on the risk map x-axis and y-axis as well as the technical resource budgeting.

During the Baseline Review part of the Design Synthesis Exercise (DSE) I was part of the group that performed the aerodynamics literature research and tried to determine what methods were suitable to be used to analyse the aerodynamic behaviour of the concepts that were considered. This presented a challenge since for hypersonic flows the non-linear aerodynamics are very difficult to capture accurately without using Computational Fluid Dynamics. In addition to doing a literature survey I made the initial risk assessment.

My main contributions towards the Mid-Term Review were part of the aerodynamic concept analysis. This consisted of deciding on the exact method to be used for the heat flux determination and performing the validation of the modified Newtonian flow and heat flux models. In addition to this I conceived the risk map elements and made the risk map featured in the Mid-Term Review report. Lastly I wrote part of the text pertaining to our approach with respect to sustainable development.

Following the Mid-Term Review I transitioned into being one of the people concerned with designing the control actuation system. This consisted of a small literature study and the conceptual design of the control actuation system. In addition to this the group roles that were initially assigned were switched around. Thus instead of being the risk engineer I became one of the editors.

Finally I was of course part of the group meetings and discussions during all of the project sessions. 

During this DSE-project I have gotten the impression our group was spent more time developing computation and analysis tools than most of the other groups. This is a thing I find enjoyable, since it encourages us to think for ourselves instead of just reproducing and rehashing what other people have already done multiple times before (like for some of the aircraft-related projects). In the end this really worked out for us and we were able to fully design our system within a week, by using the developed analysis tools.

The group cooperation has been very fruitful. We have a good mix of people interested in aerodynamics, structures and control systems, which is very useful and convenient, since most if not all group members get to work on a subject they enjoy.

Regarding my own functioning I have to be critical at times. It feels like sometimes during the day I could spend my time more efficiently, but I think that is to be expected when you are working towards a certain goal for at least eight hours per day for five days per week.

As for the organisation of the DSE I sometimes find the guidance with respect to deliverable items lacking. Multiple definitions exist on Blackboard of what should be included in several deliverables and what some of these reports actually entail. Furthermore the communication flow from the DSE organisation could be improved, for example when uploading new versions of the DSE schedule.