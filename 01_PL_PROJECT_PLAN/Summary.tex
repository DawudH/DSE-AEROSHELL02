\section*{Summary}\label{cha:summary}%Notes: Stand alone, consise and do not explain anything.
-Requirements
-Description of the entire system (the Global Picture)

\subsection{Requirements}
In this section the top-level requiremets are stated. They can be found in table \ref{tab:requirements}.

\begin{table}[H]
	\caption {Requirement}
    \begin{tabular}{|l|l|}
    \hline
    Code          & Description                                                                                                      \\ \hline
    CIA-Sys-A01-1 & The re-entry vehicle shall be able to cope with an entry velocity of seven kilometers per second.                \\ \hline
    CIA-Sys-A01-2 & The inflated aeroshall shall have a maximum diameter of 12 meters.                                               \\ \hline
    CIA-Sys-A01-3 & The diameter of the launcher fairing shall be 5 meters.                                                          \\ \hline
    CIA-Sys-A01-4 & The maximum entry mass of the re-entry vehicle shall be 10,00 kilograms.                                         \\ \hline
    CIA-Sys-A01-5 & The hypersonic deceleration system mass shall not be havier than ten percent of the total re-entry vehicle mass. \\ \hline
    CIA-Sys-A01-6 & The control system shall have a maximum failure probability of 5.0e-4.                                           \\ \hline
    CIA-Sys-A01-7 & The maximum allowable loads on the re-entry vehicle shall be 3 earth g's in each axis                            \\ \hline
    CIA-Sys-A01-8 & The re-entry vehicle shall have a maximal aerobraking duration of ten days.                                      \\ \hline
    \end{tabular}
    \label{tab:requirements}
\end{table}

\subsection{System description}

