\section*{Summary}\label{cha:summary}
One of the solutions that fulfil the need to carry human payload to Mars is a \gls{cia}. The project has to demonstrate the feasibility of such a controllable inflatable aeroshell. The design of this product can be considered as complex. Therefore Systems Engineering (SE) methods and tools can be used to simplify the design process. The initial steps are the organization of the group as a team and subsequently organizing the project work itself. These steps are summarized in this Project Plan.

The first SE-tool that has been used is the \gls{obs} aiming to divide the project team in tasks, functions and responsibilities. Examples of divisions are the chairman, secretary and systems engineer. This is then followed by assigning different team members to different tasks. A \gls{wbs} is utilized to identify the required activities during the project. Here the project work is divided into phases to achieve a high-level division of the work. Where the \gls{wbs} identifies the required activities, the \gls{wfd} will indicate the sequence of these activities. This diagram shows the logic flow of the activities. It also specifies the milestones that will close the different project phases. With the sequence of the required activities known, the expected human resources, thus time, can be allocated to the different activities resulting in a Gantt chart. 

The previously mentioned methods and tools are used to divide the project work itself. However, other aspects are also important during the design process, such as risk and sustainability. The process is exposed to several risks such as schedule overruns and insufficient technical performance. In order to reduce the risk a risk mitigation plan is made, which uses risk mapping as a qualitative method and Technical Resource Budgeting (TRB) as a quantitative method. Within this mission, sustainability will not lead the design process since the total impact of a single interplanetary mission on the environment can be assumed to be very small.

For future work on can refer to the WFD or Gantt chart. First a literature study is done to get familiarity with the subject. Then by the time of the Mid-term review the five generated conceptual designs are traded to choose one design that shall be developed further. The final review will have developed this design such that it can be analyzed at a deeper level of performance, which could enable the group to recommend the design as feasible for the needed controllable inflatable aeroshell.