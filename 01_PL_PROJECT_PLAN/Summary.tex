\section*{Summary}\label{cha:summary}%Notes: Stand alone, consise and do not explain anything.
-Requirements
-Description of the entire system (the Global Picture)

Ever since successful manned missions to the moon, then next step is a manned mission to mars. Re-entry in the atmosphere of mars induces great aerodynamic loads which lead to high thermal and structural loads on the structure. Where a spacecraft structure can be designed for very high loads, the human body is very limited in the loads it can endure. This creates the need for a system that can perform a re-entry on mars while keeping loads within the limits of what the human body can endure. A very promising design to perform such a re-entry is an inflatable guidable re-entry vehicle. Three designs have been made and tested by NASA and the next will be launched in 2016. \cite{irve2}(add 2 more ref. NASA) 

\subsection{Requirements}
In this section the top-level requirements are stated. They can be found in table \ref{tab:requirements}.

\begin{table}[H]
	\caption{Requirement}
	\begin{tabular}{|p{0.18\textwidth}|p{0.77\textwidth}|}
    \hline
    Requirement ID          & Description                                                                                                      \\ \hline \hline
    CIA-Sys-A01-1 & The re-entry vehicle shall be able to cope with an entry velocity of seven kilometers per second.                \\ \hline
    CIA-Sys-A01-2 & The inflated aeroshell shall have a maximum diameter of 12 meters.                                               \\ \hline
    CIA-Sys-A01-3 & The diameter of the launcher fairing shall be 5 meters.                                                          \\ \hline
    CIA-Sys-A01-4 & The maximum entry mass of the re-entry vehicle shall be 10,000 kilograms.                                         \\ \hline
    CIA-Sys-A01-5 & The hypersonic deceleration system mass shall not be heavier than ten percent of the total re-entry vehicle mass. \\ \hline
    CIA-Sys-A01-6 & The control system shall have a maximum failure probability of 5.0e-4.                                           \\ \hline
    CIA-Sys-A01-7 & The maximum allowable loads on the re-entry vehicle shall be 3 Earth g's in each axis                            \\ \hline
    CIA-Sys-A01-8 & The re-entry vehicle shall have a maximal aerobraking duration of ten days.                                      \\ \hline
    \end{tabular}
    \label{tab:requirements}
\end{table}

\subsection{System description}
Previous missions comparable to the present mission have been performed by NASA, demonstrating the controllability of the 
