\section*{Summary}\label{cha:summary}
To fulfill the need to carry human payload to Mars one solution is a \gls{cia}. The project has to demonstrate the feasibility of such a controllable inflatable aeroshell. The design of this product can be considered as complex. Therefore Systems Engineering (SE) methods and tools can be used to simplify the design process. The initial steps are the organization of the group as a team and subsequently organizing the project work itself. These steps are summarized in this Project Plan.

The first SE-tool that has been used is the Organizational Breakdown Structure (OBS) to assign tasks, functions and responsibilities to the project team. Examples of managerial functions are the chairman, secretary and systems engineer. This is then followed by assigning different team members to different technical functions: the structures, aerodynamic, thermodynamic and control departments. Each department has two or three team members. A Work Breakdown Structure (WBS) is utilized to identify the required activities during the project. Here the project work is divided into phases to achieve a high-level division of the work. Where the WBS identifies the required activities, the Work Flow Diagram (WFD) indicates the sequence of these activities. This diagram shows the logic flow of the activities. It also specifies the milestones that will close the different project phases. With the sequence of the required activities known, human resources, thus time, can be allocated to the different activities resulting in a Gantt chart. 

The previously mentioned methods and tools are used to divide the project work itself. However, other aspects are also important during the design process, such as risk and sustainability. The process is exposed to several risks, primarily cost and schedule overruns and insufficient technical performance. In order to identify, analyze and manage risks a risk mitigation plan is made, which uses risk mapping as a qualitative method and Technical Resource Budgetting (TRB) as a quantitative method. Within this mission, sustainability will not lead the design process since the total impact of a single interplanetary mission on the environment is relatively small. It will, however, play a role when faced with design choices. If budgets and technical performance allows, a sustainable design is the preferred one.

For future work one can refer to the WFD or Gantt chart. First a literature study is done to get familiarity with the subject. Then by the time of the \gls{mtr} the five generated conceptual designs are traded to choose one final design that shall be developed further. After further, more detailed, analysis of this final design a recommendation can be made as to the feasibility of the design.

%\footnote{Title page figure from (URL): https://gcd.larc.nasa.gov/projects/archived-projects-2/hypersonic-inflatable-aerodynamic-decelerator. Accessed 20 April %2015]}
