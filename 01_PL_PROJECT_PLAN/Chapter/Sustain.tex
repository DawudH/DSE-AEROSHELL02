\section{Approach with respect to sustainable development}\label{cha:sustain}
Sustainable development in engineering means that the design, production, operation and disposal of a product should be done in a sustainable way. In this case sustainable means that energy and resources are used in a manner that does not threaten the environment or the needs of future generations \cite{sustain}. In this chapter the general approach with respect to the sustainable development of the controllable inflatable aeroshell is briefly discussed.

Even though sustainability is becoming more important in engineering, it is of less importance in these kind of space missions. The reason for this is that the proposed mission is a single mission and therefore its total impact will be relatively small. For example, it is acceptable that the production of the space vehicle is less sustainable than the production of one small passenger aircraft, since the aircraft is produced in large numbers whereas only one space vehicle is produced. It can therefore be said that sustainability will not be the design driver for the controllable inflatable aeroshell. Of course, sustainable methods are preferred when they do not add much costs and very unsustainable methods are to be avoided. It must be noted that even though a space mission itself may never be fully sustainable, advances towards more sustainable development can still be made. This design aims at a mass reduction with respect to more conventional designs. As such, if a mission is performed, this aeroshell design will help to increase the sustainability of the whole mission. For example, a lighter than conventional design will lower the launch emissions and is as such helping towards a more sustainable world.

Some examples of sustainable methods within the aeroshell design can however also be mentioned. In the process of producing the aeroshell unnecessary polluting methods that threaten the natural environment should be avoided. Also interplanetary forward contamination should be prevented. In this case it means that life and other forms of contamination should not be transferred from Earth to Mars. In practice this is already standard procedure. Another, less important, form of sustainability is the avoidance of space debris in the atmospheres of Earth and Mars.

Therefore sustainability will not be a leading driver for the design of the controllable inflatable aeroshell, as long as very unsustainable methods are avoided. The latter is effected by a preference of sustainable methods in choices for production and operation of the aeroshell.