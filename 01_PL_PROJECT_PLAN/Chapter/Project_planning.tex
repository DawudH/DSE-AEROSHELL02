\section{Risk Management}\label{cha:plan}
% Schedule of the OS design project, indicating the project phasing and the planning for the delivery of the items presented in a bar chart
During the course of this project several risks are present. The risk of a certain occurrence is defined as the product of the probability of its occurrence and its consequence. In order to prevent schedule overruns and to guarantee the technical performance of the system to be designed these risks will have to be identified and managed properly. This will be done with the methods discussed in the subsequent sections. First section \ref{subsec:riskmap} will present the qualitative method of risk mapping, used to determine where to allocate scarce resources. Following that the quantitative method of glfs{trb} will be used to account for uncertainty and possible growth of Technical Performance Measurements (TPMs). These methods have been taken from \cite{nasasehandbook}.

\subsection{Risk mapping}
\label{subsec:riskmap}
During the conceptual design phase the elements of each of the concepts to be considered will be included in a risk map. From this risk map a comparison between the risks involved in the different conceptual designs can be made. First the elements of each concept will be identified. Secondly these will be included into the risk map. This risk map consists of a table with on the X-axis the consequence of failure of each element. Ordinal scales are used and these are thereby rated qualitatively from low to high as `negligible', `marginal', `critical' or `catastrophic'. \\
\noindent Negligible consequences barely influence the functioning of the system. Marginal consequences present inconveniences in performing the design mission, possibly with a small reduction in technical performance. However, the system will still be able to complete its primary mission with some (minor) adjustments. System performance is only severely compromised when a critical or catastrophic failure occurs. Critical failures strain the capabilities of the system and make mission succes questionable. A degradation in technical performance is to be expected. Catastrophic failure causes immediate mission failure or severely compromises the technical performance of the system. \\
\noindent On the Y-axis the current state of the technology of each element is presented. It is assumed during risk identification that this technology state is interchangeable with the probability of failure of the element. The technology states are rated as either 'feasible in theory', 'working laboratory model', 'based on existing non-flight engineering', 'extrapolated from existing flight design' or 'proven flight design'. Elements and technologies that have only been proven as 'feasible in theory' have an inherently higher probability of failure than a component that has already been proven to function during flight. \\
\noindent From this risk mapping table it can be seen that the elements that present the greatest risks for mission completion are those whose failure is either critical or catastrophic and whose technology has not matured enough yet, e.g. has only been proven in theory or has not functioned outside of a laboratory situation yet. To mitigate the risks these elements pose during the different design phases the fraction of resources allocated to them will be increased accordingly.
Furthermore the outcomes of the risk mapping can be used as a criterion during the concept trade-offs; if two concepts are otherwise equal in performance but the risks involved with one of them are significantly higher this concept can and will be regarded as inferior to the other.

\subsection{Technical resource budgeting}
Following the qualitative risk identification and management with risk maps quantitative risk management will be used during the concept analysis phases. This will be done using technical resource budgeting. Uncertainty factors are introduced for each TPM. These will be used to account for the uncertainty of the estimates of the TPMs during various design stages. As the eventual design matures the analysis methods used increase in complexity and thus the uncertainty of the estimates decreases. In order to account for possible resource growth (for example mass growth) and to prevent conflicts in later design stages Resource Contingency Allowances (RCA) are used. These allowances are functions of the design maturity and the types of hardware and software used; for a mature design or a design using only existing components the RCA's are low. \\
\noindent If at any point during the design process the product of the RCA factor and the estimate of a certain quantity (mass, power, etc.) exceeds the maximum allowable TPM an action needs to be taken. Possible actions include a (partial) redesign or changing the maximum allowable TPM. However, the latter is not always possible. When this occurs either the RCA factor needs to be decreased or the system needs to be (partially) redesigned. Ways to decrease the RCA factor include (sub)system testing and critically reviewing the design.