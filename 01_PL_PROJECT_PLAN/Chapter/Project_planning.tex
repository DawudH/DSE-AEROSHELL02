\section{Project planning}\label{cha:plan}
% Schedule of the OS design project, indicating the project phasing and the planning for the delivery of the items presented in a bar chart

\subsection{Risk mitigation plan}
\label{sec:riskmit}
During the course of this project several risks are present. The risk of a certain occurrence is defined as the product  In order to prevent schedule overruns and guarantee the technical performance of the system to be designed these risks will have to be identified and managed properly. This will be done with the methods discussed in the subsequent sections.

\subsubsection{Risk mapping}
During the conceptual design phase the elements of each of the concepts to be considered will be included in a risk map. From this risk map a comparison between the risks involved in the different conceptual designs can be made. First the elements of each concept will be identified. Secondly these will be included into the risk map. This risk map consists of a table with on the X-axis the consequence of failure of each element. These are rated qualitatively from low to high as 'negligible', 'marginal', 'critical' or 'catastrophic'. \\
\noindent Negligible consequences barely influence the functioning of the system. Marginal consequences present inconveniences in performing the design mission, possibly with a small reduction in technical performance. However, the system will still be able to complete its primary mission with some (minor) adjustments. System performance is only severely compromised when a critical or catastrophic failure occurs. Critical failures strain the capabilities of the system and make mission succes questionable. A degradation in technical performance is to be expected. Catastrophic failure causes immediate mission failure or severely compromises the technical performance of the system. 
\noindent On the Y-axis the current state of the technology of each element is presented. It is assumed during risk identification that this technology state is interchangeable with the probability of failure of the element. The technology states are rated as either 'feasible in theory', 'working laboratory model', 'based on existing non-flight engineering', 'extrapolated from existing flight design' or 'proven flight design'. Elements and technologies that have only been proven as 'feasible in theory' have an inherently higher probability of failure than a component that has already been proven to function during flight.
\noindent From this risk mapping table it can be seen that the elements that present the greatest hazards for mission completion are those whose failure is either critical or catastrophic and whose technology has not matured enough yet, e.g. has only been proven in theory or has not functioned outside of a laboratory situation yet. To mitigate the risks these elements pose during the different design phases the fraction of resources allocated to 

\subsubsection{Technical budgetting}