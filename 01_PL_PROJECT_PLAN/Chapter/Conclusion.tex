\section{Conclusion}\label{cha:conclusion}
The project involves the design of a complex mission, namely an entry vehicle featuring a guidable inflatable aeroshell. As such, it inherently comprises parts in intricate arrangement and requires the use of project management and systems engineering tools to guide the project team effectively. To this end, a number of steps have been taken. 

Firstly, the mission framework has been defined and the project objective statement is chosen to be the design of a controllable inflatable (re-)entry vehicle for hypersonic guided Mars entry of human class spacecraft using available launchers with nine persons in ten weeks time.  Secondly, managerial functions have been appointed to group members for the first project half. After the Mid-Term Review (MTR), taking place on May 26, these are re-appointed on the basis of performance in the project weeks leading up to this point. Thirdly, technical functions are related to the main disciplines involved in subsystem design, namely thermodynamics, aerodynamics, structures, control and atmospheric model generation $\%$ orbital mechanics (trajectory optimization). Each subsystem involves two or three persons responsible. Fourthly, work has been broken down  and sequenced by means of a Work Breakdown Structure (WBS) and Work Flow Diagram (WFD), respectively, for the allocation of human resources. The latter is depicted by a Gantt chart, set up and maintained continuously throughout the project by the planner. Fifthly, the risk management approach involves the use of risk mapping with ordinal scales, distinguishing probability and consequence of occurrence of technical risks, and the use of contingencies and allowances to account for uncertainties. Tracking risks in budgets is performed by the use of Technical Performance Measurements (TPM).

Future work shall start with a literature study to get more familiar with the subject. Other Systems Engineering methods and tools such as a Functional Flow (Block) Diagram, a Functional Breakdown, a Requirements Discovery Tree, a Budget Breakdown, Technical risk assessment and a Design Option Tree can be used to generate five conceptual design. Following this, relatively simple trade-offs can be made. Then using these trade-offs the choice for two conceptual designs can be justified, which will be developed on a more detailed level. More involved trade-offs between these two preliminary design are then made, using a set of enhanced tools. 