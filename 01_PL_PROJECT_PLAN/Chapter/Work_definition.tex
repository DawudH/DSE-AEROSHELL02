This chapter treats identification and sequencing of activities required during the project and allocation of resources to these activities. As such, it comprises a Work Breakdown Structure (WBS) and Work Flow Diagram (WFD) as primary Systems Engineering elements to categorize respectively sequence work activities. In addition, the WFD provides interrelations between steps to identify iterations in the work process where appropriate. It is essential that work activities are defined and sequenced in order to allocate resources, as depicted by the Gantt chart.

This chapter is structured as follows. The first section commences with a presentation of the WBS and accompanying discussion to justify the categorization, the second section proceeds with a presentation of the WFD, the third section discusses allocation of resources and presents the Gantt chart. The latter is accompanied by a brief discussion on milestones set to monitor project progress.

\section{Work Breakdown Structure}\label{cha:WBS}
Project work has been divided into a number of phases: project planning, literature research, mission analysis, tool development and enhancement, conceptual design of a first set of concepts, detailed design of a number of selected concepts and project close-out. This high-level work division is summarized in the work breakdown structure (WBS) displayed in Figure \ref{fig:wbs}. Whereas the sequence of activities is illustrated by the Work Flow Diagram (WFD), the WBS provides a categorization of activities. 




A number of project phases is distinguished with clearly identifiable milestones that close activities. Predominantly, these milestones are the following:
\begin{itemize}
\item[Project Plan (PP):] Finalizes provisional planning of the project, comprising: technical and managerial function appointment to team members, allocation of resources, break-down and sequencing of project work, formulation of a risk management plan and a plan for sustainable development, set-up of archiving, doucmentation and logbook and consolidating a set of project procedures.
\item[Baseline Review (BR):] Finalizes the mission analysis phase by a liaison with the customer for agreement on mission definition, requirements and planning as well as a presentation of initial concepts formulated in the conceptual design phase.
\item[Mid-Term Review (MTR):] Finalizes the first phase of concept selection, reviewing the concepts for trade-off, the trade-off process and resulting final concepts for further evaluation.
\item[Final Review (FR):] Finalizes the phase of concept selection with a technical presentation and agreement with the customer on the final design and the process by which it was reached. 
\end{itemize}

