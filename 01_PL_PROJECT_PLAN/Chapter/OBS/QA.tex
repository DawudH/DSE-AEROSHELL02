\paragraph{Editor}
Primary function is assuring consistent and high quality of all written communication, by means of proof-reading and correcting of pieces submitted by all group members. In addition, the lay-out and structure of reports and presentations is scrutinized and egalized. Strong interaction takes place with all group members with direct contributions to the written work, while open communication with documentation manager is maintained to resolve issues with the formatting of reports. In case of repeated errors by group members, the editor makes an effort to enter conversation with the repeaters in order to identify the origin of the problem and if need be to take pre-emptive action against future occurrences.
\paragraph{Verification}
Verification shall occur at multiple stages of the design. For example, in the initial stages it can be used to verify the requirements. At the end, the final product should be verified to check whether the developed product meets the requirements. Another definition given by NASA is that verification should proof that the product complies with design solution specifications and descriptive documents. Thus the responsibility is to have the product meet the specified requirements.

\paragraph{Validation}
IEEE defines validation as ''The assurance that a product, service, or system meets the needs of the customer and other identified stakeholders. It often involves acceptance and suitability with external customers.''(ref. ''IEEE V\&V.pdf'') The person in charge of validation is thus responsible for the compliance of the product with the requirements imposed by the costumers and stakeholders. Another task is to assure that the subsystem requirements contribute to accomplishing the system requirements and in the end to the top level requirements.
