\section{Introduction}\label{cha:introduction}
In view of feasibility and cost-effectiveness of future space exploration and habitation of extraterrestrial sites, such as Mars, solutions for the transportation of human payload to planetary surfaces should be increasingly cost-effective. One such solution is an inflatable aeroshell, stowable within conventional launcher configurations in undeployed condition. The design of a controllable inflatable aeroshell can be considered as complex, featuring interaction between many different disciplines. To reduce the complexity of the design problem Systems Engineering methods and tools can be used to succesfully develop the complex system. One of the important aspects in the development of this complex system is project management. The first steps in the project management process are summarized in this Project Plan.

The purpose of the Project Plan is to outline the general approach of the group to the technical and management aspects of the design project. The Plan shows how the group plans to accomplish all necessary tasks to prepare and develop the design. This initial planning consists of a number of steps, commencing with a defintion of the project objective. After outlining the project, a number of managerial and technical functions is appointed to the group members. This is followed upon by definition of work, in the form of a Work Breakdown Structure (WBS) and a Work Flow Diagram (WFD). Resources are then allocated to the project activities thusly defined by the WBS and WFD, presented in the form of a Gantt chart.

The structure of this report is as follows. In Chapter 2 the mission is described and the top-level requirements are given. An organogram is synthesized in Chapter 3. Work is broken down, sequenced and allocated in Chapter \ref{cha:workdef}. A plan for risk management is stated in Chapter \ref{cha:plan}.  In the following chapter a general approach to the sustainable development of the mission is briefly explained. The report is then concluded by Chapter \ref{cha:conclusion}. An appendix gives a detailed overview of resource allocation in the form of a Gantt chart.