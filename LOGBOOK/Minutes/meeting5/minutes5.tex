\subsubsection{Opening}
Locatie: Fellowship Meeting room 2\\
Aanwezig: HD, NR, DD, J, L, G, A, B, T, D, Su, Se\\
Verontschuldigd: - \\
Afwezig: - \\
tijd: Joost opent de vergadering om 10:03\\

\subsubsection{Agenda}
De agenda van de vergadering bestaat oorspronkelijk uit de volgende punten:
\begin{enumerate}
\item Opening
\item Agenda
\item Goedkeuring van de notulen
\item Control System DOT
\item Trade-off criteria
\item Andere vragen
\item W.V.T.T.K.
\item Rondvraag
\item Sluiting
\end{enumerate}

Hieraan worden toegevoegd:
\begin{itemize}
\item Tussen puntje 3 en 4: Baseline Report
\item Tussen puntje 5 en 6: Logbook 
\end{itemize}

\subsubsection{Goedkeuring van de notulen}
Wegens tijdgebrek heeft niet iedereen de notulen gelezen. L geeft aan dat de voorheen besproken veranderingen wel zijn doorgevoerd.\\

\subsubsection{Baseline Report}
Na het inleveren van het baseline report zijn er wat op en aanmerkingen van HD, NR en DD. Deze zijn uitvoerig besproken. Over het algemeen zijn er een aantal grote lijnen in deze opmerkingen te vinden. Zo was het grootste commentaar van toepassing op de literatuurstudie. Deze bevatte voornamelijk referenties naar bronnen, maar er werd opmerkelijk weinig uitgelegd over deze bronnen. Hierdoor is het doel van een literatuurstudie niet bereikt; namelijk het creëren van een naslagwerk, waardoor het erop naslaan van eerder bestudeerde literatuur niet meer nodig is. \\
Een ander groot kritiek punt had betrekking op de requirements. De top level requirements zijn niet SMART (specific measurable attainable realistic and timeley) bevonden en lijken open deuren in te trappen. G geeft aan dat dit gedaan is om requirements terug te linken naar hogere levels van requirements. Er wordt echter aangegeven dat hierdoor het overzicht weg is en dat er wordt gesuggereerd dat de missie bestaat uit remmen en 'de rest', hoewel er wordt verwacht dat de requirements overzichtelijk over het gehele systeem worden verdeeld.  Tevens zijn alleen de opgelegde requirements besproken in het verslag, hoewel andere requirements (e.g. sustainability requirements en requirements die gebonden zijn aan de wet) niet zijn besproken.\\
Daarnaast werd er in het verslag niet goed uitgelegd waarom het een goed idee is om een inflatable te kiezen in plaats van een solide ontwerp. B verklaart dit door aan te geven dat er nog geen officiële keuze is gemaakt voor een inflatible ontwerp en dat daarom onze visie nog heel breed is en niet per se gefocust op een solide ontwerp als definitieve oplossing voor het ontwerp vraagstuk.\\
Een ander puntje was dat het mass-budget nog niet heel accuraat was. Het vervoeren van 6 mensen naar Mars lijkt vooralsnog onhaalbaar volgens HD. Er wordt aanbevolen om deze schatting niet in berekeningen te gebruiken.\\

\subsubsection{Control System DOT}
Er wordt voorgelegd dat de groep denkt dat het beter is om het trade-off process onafhankelijk te maken van de missie tijdsduur en het control systeem. HD NR en DD vinden dit een goed idee.

\subsubsection{Trade-Off Criteria}
De groep heeft trade-off criteria opgesteld voor het analyseren van de 5 eerder gekozen concepten. Deze willen zij bespreken. De bedachte trade-off criteria zijn:

\begin{itemize}
\item Mass
\item Developement risk
\item Reliability
\item Orbit controllability
\item Deceleration capability
\end{itemize}

Er volgt een discussie over de criteria en met name over de verhouding tussen requirements en trade-off criteria. Hieruit blijkt dat requirements niet je trade-off moeten drijven, maar dat trade-off criteria er moeten zijn om te kijken welk concept, dat aan alle requirements zou moeten kunnen voldoen, beter is. Er is soms echt overlap.\\

Verder wordt er besproken dat reliability geen goede trade-off criteria is omdat reliability vaak af te kopen is met extra massa. Tevens is de massa zelf geen goede trade-off criteria omdat de massa een gegeven requirement is. Teveel massa meenemen kan niet, te weinig massa mee nemen zou zonde zijn. Er wordt geconcludeerd dat payload mass wel een goed criterium is, omdat er dan meer payload mee kan voor de zelfde missie.\\

NR vraagt nog wat het verschil is tussen orbit controlability en control. Hierop antwoord T dat orbit control vooral over de bekwaamheid van een configuratie gaat om zijn orbit te behouden, hoewel control anderzijds meer focust op het gebruik van hulpmiddelen zoals control surfaces, cg. change etc..\\

Er volgt een gesprek over de nauwkeurigheid van de baan van het voertuig met betrekking op het einde van de missie. HD geeft hierop het einde van de missie in cijfers aan, zoals lang gehoopt. De missie zal op 10 km MOLA eindigen met een snelheid van Mach 5 waarbij de positie van het voertuig in een straal van 500 meter van een doel moet kunnen zijn. Er wordt een hint gegeven dat er wel moet worden gekeken naar de trajectory van het voertuig. Daarnaast is niet bij iedereen bekend wat MOLA is, dit moet worden opgezocht als actiepuntje.\\

Vervolgens geeft T aan dat het misschien wel nodig is om een versoepeling in de requirements te krijgen . HD is benieuwd wat dit kost in termen van andere requirments en geeft aan dat een concreet voorstel nodig is om hier verder op in te kunnen gaan.\\

\subsubsection{Logbook}
NR merkt op dat hij alleen notulen in het logbook ziet, maar dat de overige puntjes zoals action items , planning en dicisions ontbreken. Hij vraagt of dingen zoals action items wel besproken worden. L geeft aan dat dit het geval is en zal deze items voortaan niet alleen opnemen in zijn schrift maar ook in het logbook verwerken. Wel geeft L aan dat dit voor hem wat tijd kost en dat het van de tijd afgaat die anders aan technische taken besteed kon worden.

\subsubsection{Andere vragen}
L vraagt of we per se getallen moeten hangen aan de criteria in de trade-off tabel, omdat dit in de Systems Engineering slides niet zo is aangegeven. DD legt uit dat dit niet altijd hoeft, je kunt bijvoorbeeld ook plusjes gebruiken, zolang het maar duidelijk is welk concept beter is en waarom. Hier gaat HD nog even op in omdat in vorige projecten de trade-off niet door de groep alleen werd gedaan, maar samen met de tutors. Er wordt besloten om over deze keuze nog even na te denken en hier later op terug te komen.\\

G vraagt hoe het zit met de personal appendix deliverable, omdat deze op blackboard staat maar nergens in het opdrachten blad. HD zegt dat dit een management samenvatting mag zijn van het logbook en dat we hier niet zo veel tijd in hoeven te steken, maar dat het wel aanwezig moet zijn.\\

B vraagt tot hoe ver het structural design uitgewerkt moet worden voor de mid term review omdat het erg ver kan gaan en er te weinig tijd is om bijvoorbeeld buckling te berekenen van elk concept. HD en NR zeggen dat de berekeningen nog niet heel diep hoeven te gaan maar dat een orde van grootte wel bekend moet zijn.\\

\subsubsection{W.V.T.T.K.}
Er was een vraag van de OSSA's om de deliverables op blackboard te zetten. NR vraagt of dit gelukt is. A zegt dat hij dit gedaan heeft, maar dat er bij de OSSA's technische problemen zijn en dat de documenten niet meer op de website staan. Hier wordt nog aan gewerkt.

\subsubsection{Rondvraag}
Er zijn geen verdere vragen.

\subsubsection{Sluiting}
De volgende vergadering zal plaatsvinden op 12-05-2015 om 10:00 in meeting room 2.\\

Joost sluit de vergadering om 12:34.\\
