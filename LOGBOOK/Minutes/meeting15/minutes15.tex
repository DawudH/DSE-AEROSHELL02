\subsubsection{Opening}
Locatie: Fellowship Meeting room 2\\
Aanwezig: HD, NR, DD, J, L, G, A, B, T, D, Su, Se\\
Verontschuldigd: \\
Afwezig: \\
Tijd: Dawud opent de vergadering om 11:05\\

\subsubsection{Agenda}
De agenda van de vergadering bestaat oorspronkelijk uit de volgende punten:
\begin{enumerate}
\item Opening
\item Agenda
\item Goedkeuring van de notulen
\item Andere vragen
\item W.V.T.T.K.
\item Rondvraag
\item Sluiting
\end{enumerate}

Aan de agenda worden verder geen punten toegevoegd.

\subsubsection{Goedkeuring van de notulen}
Er waren geen op- of aanmerkingen.

Nicalon max. temp 1100C volgens Niels. Onze sources zeggen hoger. Spaceshuttle gebruikt doeken pas bij 1000 graden. Verschil tussen smelt temp en max use temp, zonder/met loading. 
Vatten het ontwerp samen. Eerste aerocapture tot 50km alt. Parking orbit en dan final entry. Density aanpassen is dus 0.9 en 1.1 verhaal. Dan controlen om op initiele positie te komen. 
L/D 0.35 en maximale CD geoptimaliseerd. Maximale cg offset .5m voor het trimmen. Asymmetrisch. 
Nicalon redt het vooralsnog. Eventueel nextel met coating voor hogere emissivity. NR zegt dat er coatings zijn die dat aan zouden moeten kunnen, maar coating is ook weer mass. Ook zou het duur zijn, waarschijnlijk negligible t.o.v. totale kosten. Coatings kunnen heel dun zijn (tot nano).
Structural, met zylon kun je de dunst mogelijke laag pakken, kevlar wordt dikker. Inflation zorgt voor 65kPa in de inflatable. Drukverschil neemt in atmosfeer niet af, doordat de temperatuur toeneemt. IRVE pompte desondanks bij, dat komt door leakages. Inflatable wordt niet meer afgeworpen, maar deflaten. Thruster in de centerbody die vrij fors is. Geen alternatieven, parachute werkt niet. Wat met bijv. een grote parachute en dan een soyuzboost. Parachute weegt al meer dan brandstofwinst. 

Voordeel van onze aeroshell. 7km req was driving. Langere orbits zorgen voor lagere entry velocities. Wel of geen return, kost veel massa.

Iteraties in design. Sturen op orbit en aero. Hoge CD was in eerste instantie niet goed voor dde 10km mach 5 req. Die was dus veranderd naar 15km mach5. Toroids leeg laten lopen zorgt ook voor lift verlies, wat niet mag. 

\subsubsection{Andere vragen}
\textit{In welke mate moeten wij de eisen verder uitwerken?} Sub-eisen zijn niet opgelegd door HD, top-level lijkt voldoende naast: radiation, spacedebris.

Onvoorspelbaarheid van de leeggelopen inflatable. Alle failure modes moeten besproken worden voordat je een echte aanbeveling kan doen. Asymmetrie, inflatable mid-air deflaten, niet werkende inflatable etc.

\textit{C\&DH diagram} Twee dingen: data handling block diagram. En communication flow diagram. HD weet ook niet precies wat ze willen zien. Bij ons is communicatie vooral intern. Misschien dit interne wat beter uitwerken. Communications tussen vehicle en aarde, vehicle is gecovered, aarde (ground station) beetje achtergebleven volgens DD. Voor wat we tot nu toe hebben: hoe snel moeten we dingen hebben, hoevaak heb je info nodig. Redundancy bekijken. Hoe check je of een sensor nog goede informatie doorgeeft. Focus op aeroshell.

\textit{Wie zitten er bij de final review?} Meestal beperkt. Leg DSE heel kort uit, wel duidelijk uitleggen wat ons probleem is.

\subsubsection{W.V.T.T.K.}
\textbf{1.} L: Poster ook in jip-en-janneke taal? Symposium en poster gaat om winnen, verkoop het verhaal.

\textbf{2.} A: FR met zn allen, symposium 2 of 3. 

\textbf{3.} HD: Hoeveel meetings hebben we nog? FR, poster, symposium voorbereidingen? Dinsdag verslag, woensdag/donderdag een poster. Vrijdag een meeting voor laatste report dingetjes. Dinsdag meeting om symposium door te spreken. Probleempje om vrijdag te graden. 

\subsubsection{Rondvraag}
Er zijn verder geen vragen.

\subsubsection{Sluiting}
De volgende vergadering zal plaatsvinden op 22-06-2015 om 11:00 in Fellowship Meeting room 2.
\newline\newline
Dawud sluit de vergadering om 12:34.