\subsubsection{Opening}
Locatie: LR Meeting room 3\\
Aanwezig: HD, NR, DD, J, L, G, A, B, T, D, Su, Se\\
Verontschuldigd: \\
Afwezig: \\
Tijd: Dawud opent de vergadering om 11:05\\

\subsubsection{Agenda}
De agenda van de vergadering bestaat oorspronkelijk uit de volgende punten:
\begin{enumerate}
\item Opening
\item Agenda
\item Goedkeuring van de notulen
\item Andere vragen
\item W.V.T.T.K.
\item Rondvraag
\item Sluiting
\end{enumerate}

Aan de agenda worden verder geen punten toegevoegd.

\subsubsection{Goedkeuring van de notulen}
NR had een paar vragen/opmerking over onduidelijkheden in het vorige report.

NR had opgezocht dat de maximum usable temperature van Nicalon op 1100C lag. Onze sources gaven 1800C aan. NR zegt dat het belangrijk is te weten dat er dus een verschil is tussen melting temperature en maximum use temperature. Voor de duidelijkheid wordt het ontwerp alsnog snel samengevat voor NR.
\newline\newline
Voor de orbit eerst is er een aerocapture tot ongeveer 50km altitude, vervolgens een parking orbit om een goed moment te vinden voor de final entry. Om aan te tonen dat we binnen de 500m range kunnen eindigen is de density profile aangepast naar een 0.9 en 1.1 ratio om eventuele onzekerheden in de atmosfeermodellen mee te nemen. Vervolgens wordt maximaal gecontroleerd om terug op de initiele positie te komen.
\newline\newline
Voor de shape is er een L/D van 0.35 en is voor een maximale CD geoptimaliseerd. De fixed c.g. offset is 0.5m om alpha te trimmen. Verder is de aero-shape asymmetrisch.
\newline\newline
We gebruiken dus Nicalon. Alternatief zou zijn Nextel met een coating wat een hogere emissivity zou hebben. Volgens NR zijn hier wel coatings voor, maar coatings zorgen ook weer voor extra mass. Verder zou Nicalon redelijk duur zijn, maar dit is verwaarloosbaar voor spacemissions. Overigens kunnen coatings heel dun zijn volgens DD.
\newline\newline
Voor structural wordt een zo dun mogelijke laag Zylon gebruikt. Kevlar zou duurder worden. Inflation system zorgt voor 65kPa in de inflatables. Drukverschil neemt overigens in de atmosfeer niet af, aangezien de temperatuur toeneemt. IRVE pomp desondanks steeds bij, door de leakages. Inflatable zal niet meer afgeworpen worden, maar we gaan deflaten. Verder gebruiken we een vrij forse thruster in de centerbody. Hiervoor hadden we geen alternatieven. Parachute wordt te zwaar. De parachute zelf weegt al meer dan de brandstofwinst en de thruster heb je sowieso nodig.
\newline\newline
Uiteindelijk was de 7 km/s requirement was driving, omdat in de aerocapture geremd moet worden tot onder de escape velocity, wat vrij moeilijk wordt. Deze 7 km/s komt door de snelle transfer orbit (89 dagen). Langere orbits zorgen voor lagere entry velocities. Het hebben van een returnmogelijkheid zorgt voor onredelijk veel extra mass.
\newline\newline
Voor de design-iteraties wordt er gestuurd in orbit en aero. Een hoge CD was benodigd, maar was niet goed voor de 10km mach 5 requirement. Die is veranderd naar 15km op mach 5. Eventueel toroids leeg laten lopen om $C_DA$ te verlagen zorgt ook voor vermindering in lift-force wat je echt niet wil.

\subsubsection{Andere vragen}
\textit{In welke mate moeten wij de eisen verder uitwerken?} Sub-eisen zijn niet opgelegd door HD, top-level lijkt voldoende naast bijvoorbeeld radiation en space debris.
\newline\newline
Onvoorspelbaarheid van de leeggelopen inflatable. Het is namelijk onzeker wat er precies gebeurt en wat voor effect dit heeft op de stabiliteit. Alle failure modes moeten besproken worden voordat je een echte aanbeveling kan doen. Dingen als asymmetrie, inflatable mid-air deflaten, niet werkende inflatable etc. moeten aangestipt worden.
\newline\newline
\textit{C\&DH diagram} Er waren twee deliverables: data handling block diagram. En communication flow diagram. HD weet ook niet precies wat ze willen zien. Bij ons is communicatie vooral intern door de delay tussen Mars en Aarde. Misschien moet het interne wat beter uitgewerkt worden. Bij de communicatie tussen vehicle en aarde is de vehicle goed behandeld, maar de aarde (ground station) is een beetje achtergebleven volgens DD. Voor wat we tot nu toe hebben is het fijn om te weten hoe snel moeten we bepaalde info nodig hebben en hoevaak. Sommige info wil je namelijk elke 0.1s weten en andere misschien maar per minuut. Ook kan redundancy bekeken worden. Hoe check je of een sensor nog goede informatie doorgeeft.

\textit{Wie zitten er bij de final review?} Meestal beperkt. Leg het DSE heel kort uit, wel duidelijk uitleggen wat ons probleem is.

\subsubsection{W.V.T.T.K.}
\textbf{1.} L: Moet de poster ook in jip-en-janneke taal? HD: Symposium en poster gaat om winnen, verkoop het verhaal.

\textbf{2.} A: Klopt het dat FR met zn allen gepresenteerd moet worden en het symposium met met 2 of 3?

\textbf{3.} HD: Hoeveel meetings hebben we nog? De belangrijke dagen zijn als volgt: dinsdag komt het verslag, woensdag/donderdag een poster. Vrijdag een meeting om final report te bespreken. Dinsdag een meeting om symposium door te spreken. HD geeft wel aan dat het moeilijk wordt om vrijdag te graden. 

\subsubsection{Rondvraag}
Er zijn verder geen vragen.

\subsubsection{Sluiting}
De volgende vergadering zal plaatsvinden op 26-06-2015 om 11:00 in Fellowship Meeting room 2.
\newline\newline
Dawud sluit de vergadering om 12:34.