\subsubsection{Opening}
Locatie: Fellowship Meeting room 2\\
Aanwezig: HD, DD, J, L, G, A, B, T, D, Su, Se\\
Verontschuldigd: \\
Afwezig: NR \\
Tijd: Dawud opent de vergadering om 11:09 (aanvangstijd gewijzigd)\\

\subsubsection{Agenda}
De agenda van de vergadering bestaat oorspronkelijk uit de volgende punten:
\begin{enumerate}
\item Opening
\item Agenda
\item Goedkeuring van de notulen
\item Ontwerp
\item Andere vragen
\item W.V.T.T.K.
\item Rondvraag
\item Sluiting
\end{enumerate}

Aan de agenda worden verder geen punten toegevoegd.

\subsubsection{Goedkeuring van de notulen}
Er waren geen op- of aanmerkingen.

\subsubsection{Ontwerp}
BEzig geweest met ontwerp/iteraties etc. 

Begonnen met initial design. Daarvoor geoptimaliseerde shape. Orbit gekozen. Heatflux en dynamic pressure uitrekenen. Die gebruiken thermo, structures en control om massa's uit te rekenen. Daar zitten feedback loops in. Drie componenten geven totale massa. Ofwel temp te hoog, of massa te hoog. Dan opnieuw itereren. HD vraagt waar het probleem meestal was, dat was vaak te hoge flux, die omlaag brengen gaf problemen voor de control. Mach 5 op hogere altitude (15km), eerder met HD overlegd. 

Shape: rond met 12m dia. Extreme punt van de koepel beetje geoffset (1m). 10 toroids. Meer dan 10 heeft structureel gezien geen zin. Minder kan eventueel. Is voldoende om de shape te maken.

Trajectory: Zo hoog mogelijk door de atmosfeer, zo laag mogelijke heat flux. Plotjes van de trajectory worden uitgelegd. De landing range wordt bepaald door density te veranderen en maximaal control toepassen. Vervolgens kun je zien of je binnen de marge blijft. Hoe zit het met het moment wanneer je gaat banken. Wat gebeurt er als je density minder dan 15\% is? Eerder doen geen probleem, later wel een probleem omdat je dan dieper de atmosfeer in gaat. Vooralsnog getriggerd op 1g. We denken alleen bank control nodig te hebben. Zullen we goed moeten opschrijven

Thermo: op 270 kg, ook aan de achterkant wat nodig. Kijken of nicalon brittle is of niet.

Folding als een paraplu, wel wat folds omdat hij assymetric. Structurele lagen zijn wel goed op te vouwen. Lengte wordt ongeveer 3.5m. Zou makkelijk in SLS passen, capsule wordt ongeveer 6m. 

Uiteindelijk wordt de decelerator afgeworpen. Thrusters gaan door de centerbody heen d.m.v. een klep. Asymmetric shape, hoe gaat dat verticaal naar beneden? Dingen zijn een beetje buiten de scope, maar je zou er vragen over kunnen krijgen. 


Voor nicalon kijken of hij vouwbaar is en puncture resistant. Alternatieven? Coating voor nextel om hogere emissivity. 
We hebben een contingency doordat de structure temperature lager wordt geeist.
Totale mass wordt 800kg met 20\% contingency. 
Voor propellant mass gebruiken we hydrazine en meerdere thrusters. Masses uit SMAD ed gehaald. Control moments geschat met estimates voor moment of inertia enzo. 

Zijn er single points of failure? Losses minder geaccepteerd etc. Failure modes mogen er best zijn, teveel redundancy kan niet. Kijk of het veiliger kan, niet teveel tijd aan besteden. Hoeveel tijd voor de atmosfeer wordt het opgeblazen? Veel te vroeg opblazen heeft als enige nadeel micrometeorites.

Twee personen met contingency, was drie zonder contingency.
\subsubsection{Andere vragen}
\textit{Wordt er aan het einde van het ontwerp een CAD model verwacht?}\newline
Nu hebben we een schematische tekening. Vormen weten we bijv. nog niet. Hoeft niet geen CATIA-tekening te zijn, wel maten erbij zetten bij je huidige tekeningen. CAD-plaatjes kunnen evt. je symposium opleuken. Makkelijker om CAD-tekening te hebben, zijn ook makkelijker in 3DS te zetten. 

\subsubsection{W.V.T.T.K.}
HD: Hoe gaat het met tijdsplanning? Morgen moet het design in het report staan. Vrijdag komt afraffeldag. Dan zou het tekstueel af moeten zijn, maandag/dinsdag editing. Lees iedere zin met de intentie of het iets toevoegt. Zeker in de vorige report was het een beetje wollig, kijk kritisch naar de inhoud van je tekst. Veel leesbaarder en makkelijker te beoordelen.

\subsubsection{Rondvraag}

\subsubsection{Sluiting}
De volgende vergadering zal plaatsvinden op 22-06-2015 om 11:00 in Fellowship Meeting room 2.
\newline\newline
Dawud sluit de vergadering om 12:34.