\subsubsection{Opening}
Locatie: Fellowship Meeting room 2\\
Aanwezig: HD, DD, J, L, G, A, B, T, D, Su, Se\\
Verontschuldigd: \\
Afwezig: NR \\
Tijd: Dawud opent de vergadering om 11:09\\

\subsubsection{Agenda}
De agenda van de vergadering bestaat oorspronkelijk uit de volgende punten:
\begin{enumerate}
\item Opening
\item Agenda
\item Goedkeuring van de notulen
\item Ontwerp
\item Andere vragen
\item W.V.T.T.K.
\item Rondvraag
\item Sluiting
\end{enumerate}

Aan de agenda worden verder geen punten toegevoegd.

\subsubsection{Goedkeuring van de notulen}
Er waren geen op- of aanmerkingen.

\subsubsection{Ontwerp}
We zijn bezig geweest met iteraties om tot een final design te komen. Dit is als volgt tot stand gekomen:
\newline\newline
Begonnen met een initial design, welke geoptimaliseerd werd voor de shape. Vervolgens een orbit gekozen, waaruit de dynamic pressure en heat flux berekend konden worden. De gegevens werden door thermo, structures en control gebruikt om massa's uit te rekenen. De drie massa's bij elkaar is de totale massa van het heat shield. Er zitten wel feedback loops in. Probleem lag vaak bij thermo, voor 12m diameter werd het te heet. Oftewel, te hoge heat flux. Heat flux omlaag brengen gaf weer problemen voor control. Uiteindelijk gelukt met Nicalon, een silicon carbide. DD vraagt wel goed te letten of dat nieuwe materiaal wel toepasbaar is netzoals Nextel dat was (brittleness, impact resistance etc.). Ook is met HD overlegd dat het eindpunt op 15km hoogte met Mach 5 ligt.
\newline\newline
De shape is van voren rond met een 12m diameter. Van de zijkant lijkt het een beetje op een lens. Echter is de top niet precies in het midden, maar heeft het een offset van ongeveer een meter, asymmetrie dus. Er komen 10 toroids, meer dan 10 heeft structureel gezien geen zin. Minder zou kunnen en het zijn er genoeg om de shape te maken.
\newline\newline
De landing range margin wordt bepaald door de density te veranderen en maximaal control toe te passen. Vervolgens kun je zien of je binnen de marge kan blijven. We beginnen met banken bij 1g. HD vraagt waarom dit moment gekozen is. In principe is eerder banken geen probleem. Later banken wordt een probleem omdat je dan te diep door de atmosfeer gaat. Vooralsnog lijkt enkel bank control voldoende. Dit moeten we goed op gaan schrijven.
\newline\newline
Thermo is op 270kg gekomen. Aan de andere kant van het heat shield is een laagje Nextel AF-14 aangebracht. Voorkant dus van Nicalon, waarbij nog gecheckt moet worden of het netzoals Nextel goed foldable is en een goede impact toughness heeft. Het wordt opgevouwen als een soort paraplu. Door de asymmetrische vorm komen er wel vouwen/folds in, maar dit zou geen probleem moeten zijn. De structurele lagen zijn zeker goed op te vouwen. Lengte van het heat shield wordt ongeveer 3.5m in opgevouwen vorm om de crew module heen. Zou makkelijk in SLS moeten passen. De crew module zelf wordt ongeveer 6m lang. Uiteindelijk wordt de decelerator afgeworpen. Thrusters gaan door de centerbody heen d.m.v. een klep. HD vraagt hoe de asymmetrische vorm verticaal naar beneden kan vallen. Dit soort dingen zijn een beetje buiten de scope, maar er zouden vragen over kunnen komen. 
\newline\newline
De totale mass wordt 800kg met 20\% contingency. Voor de propellant mass gebruiken we hydrazine en meerdere thrusters. Masses van dit alles uit SMAD e.d. gehaald. Control moments geschat met estimates voor moment of inertia. Zo is de required control mass bepaald.
\newline\newline
HD vraagt of er single points of failure zijn? Losses worden steeds minder geaccepteerd. Failure modes mogen er best zijn, immers teveel redundancy kan niet. We moeten kijken of het veiliger kan, maar niet teveel tijd aan besteden. Verder is er nog de vraag wanneer de inflatable opgeblazen wordt. Dit moet uiteraard voordat we de atmosfeer binnenkomen, wanneer precies maakt niet uit, enkel een verlengd risico op micrometeorites. We nemen uiteindelijk twee personen mee voor onze missie.

\subsubsection{Andere vragen}
\textit{Wordt er aan het einde van het ontwerp een CAD model verwacht?}\newline
Op het moment zijn er schematische tekeningen in Visio. We denken dat een CATIA-tekening weinig zal toevoegen. HD vindt dat we wel afmetingen moeten zetten bij de huidige tekeningen. CAD-plaatjes kunnen eventueel leuk zijn voor het symposium. Ook kunnen CAD-tekeningen makkelijk overgezet worden naar andere programma's.

\subsubsection{W.V.T.T.K.}
\textbf{1.} HD vraagt hoe we met de tijdsplanning zitten. Donderdag zou het gehele design in het report moeten staan. Vrijdag moeten de laatste tekstuele dingen af zijn. Maandag en dinsdag zijn dan voor editing. HD raadt aan om bij het editen iedere zin te lezen met de vraag of het iets toevoegt. In het vorige report was er vaak veel tekst, weinig inhoud. Als we hier kritischer op zijn, wordt het veel leesbaarder en makkelijker te beoordelen.

\subsubsection{Rondvraag}
Er zijn verder geen vragen.

\subsubsection{Sluiting}
De volgende vergadering zal plaatsvinden op 22-06-2015 om 11:00 in Fellowship Meeting room 2.
\newline\newline
Dawud sluit de vergadering om 12:34.