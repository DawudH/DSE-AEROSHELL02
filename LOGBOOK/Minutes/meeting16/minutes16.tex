\subsubsection{Opening}
Locatie: Fellowship Meeting room 2\\
Aanwezig: HD, NR, DD, J, L, G, A, B, T, D, Su, Se\\
Verontschuldigd: \\
Afwezig: \\
Tijd: Dawud opent de vergadering om 11:06\\

\subsubsection{Agenda}
De agenda van de vergadering bestaat oorspronkelijk uit de volgende punten:
\begin{enumerate}
\item Opening
\item Agenda
\item Goedkeuring van de notulen
\item Bespreking Final Report
\item Andere vragen
\item W.V.T.T.K.
\item Rondvraag
\item Sluiting
\end{enumerate}

Aan de agenda worden verder geen punten toegevoegd.

\subsubsection{Goedkeuring van de notulen}
Er zijn geen op- of aanmerkingen op de notulen.

\subsubsection{Bespreking Final Report}
Mission is niet altijd hetzelfde (soms enkel 10 dagen, soms 100 dagen). Moet consequent worden toegepast. Niet duidelijk wat onze missie is. Onduidelijk welke aspecten onder ons vallen. Voren verwijzen is niet fijn. Misschien te chronologisch opgeschreven, meer interesse in het resultaat en minder de manier waarop we tot het resultaat is gekomen. 

(ZOEK OP SCOPE, EFFECT/AFFECT?, CIA of HIAD consistent gebruiken, als een afkorting maar een keer gebruikt wordt niet doen, ms^2 blz 9, hyphenation door een plaatje heen, accuracy/reliability, the paper doet niks, meer uitbreiden over sensors, hoe zit thermal aan elkaar vast, significantie)

Inhoud oke, structuur minder.

\subsubsection{Andere vragen}

\subsubsection{W.V.T.T.K.}
Grading komt evt op vrijdag de grading met DD en NR doen als Nando Timmer en anders op maandag met HD erbij.

\subsubsection{Rondvraag}
Er zijn verder geen vragen.

\subsubsection{Sluiting}
De volgende vergadering zal plaatsvinden op 30-06-2015 om 11:00 in Fellowship Meeting room 2.
\newline\newline
Dawud sluit de vergadering om 12:24.