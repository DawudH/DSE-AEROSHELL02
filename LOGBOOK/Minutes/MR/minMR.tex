\subsubsection{Opening}
Locatie: Fellowship Meeting room 2\\
Aanwezig: HD, NR, DD, J, L, G, A, B, T, D, Su, Se\\
Verontschuldigd: \\
Afwezig: - \\
tijd: Joost opent de vergadering om 16:03\\

\subsubsection{Agenda}
De agenda van de vergadering bestaat oorspronkelijk uit de volgende punten:
\begin{enumerate}
\item Opening
\item Agenda
\item Goedkeuring van de notulen
\item Trade-off criteria
\item Validatie van de ORbit Tool
\item Andere vragen
\item W.V.T.T.K.
\item Rondvraag
\item Sluiting
\end{enumerate}

Aan de agenda worden geen punten toegevoegd.

\subsubsection{Goedkeuring van de notulen}
de notulen worden goed gekeurd

\subsubsection{Trade-off criteria}
- Decelerator mass
- Developement risk
- Stability (Cmalpha)
- Deceleration time(Clalpha)

J Deceleration time Clalpha, capability om slope van de trajectory aan te passen. 

B Massa. niet meer specifieke massa's, alleen voor structures. Daarom genoodzaakt om andere vergelijkingsmaat te vinden. 

HD vergelijking: stel, we maken geen control systeem. Dan massa control = 0. dat lijkt heel goed. waar zie je dat dit slecht is. T Cmalpha geeft dit aan. Cm/Cl wordt een soort van requirement. Dus het staat zowel in de trade-off als in de requirements. zelfde verhaal als met reliability. HD Je kunt Cm en Cl afkopen met control massa. T het werkt juist de andere kant op. HD: Oke, dan werkt het. goed over nagedacht. Dit moet je goed uitleggen bij je midterm review. 

HD: Nog opsplitsen van massa's? Er lijkt iets te missen. DD het is niet zo overzichtelijk. Werkt wel met goede uitleg. Echte tijd kunnen we nog niet geven, is nog te vroeg. discussie waarom geen tijd beschikbaar. Nog erg aghankelijk van PID controller. Het kan nog niet worden aangenomen dat die PID perfect werkt. kan niet constand 3g erop houden. HD: 3g constant aanhouden is niet altijd de goede manier. Wat is de ideale trajectory? Wat wil ik / wat kan ik? los van de tool. Stel dat je nog geen control system hebt. en kan niet, dan moet trade-off aanwakkeren. T wil eigenlijk zo snel mogelijk afremmen, dus continue 3g. Dat programmeren is lastig. 

kunt eigenlijk beter aannemen dat je alpha direct kunt aanpassen dan zien welke orbit je wilt hebben. en dan krijg je een beeld welk totaal alpha verschil je nodig hebt. kun je moeilijk kiezen. Je kunt toch cl en cd rho en snelheid altijd aanpassen zodat je 3g hebt. Ja, dat kan. HD: dit klinkt als een goed idee. later de PID nog inbouwen.

Daarom beter trade-off criteria. Als het lukt gaan we dan voor de deceleration time. 


B massa's samen of appart nemen? samen, maar met toelichting in de trade-off van de 3 verschillende massa's, die niet mee doen voor de echte criteria.


Cma en Cla hangen af van M en wordt vrij betrouwbaar emt toenemende 


Thermal andere weg op geslagen. Niet meer lay-up tool. Nu heat load vergelijken. hoe nieuwe tool werkt, waarom oude tool niet werkt.


\subsubsection{Validatie tool}
T de getallen zijn heel mooi. Verificatie werkt helemaal prima, maar validatie lukt niet omdat er geen refernetie cases zijn met een vergelijkbare trajectory. HD: meer dan dat kun je nog niet doen. 

\subsubsection{Andere vragen}
Personal in het het rapport? Nee, niet nodig. Als appart document in de dropbox. met welk pagina limiet? 1 pagina it is.

Wisselen taken, na de MTR? Ja. komt nog.

\subsubsection{W.V.T.T.K.}
HD: Midterm. netjes kleden

HD: chris mokkum. Heeft paper ingediend. Komt een conferentie. Het zou leuk zijn als jullie dit volgend jaar deden. Volgende week wil hij ook wel langs komen. Je kunt vragen bij hem kwijt. 

HD mensen uit de VS. gast van NASA op het bureau. En andere gast ook bij nasa gewerkt. Nu hoogleraar. Zij willen de bachelor verbeteren.

\subsubsection{Rondvraag}
A veranderingen van zalen. ACTIEPUNT uitzoeken. En uitzoeken symposium dag. ACTIEPUNT MAILEN

\subsubsection{Sluiting}
Er is nog niet besproken wanneer de volgende vergadering zal zijn.\\

Joost sluit de vergadering om 11:07.\\

