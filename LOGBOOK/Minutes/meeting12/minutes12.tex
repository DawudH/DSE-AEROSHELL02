\subsubsection{Opening}
Locatie: Fellowship Meeting room 2\\
Aanwezig: HD, NR, DD, J, L, G, A, B, T, D, Su, Se\\
Verontschuldigd: \\
Afwezig: CM \\
Tijd: Dawud opent de vergadering om 11:03\\

\subsubsection{Agenda}
De agenda van de vergadering bestaat oorspronkelijk uit de volgende punten:
\begin{enumerate}
\item Opening
\item Agenda
\item Goedkeuring van de notulen
\item Orbit \& Control
\item Status update
\item Andere vragen
\item W.V.T.T.K.
\item Rondvraag
\item Sluiting
\end{enumerate}

Aan de agenda worden verder geen punten toegevoegd.

\subsubsection{Goedkeuring van de notulen}
Er waren geen op- of aanmerkingen.

\subsubsection{Orbit \& Control}
Tijdens de vorige meeting op vrijdag was dit ook al uitgebreid besproken met Chris. Vrijdagmiddag is er een afspraak geweest met Erwin Mooij. Daaruit volgde dat het transient deel van de bank vector verwaarloosd mag worden. HD merkt hier het verschil op tussen bank angle en slip angle. Bij een constante bank angle zou je namelijk verwachten dat de slip angle niet constant is (maar steeds groter wordt). Hierdoor zou in de bochten een grotere $g$-load moeten zijn. T zegt dat hier geen sprake van is omdat we geen constante hoogte aanhouden. De richting van de $g$ zal veranderen (in het aerodynamic reference frame), maar de magnitude niet. Dit komt doordat het body reference frame draait t.o.v. het aerodynamic reference frame. In het body reference frame zou hij dus hetzelfde moeten blijven.
\newline\newline
Vervolgens wordt $\alpha$-control vs. bank-control weer besproken. De groep legt eerst hun visie uit. Bank angle aanpassen is makkelijk omdat de moment of inertia om die as lager is en het is neutrally stable of zelfs net unstable. $\alpha$ is veel stabieler, waardoor het ook moeilijker aan te passen is. Met banking kun je het vehicle minder diep de atmosfeer in drukken (dus langzamer dalen), omdat de negatieve liftvector schuin/zijwaarts gepoint kan worden, waardoor het component dat naar beneden wijst een lagere magnitude heeft. Het voordeel van $\alpha$-control is dat de drag is te vari\"{e}re. Een nadeel is dat het een overshoot heeft in de acceleratie. Ook is het vatbaarder voor de varieteit van de atmosfeer. Met $\alpha$-control varieert de hoogte wanneer je probeert te controleren, waardoor je sneller door meer luchtlagen zal gaan dan bij bank-control. Het probleem zou dat er minder ruimte is voor correctie. Bij $\alpha$-control is het gevaar dat je zal crashen of skippen als je te laat je control toepast. Met $\alpha$-control zou je veel sneller je hoogte kunnen aanpassen t.o.v. bank-control. De vraag is dan wat je liever hebt.
\newline\newline
Het gaat erom hoe snel en hoeveel je kan (en wilt) vari\"{e}ren. Effect van $\beta$-change is veel kleiner dan $\alpha$-change. Dus snel en veel reageren ($\alpha$) of minder en gedoseerd reageren (bank). Voor de eerste pass-through in de atmosfeer lijkt bank-control sowieso beter. HD vraagt wat nu de reden dat je $\alpha$-control moet afstrepen. De enige manieren om $\alpha$ te controleren lijkt veranderende c.g.-offset en locked bodyflaps. Beiden lijken lastig te realiseren voor ons. Ook introduceer je extra loads wanneer je dit niet wilt. Beiden zouden toch feasible zijn, maar bank-control lijkt meer feasible, omdat de benodigde systemen lichter zouden zijn. Verder is bank-control ook hetgeen wat meer wordt toegepast. Tot nu toe enkel IRVE gezien met $\alpha$-control. Erwin Mooij heeft nog een paper geschreven over een combinatie van $\alpha$-control en bank-control. Verder wordt weer aangehaald dat bank-control niet zo accuraat zou zijn, zo hoefde Apollo alleen maar in zee te landen. De groep zegt dat dit niet waar was en dat er wel degelijk een bepaalde landing-area was aangewezen waarop gemikt werd. De paper van Erwin Mooij laat zien bank-control een accuraat pad kan aanhouden. Dat kunnen wij niet modelleren, omdat in de paper active control wordt gebruikt..
\newline\newline
Er wordt nog gewaarschuwd over het beslissen op basis van papers van anderen. Het is veilig, maar zal niet altijd werken. Zeker als de papers ook maar concepten zijn. Liever meer creativiteit dan bestaande dingen opleuken. Verder wordt nog benadrukt dat niet alles perfect in detail uit te werken is.

\subsubsection{Status update}
\textit{Aerodynamics}\newline
Sensitivity analysis uitgevoerd om te kijken hoe alles met elkaar schaalt/verhoudt. Belangrijke conclusie is een geskewde vorm voor een goede L/D.
\newline\newline
\textit{Structures}\newline
Forces zijn geanalyseerd. Diktes van de inflatables komen overeen met IRVE. Ook zijn grotere diameters structurally feasible. Voorkeur voor een pointed shape als je naar bendig stiffness kijkt. Een halfcone-angle van $40^{\circ}$ lijkt optimaal. Ook is er naar het inflation en deployment system gekeken.
\newline\newline
\textit{Thermodynamics}\newline
Sensitivity van TPS-mass tegen heat flux, diameter en tijd wordt/is bekeken. Langs geweest bij Ferry Schrijer met vragen over radiation van plasma in hypersonic flow en over contact resistance. Nu wordt dit gemodelleerd met een laagje met hele lage varierende conductivity. Ferry Schrijer gaf als tip om eventueel de diktes van deze lagen aan te passen.

\subsubsection{Andere vragen}
Er waren geen vragen ingediend.

\subsubsection{W.V.T.T.K.}
Er zijn W.V.T.T.K.'tjes ingediend.

\subsubsection{Rondvraag}
J vraagt in hoeverre er al infrastructuur is op Mars. HD zegt dat we mogen aannemen dat op Mars alles is om daar te overleven en de reis terug kunnen maken. Er zijn echter nog geen mensen, dus het moet simpel blijven.

NR vraagt of er al een kostenanalyse is gemaakt. Deze staat voor volgende week vrijdag gepland.

\subsubsection{Sluiting}
De volgende vergadering zal plaatsvinden op 12-06-2015 om 11:00 in Fellowship Meeting room 2.
\newline\newline
Dawud sluit de vergadering om 12:26.