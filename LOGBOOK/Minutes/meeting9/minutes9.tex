\subsubsection{Opening}
Locatie: Fellowship Meeting room 1\\
Aanwezig: HD, NR, DD, J, L, G, A, B, T, D, Su, Se\\
Verontschuldigd: - \\
Afwezig: - \\
Tijd: Dawud opent de vergadering om 10:08\\

\subsubsection{Agenda}
De agenda van de vergadering bestaat oorspronkelijk uit de volgende punten:
\begin{enumerate}
\item Opening
\item Agenda
\item Goedkeuring van de notulen
\item Bespreking Mid Term Review
\item Andere vragen
\item W.V.T.T.K.
\item Rondvraag
\item Sluiting
\end{enumerate}

Aan de agenda zijn verder geen punten toegevoegd.

\subsubsection{Goedkeuring van de notulen}
Notulen van de vorige meeting (woensdag 27-08-2015) zullen via de mail verstuurd worden.
De notulen worden verder goed gekeurd.


\subsubsection{Bespreking mid term review}
Vorige meeting zijn we tot Chapter 8 gekomen.
\newline\newline
\textit{Chapter 9}\newline
HD komt de side slip angle $\beta$ voor het eerst tegen op blz. 40. Hij vraagt zich af hoe de roll/pitch/yaw dan werkt in de orbit. D zegt tot nu toe enkel 2D motion bekeken te hebben. Sideslip $\beta$ wordt tot nu toe als te complex gezien. HD waarschuwt dat iets niet weggelaten mag worden omdat het te moeilijk is voor de huidige fase van ontwerp. Detail-ontwerp zou ook dit soort dingen moeten bevatten, desnoods versimpel je het probleem. Zelfde geldt voor subsystemen. Je wilt weten wat je ongeveer meeneemt en hoeveel het ongeveer weegt etc.
\newline
NR vraagt of er een minimum Mach-number is voor de Newtonian flow theory. Se bevestigt dat het alleen in het hypersonic-regime geldt (M>5). HD complimenteert de documentie van de V\&V, maar waarschuwt dat het niet te veel aandacht moet krijgen in het report. NR geeft aan dat validatie teruggekoppeld moet worden met de aannames. HD had graag geweten hoe de vorm van de concepts in Figure 33 is gedefinieerd.
\newline
HD merkt ook op dat $C_M$, $C_MA$, $C_{M_\alpha}$ worden door elkaar gebruikt, zelfde geldt voor de gebruikte units. De groep geeft aan dat dit komt door last-minute aanpassingen.
\newline\newline
\textit{Chapter 10}\newline
NR merkt op dat er iets mis is gegaan met de editing. Soms staan zinnen dubbel. De groep geeft aan dat er wat problemen waren met het file-exchange program (gitHub). HD vindt dat de beschrijving van de tool aan de lange kant is. Hij had graag meer uitleg over de ontwerpen zelf gezien. Dus meer resultaten van de tool zelf. NR waarschuwt de groep voor het feit dat er waarschijnlijk veel uitgezocht moet worden voor het uiteindelijke ontwerp. Tools zullen niet alle informatie geven. HD is bang dat er te veel focus op de tools ligt en dat de elementaire ontwerp kennis wellicht nog ontbreekt. Het gaat om het begrip van het ontwerp en de outputs van de tools zou niet blind aangenomen mogen worden.
\newline
HD vraagt waar de 14\% backshell mass van Steinfeldt vandaan komt en of het toepasbaar is op onze missie. A geeft aan dat het niet per se om het getal gaat, maar vooral om aan te tonen dat het rigid concept uiteindelijk te zwaar wordt. Het is volgens HD niet duidelijk wat er gebeurt bij het groter/kleiner maken van het vehicle.
\newline
HD adviseert om ook zeer goed na te denken over het ontwerp. Er wordt van alles gevraagd tijdens de Final Review en het Symposium en ze kunnen daarin best hard zijn.
\newline
HD merkt op dat alle concepten worden op eenzelfde trajectory vergeleken. Maar dit is al fout. Mid Term Review lijkt te vroeg te zijn geweest. We hebben als groep besloten om het zo te doen, aangezien er op dat moment niet meer data over verschillende trajectories beschikbaar was en het waarschijnlijk niet mogelijk was om voor de Mid Term dit nog nauwkeuriger te doen. HD vraagt of we als groep weten wat het kritieke punt is voor het halen van deadlines. Dit lijkt vooral over trajectories te gaan. D geeft aan dat een trajectory bepaald kan worden, maar dat er feedback nodig is van de thermal tool.
\newline
NR vraagt waarom de aerial thickness is gegeven en waarom sommige waardes niet zijn gegeven. Na verklaringen van A en B zegt hij dat er best geschatte waardes gegeven mogen worden als de waardes niet te vinden zijn in Table 7. Sowieso kunnen sommige kolommen in Table 7 weggelaten worden. Verder had een maximum temperature kolom handig geweest bij Table 7.
\newline
HD geeft aan dat communicatie bij problemen heel belangrijk is. Geldt ook voor buiten het DSE.
\newline
NR vraagt hoe isotrope en anisotrope materialen worden meegenomen. B zegt dat daar meer uitgewerkt moet worden. HD vindt dat er betere uitleg moet zijn over hoe de verdere figuren gebruikt kunnen worden in een ontwerp.
\newline\newline
\textit{Chapter 11}\newline
NR merkt op dat Table 8 met material properties voor structural dubbelop is. Verder vraagt hij of de thermo-groep weet wat de warmte doet met de inflation gas. B geeft aan dat het niet veel uit zal maken, zolang de temperatuur niet te hoog wordt. Ook vraagt hij of er naar contact resistance wordt gekeken, dus de resistance tussen de layers. Su zegt dat het model hier geen rekening mee houdt, maar dat de validatie hier kan laten zien of dit een plausibele aanname is.
\newline
NR adviseert goed te kijken naar de radiation. Nu wordt er enkel $T_\infty$ gebruikt. DD noemt als voorbeelden de infrarood-straling (IR) van Mars en het albedo. Verder vraagt NR of peak heat load van de ballute voor de payload module of de inflatable geldt. L zegt dat het voor allebei geldt.
\newline\newline
\textit{Chapter 12}\newline
HD vraagt waarom number 10 (Ability of structure to withstand mission loads) TRL1 heeft. B zegt dat dit inderdaad een hogere TRL-level heeft, omdat IRVE op aarde is getest. Een fout dus.
\newline\newline
\textit{Chapter 13}\newline
Besproken bij de review.
\newline\newline
\textit{Overig}\newline
Verder zijn er geen comments meer. 
\subsubsection{Andere vragen}
Er zijn geen verdere vragen ingediend.

\subsubsection{W.V.T.T.K.}

\textbf{1.} D vraagt wanneer de vergadering is waarin de groep haar beoordeling krijgt. HD zegt dat er later vandaag een afspraak staat gepland met Nando Timmer. Met de groep wordt het groepscijfer gegeven en dan wordt het invididuele cijfer in een minuut of twee per persoon medegedeeld. Dit kan bijvoorbeeld woensdag.

\textbf{2.} D vraagt of het nodig is om nog twee vergaderingen te hebben deze week? T meent dat een eventuele statusupdate wel handig op woensdag. Er wordt afgesproken om tussen 13:30 en 14:00 een vergadering in te plannen.

\textbf{3.} T vraagt of de personal appendix en peer review zijn gelezen. HD heeft het niet heel goed gelezen. Hij vraagt of de groep het nagekeken wil hebben. L zegt dat hij het liever wel ziet dat het gelezen wordt, zeker omdat er aangegeven is dat we een gesloten groep zijn. Dit soort dingen kunnen dan misschien meer inzicht geven. HD belooft het voor woensdag te lezen.

\textbf{4.} HD hoe het ging met de Amerikanen? Se vertelt dat we ze hebben uitgelegd wat de DSE is en wat wij daarin als opdracht hebben gekregen. Daarna hebben we per department individueel uitgelegd wat we gedaan hebben. Over het algemeen is dit wel goed gegaan.

\textbf{5.} NR vraagt aan HD wat de diepgang was van de structural analysis twee jaar geleden. HD zeg dat ze de loads op hun structure bekeken en een globaal idee hadden wat ze doen op hun structure. B legt vervolgens uit hoe de structural department dit tot nu toe heeft gedaan. Diepgang is volgens HD tot nu toe voldoende. HD adviseert vooral naar de subsystemen te kijken, het is namelijk ook gewoon een deliverable.

\subsubsection{Rondvraag}
Er zijn verder geen vragen.

\subsubsection{Sluiting}
De volgende vergadering zal plaatsvinden op 03-06-2015 om 13:30 in Fellowship Instruction room 1.
\newline\newline
Dawud sluit de vergadering om 12:18.