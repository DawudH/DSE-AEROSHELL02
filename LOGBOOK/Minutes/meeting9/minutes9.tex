\subsubsection{Opening}
Locatie: Fellowship Meeting room 1\\
Datum: 01-06-2015\\
Aanwezig: HD, NR, DD, J, L, G, A, B, T, D, Su, Se\\
Verontschuldigd: - \\
Afwezig: - \\
Tijd: Dawud opent de vergadering om 10:08\\

\subsubsection{Agenda}
De agenda van de vergadering bestaat oorspronkelijk uit de volgende punten:
\begin{enumerate}
\item Opening
\item Agenda
\item Goedkeuring van de notulen
\item Bespreking Mid Term Review
\item Andere vragen
\item W.V.T.T.K.
\item Rondvraag
\item Sluiting
\end{enumerate}

Aan de agenda zijn verder geen punten toegevoegd.

\subsubsection{Goedkeuring van de notulen}
Notulen van de vorige meeting (woensdag 27-08-2015) zullen via de mail verstuurd worden.
De notulen worden verder goed gekeurd.


\subsubsection{Bespreking mid term review}
Vorige meeting zijn we tot Chapter 8 gekomen.
\newline\newline
\textit{Chapter 9}\newline
HD ziet de side slip angle $\beta$ voor het eerst tegen op blz. 40. Hij vraagt zich af hoe de roll/pitch/yaw dan werkt. D zegt tot nu toe enkel 2D motion bekeken te hebben. Sideslip $\beta$ wordt als te complex gezien. HD waarschuwt dat iets niet weggelaten mag worden omdat het te moeilijk is. Detail-ontwerp moet dit soort dingen bevatten. Zelfde geldt voor subsystemen. Wat nemen we mee, hoeveel weegt het...

NR: is er een minimum Mach voor Newtonian flow theory?
Se: Hypersonic, dus M>5.
HD: V&V goed gedocumenteerd, maar pas op dat het niet te veel aandacht krijgt in het report. 

NR: Belangrijkste validatie-case IRVE, verschilt met 19\%. Er wordt niet gezegd waar dit door komt. Terugkoppelen met aannames.

HD: Vorm van de concepts in Figure 33 niet gedefinieerd.
NR: Wat is de wake van de ballute? (niet belangrijk)

HD: $C_M$, $C_MA$, $C_{M_\alpha}$ worden door elkaar gebruikt. Let ook op de units.
Groep: Laatste moment nog aanpassingen doorgevoerd.
\newline\newline
\textit{Chapter 10}\newline
NR: Er lijkt iets verkeerd te zijn. Soms staan dingen dubbel.
Groep: Er waren wat problemen met ons file-exchange program.
HD: Beschrijving van de tool is aan de lange kant. Meer uitleg over de ontwerpen zelf (resultaat van de tool).
NR: Wees je bewust dat je nog veel moet uitzoeken voor je uiteindelijke ontwerp. Tools geven niet alle informatie.
HD: Er is veel focus op de tool. Het lijkt alsof jullie nog niet de elementaire ontwerp kennis hebben. De ontwikkeling van tools is wellicht iets te vroeg geweest. Het gaat om het begrijpen van het ontwerp en niet blind aannemen van de output van de tools.

HD: Waar komt die 14\% van Steinfeldt vandaan? Is dit getal toepasbaar op onze missie?
A: Het gaat niet per se om het getal. Het stuk gaat er meer om dat het rigid concept uiteindelijk te zwaar wordt.
HD: Het is dus niet duidelijk wat er gebeurt bij het groter/kleiner maken van het vehicle.

HD: Op blz.60, welke dynamic pressure gaat het hierover? Waar komen 2400 en 300 vandaan?
D: Dit is een resultaat van de orbit. Minder dan 3000.
HD: Denk goed na over je ontwerp. Het wordt zeker gevraagd tijdens FR en Symposium. Denk over alles na.

NR: Hier en daar typos.

HD: Alle concepten worden op eenzelfde trajectory vergeleken. Maar dit is al fout. MR lijkt te vroeg te zijn geweest.
B: We hebben dit zo moeten doen, aangezien er op dat moment niet meer data over verschillende trajectories beschikbaar was. 
G: We hebben hier voor gekozen, omdat we zagen dat het niet mogelijk zou zijn om voor het MR dit nog nauwkeuriger te doen.
De groep heeft met elkaar afgesproken om voor vrijdag de tools af te hebben.
HD: Hebben jullie door wat het kritieke punt is voor het halen van deadlines. Dit lijkt vooral over trajectories te gaan.
D: We kunnen een trajectory bepalen, maar hebben feedback nodig van de thermal tool.

NR: Waarom geef je een aerial density? Is er maar één dikte mogelijk? Hoe zit het met bijvoorbeeld een fiber.
A: Ik weet niet hoe het precies zit. Minimale dikte is een van de overwegingen.
B: Naamgeving is misschien fout, minimum aerial density had beter geweest.
NR: Waarom dan niet gewoon minimum thickness? En waarom staan er sommige waardes niet bij?
B: Die waardes konden we niet vinden.
NR: Je mag best een geschatte waarde aan hangen.

HD: Weten jullie al iets van materiaalkeuze bij je structuur of TPS?
L: Tot nu toe is de nadruk geweest op het maken van een tool die de materiaalkeuze variabel houdt. 
B: De structurele materialen zijn toepasbaar.
NR: Waarom staat de maximum temp niet in Table 7?
B: Die had er beter bij gemoeten.

Communicatie bij problemen is heel belangrijk. Geldt zeker ook voor buiten het DSE.

NR: Isotrope en anisotrope materialen, hoe worden die meegenomen?
B: Daar wordt weinig over vermeld. Daar zullen we dieper op in moeten gaan.

HD: Bij 5m lijkt er geen massa te zijn.
B: Dit komt doordat dan de centerbody 5m is en er dus geen inflatable is.

HD: Foutje met Fig 45/46

HD: Hoe moeten de plaatjes gebruikt worden? Dat had in de conclusie gemoeten.
\newline\newline
\textit{Chapter 11}\newline
NR: Wall temperature is stagnation temperature?
L: Ja

NR: Tabel met material properties is voor structural dubbelop.

NR: Enig idee wat de warmte doet met het gas?

NR: vragen over validatie
NR: Kijken jullie naar contact resistance, aangezien Del Corso dit belangrijk noemt.
Su: Nee, het model houdt hier geen rekening mee.

HD: Wat doen de kinkjes in figure 55.

NR: Kijk goed naar de radiation, gebruik je enkel $T_\infty$?
DD: Gebruik bijvoorbeeld Mars IR, albedo.

NR: Peak heat load van de ballute, geldt die voor de payload module of the inflatable?
L: Voor allebei.

\newline\newline
\textit{Chapter 12}\newline
HD: Waarom heeft number 10 (Ability of structure to withstand mission loads) TRL1?
B: Deze heeft inderdaad een hogere TRL-level, omdat IRVE op aarde is getest. Een fout dus.
\newline\newline
\textit{Chapter 13}\newline
Besproken bij de review.
\newline\newline
\textit{Overig}\newline
Verder zijn er geen comments over 
\subsubsection{Andere vragen}
Er zijn geen verdere vragen ingediend.

D: Wanneer is de vergadering waarin wij de beoordeling krijgen?
HD: Vandaag afspraak met Nando Timmer. Met de groep het groepscijfer en dan invididueel een minuut of twee. Dit kan bijvoorbeeld woensdag.

D: Is het nodig om nog twee vergaderingen te hebben?
T: Woensdag is een eventuele statusupdate wel handig.
13:30-14:00 volgende meeting.

T: Hebben jullie de personal appendix en peer review gelezen?
HD: Niet heel goed gelezen. Willen jullie dit nagekeken hebben?
L: Liefst willen we dat jullie het wel lezen. Jullie geven aan dat we een gesloten groep zijn, dit soort dingen kunnen misschien meer inzicht geven.
HD: Oke, zullen we doen voor woensdag.

\subsubsection{W.V.T.T.K.}
\textbf{1.} HD: Hoe ging het met de Amerikanen? 
Se: We hebben ze uitgelegd wat de DSE is en wat wij daarin als opdracht hebben gekregen. Daarna per department individueel uitgelegd wat gedaan hebben. Is allemaal goed gegaan.

\textbf{2.} NR: Wat hebben ze twee jaar geleden gedaan? HD: Loads op je structure en een globaal idee wat ze doen op je structure. B legt uit hoe de structural department dit tot nu toe heeft gedaan. Diepgang is tot nu toe voldoende. Ga vooral naar je subsystemen kijken, is ook een deliverable.

\subsubsection{Rondvraag}


\subsubsection{Sluiting}
De volgende vergadering zal plaatsvinden op 03-06-2015 om 13:xx in Fellowship Meeting room 2.
\newline\newline
Dawud sluit de vergadering om 12:18.