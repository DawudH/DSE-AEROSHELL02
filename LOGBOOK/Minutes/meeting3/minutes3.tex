\subsubsection{Call to order}
Locatie: Fellowship Meeting room 2\\
tijd: Joost opened the meeting at 16:01\\


\subsubsection{Agenda}
J: Ik wil tussen puntje twee en drie nog een kopje voor het goedkeuren van de notulen toevoegen. Volgende keer staat die er standaard in.\\

\subsubsection{Acceptance of minutes}
G: Bij het antwoord op mijn vraag over het snelheidsregime waarin we moeten ontwerpen miste dat we ook voor supersoon moeten ontwerpen.\\
L:Ik zal het aanpassen.\\

\subsubsection{Action points}
J: Een action van de vorige vergadering was het afmaken van het verbeterde project plan. Is dit naar tevredenheid gebeurt?\\
HD: Ik wist niet dat jullie hier verder commentaar op wouden hebben.\\
J: Ik vraag het meer omdat het vorige keer een action Item was.\\
HD: Ik heb het diagonaal doorgelezen en ik zag dat de veranderingen waren toegepast, verder heb ik geen commentaar.\\
DD: Ik heb mijn deel nog eens doorgelezen en alles was gewoon doorgevoerd, dus het is wel goed. Maar ik zag nog wel een typefoutje in de mission statement. en er was ook een foutje in de lay-out waarbij tekst door een tabel heen liep.\\

\subsubsection{Requirments}
\textit{Q1: T: met betrekking tot de g-loads, mag je 3 aardse g's in elke richting trekken of alleen in een richting?}\\
HD: Het gaat om de lengte van de vector, dus in een richting. Als de astronauten voor 10 dagen lang meer dan 3 g moeten dragen vinden ze dat niet fijn.\\
Se: We kunnen wel een budget maken over de tijd van de gehele missie, zodat we rekening kunnen houden met eventuele groei.\\
HD: Nee, dat is niet de bedoeling.

\textit{Q2: T: En hoe zit het met peak g loads?}\\
G: Moeten we ons altijd aan de 3 g requirement of we voor korte tijd ook meer g's dragen?
HD: In principe moet je gewoon ontwerpen voor de 3 g requirement, maar ik ben als klant wel geïnteresseerd in andere opties. Als jullie met een bepaalde trajectory komen die over het algemeen veel beter is maar wel heel even wat hogere g's moet kunnen hebben dan heb ik daar wel oor voor. Maar, het moet geen onderhandel punt worden, nogmaals, het is in principe de bedoeling dat je voor 3 g requirement ontwerpt.


\subsubsection{Survey}
\textit{Q1: Se: Ik heb een familie situatie en het zou kunnen zijn dat ik binnenkort wat uren ga missen. Wat is uw beleid hierin? Ik wil in principe wel gewoon mijn uren goed maken.}\\
HD: Shit happens. Kijk, daar kunnen we gewoon mee omgaan. Weg blijven moet alleen geen excuus worden om je werk niet te doen. Ik zeg niet dat het zo is, maar houdt er rekening mee. Ik had twee jaar geleden een DSE groepje waarbij iemand mazelen had en thuis moest blijven en dat hebben we prima kunnen regelen via Skype, dus er is altijd wel een oplossing te vinden. Je moet wel altijd laten weten aan Joris Melkert dat je weg bent, want hij is hier nogal streng in. Hij hoort dit soort dingen ook graag van te voren. Daarnaast moeten team members niet gewoon het werk van de persoon opvangen die weg gaat, maar het aankaarten als dat nodig is.\\

\textit{Q2: J: Ik had nog een mailtje gestuurd over MARS-Gram, maar misschien kunnen we dat zo oplossen.}\\
HD: Daar kunnen we na de meeting nog wel samen naar kijken.\\

\textit{Q3: L: Over het thermodynamica gedeelte van de literatuur studie, hoe diep moeten we dat nu uitzoeken? want het wordt of vrij makkelijk of juist veel te moeilijk. Ik weet echt nog niet hoe ik bijvoorbeeld een berekening kan maken van materiaal wat zichzelf aan het opofferen is, omdat de boundary conditions gewoon veranderen.}\\
HD: Zo veel detail is nu nog echt niet nodig. Je zou nu heel oppervlakkig moeten weten wat voor componenten er zijn en hoe het design beperkt wordt. Later komen de heftige berekeningen nog wel.\\
HD: Het is ook wel het geval met deze DSE opdracht dat het niet zo een twee drie het boekje volgen is. Je kunt niet even de bibliotheek inlopen en het juiste design punt opzoeken in bijvoorbeeld een Raymer boek. Je moet zelf echt een beetje gaan nadenken over het ontwerp.\\
L: Dus het is een beetje de jazz onder de DSE opdrachten?\\
HD: Hoe bedoel je? Daar houd ook niemand van?\\
L: Nee, in de zin dat je zelf moet improviseren en niet zomaar het boekje kunt volgen.\\
NR: Ik ben zelf bezig geweest met high temperature composites en ik kan je wel een tip geven. Niet alleen het materiaal kan een aanduiding geven van je thermal protection, maar ook de plaats kan een indicatie geven. denk aan de tegels op de space shuttle.\\

Zoekwoord: Requirements\\

\subsubsection{W.F.C.T.T.T}
HD: Hoe staat het met de planning? Hebben jullie een beetje het idee dat jullie op stoom zijn gekomen?\\
Allen: Ja hoor.\\
L: Ja, zeker na de literatuur studie, dan krijgt de opdracht steeds meer een bepaalde vorm.\\
T: We zijn gewoon lekker bezig en soms zie je mensen ook een eureka moment krijgen. 'OO, zo zit dat in elkaar' hoor je ze dan denken.\\

NR: Tijdens jullie dagelijkse meeting, bespreken jullie dan ook wat er inhoudelijk is gebeurt?\\
T: Ja, dat bespreken we.\\
NR: Houden jullie ook rekening dat jullie moeten gaan presenteren en ook nog een presentatie moeten maken?\\
T: Ja, dat staat ook in onze planning.

\subsubsection{Survey}
Er zijn geen verdere vragen

\subsubsection{Adjournment}
De volgende vergadering zal plaatsvinden op 01-05-2015 om 10:30 in meeting room 2.\\

Joost sluit de vergadering om 16:24.\\
