\subsubsection{Opening}
Locatie: Fellowship Meeting room 2\\
Aanwezig: HD, NR, DD, J, L, G, A, B, T, D, Su, Se\\
Verontschuldigd: - \\
Afwezig: - \\
tijd: Dawud opent de vergadering om 11:03\\

\subsubsection{Agenda}
De agenda van de vergadering bestaat oorspronkelijk uit de volgende punten:
\begin{enumerate}
\item Opening
\item Agenda
\item Goedkeuring van de notulen
\item Vernieuwde team indeling
\item Onduidelijkheden final deliverables
\item Logboek
\item Peer review
\item Bespreking mid term review
\item Andere vragen
\item W.V.T.T.K.
\item Rondvraag
\item Sluiting
\end{enumerate}

Aan de agenda worden geen punten toegevoegd.

\subsubsection{Goedkeuring van de notulen}
T merkt op dat $C_{L_\alpha}$ is aangepast sinds de laatste vergadering. Dit hoeft niet aangepast worden in de notulen, aangezien de aanpassing is gebeurd na de vergadering.
\newline\newline
De notulen worden verder goed gekeurd.


\subsubsection{Vernieuwde team indeling}
De groep heeft zich opnieuw ingedeeld over de verschillende organisatorische taken. T.o.v. de eerste helft van de DSE zijn de rollen met verification- en validation-taken vervallen. Het tekort aan taken is opgevuld door de secretaris te ondersteunen met een aparte notulist. Verder zijn er twee editors i.p.v. één voor het laatste report.
\newline\newline
De vernieuwde team indeling is als volgt:
\begin{tabular}{ll}
	\textbf{Rol}	&	\textbf{Persoon}\\
	Chairman & Dawud Hage\\
	Secretaris & Guido van Koppenhagen\\
	Notulist & Suthes Balasooriyan\\
	Editor I & Twan Keijzer\\
	Editor II & Sebastiaan van Schie\\
	Risk Engineer & Lucas Mathijssen\\
	Systems Engineer & Bj\"{o}rn van Dongen\\
	Planner & Joost van Meulenbeld\\
	Documentation manager & Alexander van Oostrum\\
\end{tabular}

\subsubsection{Onduidelijkheden final deliverables}
De groep is van mening dat de lijst met delivarbles voor het Final Report veel onduidelijkheden bevat, aangezien niet alles toepasbaar is op de missie. Een voorbeeld hiervan zijn de spacecraft systems characteristics (data rates, link budget etc.) uit de lijst van deliverables. HD geeft aan dat dit voorgeschreven deliverables zijn en dat het belangrijk is dat het in ieder geval over nagedacht moet worden en het moet vermeld worden in het Final Report. Dit mag eventueel beknopt zijn. De groep vindt ook dat sommige deliverables (material and astrodynamics characteristics) uit het Mid Term Report juist van groot belang zijn voor het Final Report en toegevoegd zouden moeten worden.

\subsubsection{Logboek}
NR zegt dat de laatste update dateert van 6 mei. Hij vindt het een nuttige deliverable en vraagt hoe dit kan. L zegt dat hij hierin tekort is geschoten en dat het inderdaad beter had gemoeten. HD haalt de laatste afspraak aan waarin de eisen van het logboek zeer laag waren gesteld. Het invullen hiervan mag geen last worden. De groep zal vanaf nu het logboek op de dropbox direct aanpassen, zodat het direct inzichtelijk is.

\subsubsection{Peer review}
NR herinnert de groep dat de deadline van de peer review vandaag (27 mei red.) is. Ook wordt gezegd dat er gebruik gemaakt kan worden van private en public comments. HD raadt de groep aan eerlijk te zijn, zodat een ieder zich kan verbeteren op zijn minder goede punten.

\subsubsection{Bespreking mid term review}
D zegt dat het waarschijnlijk niet lukt om alles te bespreken en stelt voor om tot de lunch (12:30) door te gaan en de rest voor de volgende meeting (28 mei red.) te laten. DD geeft aan niet te kunnen op vrijdag 28 mei, hij zal zijn overige punten op donderdag (27 mei) aan de groep geven. 
\newline
Het taalgebruik is goed volgens NR en HD vindt dat de summary een goede structuur heeft.
\newline\newline
\textit{List of symbols}\newline
Wanneer een unit als breuk [teller/noemer] in de tekst staat, wordt de regelafstand een beetje raar. Dit is volgens B (editor) gedaan om de leesbaarheid van langere units te waarborgen. 
\newline\newline
\textit{Introduction}\newline
NR las dat concept mass is opgedeeld in slechts drie delen. Hij vraagt zich af of de payload mass dan niet is weggelaten. Su zegt dat hier decelerator mass i.p.v. concept mass beter op zijn plaats was geweest.
\newline\newline
\textit{Chapter 2}\newline
Het stuk over de launch en interplanetry flight (dus voor de missie) is nogal kort. HD vraagt zich af of er genoeg is nagedacht over de implicaties van deze fases op de missie.
HD: launch/interplanetary flight nogal kort. Nagedacht over implicaties van deze fase op onze missie?\newline
DD vraagt of er nog andere requirements voor Phase 4. HD zegt dat er niets concreets uitgewerkt moet worden, maar dat er wel een plan moet zijn.\newline
HD vraagt of er nog naar het aantal astronauts is gekeken. G zegt dat dit laatste tijd niet echt meer bekeken is, maar stelt wel dat de groep snapt dat het er minder worden dan eerder gedacht bij de baseline review (toen waren het er zes).\newline
NR heeft een algemene comment over de alinea's. Hij raadt aan om er meer te maken, aangezien dat fijner leest dan een lange lap tekst.
\newline\newline
\textit{Chapter 3}\newline
Weinig inhoudelijk veranderd, geen comments verder.
\newline\newline
\textit{Chapter 4}\newline
Inhoudelijk hetzelfde gebleven. DD zegt dat het beter is om eerst het grote voordeel te noemen van een aeroshell voor sustainability. Nu wordt eerst gezegd dat het tamelijk onbelangrijk is en dan pas wordt er een voordeel genoemd.
\newline\newline
\textit{Chapter 5}\newline
HD vindt het hoofdstuk nogal leeg. Er wordt niet veel duidelijk in het hoofdstuk, dingen worden later in het report pas duidelijk. Er wordt weinig verwezen naar het verdere report. Hoofdstuk had als goede introductie voor de rest van het technische deel kunnen dienen.\newline
NR heeft een opmerking over een te grote $C_{m_\alpha}$, dat is namelijk ook niet heel goed voor de stabiliteit.
\newline\newline
\textit{Chapter 6}\newline
Er zijn complimenten voor de schetsen. DD vraagt zich af of het inderdaad eerlijk is om de control los te koppelen als trade-off. De groep geeft aan hier lang over nagedacht te hebben en daarop haar besluit heeft gemaakt.\newline
NR vraagt of een isotensoid met ram-air opgeblazen moet worden. A zegt dat de grootte van de inflatable van een isotensoid dit noodzakelijk maakt. Anders zou er onredelijk veel inflation gas mee moeten bij de launch. NR zegt wel dat er dan rekening gehouden moet worden met het feit dat de isotensoid niet is opgeblazen als het de eerste keer de atmosfeer binnenkomt.\newline
HD vraagt aan de groep of ze nog steeds achter de keuze staan om het control system zo laat te kiezen. T geeft aan dat per concept de mogelijke control systemen bekeken zijn. De grootte hiervan is nog onbekend en zal in de volgende fase behandeld worden. HD zegt dat dit onvoldoende in het report staat.\newline
Er is een vraag over noodzakelijkheid van de connector in de tension cone FBD B zegt het een structureel onderdeel is.\newline
Er wordt opgemerkt dat Tabel 3 waarschijnlijk iets later in het hoofdstuk had gekund.
DD: Concept control systems had eerder in het hoofdstuk gemoeten bij Tabel 3.
\newline\newline
\textit{Chapter 7}\newline
DD zegt dat de titel \textit{Subsystem design interfaces} de lading van het stuk niet helemaal dekt. Dit had beter \textit{Design interfaces} genoemd kunnen worden. Elementen in de N2-chart zijn namelijk geen subsystems.\newline
HD vraagt zich af of de groep wel genoeg weet over subsystems. A zegt dat zolang er geen final concept was, er vaak niet niet op subsysteem-niveau geanalyseerd kon worden.\newline
DD zegt dat het figuur met de \textit{System communication flow} suggereert dat de ADCS-computer het enige redundant component is. B stelt dat vermeld had moeten worden dat we ze fail-safe willen uitvoeren.
\newline\newline
\textit{Chapter 8}\newline
NR vindt dat het goeds dat de assumptions apart vermeld zijn. Er volgt een discussie over hoe de snelheid is gedefinieerd. HD vraagt zich af hoe de hoek van de incoming hyperbolic orbit is bepaald. Uitleg van Dawud/Twan is afdoende, maar HD vind dat dit niet duidelijk genoeg in het report staat. Het is onduidelijk wat de flight path vector is in Phase I.\newline
HD vraagt wat de bedoeling is met het control system. De referentie moet al bepaald zijn. D zegt dat ze tot nu toe zijn we bezig geweest met het vinden van een pad dat als referentie kan dienen. HD zegt dat ze bewust moeten zijn wat de referentie wordt. Op acceleratie (3g) sturen werkt blijkbaar niet.\newline
Er staat geen referentie voor de eigenschappen van Mars. Verder heeft DD een vraag m.b.t. de discretisation error en starting offset.\newline
DD vraagt ook of er andere pertubaties zoals J2-effect worden meegenomen. D geeft aan dat dit waarschijnlijk te veel werk gaat worden.\newline
Verder is er nog een opmerking over de vertical as van Figuur 23. DD raadt aan om break-symbols te gebruiken in grafieken. NR mist verder korte conclusies na elk technisch hoofdstuk.
\newline\newline
De rest van het report zal tijdens de volgende meeting (vrijdag 29 mei) besproken worden.
\subsubsection{Andere vragen}
Er zijn geen verdere vragen ingediend.

\subsubsection{W.V.T.T.K.}
\textbf{1.} HD herinnert de groep aan de komst van studenten van de Purdue University. Hij vraagt of de groep iets heeft bedacht voor de student(en). Hij vraagt om in ieder geval een introductie te geven over de hoe de DSE zelf werkt binnen de bachelor. Ook zou ons project binnen de DSE ge\"{i}ntroduceerd moeten worden en we kunnen eventueel laten zien hoe we de conceptuele fase zijn doorlopen.
\newline\newline
\textbf{2.} L vraagt of de groep er op 6 juli (tweede grading dag) bij moet zijn. HD zegt dat het symposium het laatste verplichte onderdeel is. Maar dat de grading meeting tussen de tutor, coaches en een DSE-co\"{o}rdinator op ofwel vrijdag 3 juli ofwel maandag 6 juli plaats vindt. Daarna zou de groep eerst als groep het groepscijfer en daarna individueel het persoonlijke cijfer kunnen ontvangen. In dat opzicht heeft 3 juli de voorkeur.\newline \underline{Actiepunt: Joris Melkert mailen om op vrijdag 3 juli de grading meeting te plannen.}
\newline\newline
\textbf{3.} Su vraagt hoe het precies zit met de Mid-Term grading. HD geeft aan dat de meeting met Vincent Br\"{u}gemann nog moet plaatsvinden. Hij benadrukt dat het cijfer een indicatie is naar waar het uiteindelijke cijfer toe gaat, aangezien er ook beoordeeld wordt op onderdelen die nog moeten komen. G vraagt of de grading sheet inzichtelijk is. Dit is niet zo, in de persoonlijke meetings worden de verbeterpunten aangegeven.

\subsubsection{Rondvraag}
B: Kan NR helpen met het bekijken voor het plan voor de structural analysis.

\subsubsection{Sluiting}
De volgende vergadering zal plaatsvinden op 29-05-2015 om 11:00 in Fellowship Meeting room 2.
\newline\newline
Dawud sluit de vergadering om 12:49.