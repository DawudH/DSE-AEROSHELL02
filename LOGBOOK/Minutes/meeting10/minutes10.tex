\subsubsection{Opening}
Locatie: Fellowship Instruction room 1\\
Aanwezig: HD, NR, DD, J, L, G, A, B, T, D, Su, Se\\
Verontschuldigd: - \\
Afwezig: - \\
Tijd: Dawud opent de vergadering om 13:35\\

\subsubsection{Agenda}
De agenda van de vergadering bestaat oorspronkelijk uit de volgende punten:
\begin{enumerate}
\item Opening
\item Agenda
\item Goedkeuring van de notulen
\item Mid-term grading
\item Status update huidig werk
\item Plan van aanpak toekomstig werk
\item Andere vragen
\item W.V.T.T.K.
\item Rondvraag
\item Sluiting
\end{enumerate}

Aan de agenda zijn verder geen punten toegevoegd.

\subsubsection{Goedkeuring van de notulen}
Opmerking over notulen van 27 mei: actiepunt over Joris Melkert was niet nodig, dit was verkeerd begrepen door Su (notulist).
Opmerking over notulen algemeen: De conclusie van discussies moet meer naar voren komen. Het is niet van belang om te notuleren wie wat precies gezegd heeft.
Verder geen op- of aanmerkingen.

\subsubsection{Mid-term grading}
Voorlopig groepscijfer 6. Met kanttekening dat dit cijfer een beetje vertekend is en dat de groep eigenlijk op een hoger cijfer zit, maar dit kan niet naar voren komen met het gegeven gradingsheet. Het groepscijfer is opgedeeld in drie onderdelen: Design, Process en Communication. De weging per onderdeel is niet gelijk, design weegt het zwaarst (ongeveer 40\%).
\newline\newline
Vooral de grading van het design is laag uitgevallen. Verklaring daarvoor is dat de focus vooral op het ontwikkelen van tools heeft gelegen. De groep is nog niet met echte waardes/schatting voor een eventueel design gekomen. Vooral de subsystemen zijn gebrekkig behandeled. Het control system mist bijvoorbeeld volledig. De sensitivity analysis en Verification \& Validations is daarentegen wel van goed niveau. Voor de V\&V kunnen eventuele afwijkingen wel beter beschreven/verklaard worden. Om relatief makkelijk meer punten te kunnen scoren op het grading sheet is het handig om de market and cost analysis uitgebreider te doen, door bijvoorbeeld een kostenplaatje aan de missie te hangen. Verder ontbreekt een goede requirements flowdown vanaf de top-level requirements nog.
\newline\newline
Het proces-deel is redelijk goed. Een puntje van kritiek is Use of External Resources. Kennis die we als groep niet hebben zou eigenlijk van buiten gehaald moeten worden. Er wordt aangeraden om anderen (niet alleen HD, NR en DD) te betrekken bij ons werk, ondanks dat we een gesloten groep zijn.
\newline\newline
Ook het communication-deel is redelijk goed. Weer ontbreekt hier communication with external staff. Er is tevredenheid over de meetings, de gebruikte terminologie en manier van refereren in het report. Ability to answer staff questions is dan weer wat minder. Als groep zijn we niet heel vatbaar voor de feedback die we krijgen. Vaak is het zo dat we als groep komen met oplossingen/idee\"{e}n en daar graag aan vast blijven houden en proberen vervolgens kritiek te weerleggen. Dit laat wel zien dat er eenheid binnen de groep heerst, wat wel een redelijk sterk punt is.
\newline\newline
Het groepscijfer telt voor 40\% mee en het individuele cijfer voor 60\%. Gemiddeld zitten we rond de 6-6.5, waarbij gezegd mag worden dat er veel ruimte is om omhoog te gaan. De tutor en coaches zijn tevreden over de groep als geheel. Een puntje waar iedereen als individu kan winnen is het begrip van de tools (understanding of subject matter). De attitude, job performance, initiative en communication zijn goed.
\newline\newline
B stelt een vraag m.b.t. het design wat eventueel gebrekkig kan zijn in de structures groep. Tot nu toe is er enkel een load/truss-analysis gedaan. Dit is in principe enkele analyse, dus de vraag is of er dan ook nog daadwerkelijk gedesigned moet worden. HD zegt dat er in een design moet komen zodat er een idee is hoe alles eruit gaat zien en in elkaar gaat passen. L vraagt of er nog extreem negatieve punten waren. Die waren er niet. Er mag geconcludeerd worden dat het cijfer wat laag is uitgevallen, mede door het gradingsheet. Maar er is zeker ruimte voor verbetering in de voorgenoemde punten.


\subsubsection{Status update huidig werk}
D vertelt dat we tot nu toe bezig zijn geweest met de uitwerking van tools, welke in principe vrijdag af moeten zijn. We hebben als groep interne meetings gehad over wat uit de tools moet rollen en hoe we die gaan koppelen met andere tools. Alle departments zullen kort vermelden wat ze gedaan hebben:

\textit{Aerodynamics}\newline
G vertelt dat de tool niet veel veranderd is t.o.v. het Mid Term Review. Er is nog een optimalisatie ingebouwd die zoekt naar \textit{een} optimale vorm. Nadruk op \textit{een} optimale vorm, aangezien het nog niet bekend is welke parameter van de vorm het design optimaal werkt. De optimalisatie werkt, volgende week zal bekeken worden wat het beste is. NR vraagt of de tools automatisch aan elkaar gekoppeld zijn of dat er datasets overgedragen worden. Er wordt vervolgens uitgelegd dat de datasets dus worden overgedragen. Orbit geeft bijvoorbeeld een dynamic pressure aan aero, waarna aero bijvoorbeeld heatflux aan thermo geeft. HD geeft aan dat het belangrijk is om te weten wat de relatie is tussen de design parameters en performance parameters. G bevestigt dat deze kennis nodig is om de gewichten aan de optimalisatietool te geven. HD zegt ook dat er (nog) geen optimaal design punt kan zijn. Het is beter om een initi\"{e}le waarde te hebben. Ook zonder optimalisatietool zouden we moeten kunnen zeggen welke richting we een bepaalde design parameter moeten veranderen om een bepaalde performance te verkrijgen. (Weer die link tussen design en performance parameters dus)
D: We zijn bezig met de uitwerking van tools, moet vrijdag af zijn. Discussies gehad wat uit tools moeten komen en hoe we die interfacen met andere tools. We gaan even de alle departments langs.
\newline\newline
\textit{Orbit}\newline
T vertelt dat er nu ook bank-control is ge\"{i}mplementeerd. De lift-vector wordt nu gepointed, de manier waarop is niet gegeven. Daar zal extern voor gezorgd moeten worden (door control). Voordeel is dat er minder afhankelijkheid is van density zoals bij enkel alpha-control. NR vraagt of er nu op hoogte gestuurd wordt. T bevestigt dat dit gebeurt m.b.v. alpha en sideslip-angle.
\newline\newline
\textit{Control}\newline
Se zegt nu bezig te zijn met de static stability en dat er problemen met het oplossen van drie vergelijkingen (singular matrix). NR vraagt hoe de control-surfaces gedimensioneerd zullen worden. D vertelt dat dit zal gebeuren met behulp van de benodigde momenten die uit de berekening zullen komen.
\newline\newline
\textit{Thermodynamics}
L vertelt dat er de 1D-tool verbeterd is. Oorspronkelijk werd een FTCS-scheme gebruikt, nu Crank-Nicolson ge\"{i}mplementeerd, welke als voordeel onvoorwaardelijke stabiliteit heeft. In eerste instantie wilden we een 3D-tool, maar dit heeft niet de prioriteit gekregen. De prioriteit ligt momenteel op het produceren van een design. Er wordt gedesigned op het stagnation point. Als de TPS faalt heb je twee keuzes: Andere lay-up of andere orbit. HD denkt dat de TPS de missie gaat defini\"{e}ren. L benadrukt dat de focus hier inderdaad op design ligt en niet op een mooie 3D-tool.
\newline\newline
\textit{Structures}\newline
A legt uit dat er bij structures minder sprake van een tool is. Het plan is om alle structele delen te bekijken: inflatables, centerbody, connection en inflation system. Er is een load analysis gedaan met de inflatables gemodelleerd als een truss-structure. Die zal gebruikt worden om het mass-model uit het Mid Term Review te verbeteren. Interne stressen worden vooralsnog niet bekeken, meer nadruk op systemen. Bij de inflatables is de vorm al vastgezet (dimensies niet) en zullen worden vastgehouden met straps. De voornaamste reden om de loads te bekijken is om te kunnen zeggen of iets feasible is of niet. Verder is het aantal toroids nog volledig variabel. G merkt op dat er wel een minimum zal zijn door de door aerodynamics opgelegde vorm. NR vraagt zich af wat er gebeurt bij de voids tussen de inflatables en TPS. In principe is dat gewoon geregeld doordat alles onder druk staat, enkel bij te weinig toroids zal het een probleem opleveren.
\newline\newline
HD vraagt of er al een diameter bekend is. De groep geeft aan dat dit nog niet bekend is, maar dat het volgende week bekend zal worden als er daadwerkelijk gedesigned gaat worden. HD zegt ook dat er best buiten de opgelegde 12m gedacht mag worden.

\subsubsection{Plan van aanpak toekomstig werk}
Gisteren (dinsdag 2 juni) een poging gewaagd om alle subsystemen die we gaan behandelen op een lijstje te zetten. Hierop staan welke subsystemen wij denken nodig te hebben, ook vanuit de deliverables gezien. Het is opgedeeld in delen over de decelerator, crew module en mission phase.
\newline
B vertelt dat we tot nu toe vooral bezig geweest met de tools voor de decelerator subsystems. Het ontwerp met deze tools zal uiteindelijk afdoende zijn voor het Final Report. Voor de subsystemen in de crew module is het plan om het volume, gewicht en eventueel locatie en power van de subsystems te bepalen/schatten om zo een completer beeld van het ontwerp te kunnen schetsen. We hadden al een hoofdstuk gewijd aan de mission phases in het Mid Term Report. We willen dit uitgebreider gaan doen, dus meer over de launch en meer over de interplanetary transfer. En dan vooral benadrukken wat voor invloed deze fases hebben op ons deel van de missie. (dit was ook een puntje van kritiek in de bespreking van het Mid Term Report).

\subsubsection{Andere vragen}
G vraagt of mogen aannemen dat we, zoals ook het plan is voor de Orion, met een transport capsule de interplanetary transfer doen. De transport capsule biedt dan een leefomgeving tijdens de langdurige transfer voor de astronauten. Dit verlaagt de eisen voor bijvoorbeeld lifesupport enorm. Nadelen zijn bijvoorbeeld een extra launch voor de transport capsule (naast het re-entry vehicle) en docking-requirements. HD zegt dat de groep daarover eerst een analyse moet doen en vervolgens met voorstellen moet komen. NR hint ook om te kijken naar de sterke punten van een aeroshell.
\newline
NR merkt vervolgens op dat dit soort vragen eigenlijk veel eerder hadden moeten komen. Als groep zijn we een beetje verkeerd om bezig. We zijn volgens HD iets te vroeg en te diep in de tools gedoken. Dit soort vragen zijn eigenlijk key drivers voor je design. Zo zal het veel uitmaken voor je design of je zo'n transport capsule mee krijgt of niet. Voorlopig mag er wel aangenomen worden dat er op Mars al spul aanwezig is, wel moeten de astronauten terug naar Aarde kunnen gaan.

\subsubsection{W.V.T.T.K.}

\textbf{1.} NR geeft aan vrijdag niet aanwezig te kunnen zijn bij de vergadering. Vrijdag komt Chris Mockel. Hij heeft twee jaar geleden de DSE gedaan. Chris Mockel is op dit moment bezig met zijn master en houdt zich vooral bezig met astrodynamica.

\subsubsection{Rondvraag}
Er zijn verder geen vragen.

\subsubsection{Sluiting}
De volgende vergadering zal plaatsvinden op 05-06-2015 om 11:00 in Fellowship Meeting room 2.
\newline\newline
Dawud sluit de vergadering om 15:00.