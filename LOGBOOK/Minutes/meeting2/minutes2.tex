\subsubsection{Call to order}
Locatie: Fellowship Meeting room 1\\
tijd: Joost opened the meeting at 14:00\\

\subsubsection{Agenda}
J: Ik wil tussen puntje twee en drie nog een kopje voor het goedkeuren van de notulen toevoegen.\\

\subsubsection{Acceptance of minutes}
De notulen zijn door iedereen goed gelezen, bravo.\\

N: Mijn naam is verkeerd gespeld in de notulen, het is namelijk Reurings en niet Reuring. Dit staat volgens mij ook fout in de reader. Maar verder is het wel goed.\\

L: Ik zal het aanpassen.\\

\subsubsection{Review of Project Plan}
HD: Jullie hebben het werk mooi op tijd ingeleverd. Dat is goed, want met een goed project plan valt of staat een project.\\

J: We hebben het rapport van te voren ingeleverd. Hebben jullie nog opmerkingen over het verslag die wij kunnen doorvoeren?\\

HD: Ik heb naar het verslag gekeken en ik heb nog een paar puntjes.\\
\textit{1: HD: De preface, nou nou.. dit ziet er een beetje uit als slijmen naar de begeleider. Als klant zit ik hier niet echt op te wachten. houd het vooral zakelijk.}\\

\textit{2: HD: In de introduction bij de purpose of the project plan geven jullie aan een general approach te hebben. maar dit project is juist heel specifiek, daar stoorde ik mij aan.}\\
Su: Het gaat eigenlijk meer over onze aanpak van dit project.\\
HD: Zou je bij het ontwerpen van een reactor het zelfde plan van aanpak hebben? Ik hoop het niet!\\
T: Wat Suthes bedoeld is dat het gaat over het al omvattende plan binnen dit specifieke onderwerp.\\

ZOEKWOORD: STRAWMAN CONCEPT\\

\textit{3: HD: De objective statement geeft wel heel specifiek aan dat er gebruik gemaakt moet worden van bestaande launchers, dit trekt de aandacht weg van het eigenlijke doel.}\\
L: Dit is om duidelijk aan onze klant aan te geven wat wel en niet binnen onze taakomschrijving valt.\\
HD: En andere requirements niet? dan zou je ze er allemaal in moeten steken.\\
NR: Zorg dat je statement SMART is (Specific Measurable Attainable Realistic and Time dependent).\\

\textit{4: NR: Er missen in die sectie ook referenties. zo staat er bijvoorbeeld een hele mooie mission tabel, maar er staat nergens een referentie waar de data vandaan komt.}\\
T: Die staat geloof ik wel ergens in de tekst.
HD: Wees secuur! het is hier niet helemaal duidelijk. \\
NR: Wees er zelf erg bewust van wat je opschrijft, niet verwijzen wordt gezien als plagiaat en kan grote gevolgen hebben.\\
HD: dus elke keer als er een statement wordt gemaakt in je tekst zonder dat deze is uitgelegd, moet deze statement van jezelf komen of ergens anders vandaan komen. In het laatste geval is een echt referentie nodig! Wees er heel voorzichtig mee.\\

\textit{5: DD: jullie mission need statement is vreemd geformuleerd. Nu lijkt het alsof jullie zelf gaan demonstreren dat het werkt, hoewel jullie alleen maar het design verzorgen.}\\
All: We zullen het aanpassen.\\

\textit{6: DD: In section 2.1 staat ook ergens aircraft i.p.v. spcecraft}

\textit{7: HD: In section 2.2 begint jullie requirement code altijd met CIA en SYS, deze code is daardoor onnodig lang.}\\
L: Dat is zo geformuleerd omdat we eerst wouden werken met meerdere concepten, maar dat is nu niet meer nodig. Tevens staat CIA altijd aangegeven zodat de klant direct kan zien over welk project het gaat, van de vele projecten waarin hij of zij geïnvesteerd heeft. Ik zal het aanpassen.\\

\textit{8: HD: Requirement 5 is ook niet echt duidelijk}\\
A: Deze requirement is overgenomen uit de reader en is sowieso vaag.\\
HD: Wat was er niet duidelijk dan?\\
A: Ik snap niet waarom deze requirement zo beschreven is. de massa van de heat shield mag maar 10\% van het totale gewicht, waarbij het totale gewicht 10,000 kg is. Maar deze gewichten staan helemaal niet vast. Moeten de ene mee schuiven met de ander als de gewichten veranderen?\\
HD: Ja, dat klopt en die 10\% is een maximum gewicht. Ik hoor het graag als er geschoven moet worden met de requirement.\\
NR: Het uiteindelijke gewicht zal zijn hangt af van hoe het zich in de praktijk zal uitwijzen.\\
Hd: Het grote lijn idee is in ieder geval dat je voertuig aan het begin van de re-entry een gewicht van 10,000 kg heeft. Deze massa moet worden afgeremd.\\

\textit{9: HD: bij de OBS zie ik nergens een astrodynamics department?}\\
T: We hebben hierover nagedacht, dit valt onder de orbital- en control department.\\

\textit{10: NR: In de tekst staat niet goed beschreven wat Verification en Validation is. Weten jullie hoe en wat het is?}\\
T: Verification heeft meer met je eigen requirements te maken, hoewel validation meer over requirements van de customer gaat.\\
L: Zoals ik het zie gaat verification erover of je het product correct maakt en validation erover of het product wel de waarheid simuleert.\\
HD: Dit moeten jullie nog wel even aanpassen. Verification gaat over de vraag of je het goed doet en validation of je het goede doet.\\

\textit{11: HD: Ik zag ook dat er nog steeds maar een planner is en maar een persoon op systems engineering. Hier hebben we het vorige keer nog over gehad! Waarom is er niets veranderd?}
G: In principe werken we er met meer mensen aan, maar er is een eind verantwoordelijke.\\
DD: Op deze manier valt er wel heel veel verantwoordelijkheid op de SE. Zorg dat er goed ondersteuning is, die heb je nodig.\\
T: V\& V team geeft ook ondersteuning aan de SE. SE let erop of het gebeurt, V\& V of het wel goed gebeurt.\\
HD: En wat doet de planner als jullie je deadlines niet halen.\\
A: Het idee is dat we niet gaan achterlopen, daar is de planning voor.\\
NR: Hier moet je mee uitkijken, dit gaat sowieso wel gebeuren en als het gebeurt schuift het werk op en dat gaat altijd ten koste van het schrijfwerk. Dat zal de editor niet zo leuk vinden.\\
T: In principe moet de planner dan op tijd aan de bel trekken.\\
D: Ja, maar we moeten ook beschrijven hoe en wanneer.\\
A: Zo vroeg mogelijk.\\
NR: Dus jij houdt overzicht over de planning en stuurt dagelijks bij?\\
A:Ja, aan het einde van de dag, na onze group meeting.\\

\textit{12: HD: Over de WBS: Jullie hebben ook tijd ingeplant voor het ontwikkelen van tools, dat is heel goed! dat wordt vaak vergeten.}\\
NR: Ik vond het juist wel een beetje de andere kant opslaan. Er was wel veel tijd vrijgehouden voor tools, maar weinig tijd voor het daadwerkelijke design. De twee kunnen niet zonder elkaar.\\
DD: Gaan jullie alles zelf ontwikkelen?\\
D: Een combinatie van.\\
DD: Het lijkt nu inderdaad veel werk om de tools te ontwikkelen, waar je kunt uitbesteden zou ik uitbesteden.\\

\textit{13: HD: Ik merk dat jullie vaak wel weten wat jullie willen gaan doen, maar dat het niet duidelijk staat opgeschreven. Als dit in een keer duidelijk wordt gedaan scheelt het een hoop discussie tijd! (goede tip)}\\

\textit{14 NR: Ik zie ook niet zo goed waar de iteration in jullie design proces staan weergegeven.}\\
G: In het detailed design.\\

\textit{15: HD: Jullie workpackages in de WBS komen niet overeen met de workpackages in de Gantt-chart.}\\
B: Dat komt omdat de WBS vrij globaal is en de Gantt-chart niet. We volgen niet de workpackages omdat we zo beter kunnen schuiven in de gantt-chart als dat nodig blijkt.\\
HD: Het is gebruikelijk om wel de zelfde workpackages te hebben. Het moet naadloos in elkaar over kunnen lopen.\\

\textit{16: HD: Verder is er volgens mij inhoudelijk niets mis mee.}\\

\textit{17: NR: Bij sustainability moet je niet zeggen dat het niet belangrijk is als je er net een hele pagina aan hebt gewijd. Dan ben je jezelf aan het onderuit halen.}\\
DD: Er er miste ook COOSPAR.\\
Se: Zal het aanpassen.\\

\textit{18: NR: In je referentie lijst moet je wel schrijven of iets een thesis /journal of een article is, zodat ik weet ik moet benaderen als ik verdere informatie wil hebben.}\\
HD: Gebruiken jullie bibtech? weten jullie hoe dat werkt? Je kunt ook "mandolate" gebruiken voor het managen van je referenties, dan krijg je geen dubbele referenties in je lijst, etc. Dat werkt het beste uit mijn ervaring.\\
DD: Je kunt ook de bibtech code direct van google scholar halen.\\
D: Ik zal kijken of ik dat kan oplossen.\\

\textit{19: HD: Dit was veel kritiek, maar jullie hoeven niet te denken dat jullie het slecht hebben gedaan, inhoudelijk was er niets mis met het werk.}\\

J: Bedankt voor de feedback.\\


\subsubsection{Deliverables}
J:Er is een mismatch in de deliverable tabel. Op blackboard zijn er updates en het is niet duidelijk wat we nou moeten inleveren.\\
HD: Ja, ik heb het gezien, heel vervelend. Maar de tabel in de reader is leidend.\\

A: Over de baseline review. wat houdt het precies in? Ik zag dat we een agenda moeten voorbereiden, maar wat moet erin?\\
HD: Zoals we nu om de tafel hebben gezeten voor het project plan, zullen we tijdens de baseline review om de tafel gaan zitten voor jullie baseline report. Het is net iets formeler. Jullie moeten dan ook presenteren, niet iedereen, maar 3 tot 4 mensen (Uiteindelijk moet iedereen een keer gepresenteerd hebben). Jullie moeten ook zelf voor een zaal zorgen en jullie moeten er ook voor zorgen dat jullie goed voorbereid zijn. Hiermee bedoel ik dat de computer al klaar moet staan en dat jullie gelijk kunnen beginnen. Daar beoordelen we jullie op.\\
J: Is er een pagina limiet voor de baselinereview?\\
HD: Blijf ongeveer onder de 50 pagina's.

L: Over het logbook. Dat staat nergens als een deliverable, maar het is wel een officieel document, hoe zit dat?\\
HD: Jullie hoeven het niet in te leveren, maar ik wil het wel in kunnen zien voor het geval er iets fout is gegaan, omdat het design proces hier in is vast gelegd samen met alle belangrijke beslissingen etc..\\

\subsubsection{Questions}
\textit{Q1: B: Over de geometrie van ons voertuig. moeten we ook voor de payload ontwerpen?}\\
HD: In principe moet je ook je payload designen. Als je dit niet weet kan dat later een groot sneeuwbaleffect veroorzaken.\\

\textit{Q2: G: Ik vraag me af waar onze missie eindigt en met name over welke orde van grootte dit gaat. In andere woorden, moeten we ook ontwerpen voor het subsone regime?}\\
HD: Nee, dan is jullie missie al afgelopen. Jullie hoeven enkel te ontwerpen voor de hypersone vlucht.\\

\subsubsection{W.F.C.T.T.T}
L: Een vriendin van mij zit in een ander DSE groepje, en zij willen graag ook gebruik maken van MARS-Gram, kan dat?\\
HD: Dat is goed, maar onder bepaalde voorwaarden. Ik heb van NASA goedkeuring gekregen voor het gebruik binnen de DSE, maar het programma mag daarbuiten niet gebruikt worden.\\

\subsubsection{Survey}
\textit{DD: Hebben jullie de Belbin test nog gemaakt? Wat was het resultaat?}\\
T: We hebben de test gedaan en we hebben van bijna alle karakters wel iemand, behalve twee karakters geloof ik.\\
D: We kunnen het resultaat wel doorsturen.\\
HD: Pas op, dit kan een valkuil zijn. ga niet alleen focussen op de taken waar je goed in bent, maar doe alles wat nodig is.\\

\subsubsection{Adjournment}
De volgende vergadering zal plaatsvinden op 29-04-2015 om 16:00 in meeting room 2.\\

Joost sluit de vergadering om 11:24. iedereen is zeer opgelucht.\\
