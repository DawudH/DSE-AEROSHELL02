\subsubsection{Opening}
Locatie: LR Meeting room 2 (locatie gewijzigd)\\
Aanwezig: HD, NR, DD, J, L, G, A, B, T, D, Su, Se\\
Verontschuldigd: \\
Afwezig: CM \\
Tijd: Dawud opent de vergadering om 09:09 (aanvangstijd gewijzigd)\\

\subsubsection{Agenda}
De agenda van de vergadering bestaat oorspronkelijk uit de volgende punten:
\begin{enumerate}
\item Opening
\item Agenda
\item Goedkeuring van de notulen
\item Status update
\item Andere vragen
\item W.V.T.T.K.
\item Rondvraag
\item Sluiting
\end{enumerate}

Aan de agenda worden verder geen punten toegevoegd.

\subsubsection{Goedkeuring van de notulen}
Er waren geen op- of aanmerkingen.

\subsubsection{Status update}
Afgelopen twee dagen zijn we bezig geweest met de subsystems in de crew module en de rest van de missie.
\newline
Voor de launch gaan we eerst naar LEO, waar evt. gedocked wordt. Van LEO naar een transfer orbit. Verder is er gekeken naar launch vibrations, diameter en maximale volume voor in de launcher. Dat wordt inderdaad vijf meter diameter en 10 meter hoog.
\newline
Bij de interplanetary flight is naar verschillende transfer orbits gekeken welke verschillende duur hebben en verschillende deltaV-budget vereisen.
\newline
G vertelt dat voor de heenweg geen habitatcapsule nodig is. Je hebt wel een earth-return vehicle nodig voor de terugweg naar Aarde. Dit betekent minimaal twee launches. De eerste voor de heenweg moet met SLS (Saturn V niet toereikend) om de juiste snelle transfer orbit te kunnen pakken. De tweede kan met een kleinere, omdat de earth-return vehicle niet per se supersnel bij Mars moet aankomen. HD geeft als tip alles zo op te schrijven zodat het zodat het geen extra vragen oproept. Misschien een beetje buiten de scope, maar er moet wel over nagedacht zijn.
\newline
We kunnen bij problemen niet terug naar Aarde. De entry-vehicle wordt in een parking orbit gezet om een goede landingsmogelijkheid te vinden. Eerst een aerocapture en dan ongeveer 100kg aan brandstof om een deltaV te krijgen voor de parking orbit. We weten nog niet precies welke thrusters we hiervoor willen gebruiken.
\newline
Voor de landing (of terminal descent) loont het gebruik van een parachute niet. Het is lichter om thruster fuel mee te nemen (200kg tegen 280kg). De aeroshell is gewoon naar beneden gericht met een kleine parachute (drogue) voor de stabiliteit. De aeroshell wordt afgeworpen. Ook hier is het nog niet duidelijk welke thrusters gebruikt gaan worden. Hier moeten we wel op gaan letten.
\newline
Voor de terugkeer is aangenomen dat er een raket op Mars de astronauten naar het earth-return vehicle, die al om Mars cirkelt.
\newline\newline
Verder zijn we bezig geweest met de subsystems van de crewmodule. We hebben zoveel mogelijk massa en volume toegekend aan ADCS, Power, Structures, C\&DH (communications and data handling), operational items en Thermal control. Veel is uit referenties gehaald. NR vraagt of er subsystems zijn met een te hoge massa. Voroalsnog niet, vandaag zal de packaging van alle subsystems in de cre module plaats vinden.
\newline\newline
Vervolg van de control-opties van woensdag 10 juni:
\newline
Control met vari\"{e}rende offset benodigt hele grote actuators (zeker bij enkel $\alpha$-control). Alleen de pompen in Apollo waren al 120kg/stuk en die konden minder massa verplaatsen. Dit zijn dan alleen de pompen en we moeten onder de 1000 kg blijven. CG offset benodigd bij bank is kleiner, omdat bank neutrally stable is, dus een minder groot moment is nodig. We willen wel een initial offset voor $\alpha$-trim. We hebben nog niet gekeken naar cg-shift door andere massa-verdeling. 
\newline
Bodyflaps werken niet buiten de atmosfeer (netzoals cgoffset). Je hebt dus sowieso thrusters nodig. Voordeel is dat ze niets kosten als ze gelocked zijn. Wel lastig te maken aan het einde van inflatable.
\newline
Thrusters werken buiten atmosfeer en is proven concepts. Nadeel is massaverbruik over tijd. Meestal hydrazine thrusters wat niet fijn is als het misgaat. Deze zitten echter ook op Orion, dus het kan wel gewoon.
\newline\newline
Er is in principe nog niet gekozen voor een control system, maar we neigen naar een skewed shape, fixed c.g. offset met thrusters.
\newline\newline
Er wordt even gepraat over Limit Cycle Oscillation, welke voor ongewenste vibraties kunnen zorgen. Is voor ons vrijwel niet te analyseren

\subsubsection{Andere vragen}
Er waren geen vragen ingediend.

\subsubsection{W.V.T.T.K.}
NR geeft aan sowieso woensdag er niet bij te zijn en vrijdag misschien niet bij kan zijn.

D groep geeft aan vrijdag (19 juni) liever geen vergadering te plannen. HD kon dan sowieso niet, dus komt goed uit. Deze meeting zal verplaatst worden naar maandag 22-06-2015. Liefst zelfde tijd (11:00-12:30)

Voorkant van het final report mag je zelf bepalen, de rest moet in zwart-wit leesbaar zijn.

\subsubsection{Rondvraag}
- HD vertelt dat er woensdag 17 juni een C\&S barbecue is - 

\subsubsection{Sluiting}
De volgende vergadering zal plaatsvinden op 17-06-2015 om 11:00 in Fellowship Meeting room 2.
\newline\newline
Dawud sluit de vergadering om 10:17.