\subsubsection{Opening}
Locatie: LR Meeting room 2 (locatie gewijzigd)\\
Aanwezig: HD, NR, DD, J, L, G, A, B, T, D, Su, Se\\
Verontschuldigd: \\
Afwezig: CM \\
Tijd: Dawud opent de vergadering om 09:09 (aanvangstijd gewijzigd)\\

\subsubsection{Agenda}
De agenda van de vergadering bestaat oorspronkelijk uit de volgende punten:
\begin{enumerate}
\item Opening
\item Agenda
\item Goedkeuring van de notulen
\item Status update
\item Andere vragen
\item W.V.T.T.K.
\item Rondvraag
\item Sluiting
\end{enumerate}

Aan de agenda worden verder geen punten toegevoegd.

\subsubsection{Goedkeuring van de notulen}
Er waren geen op- of aanmerkingen.

\subsubsection{Status update}
Afgelopen twee dagen bezig met subsystems in de crew module en de rest van de missie.

A: Launch naar LEO, daar evt. docken. Van LEO naar transfer orbit. Launch vibrations, diameter en max. volume bekeken. HD komen er nog extra reqs bij? We hebben nu een maximale volume (dus maximale hoogte). Wordt 5dia bij 10hoog. Saturn V was niet toereikend.

D: Interplanetary flight, hoelang het duurt. dV-budget, orbits die gebruikt worden.

G: Heenweg geen habitatcapsule. Je hebt een earth-return vehicle nodig voor de terugweg naar Aarde. HD: hoeveel SLS heb je nodig? In principe 2, maar voor de andere heb je niet per se SLS nodig (je kan langere orbits). Wordt 2 lanceringen. Schrijf het zo op dat het geene xtra vragen oproept. Buiten de scope, maar wel over nadenken.

 

HD kunnen we evt. terug naar Aarde? Nee, we gaan voor een parking orbit om een goede landings mogelijkheid te oakken.  Eerst een aerocapture en dan ong. 100kg brandstof om de parking orbit te bereiken. HD vraagt welke thrusters we hiervoor gebruiken.
launch windows elke 2jaar. Hoe wil je density gaan meten? Dat doen we tijdens de aerocapture.

Landing: Terminal descent. Gebruik van parachute loont niet, beter om thruster fuel mee te nemen (200kg tegen 280kg). Aeroshell naar beneden, met kleine parachute voor stabiliteit. Afremmen met thruster, aeroshell afwerpen. Je hebt sowieso een thruster nodig. Vergeet motoren niet.

Raket vanaf Mars, dan een terugtransferhabitatmodule. 

D": Verder zijn we bezig geweest de subsystems van de crewmodule. 
ADCS
Power, solar arrays ipv nuclear. Power requirements per subsystem lastig te vinden. Uiteindelijk massa uit paper gehaald. Ong. 300 kg.
NR: massa die budget overschrijdt? Dat gaan we vandaag bekijken bij de packaging.
Structures: vergelijking Orion en Apollo. Heatshield erafgehaald. 1700kg ong.
Maximale AoA door te lengte van bus. Geen backshield nodig als dit gebeurt.
Communicatie: Deep space network. K-band voor veel data. Massaschatting was lastig, opschaling van Mars Odyssee. Wordt power meegenomen? L: dit gaat voor verschillende subsystemen lastig te bekijken zijn. 
Thermal, wordt 1080kg, 480 uit paper, 600 uit SMAD.

c.g. offset benodigt hele grote actuators (zeker bij enkel $\alpha$-control). Alleen de pompen waren 120kg/stuk en die konden minder massa verplaatsen. Dit zijn alleen pompen, wordt uiteindelijk veel te zwaar omdat het binnen die 1000kg moet zijn. c.g. offset bij bank kleiner, omdat bank neutrally stable is, dus minder groot moment benodigd is. 40cm cg, voor 2graden in 1 sec is flink. Hoe krijg een lift vector? Of asymmetrische vorm, of c.g. offset. Huidige aerodynamische skewed vorm zorgt voor L/D 0.3-0.5. Met de straps kun je asymmetrische vormen maken. We hebben nog niet gekeken naar cg-shift door andere massa-verdeling. 

Bodyflaps werken niet buiten de atmosfeer (netzoals cgoffset). Je hebt dus sowieso thrusters nodig. Als ze gelocked zijn, zijn ze gratis. Lastig te maken aan het einde van inflatable. 
Trillingen bij asymmetric al bekeken? Eventuele intabiliteit bij die vorm? Limit cycle oscillation bijna niks over te zeggen met simpele analyses. Symmetrisch minder last van. Dat kan evt. met een vaste cg offset, maar asymmetric geeft meer vrijheid. (LCO tegengaan met thrusters? Vorm van flutter zonder divergentie. Van te voren niets over te zeggen)

Thrusters werken buiten atmosfeer, al gebruikt. Nadeel is massaverbruik per tijd. Meestal hydrazine thrusters, zitten ook op Orion. Niet continu te gebruiken, dus lastig voor $\alpha$. 
HD: is er al gekozen? Neigt naar fixed c.g. offset met thrusters. Dus zonder skewed? Nee met skewed. Cg offset als het nodig is. Kritisch cg locatie bij launch?


\subsubsection{Andere vragen}
Er waren geen vragen ingediend.

\subsubsection{W.V.T.T.K.}
NR is sowieso woensdag er niet bij, vrijdag misschien niet.

We hebben liever vrijdag geen vergadering. HD kan dan niet. Verplaatsen naar 22-06-2015. Liefst zelfde tijd (11:00-12:30)

Voorkant mag je zelf bepalen, rest in zwart-wit leesbaar.

\subsubsection{Rondvraag}
HD woensdag 17 juni C\&S barbecue. 

\subsubsection{Sluiting}
De volgende vergadering zal plaatsvinden op 17-06-2015 om 11:00 in Fellowship Meeting room 2.
\newline\newline
Dawud sluit de vergadering om 10:17.