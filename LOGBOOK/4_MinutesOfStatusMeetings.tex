\section{Minutes of Status Meetings}
In this section all minutes of the status meetings will be archived. For these minutes, Abbreviation will be used for the speakers. the abrivations can be found in table \ref{tab:Abb} below.

\begin{table}[H]
	\caption {Abbreviations of names}
	\centering
    \begin{tabular}{|l|l|}
    \hline
    Abbreviation for person & Full name      \\ \hline
    HD                      & Herman Damveld \\ \hline
    NR                      & Niels Reurings \\ \hline
    DD                      & Dennis Dolkens \\ \hline
    J                       & Joost          \\ \hline
    L                       & Lucas          \\ \hline
    A                       & Alexander      \\ \hline
    G                       & Guido          \\ \hline
    D                       & Dawud          \\ \hline
    Se                      & Sebastiaan     \\ \hline
    B                       & Bj\"{o}rn      \\ \hline
    Su                      & Suthes         \\ \hline
    T                       & Twan           \\ \hline
    \end{tabular}
    \label{tab:Abb}
\end{table}

%\subsection{Kick-off, 20-04-2015 9:00}


\subsection{Status Meeting 1, 22-04-2015 14:00}
Location: Fellowship Meeting room 1\\
Time: Joost opened the meeting at 14:00\\
SECRET CODE1: yes, there is a secret code

\subsubsection{Agenda}
J: Add a set of questions between points 5 and 6 in the agenda.\\
J: Additional question: Should the communication be in English or in Dutch? 
HD: As long as there is no external or formal communication, Dutch communication may be used.

\subsubsection{Organizational task devision}
D: task division is as is shown in table \ref{tab:tasks}, where the left part of the table describes the distribution of the general work division, while the right columns describe the technical task division.

\begin{table}[H]
	\caption {Task distribution}
    \begin{tabular}{|p{0.25\textwidth}|p{0.2\textwidth}|p{0.25\textwidth}|p{0.2\textwidth}|}
    \hline
    General task division                 & Names      & Technical task devision & Names                    \\ \hline \hline
    Chairman (CH)                         & Joost      & Structures              & Bj\"{o}rn; Alexander         \\ \hline
    Secretary (S)                         & Lucas      & Aerodynamics            & Guido; Joost; Sebastiaan \\ \hline
    Planning (Pl)                         & Alexander  & Thermodynamics                 & Suthes; Lucas            \\ \hline
    Systems Engineer (SE)                 & Guido      & Control                 & Dawus; Twan              \\ \hline
    Documentation and Archiving ( D \& A) & Dawud      & ~                       & ~                        \\ \hline
    Risk manager (YOLO)                   & Sebastiaan & ~                       & ~                        \\ \hline
    Editor (Ed)                           & Bj\"{o}rn      & ~                       & ~                        \\ \hline
    Verification (Ver)                    & Suthes     & ~                       & ~                        \\ \hline
    Validation (Val)                      & Twan       & ~                       & ~                        \\ \hline
    \end{tabular}
    \label{tab:tasks}
\end{table}

HD: You should make sure that you are able to switch between departments, because you cannot yet know how much work a department is going to have. It can be easy if you have people in multiple technical departments, such that there is not time loss in getting acquainted with a new topic if switching occurs.\\
G: As a SE I will be responsible for the overview and make sure that this will not be a problem.\\
HD: And who is the person that keeps a clear overview of the latest parameter numbers? Make sure that there is a clear description of the tasks and that no work is left undone. You can for instance keep the parameter values in an excel sheet.\\
L: We could also keep the latest variables as an input file in MATLAB, that runs our tools. \\
NR: Then keep in mind that you need time logs, such that there is no confusion about changes in the input by a person.\\

\subsubsection{upcomming deadlines}
J: We moved the deadline for the baseline review to may 1st, because of the free days after the weekend must be free and not used for checking. \\
HD: OK.\\
L: Since our project plan is almost finished, can we send the report today, so that you can check it tomorrow? In that way we can discuss it on Friday and we will still have time to correct the last things.\\
HD: I don't have much time. I can see what I can do.
NR \& DD: We can have a look. If you can get an advantage this way, I would seartenly make use of it if I were you.\\

\subsubsection{Questions}
Q1: \textit{L: The requirements state that the tutor and coaches should always be able to look at the latest version of the logbook, how shall we realise this?}\\
HD: We can use any cloud storage, but I would prefer If you have a look at 'surfspot'.\\
L: What should we put on the storage cloud?\\
HD: General deliverables, like the logbook, agendas and deliverables. No large datasets. Also note that you need to print certain things. \\

Q2: \textit{Se: How detailed should the risk mitigation plan be?}\\
HD: For the baseline review the risk should be evaluated very generally. We want to see that the group understands the requirements and what the driving requirements will be. cost for example is not a requirement and hence it shouldn't be elaborated extensively, otherwise you did something wrong. Also, You don't have to write a plan for the complete mission, you must focus on the re-entry. You don't have to design the capsule, you can find data on this from similar missions. Examples are the Orion and Apollo mission from NASA.\\

Q3: \textit{L:What is the starting point of the mission, will we start at launch or in the atmosphere, or somewhere in between?}\\
HD: In general, the begin point of your mission is a speed of 7 km/s at an altitude I will provide you with on a late time.\\
D: What about the orientation, is that free to choose?\\
HD: You will see that there are only few options for a human flight to land on mars, so this will automatically provide you with a requirement.\\

Q4: \textit{Su: How detailed should the sustainability part be, since it will determine 10\% of our symposium grade?}\\
HD: This is not a very important part in the design, because we cannot bring extra mass just for the use of sustainable materials. This will induce a large snowball effect, such that more propellant is needed. Burning more propellant is not at all sustainable. \\
NR: You can focus on things like space debris, materials used and how they are produced.\\
DD: Or on life forms that you will bring to space that might cause damage, etc.\\
HD: So in general, don't underestimate it. Treat it with respect, but don't make it a large part of your design.\\

Q5:\textit{L: We had some problems with the code for the requirements, but it is solved now}\\
HD: Now I am curious.\\
L:described the code: Mission-Concept-number-subsystem-number.\\
NR: It's nice that you did it, but a bit a waste of your energy.\\
HD: By the way, you don't need two concepts in the end. That's something we did two years ago with the same project, but I wanted to remove it from the project this time. I must have forgotten it. So after the mid-term review, you have to consider only one design.\\

\subsubsection{W.F.C.T.T.T}
No comments
SECRET CODE2: It's raining meatballs

\subsubsection{Survey}
A:-\\ 
B:-\\
Su:-\\
D:-\\
\textit{NR: For the editor, it might be very stressful to do proofreading, because a lot of people might submit their parts last minute, don't you want more than one person in this department?} \\
B: The task division only states final responsibility, but overall more than one person will work on this.\\
T:-\\
Se:-\\
\textit{J: I am a literature study right now and I see that a lot of questions in the literature are not solved yet, how should we proceed with this?}\\
HD: You should only focus on your own design, but I want to make the problem a bit broader than it is. I want you to think of other concepts than only aero shells. You need to incorporate this in your design option tree.\\
T: So should our mission statement change? Because we have stated that we will only look at aero shells.
HD: No, I just want you to apply broader knowledge during the concept development, such that you will get a better understanding of the mission.\\
\textit{G: Should we also design the landing, or where will our mission end?}\\
HD: No, that is outside the scope of this exercise. I will later send you a final altitude with a corresponding speed that will indicate the end of your mission.\\

HD: Yesterday I had contact with a very smart student who did this same project two years ago. He indicated that he was willing to help you with questions about the project. I will later send you his name and contact information. If I were you, I would not refuse this offer.\\

\subsubsection{Adjournment}
The next meeting will be Friday 24-04-2014 at 10:00 in Meeting room 1.\\

Adjournment at 15:00

\subsection{Status Meeting 2, 24-04-2015 14:00}
\subsubsection{Call to order}
Locatie: Fellowship Meeting room 1\\
tijd: Joost opened the meeting at 14:00\\

\subsubsection{Agenda}
J: Ik wil tussen puntje twee en drie nog een kopje voor het goedkeuren van de notulen toevoegen.\\

\subsubsection{Acceptance of minutes}
De notulen zijn door iedereen goed gelezen, bravo.\\

N: Mijn naam is verkeerd gespeld in de notulen, het is namelijk Reurings en niet Reuring. Dit staat volgens mij ook fout in de reader. Maar verder is het wel goed.\\

L: Ik zal het aanpassen.\\

\subsubsection{Review of Project Plan}
HD: Jullie hebben het werk mooi op tijd ingeleverd. Dat is goed, want met een goed project plan valt of staat een project.\\

J: We hebben het rapport van te voren ingeleverd. Hebben jullie nog opmerkingen over het verslag die wij kunnen doorvoeren?\\

HD: Ik heb naar het verslag gekeken en ik heb nog een paar puntjes.\\
\textit{1: HD: De preface, nou nou.. dit ziet er een beetje uit als slijmen naar de begeleider. Als klant zit ik hier niet echt op te wachten. houd het vooral zakelijk.}\\

\textit{2: HD: In de introduction bij de purpose of the project plan geven jullie aan een general approach te hebben. maar dit project is juist heel specifiek, daar stoorde ik mij aan.}\\
Su: Het gaat eigenlijk meer over onze aanpak van dit project.\\
HD: Zou je bij het ontwerpen van een reactor het zelfde plan van aanpak hebben? Ik hoop het niet!\\
T: Wat Suthes bedoeld is dat het gaat over het al omvattende plan binnen dit specifieke onderwerp.\\

ZOEKWOORD: STRAWMAN CONCEPT\\

\textit{3: HD: De objective statement geeft wel heel specifiek aan dat er gebruik gemaakt moet worden van bestaande launchers, dit trekt de aandacht weg van het eigenlijke doel.}\\
L: Dit is om duidelijk aan onze klant aan te geven wat wel en niet binnen onze taakomschrijving valt.\\
HD: En andere requirements niet? dan zou je ze er allemaal in moeten steken.\\
NR: Zorg dat je statement SMART is (Specific Measurable Attainable Realistic and Time dependent).\\

\textit{4: NR: Er missen in die sectie ook referenties. zo staat er bijvoorbeeld een hele mooie mission tabel, maar er staat nergens een referentie waar de data vandaan komt.}\\
T: Die staat geloof ik wel ergens in de tekst.
HD: Wees secuur! het is hier niet helemaal duidelijk. \\
NR: Wees er zelf erg bewust van wat je opschrijft, niet verwijzen wordt gezien als plagiaat en kan grote gevolgen hebben.\\
HD: dus elke keer als er een statement wordt gemaakt in je tekst zonder dat deze is uitgelegd, moet deze statement van jezelf komen of ergens anders vandaan komen. In het laatste geval is een echt referentie nodig! Wees er heel voorzichtig mee.\\

\textit{5: DD: jullie mission need statement is vreemd geformuleerd. Nu lijkt het alsof jullie zelf gaan demonstreren dat het werkt, hoewel jullie alleen maar het design verzorgen.}\\
All: We zullen het aanpassen.\\

\textit{6: DD: In section 2.1 staat ook ergens aircraft i.p.v. spcecraft}

\textit{7: HD: In section 2.2 begint jullie requirement code altijd met CIA en SYS, deze code is daardoor onnodig lang.}\\
L: Dat is zo geformuleerd omdat we eerst wouden werken met meerdere concepten, maar dat is nu niet meer nodig. Tevens staat CIA altijd aangegeven zodat de klant direct kan zien over welk project het gaat, van de vele projecten waarin hij of zij geïnvesteerd heeft. Ik zal het aanpassen.\\

\textit{8: HD: Requirement 5 is ook niet echt duidelijk}\\
A: Deze requirement is overgenomen uit de reader en is sowieso vaag.\\
HD: Wat was er niet duidelijk dan?\\
A: Ik snap niet waarom deze requirement zo beschreven is. de massa van de heat shield mag maar 10\% van het totale gewicht, waarbij het totale gewicht 10,000 kg is. Maar deze gewichten staan helemaal niet vast. Moeten de ene mee schuiven met de ander als de gewichten veranderen?\\
HD: Ja, dat klopt en die 10\% is een maximum gewicht. Ik hoor het graag als er geschoven moet worden met de requirement.\\
NR: Het uiteindelijke gewicht zal zijn hangt af van hoe het zich in de praktijk zal uitwijzen.\\
Hd: Het grote lijn idee is in ieder geval dat je voertuig aan het begin van de re-entry een gewicht van 10,000 kg heeft. Deze massa moet worden afgeremd.\\

\textit{9: HD: bij de OBS zie ik nergens een astrodynamics department?}\\
T: We hebben hierover nagedacht, dit valt onder de orbital- en control department.\\

\textit{10: NR: In de tekst staat niet goed beschreven wat Verification en Validation is. Weten jullie hoe en wat het is?}\\
T: Verification heeft meer met je eigen requirements te maken, hoewel validation meer over requirements van de customer gaat.\\
L: Zoals ik het zie gaat verification erover of je het product correct maakt en validation erover of het product wel de waarheid simuleert.\\
HD: Dit moeten jullie nog wel even aanpassen. Verification gaat over de vraag of je het goed doet en validation of je het goede doet.\\

\textit{11: HD: Ik zag ook dat er nog steeds maar een planner is en maar een persoon op systems engineering. Hier hebben we het vorige keer nog over gehad! Waarom is er niets veranderd?}
G: In principe werken we er met meer mensen aan, maar er is een eind verantwoordelijke.\\
DD: Op deze manier valt er wel heel veel verantwoordelijkheid op de SE. Zorg dat er goed ondersteuning is, die heb je nodig.\\
T: V\& V team geeft ook ondersteuning aan de SE. SE let erop of het gebeurt, V\& V of het wel goed gebeurt.\\
HD: En wat doet de planner als jullie je deadlines niet halen.\\
A: Het idee is dat we niet gaan achterlopen, daar is de planning voor.\\
NR: Hier moet je mee uitkijken, dit gaat sowieso wel gebeuren en als het gebeurt schuift het werk op en dat gaat altijd ten koste van het schrijfwerk. Dat zal de editor niet zo leuk vinden.\\
T: In principe moet de planner dan op tijd aan de bel trekken.\\
D: Ja, maar we moeten ook beschrijven hoe en wanneer.\\
A: Zo vroeg mogelijk.\\
NR: Dus jij houdt overzicht over de planning en stuurt dagelijks bij?\\
A:Ja, aan het einde van de dag, na onze group meeting.\\

\textit{12: HD: Over de WBS: Jullie hebben ook tijd ingeplant voor het ontwikkelen van tools, dat is heel goed! dat wordt vaak vergeten.}\\
NR: Ik vond het juist wel een beetje de andere kant opslaan. Er was wel veel tijd vrijgehouden voor tools, maar weinig tijd voor het daadwerkelijke design. De twee kunnen niet zonder elkaar.\\
DD: Gaan jullie alles zelf ontwikkelen?\\
D: Een combinatie van.\\
DD: Het lijkt nu inderdaad veel werk om de tools te ontwikkelen, waar je kunt uitbesteden zou ik uitbesteden.\\

\textit{13: HD: Ik merk dat jullie vaak wel weten wat jullie willen gaan doen, maar dat het niet duidelijk staat opgeschreven. Als dit in een keer duidelijk wordt gedaan scheelt het een hoop discussie tijd! (goede tip)}\\

\textit{14 NR: Ik zie ook niet zo goed waar de iteration in jullie design proces staan weergegeven.}\\
G: In het detailed design.\\

\textit{15: HD: Jullie workpackages in de WBS komen niet overeen met de workpackages in de Gantt-chart.}\\
B: Dat komt omdat de WBS vrij globaal is en de Gantt-chart niet. We volgen niet de workpackages omdat we zo beter kunnen schuiven in de gantt-chart als dat nodig blijkt.\\
HD: Het is gebruikelijk om wel de zelfde workpackages te hebben. Het moet naadloos in elkaar over kunnen lopen.\\

\textit{16: HD: Verder is er volgens mij inhoudelijk niets mis mee.}\\

\textit{17: NR: Bij sustainability moet je niet zeggen dat het niet belangrijk is als je er net een hele pagina aan hebt gewijd. Dan ben je jezelf aan het onderuit halen.}\\
DD: Er er miste ook COOSPAR.\\
Se: Zal het aanpassen.\\

\textit{18: NR: In je referentie lijst moet je wel schrijven of iets een thesis /journal of een article is, zodat ik weet ik moet benaderen als ik verdere informatie wil hebben.}\\
HD: Gebruiken jullie bibtech? weten jullie hoe dat werkt? Je kunt ook "mandolate" gebruiken voor het managen van je referenties, dan krijg je geen dubbele referenties in je lijst, etc. Dat werkt het beste uit mijn ervaring.\\
DD: Je kunt ook de bibtech code direct van google scholar halen.\\
D: Ik zal kijken of ik dat kan oplossen.\\

\textit{19: HD: Dit was veel kritiek, maar jullie hoeven niet te denken dat jullie het slecht hebben gedaan, inhoudelijk was er niets mis met het werk.}\\

J: Bedankt voor de feedback.\\


\subsubsection{Deliverables}
J:Er is een mismatch in de deliverable tabel. Op blackboard zijn er updates en het is niet duidelijk wat we nou moeten inleveren.\\
HD: Ja, ik heb het gezien, heel vervelend. Maar de tabel in de reader is leidend.\\

A: Over de baseline review. wat houdt het precies in? Ik zag dat we een agenda moeten voorbereiden, maar wat moet erin?\\
HD: Zoals we nu om de tafel hebben gezeten voor het project plan, zullen we tijdens de baseline review om de tafel gaan zitten voor jullie baseline report. Het is net iets formeler. Jullie moeten dan ook presenteren, niet iedereen, maar 3 tot 4 mensen (Uiteindelijk moet iedereen een keer gepresenteerd hebben). Jullie moeten ook zelf voor een zaal zorgen en jullie moeten er ook voor zorgen dat jullie goed voorbereid zijn. Hiermee bedoel ik dat de computer al klaar moet staan en dat jullie gelijk kunnen beginnen. Daar beoordelen we jullie op.\\
J: Is er een pagina limiet voor de baselinereview?\\
HD: Blijf ongeveer onder de 50 pagina's.

L: Over het logbook. Dat staat nergens als een deliverable, maar het is wel een officieel document, hoe zit dat?\\
HD: Jullie hoeven het niet in te leveren, maar ik wil het wel in kunnen zien voor het geval er iets fout is gegaan, omdat het design proces hier in is vast gelegd samen met alle belangrijke beslissingen etc..\\

\subsubsection{Questions}
\textit{Q1: B: Over de geometrie van ons voertuig. moeten we ook voor de payload ontwerpen?}\\
HD: In principe moet je ook je payload designen. Als je dit niet weet kan dat later een groot sneeuwbaleffect veroorzaken.\\

\textit{Q2: G: Ik vraag me af waar onze missie eindigt en met name over welke orde van grootte dit gaat. In andere woorden, moeten we ook ontwerpen voor het subsone regime?}\\
HD: Nee, dan is jullie missie al afgelopen. Jullie hoeven enkel te ontwerpen voor de hypersone vlucht.\\

\subsubsection{W.F.C.T.T.T}
L: Een vriendin van mij zit in een ander DSE groepje, en zij willen graag ook gebruik maken van MARS-Gram, kan dat?\\
HD: Dat is goed, maar onder bepaalde voorwaarden. Ik heb van NASA goedkeuring gekregen voor het gebruik binnen de DSE, maar het programma mag daarbuiten niet gebruikt worden.\\

\subsubsection{Survey}
\textit{DD: Hebben jullie de Belbin test nog gemaakt? Wat was het resultaat?}\\
T: We hebben de test gedaan en we hebben van bijna alle karakters wel iemand, behalve twee karakters geloof ik.\\
D: We kunnen het resultaat wel doorsturen.\\
HD: Pas op, dit kan een valkuil zijn. ga niet alleen focussen op de taken waar je goed in bent, maar doe alles wat nodig is.\\

\subsubsection{Adjournment}
De volgende vergadering zal plaatsvinden op 29-04-2015 om 16:00 in meeting room 2.\\

Joost sluit de vergadering om 11:24. iedereen is zeer opgelucht.\\


\subsection{Status Meeting 3, 29-04-2015 16:00}
\subsubsection{Call to order}
Locatie: Fellowship Meeting room 2\\
tijd: Joost opened the meeting at 16:01\\


\subsubsection{Agenda}
J: Ik wil tussen puntje twee en drie nog een kopje voor het goedkeuren van de notulen toevoegen. Volgende keer staat die er standaard in.\\

\subsubsection{Acceptance of minutes}
G: Bij het antwoord op mijn vraag over het snelheidsregime waarin we moeten ontwerpen miste dat we ook voor supersoon moeten ontwerpen.\\
L:Ik zal het aanpassen.\\

\subsubsection{Action points}
J: Een action van de vorige vergadering was het afmaken van het verbeterde project plan. Is dit naar tevredenheid gebeurt?\\
HD: Ik wist niet dat jullie hier verder commentaar op wouden hebben.\\
J: Ik vraag het meer omdat het vorige keer een action Item was.\\
HD: Ik heb het diagonaal doorgelezen en ik zag dat de veranderingen waren toegepast, verder heb ik geen commentaar.\\
DD: Ik heb mijn deel nog eens doorgelezen en alles was gewoon doorgevoerd, dus het is wel goed. Maar ik zag nog wel een typefoutje in de mission statement. en er was ook een foutje in de lay-out waarbij tekst door een tabel heen liep.\\

\subsubsection{Requirments}
\textit{Q1: T: met betrekking tot de g-loads, mag je 3 aardse g's in elke richting trekken of alleen in een richting?}\\
HD: Het gaat om de lengte van de vector, dus in een richting. Als de astronauten voor 10 dagen lang meer dan 3 g moeten dragen vinden ze dat niet fijn.\\
Se: We kunnen wel een budget maken over de tijd van de gehele missie, zodat we rekening kunnen houden met eventuele groei.\\
HD: Nee, dat is niet de bedoeling.

\textit{Q2: T: En hoe zit het met peak g loads?}\\
G: Moeten we ons altijd aan de 3 g requirement of we voor korte tijd ook meer g's dragen?
HD: In principe moet je gewoon ontwerpen voor de 3 g requirement, maar ik ben als klant wel geïnteresseerd in andere opties. Als jullie met een bepaalde trajectory komen die over het algemeen veel beter is maar wel heel even wat hogere g's moet kunnen hebben dan heb ik daar wel oor voor. Maar, het moet geen onderhandel punt worden, nogmaals, het is in principe de bedoeling dat je voor 3 g requirement ontwerpt.


\subsubsection{Survey}
\textit{Q1: Se: Ik heb een familie situatie en het zou kunnen zijn dat ik binnenkort wat uren ga missen. Wat is uw beleid hierin? Ik wil in principe wel gewoon mijn uren goed maken.}\\
HD: Shit happens. Kijk, daar kunnen we gewoon mee omgaan. Weg blijven moet alleen geen excuus worden om je werk niet te doen. Ik zeg niet dat het zo is, maar houdt er rekening mee. Ik had twee jaar geleden een DSE groepje waarbij iemand mazelen had en thuis moest blijven en dat hebben we prima kunnen regelen via Skype, dus er is altijd wel een oplossing te vinden. Je moet wel altijd laten weten aan Joris Melkert dat je weg bent, want hij is hier nogal streng in. Hij hoort dit soort dingen ook graag van te voren. Daarnaast moeten team members niet gewoon het werk van de persoon opvangen die weg gaat, maar het aankaarten als dat nodig is.\\

\textit{Q2: J: Ik had nog een mailtje gestuurd over MARS-Gram, maar misschien kunnen we dat zo oplossen.}\\
HD: Daar kunnen we na de meeting nog wel samen naar kijken.\\

\textit{Q3: L: Over het thermodynamica gedeelte van de literatuur studie, hoe diep moeten we dat nu uitzoeken? want het wordt of vrij makkelijk of juist veel te moeilijk. Ik weet echt nog niet hoe ik bijvoorbeeld een berekening kan maken van materiaal wat zichzelf aan het opofferen is, omdat de boundary conditions gewoon veranderen.}\\
HD: Zo veel detail is nu nog echt niet nodig. Je zou nu heel oppervlakkig moeten weten wat voor componenten er zijn en hoe het design beperkt wordt. Later komen de heftige berekeningen nog wel.\\
HD: Het is ook wel het geval met deze DSE opdracht dat het niet zo een twee drie het boekje volgen is. Je kunt niet even de bibliotheek inlopen en het juiste design punt opzoeken in bijvoorbeeld een Raymer boek. Je moet zelf echt een beetje gaan nadenken over het ontwerp.\\
L: Dus het is een beetje de jazz onder de DSE opdrachten?\\
HD: Hoe bedoel je? Daar houd ook niemand van?\\
L: Nee, in de zin dat je zelf moet improviseren en niet zomaar het boekje kunt volgen.\\
NR: Ik ben zelf bezig geweest met high temperature composites en ik kan je wel een tip geven. Niet alleen het materiaal kan een aanduiding geven van je thermal protection, maar ook de plaats kan een indicatie geven. denk aan de tegels op de space shuttle.\\

Zoekwoord: Requirements\\

\subsubsection{W.F.C.T.T.T}
HD: Hoe staat het met de planning? Hebben jullie een beetje het idee dat jullie op stoom zijn gekomen?\\
Allen: Ja hoor.\\
L: Ja, zeker na de literatuur studie, dan krijgt de opdracht steeds meer een bepaalde vorm.\\
T: We zijn gewoon lekker bezig en soms zie je mensen ook een eureka moment krijgen. 'OO, zo zit dat in elkaar' hoor je ze dan denken.\\

NR: Tijdens jullie dagelijkse meeting, bespreken jullie dan ook wat er inhoudelijk is gebeurt?\\
T: Ja, dat bespreken we.\\
NR: Houden jullie ook rekening dat jullie moeten gaan presenteren en ook nog een presentatie moeten maken?\\
T: Ja, dat staat ook in onze planning.

\subsubsection{Survey}
Er zijn geen verdere vragen

\subsubsection{Adjournment}
De volgende vergadering zal plaatsvinden op 01-05-2015 om 10:30 in meeting room 2.\\

Joost sluit de vergadering om 16:24.\\


\subsection{Status Meeting 4, 01-05-2015 10:30}
\subsubsection{Call to order}
Locatie: Fellowship Meeting room 2\\
Aanwezig: HD, NR, DD, J, L, G, A, B, T, D, Su, Se\\
Verontschuldigd: - \\
Afwezig: - \\
tijd: Joost opent de vergadering om 10:31\\

\subsubsection{Agenda}
De dagorde wordt goedgekeurd\\

\subsubsection{Acceptance of minutes}
HD geeft aan dat de notulen zakelijker moet worden geschreven, na het zien van een ontwikkeling over de voorgaande notulen. L gaf aan dat dit is gebeurt om de notulen interessanter te maken, maar zal het aanpassen.\\

\subsubsection{Questions}
\textit{Q1: J: Er is een requirement over de reliability van het control systeem, maar wat houdt het control systeem dan in?}\\
Er volgt een interessante discussie over de reliability van een subsysteem. Er wordt geconcludeerd dat de reliability van een subsysteem wordt gebruikt om de functionaliteit vast te leggen en dat zij volgt uit de reliability van het totale systeem. Specifiek voor het control subsysteem geldt dat de reliabiltity afhangt van de huidige technologie evenals een iteratief process tussen external loads, thermal loads en het control subsysteem.\\

\textit{Q2: Se: Welke elementen moeten er in een risk map, aangezien het een levend document is en we nog in een vroeg stadium zitten}\\
HD geeft aan dat er nog niet veel in moet staan omdat het project nog in een vroeg stadium verkeerd, waardoor nog niet veel bekend is over subsystemen. NR voegt hieraan toe dat het nu vooral een handige tool is om key-aspacts te vinden die later in het proces veel werk uren zullen vergen.\\

\subsubsection{W.F.C.T.T.T}
HD vraagt of iemand zijn ringmap heeft gevonden. Dit was niet het geval.\\

J geeft aan dat MARS-Gram nu wel werkt.\\ 
\subsubsection{Survey}
\textit{Q1: DD: Hoe gaat het met het baseline report?}\\
Er wordt aangegeven door de groep dat het goed gaat met de voortgang van het baseline report. Vandaag zal het afgerond worden. Er zal er nog gewerkt worden aan de presentatie, een stukje over de trajectory tool en aan editing. Hierop  geeft HD aan dat de presentatie goed voorbereid moet zijn (zorgen dat alle spullen klaar staan) en dat mensen die niet van presenteren houden toch even moeten doorzetten.\\

\subsubsection{Adjournment}
De volgende vergadering zal plaatsvinden op 08-05-2015 om 10:00 in meeting room 2.\\

Joost sluit de vergadering om 10:57.\\



\subsection{Status Meeting 5, 08-05-2015 10:00}
\subsubsection{Opening}
Locatie: Fellowship Meeting room 2\\
Aanwezig: HD, NR, DD, J, L, G, A, B, T, D, Su, Se\\
Verontschuldigd: - \\
Afwezig: - \\
tijd: Joost opent de vergadering om 10:03\\

\subsubsection{Agenda}
De agenda van de vergadering bestaat oorspronkelijk uit de volgende punten:
\begin{enumerate}
\item Opening
\item Agenda
\item Goedkeuring van de notulen
\item Control System DOT
\item Trade-off criteria
\item Andere vragen
\item W.V.T.T.K.
\item Rondvraag
\item Sluiting
\end{enumerate}

Hieraan worden toegevoegd:
\begin{itemize}
\item Tussen puntje 3 en 4: Baseline Report
\item Tussen puntje 5 en 6: Logbook 
\end{itemize}

\subsubsection{Goedkeuring van de notulen}
Wegens tijdgebrek heeft niet iedereen de notulen gelezen. L geeft aan dat de voorheen besproken veranderingen wel zijn doorgevoerd.\\

\subsubsection{Baseline Report}
Na het inleveren van het baseline report zijn er wat op en aanmerkingen van HD, NR en DD. Deze zijn uitvoerig besproken. Over het algemeen zijn er een aantal grote lijnen in deze opmerkingen te vinden. Zo was het grootste commentaar van toepassing op de literatuurstudie. Deze bevatte voornamelijk referenties naar bronnen, maar er werd opmerkelijk weinig uitgelegd over deze bronnen. Hierdoor is het doel van een literatuurstudie niet bereikt; namelijk het creëren van een naslagwerk, waardoor het erop naslaan van eerder bestudeerde literatuur niet meer nodig is. \\
Een ander groot kritiek punt had betrekking op de requirements. De top level requirements zijn niet SMART (specific measurable attainable realistic and timeley) bevonden en lijken open deuren in te trappen. G geeft aan dat dit gedaan is om requirements terug te linken naar hogere levels van requirements. Er wordt echter aangegeven dat hierdoor het overzicht weg is en dat er wordt gesuggereerd dat de missie bestaat uit remmen en 'de rest', hoewel er wordt verwacht dat de requirements overzichtelijk over het gehele systeem worden verdeeld.  Tevens zijn alleen de opgelegde requirements besproken in het verslag, hoewel andere requirements (e.g. sustainability requirements en requirements die gebonden zijn aan de wet) niet zijn besproken.\\
Daarnaast werd er in het verslag niet goed uitgelegd waarom het een goed idee is om een inflatable te kiezen in plaats van een solide ontwerp. B verklaart dit door aan te geven dat er nog geen officiële keuze is gemaakt voor een inflatible ontwerp en dat daarom onze visie nog heel breed is en niet per se gefocust op een solide ontwerp als definitieve oplossing voor het ontwerp vraagstuk.\\
Een ander puntje was dat het mass-budget nog niet heel accuraat was. Het vervoeren van 6 mensen naar Mars lijkt vooralsnog onhaalbaar volgens HD. Er wordt aanbevolen om deze schatting niet in berekeningen te gebruiken.\\

\subsubsection{Control System DOT}
Er wordt voorgelegd dat de groep denkt dat het beter is om het trade-off process onafhankelijk te maken van de missie tijdsduur en het control systeem. HD NR en DD vinden dit een goed idee.

\subsubsection{Trade-Off Criteria}
De groep heeft trade-off criteria opgesteld voor het analyseren van de 5 eerder gekozen concepten. Deze willen zij bespreken. De bedachte trade-off criteria zijn:

\begin{itemize}
\item Mass
\item Developement risk
\item Reliability
\item Orbit controllability
\item Deceleration capability
\end{itemize}

Er volgt een discussie over de criteria en met name over de verhouding tussen requirements en trade-off criteria. Hieruit blijkt dat requirements niet je trade-off moeten drijven, maar dat trade-off criteria er moeten zijn om te kijken welk concept, dat aan alle requirements zou moeten kunnen voldoen, beter is. Er is soms echt overlap.\\

Verder wordt er besproken dat reliability geen goede trade-off criteria is omdat reliability vaak af te kopen is met extra massa. Tevens is de massa zelf geen goede trade-off criteria omdat de massa een gegeven requirement is. Teveel massa meenemen kan niet, te weinig massa mee nemen zou zonde zijn. Er wordt geconcludeerd dat payload mass wel een goed criterium is, omdat er dan meer payload mee kan voor de zelfde missie.\\

NR vraagt nog wat het verschil is tussen orbit controlability en control. Hierop antwoord T dat orbit control vooral over de bekwaamheid van een configuratie gaat om zijn orbit te behouden, hoewel control anderzijds meer focust op het gebruik van hulpmiddelen zoals control surfaces, cg. change etc..\\

Er volgt een gesprek over de nauwkeurigheid van de baan van het voertuig met betrekking op het einde van de missie. HD geeft hierop het einde van de missie in cijfers aan, zoals lang gehoopt. De missie zal op 10 km MOLA eindigen met een snelheid van Mach 5 waarbij de positie van het voertuig in een straal van 500 meter van een doel moet kunnen zijn. Er wordt een hint gegeven dat er wel moet worden gekeken naar de trajectory van het voertuig. Daarnaast is niet bij iedereen bekend wat MOLA is, dit moet worden opgezocht als actiepuntje.\\

Vervolgens geeft T aan dat het misschien wel nodig is om een versoepeling in de requirements te krijgen . HD is benieuwd wat dit kost in termen van andere requirments en geeft aan dat een concreet voorstel nodig is om hier verder op in te kunnen gaan.\\

\subsubsection{Logbook}
NR merkt op dat hij alleen notulen in het logbook ziet, maar dat de overige puntjes zoals action items , planning en dicisions ontbreken. Hij vraagt of dingen zoals action items wel besproken worden. L geeft aan dat dit het geval is en zal deze items voortaan niet alleen opnemen in zijn schrift maar ook in het logbook verwerken. Wel geeft L aan dat dit voor hem wat tijd kost en dat het van de tijd afgaat die anders aan technische taken besteed kon worden.

\subsubsection{Andere vragen}
L vraagt of we per se getallen moeten hangen aan de criteria in de trade-off tabel, omdat dit in de Systems Engineering slides niet zo is aangegeven. DD legt uit dat dit niet altijd hoeft, je kunt bijvoorbeeld ook plusjes gebruiken, zolang het maar duidelijk is welk concept beter is en waarom. Hier gaat HD nog even op in omdat in vorige projecten de trade-off niet door de groep alleen werd gedaan, maar samen met de tutors. Er wordt besloten om over deze keuze nog even na te denken en hier later op terug te komen.\\

G vraagt hoe het zit met de personal appendix deliverable, omdat deze op blackboard staat maar nergens in het opdrachten blad. HD zegt dat dit een management samenvatting mag zijn van het logbook en dat we hier niet zo veel tijd in hoeven te steken, maar dat het wel aanwezig moet zijn.\\

B vraagt tot hoe ver het structural design uitgewerkt moet worden voor de mid term review omdat het erg ver kan gaan en er te weinig tijd is om bijvoorbeeld buckling te berekenen van elk concept. HD en NR zeggen dat de berekeningen nog niet heel diep hoeven te gaan maar dat een orde van grootte wel bekend moet zijn.\\

\subsubsection{W.V.T.T.K.}
Er was een vraag van de OSSA's om de deliverables op blackboard te zetten. NR vraagt of dit gelukt is. A zegt dat hij dit gedaan heeft, maar dat er bij de OSSA's technische problemen zijn en dat de documenten niet meer op de website staan. Hier wordt nog aan gewerkt.

\subsubsection{Rondvraag}
Er zijn geen verdere vragen.

\subsubsection{Sluiting}
De volgende vergadering zal plaatsvinden op 12-05-2015 om 10:00 in meeting room 2.\\

Joost sluit de vergadering om 12:34.\\


\subsection{Status Meeting 6, 12-05-2015 10:00}
\subsubsection{Opening}
Locatie: Fellowship Meeting room 2\\
Aanwezig: HD, DD, J, L, G, A, B, T, D, Su, Se\\
Verontschuldigd: NR \\
Afwezig: - \\
tijd: Joost opent de vergadering om 16:03\\

\subsubsection{Agenda}
De agenda van de vergadering bestaat oorspronkelijk uit de volgende punten:
\begin{enumerate}
\item Opening
\item Agenda
\item Goedkeuring van de notulen
\item Trade-off criteria
\item Validatie van de ORbit Tool
\item Andere vragen
\item W.V.T.T.K.
\item Rondvraag
\item Sluiting
\end{enumerate}

Aan de agenda worden geen punten toegevoegd.

\subsubsection{Goedkeuring van de notulen}
de notulen worden goed gekeurd

\subsubsection{Trade-off criteria}
- Decelerator mass
- Developement risk
- Stability (Cmalpha)
- Deceleration time(Clalpha)

J Deceleration time Clalpha, capability om slope van de trajectory aan te passen. 

B Massa. niet meer specifieke massa's, alleen voor structures. Daarom genoodzaakt om andere vergelijkingsmaat te vinden. 

HD vergelijking: stel, we maken geen control systeem. Dan massa control = 0. dat lijkt heel goed. waar zie je dat dit slecht is. T Cmalpha geeft dit aan. Cm/Cl wordt een soort van requirement. Dus het staat zowel in de trade-off als in de requirements. zelfde verhaal als met reliability. HD Je kunt Cm en Cl afkopen met control massa. T het werkt juist de andere kant op. HD: Oke, dan werkt het. goed over nagedacht. Dit moet je goed uitleggen bij je midterm review. 

HD: Nog opsplitsen van massa's? Er lijkt iets te missen. DD het is niet zo overzichtelijk. Werkt wel met goede uitleg. Echte tijd kunnen we nog niet geven, is nog te vroeg. discussie waarom geen tijd beschikbaar. Nog erg aghankelijk van PID controller. Het kan nog niet worden aangenomen dat die PID perfect werkt. kan niet constand 3g erop houden. HD: 3g constant aanhouden is niet altijd de goede manier. Wat is de ideale trajectory? Wat wil ik / wat kan ik? los van de tool. Stel dat je nog geen control system hebt. en kan niet, dan moet trade-off aanwakkeren. T wil eigenlijk zo snel mogelijk afremmen, dus continue 3g. Dat programmeren is lastig. 

kunt eigenlijk beter aannemen dat je alpha direct kunt aanpassen dan zien welke orbit je wilt hebben. en dan krijg je een beeld welk totaal alpha verschil je nodig hebt. kun je moeilijk kiezen. Je kunt toch cl en cd rho en snelheid altijd aanpassen zodat je 3g hebt. Ja, dat kan. HD: dit klinkt als een goed idee. later de PID nog inbouwen.

Daarom beter trade-off criteria. Als het lukt gaan we dan voor de deceleration time. 


B massa's samen of appart nemen? samen, maar met toelichting in de trade-off van de 3 verschillende massa's, die niet mee doen voor de echte criteria.


Cma en Cla hangen af van M en wordt vrij betrouwbaar emt toenemende 


Thermal andere weg op geslagen. Niet meer lay-up tool. Nu heat load vergelijken. hoe nieuwe tool werkt, waarom oude tool niet werkt.


\subsubsection{Validatie tool}
T de getallen zijn heel mooi. Verificatie werkt helemaal prima, maar validatie lukt niet omdat er geen refernetie cases zijn met een vergelijkbare trajectory. HD: meer dan dat kun je nog niet doen. 

\subsubsection{Andere vragen}
Personal in het het rapport? Nee, niet nodig. Als appart document in de dropbox. met welk pagina limiet? 1 pagina it is.

Wisselen taken, na de MTR? Ja. komt nog.

\subsubsection{W.V.T.T.K.}
HD: Midterm. netjes kleden

HD: chris mokkum. Heeft paper ingediend. Komt een conferentie. Het zou leuk zijn als jullie dit volgend jaar deden. Volgende week wil hij ook wel langs komen. Je kunt vragen bij hem kwijt. 

HD mensen uit de VS. gast van NASA op het bureau. En andere gast ook bij nasa gewerkt. Nu hoogleraar. Zij willen de bachelor verbeteren.

\subsubsection{Rondvraag}
A veranderingen van zalen. ACTIEPUNT uitzoeken. En uitzoeken symposium dag. ACTIEPUNT MAILEN

\subsubsection{Sluiting}
Er is nog niet besproken wanneer de volgende vergadering zal zijn.\\

Joost sluit de vergadering om 11:07.\\



\subsection{Status Meeting 7, 22-05-2015 16:00}
\subsubsection{Opening}
Locatie: Fellowship Meeting room 2\\
Aanwezig: HD, DD, J, L, G, A, B, T, D, Su, Se\\
Verontschuldigd: NR \\
Afwezig: - \\
tijd: Joost opent de vergadering om 16:03\\

\subsubsection{Agenda}
De agenda van de vergadering bestaat oorspronkelijk uit de volgende punten:
\begin{enumerate}
\item Opening
\item Agenda
\item Goedkeuring van de notulen
\item Trade-off criteria
\item Validatie van de Orbit Tool
\item Andere vragen
\item W.V.T.T.K.
\item Rondvraag
\item Sluiting
\end{enumerate}

Aan de agenda worden geen punten toegevoegd.

\subsubsection{Goedkeuring van de notulen}
de notulen worden goed gekeurd

\subsubsection{Trade-off criteria}
De trade-off criteria zijn ietwat nogmaals besproken door het team en de volgende resultaten worden voorgesteld:

- \textbf{Decelerator mass}\\
- \textbf{Developement risk}\\
- \textbf{Stability (Cmalpha)}\\
- \textbf{Deceleration time(Clalpha)}\\

Voor het vergelijken van concepten zijn bijbehorende parameters toegewezen. Voor decelleration mass wordt gebruik gemaakt van relatieve massa's ten opzichte van de stacked torroid. Hiervoor is gekozen omdat in dit stadium nog niet voor alle concepten een specifiek massa getal is te plakken op het gehele concept.  Bij de deceleration time is voor de parameter  Clalpha gekozen omdat deze aangeeft wat de capaciteit is van het voertuig om de slope van de trajectory aan te passen. \\

Op het voorstel gaat HD in en maakt de volgende vergelijking: stel, we maken geen control systeem. Dan massa control = 0. dat lijkt heel goed. waar zie je dat dit slecht is? T reageert en zegt dat Cmalpha dit aangeeft. Dit wordt besproken en er volgt uit dat Cm/Cl een betere maatstaaf is. HD is echter nog niet overtuigd omdat hij vindt dat Cm en Cl af te kopen zijn met control massa. Maar T verteld dat het juist de andere kant op werkt: met een total massa en vorm kun je Cm en Cl krijgen. HD is het hier na een korte discussie ook mee eens en zegt dat deze redenatie ook goed naar voren moet komen tijdens de Midterm Review.\\

Het volgende punt wat nog vragen opwekt is de massa. HD vraagt zich af of de massa's nog worden opgesplitst in verschillende secties, omdat het nu lijkt alsof er nog iets mist. DD is het hier ook mee eens. B verteld dat er inderdaad een opsplitsing gemaakt kan worden in TPS-, structural- en control massa. Na dit te hebben besproken worden deze deel massa's aangenomen als deelkopjes onder de massa trade-off.\\

 Over de deceleration time wordt ook nog getwist. De groep kan hier nog niet echte een specifieke tijd aan geven. Er volgt een discussie waarom deze tijd er nog niet is. Het blijkt dat deze erg afhankelijk van PID controller, en die controller werkt nog niet goed. kan niet constant 3g erop houden. HD verteld dat 3g constant aanhouden niet altijd de goede manier is. Het probleem zou eens van een andere hoek beken moeten worden: bedenk eerst wat voor trajectory je wilt/ wat je moet kunnen en ga dan kijken of je dat kunt halen. Stel dat je nog geen control system hebt. en kan niet, dan moet trade-off aanwakkeren. T ziet dit anders, want je wilt eigenlijk zo snel mogelijk afremmen, dus continue 3g zou dan de ideale oplossing zijn. Dat programmeren is lastig. HD her formuleert zijn statement: Je kunt eigenlijk beter aannemen dat je alpha direct kunt aanpassen en dan zien welke orbit je wilt hebben. Dan krijg je een totaal beeld van welk alpha verschil je nodig hebt. Dit idee wordt besproken en het blijkt mogelijk door Cl en Cd van richting te veranderen. later kan dan de PID-controller nog ingebouwd worden.\\
 
Er wordt besloten dat de deceleration time parameter voorlopig wordt weergegeven door de eerder genoemde parameter. Echter, als het hierboven genoemde control systeem zonder PID-controller nog kan worden geïmplementeerd voor de Midterm Review, dan zou de voorkeur veranderen naar een daadwerkelijke deceleration time.\\

Na de parameter discussie volgen er nog enkele vragen. de eerste gaat over de waardes van de aerodynamische coëfficiënten. Zijn deze waardes wel betrouwbaar voor hoge snelheden? J legt uit dat de betrouwbaarheid van Cma en Cla schalen met het mach nummer en juist betrouwbaarder worden met hogere snelheid. De theorie kan niet meer worden toegepast onder Mach 5.\\

De tweede vraag gaat over thermal mass.L geeft aan dat de thermodynamics group een andere weg is op geslagen. Nu wordt er geen gebruik meer gemaakt van de lay-up tool, maar van een heat load. L en Su leggen uit wat voor een probleem is opgedoken in de tool en dat het gezien de korte tijd beter leek om toch maar resultaten in een andere vorm te leveren (hierbij verwijzen zij naar een eerder gesprek met NR), om zo toch nog wat nuttigs te kunnen zeggen de TPS-massa. De heat load geeft aan hoeveel energie het schild in totaal moet kunnen dragen en daarom is voor deze waarde gekozen als trade-off massa. De Tutors zijn het hier onder de omstandigheden mee eens maar vinden het wel jammer en geven aan dat het fijn zou zijn als de fout toch nog gevonden kan worden voor de Midterm Review.\\


\subsubsection{Validatie tool}
T heeft dit punt aangevraagd op de agenda omdat hij geen referentie cases kan vinden om de tool mee te valideren. Er bestaan geen vergelijkbare uitgevoerde missie trajectories. HD gaat hier op in en zegt dat het in dat gewoon niet mogelijk is om de tool te valideren. Dat gaat u eenmaal niet met pionier werk.\\

\subsubsection{Andere vragen}
\textit{Q1: B: Moet de personal appendix in het het rapport? en wat is de pagina limiet?} \\
HD zegt dat het niet nodig is om dit document in het raport te steken. Als de beschrijvingen in een los document komen te staan dan is dat ook goed. De limiet is ongeveer 1 pagina.\\

\textit{Q2: J: Moeten we na de Midterm Review nog wisselen van taken?}\\
HD zegt dat dit inderdaad moet en dat er nog verdere details volgen.\\

\subsubsection{W.V.T.T.K.}
HD merkt op dat de groep zich gepast moet kleden voor de Midterm Review.\\

HD Merkt ook op dat hij weer contact heeft gehad met Chris Mokkum. Hij heeft een paper ingediend over zijn DSE twee jaar geleden en dat mag hij gaan presenteren op een conferentie. hij gaf aan dat hij het leuk zou vinden als de huidige groep dat volgend jaar zou doen. Volgende week wil hij ook wel langs komen. Alle prangende vragen kun je ook bij hem kwijt.\\

HD geeft ook een toelichting over de studenten die vanuit de VS. naar Delft komen. HD zat een tijdje met een man die nu bij NASA werkt op het kantoor. Een collega van deze meneer werkt nu niet meer bij NASA, maar is hoogleraar en hij is geïnteresseerd in het verbeteren van zijn bachelor curriculum. Delft is voor hem een voorbeeld. Daarom komen studenten naar mogelijke verbeteringen. HD geeft aan dat hij het leuk zou vinden als de groep een zo'n groepje zou willen ontvangen. Er wordt bevestigd dat dit ook al is besproken met de OSSA's.\\

\subsubsection{Rondvraag}
\textit{Q1: A: Vanaf nu is het mogelijk om online te kijken of er nog zalen beschikbaar zijn. Is het niet verstandig om hier eens naar te kijken?}
Dit wordt door de secretaris verder uitgezocht.

\textit{Q2: A: Ik zag ook dat we een mail hebben ontvangen over het geven van een voorkeur m.b.t. het symposium. Kan dit worden uitgezocht?}
De secretaris zal dit verder uitzoeken.


\subsubsection{Sluiting}
Er is nog niet besproken wanneer de volgende vergadering zal zijn.\\

Joost sluit de vergadering om 17:10.\\




\subsection{Status Meeting 8, 27-05-2015 11:00}
\subsubsection{Opening}
Locatie: Fellowship Meeting room 2\\
Aanwezig: HD, NR, DD, J, L, G, A, B, T, D, Su, Se\\
Verontschuldigd: - \\
Afwezig: - \\
tijd: Dawud opent de vergadering om 11:03\\

\subsubsection{Agenda}
De agenda van de vergadering bestaat oorspronkelijk uit de volgende punten:
\begin{enumerate}
\item Opening
\item Agenda
\item Goedkeuring van de notulen
\item Vernieuwde team indeling
\item Onduidelijkheden final deliverables
\item Logboek
\item Peer review
\item Bespreking mid term review
\item Andere vragen
\item W.V.T.T.K.
\item Rondvraag
\item Sluiting
\end{enumerate}

Aan de agenda worden geen punten toegevoegd.

\subsubsection{Goedkeuring van de notulen}
T merkt op dat $C_{L_\alpha}$ is aangepast sinds de laatste vergadering. Dit hoeft niet aangepast worden in de notulen, aangezien de aanpassing is gebeurd na de vergadering.
\newline\newline
De notulen worden verder goed gekeurd.


\subsubsection{Vernieuwde team indeling}
De groep heeft zich opnieuw ingedeeld over de verschillende organisatorische taken. T.o.v. de eerste helft van de DSE zijn de rollen met verification- en validation-taken vervallen. Het tekort aan taken is opgevuld door de secretaris te ondersteunen met een aparte notulist. Verder zijn er twee editors i.p.v. één voor het laatste report.
\newline\newline
De vernieuwde team indeling is als volgt:
\begin{tabular}{ll}
	\textbf{Rol}	&	\textbf{Persoon}\\
	Chairman & Dawud Hage\\
	Secretaris & Guido van Koppenhagen\\
	Notulist & Suthes Balasooriyan\\
	Editor I & Twan Keijzer\\
	Editor II & Sebastiaan van Schie\\
	Risk Engineer & Lucas Mathijssen\\
	Systems Engineer & Bj\"{o}rn van Dongen\\
	Planner & Joost van Meulenbeld\\
	Documentation manager & Alexander van Oostrum\\
\end{tabular}

\subsubsection{Onduidelijkheden final deliverables}
De groep is van mening dat de lijst met delivarbles voor het Final Report veel onduidelijkheden bevat, aangezien niet alles toepasbaar is op de missie. Een voorbeeld hiervan zijn de spacecraft systems characteristics (data rates, link budget etc.) uit de lijst van deliverables. HD geeft aan dat dit voorgeschreven deliverables zijn en dat het belangrijk is dat het in ieder geval over nagedacht moet worden en het moet vermeld worden in het Final Report. Dit mag eventueel beknopt zijn. De groep vindt ook dat sommige deliverables (material and astrodynamics characteristics) uit het Mid Term Report juist van groot belang zijn voor het Final Report en toegevoegd zouden moeten worden.

\subsubsection{Logboek}
NR zegt dat de laatste update dateert van 6 mei. Hij vindt het een nuttige deliverable en vraagt hoe dit kan. L zegt dat hij hierin tekort is geschoten en dat het inderdaad beter had gemoeten. HD haalt de laatste afspraak aan waarin de eisen van het logboek zeer laag waren gesteld. Het invullen hiervan mag geen last worden. De groep zal vanaf nu het logboek op de dropbox direct aanpassen, zodat het direct inzichtelijk is.

\subsubsection{Peer review}
NR herinnert de groep dat de deadline van de peer review vandaag (27 mei red.) is. Ook wordt gezegd dat er gebruik gemaakt kan worden van private en public comments. HD raadt de groep aan eerlijk te zijn, zodat een ieder zich kan verbeteren op zijn minder goede punten.

\subsubsection{Bespreking mid term review}
D zegt dat het waarschijnlijk niet lukt om alles te bespreken en stelt voor om tot de lunch (12:30) door te gaan en de rest voor de volgende meeting (28 mei red.) te laten. DD geeft aan niet te kunnen op vrijdag 28 mei, hij zal zijn overige punten op donderdag (27 mei) aan de groep geven. 
\newline
Het taalgebruik is goed volgens NR en HD vindt dat de summary een goede structuur heeft.
\newline\newline
\textit{List of symbols}\newline
Wanneer een unit als breuk [teller/noemer] in de tekst staat, wordt de regelafstand een beetje raar. Dit is volgens B (editor) gedaan om de leesbaarheid van langere units te waarborgen. 
\newline\newline
\textit{Introduction}\newline
NR las dat concept mass is opgedeeld in slechts drie delen. Hij vraagt zich af of de payload mass dan niet is weggelaten. Su zegt dat hier decelerator mass i.p.v. concept mass beter op zijn plaats was geweest.
\newline\newline
\textit{Chapter 2}\newline
Het stuk over de launch en interplanetry flight (dus voor de missie) is nogal kort. HD vraagt zich af of er genoeg is nagedacht over de implicaties van deze fases op de missie.
HD: launch/interplanetary flight nogal kort. Nagedacht over implicaties van deze fase op onze missie?\newline
DD vraagt of er nog andere requirements voor Phase 4. HD zegt dat er niets concreets uitgewerkt moet worden, maar dat er wel een plan moet zijn.\newline
HD vraagt of er nog naar het aantal astronauts is gekeken. G zegt dat dit laatste tijd niet echt meer bekeken is, maar stelt wel dat de groep snapt dat het er minder worden dan eerder gedacht bij de baseline review (toen waren het er zes).\newline
NR heeft een algemene comment over de alinea's. Hij raadt aan om er meer te maken, aangezien dat fijner leest dan een lange lap tekst.
\newline\newline
\textit{Chapter 3}\newline
Weinig inhoudelijk veranderd, geen comments verder.
\newline\newline
\textit{Chapter 4}\newline
Inhoudelijk hetzelfde gebleven. DD zegt dat het beter is om eerst het grote voordeel te noemen van een aeroshell voor sustainability. Nu wordt eerst gezegd dat het tamelijk onbelangrijk is en dan pas wordt er een voordeel genoemd.
\newline\newline
\textit{Chapter 5}\newline
HD vindt het hoofdstuk nogal leeg. Er wordt niet veel duidelijk in het hoofdstuk, dingen worden later in het report pas duidelijk. Er wordt weinig verwezen naar het verdere report. Hoofdstuk had als goede introductie voor de rest van het technische deel kunnen dienen.\newline
NR heeft een opmerking over een te grote $C_{m_\alpha}$, dat is namelijk ook niet heel goed voor de stabiliteit.
\newline\newline
\textit{Chapter 6}\newline
Er zijn complimenten voor de schetsen. DD vraagt zich af of het inderdaad eerlijk is om de control los te koppelen als trade-off. De groep geeft aan hier lang over nagedacht te hebben en daarop haar besluit heeft gemaakt.\newline
NR vraagt of een isotensoid met ram-air opgeblazen moet worden. A zegt dat de grootte van de inflatable van een isotensoid dit noodzakelijk maakt. Anders zou er onredelijk veel inflation gas mee moeten bij de launch. NR zegt wel dat er dan rekening gehouden moet worden met het feit dat de isotensoid niet is opgeblazen als het de eerste keer de atmosfeer binnenkomt.\newline
HD vraagt aan de groep of ze nog steeds achter de keuze staan om het control system zo laat te kiezen. T geeft aan dat per concept de mogelijke control systemen bekeken zijn. De grootte hiervan is nog onbekend en zal in de volgende fase behandeld worden. HD zegt dat dit onvoldoende in het report staat.\newline
Er is een vraag over noodzakelijkheid van de connector in de tension cone FBD B zegt het een structureel onderdeel is.\newline
Er wordt opgemerkt dat Tabel 3 waarschijnlijk iets later in het hoofdstuk had gekund.
DD: Concept control systems had eerder in het hoofdstuk gemoeten bij Tabel 3.
\newline\newline
\textit{Chapter 7}\newline
DD zegt dat de titel \textit{Subsystem design interfaces} de lading van het stuk niet helemaal dekt. Dit had beter \textit{Design interfaces} genoemd kunnen worden. Elementen in de N2-chart zijn namelijk geen subsystems.\newline
HD vraagt zich af of de groep wel genoeg weet over subsystems. A zegt dat zolang er geen final concept was, er vaak niet niet op subsysteem-niveau geanalyseerd kon worden.\newline
DD zegt dat het figuur met de \textit{System communication flow} suggereert dat de ADCS-computer het enige redundant component is. B stelt dat vermeld had moeten worden dat we ze fail-safe willen uitvoeren.
\newline\newline
\textit{Chapter 8}\newline
NR vindt dat het goeds dat de assumptions apart vermeld zijn. Er volgt een discussie over hoe de snelheid is gedefinieerd. HD vraagt zich af hoe de hoek van de incoming hyperbolic orbit is bepaald. Uitleg van Dawud/Twan is afdoende, maar HD vind dat dit niet duidelijk genoeg in het report staat. Het is onduidelijk wat de flight path vector is in Phase I.\newline
HD vraagt wat de bedoeling is met het control system. De referentie moet al bepaald zijn. D zegt dat ze tot nu toe zijn we bezig geweest met het vinden van een pad dat als referentie kan dienen. HD zegt dat ze bewust moeten zijn wat de referentie wordt. Op acceleratie (3g) sturen werkt blijkbaar niet.\newline
Er staat geen referentie voor de eigenschappen van Mars. Verder heeft DD een vraag m.b.t. de discretisation error en starting offset.\newline
DD vraagt ook of er andere pertubaties zoals J2-effect worden meegenomen. D geeft aan dat dit waarschijnlijk te veel werk gaat worden.\newline
Verder is er nog een opmerking over de vertical as van Figuur 23. DD raadt aan om break-symbols te gebruiken in grafieken. NR mist verder korte conclusies na elk technisch hoofdstuk.
\newline\newline
De rest van het report zal tijdens de volgende meeting (vrijdag 29 mei) besproken worden.
\subsubsection{Andere vragen}
Er zijn geen verdere vragen ingediend.

\subsubsection{W.V.T.T.K.}
\textbf{1.} HD herinnert de groep aan de komst van studenten van de Purdue University. Hij vraagt of de groep iets heeft bedacht voor de student(en). Hij vraagt om in ieder geval een introductie te geven over de hoe de DSE zelf werkt binnen de bachelor. Ook zou ons project binnen de DSE ge\"{i}ntroduceerd moeten worden en we kunnen eventueel laten zien hoe we de conceptuele fase zijn doorlopen.
\newline\newline
\textbf{2.} L vraagt of de groep er op 6 juli (tweede grading dag) bij moet zijn. HD zegt dat het symposium het laatste verplichte onderdeel is. Maar dat de grading meeting tussen de tutor, coaches en een DSE-co\"{o}rdinator op ofwel vrijdag 3 juli ofwel maandag 6 juli plaats vindt. Daarna zou de groep eerst als groep het groepscijfer en daarna individueel het persoonlijke cijfer kunnen ontvangen. In dat opzicht heeft 3 juli de voorkeur.\newline \underline{Actiepunt: Joris Melkert mailen om op vrijdag 3 juli de grading meeting te plannen.}
\newline\newline
\textbf{3.} Su vraagt hoe het precies zit met de Mid-Term grading. HD geeft aan dat de meeting met Vincent Br\"{u}gemann nog moet plaatsvinden. Hij benadrukt dat het cijfer een indicatie is naar waar het uiteindelijke cijfer toe gaat, aangezien er ook beoordeeld wordt op onderdelen die nog moeten komen. G vraagt of de grading sheet inzichtelijk is. Dit is niet zo, in de persoonlijke meetings worden de verbeterpunten aangegeven.

\subsubsection{Rondvraag}
B: Kan NR helpen met het bekijken voor het plan voor de structural analysis.

\subsubsection{Sluiting}
De volgende vergadering zal plaatsvinden op 29-05-2015 om 11:00 in Fellowship Meeting room 2.
\newline\newline
Dawud sluit de vergadering om 12:49.

\subsection{Status Meeting 9, 01-06-2015 10:00}
\subsubsection{Opening}
Locatie: Fellowship Meeting room 1\\
Aanwezig: HD, NR, DD, J, L, G, A, B, T, D, Su, Se\\
Verontschuldigd: - \\
Afwezig: - \\
Tijd: Dawud opent de vergadering om 10:08\\

\subsubsection{Agenda}
De agenda van de vergadering bestaat oorspronkelijk uit de volgende punten:
\begin{enumerate}
\item Opening
\item Agenda
\item Goedkeuring van de notulen
\item Bespreking Mid Term Review
\item Andere vragen
\item W.V.T.T.K.
\item Rondvraag
\item Sluiting
\end{enumerate}

Aan de agenda zijn verder geen punten toegevoegd.

\subsubsection{Goedkeuring van de notulen}
Notulen van de vorige meeting (woensdag 27-08-2015) zullen via de mail verstuurd worden.
De notulen worden verder goed gekeurd.


\subsubsection{Bespreking mid term review}
Vorige meeting zijn we tot Chapter 8 gekomen.
\newline\newline
\textit{Chapter 9}\newline
HD komt de side slip angle $\beta$ voor het eerst tegen op blz. 40. Hij vraagt zich af hoe de roll/pitch/yaw dan werkt in de orbit. D zegt tot nu toe enkel 2D motion bekeken te hebben. Sideslip $\beta$ wordt tot nu toe als te complex gezien. HD waarschuwt dat iets niet weggelaten mag worden omdat het te moeilijk is voor de huidige fase van ontwerp. Detail-ontwerp zou ook dit soort dingen moeten bevatten, desnoods versimpel je het probleem. Zelfde geldt voor subsystemen. Je wilt weten wat je ongeveer meeneemt en hoeveel het ongeveer weegt etc.
\newline
NR vraagt of er een minimum Mach-number is voor de Newtonian flow theory. Se bevestigt dat het alleen in het hypersonic-regime geldt (M>5). HD complimenteert de documentie van de V\&V, maar waarschuwt dat het niet te veel aandacht moet krijgen in het report. NR geeft aan dat validatie teruggekoppeld moet worden met de aannames. HD had graag geweten hoe de vorm van de concepts in Figure 33 is gedefinieerd.
\newline
HD merkt ook op dat $C_M$, $C_MA$, $C_{M_\alpha}$ worden door elkaar gebruikt, zelfde geldt voor de gebruikte units. De groep geeft aan dat dit komt door last-minute aanpassingen.
\newline\newline
\textit{Chapter 10}\newline
NR merkt op dat er iets mis is gegaan met de editing. Soms staan zinnen dubbel. De groep geeft aan dat er wat problemen waren met het file-exchange program (gitHub). HD vindt dat de beschrijving van de tool aan de lange kant is. Hij had graag meer uitleg over de ontwerpen zelf gezien. Dus meer resultaten van de tool zelf. NR waarschuwt de groep voor het feit dat er waarschijnlijk veel uitgezocht moet worden voor het uiteindelijke ontwerp. Tools zullen niet alle informatie geven. HD is bang dat er te veel focus op de tools ligt en dat de elementaire ontwerp kennis wellicht nog ontbreekt. Het gaat om het begrip van het ontwerp en de outputs van de tools zou niet blind aangenomen mogen worden.
\newline
HD vraagt waar de 14\% backshell mass van Steinfeldt vandaan komt en of het toepasbaar is op onze missie. A geeft aan dat het niet per se om het getal gaat, maar vooral om aan te tonen dat het rigid concept uiteindelijk te zwaar wordt. Het is volgens HD niet duidelijk wat er gebeurt bij het groter/kleiner maken van het vehicle.
\newline
HD adviseert om ook zeer goed na te denken over het ontwerp. Er wordt van alles gevraagd tijdens de Final Review en het Symposium en ze kunnen daarin best hard zijn.
\newline
HD merkt op dat alle concepten worden op eenzelfde trajectory vergeleken. Maar dit is al fout. Mid Term Review lijkt te vroeg te zijn geweest. We hebben als groep besloten om het zo te doen, aangezien er op dat moment niet meer data over verschillende trajectories beschikbaar was en het waarschijnlijk niet mogelijk was om voor de Mid Term dit nog nauwkeuriger te doen. HD vraagt of we als groep weten wat het kritieke punt is voor het halen van deadlines. Dit lijkt vooral over trajectories te gaan. D geeft aan dat een trajectory bepaald kan worden, maar dat er feedback nodig is van de thermal tool.
\newline
NR vraagt waarom de aerial thickness is gegeven en waarom sommige waardes niet zijn gegeven. Na verklaringen van A en B zegt hij dat er best geschatte waardes gegeven mogen worden als de waardes niet te vinden zijn in Table 7. Sowieso kunnen sommige kolommen in Table 7 weggelaten worden. Verder had een maximum temperature kolom handig geweest bij Table 7.
\newline
HD geeft aan dat communicatie bij problemen heel belangrijk is. Geldt ook voor buiten het DSE.
\newline
NR vraagt hoe isotrope en anisotrope materialen worden meegenomen. B zegt dat daar meer uitgewerkt moet worden. HD vindt dat er betere uitleg moet zijn over hoe de verdere figuren gebruikt kunnen worden in een ontwerp.
\newline\newline
\textit{Chapter 11}\newline
NR merkt op dat Table 8 met material properties voor structural dubbelop is. Verder vraagt hij of de thermo-groep weet wat de warmte doet met de inflation gas. B geeft aan dat het niet veel uit zal maken, zolang de temperatuur niet te hoog wordt. Ook vraagt hij of er naar contact resistance wordt gekeken, dus de resistance tussen de layers. Su zegt dat het model hier geen rekening mee houdt, maar dat de validatie hier kan laten zien of dit een plausibele aanname is.
\newline
NR adviseert goed te kijken naar de radiation. Nu wordt er enkel $T_\infty$ gebruikt. DD noemt als voorbeelden de infrarood-straling (IR) van Mars en het albedo. Verder vraagt NR of peak heat load van de ballute voor de payload module of de inflatable geldt. L zegt dat het voor allebei geldt.
\newline\newline
\textit{Chapter 12}\newline
HD vraagt waarom number 10 (Ability of structure to withstand mission loads) TRL1 heeft. B zegt dat dit inderdaad een hogere TRL-level heeft, omdat IRVE op aarde is getest. Een fout dus.
\newline\newline
\textit{Chapter 13}\newline
Besproken bij de review.
\newline\newline
\textit{Overig}\newline
Verder zijn er geen comments meer. 
\subsubsection{Andere vragen}
Er zijn geen verdere vragen ingediend.

\subsubsection{W.V.T.T.K.}

\textbf{1.} D vraagt wanneer de vergadering is waarin de groep haar beoordeling krijgt. HD zegt dat er later vandaag een afspraak staat gepland met Nando Timmer. Met de groep wordt het groepscijfer gegeven en dan wordt het invididuele cijfer in een minuut of twee per persoon medegedeeld. Dit kan bijvoorbeeld woensdag.

\textbf{2.} D vraagt of het nodig is om nog twee vergaderingen te hebben deze week? T meent dat een eventuele statusupdate wel handig op woensdag. Er wordt afgesproken om tussen 13:30 en 14:00 een vergadering in te plannen.

\textbf{3.} T vraagt of de personal appendix en peer review zijn gelezen. HD heeft het niet heel goed gelezen. Hij vraagt of de groep het nagekeken wil hebben. L zegt dat hij het liever wel ziet dat het gelezen wordt, zeker omdat er aangegeven is dat we een gesloten groep zijn. Dit soort dingen kunnen dan misschien meer inzicht geven. HD belooft het voor woensdag te lezen.

\textbf{4.} HD hoe het ging met de Amerikanen? Se vertelt dat we ze hebben uitgelegd wat de DSE is en wat wij daarin als opdracht hebben gekregen. Daarna hebben we per department individueel uitgelegd wat we gedaan hebben. Over het algemeen is dit wel goed gegaan.

\textbf{5.} NR vraagt aan HD wat de diepgang was van de structural analysis twee jaar geleden. HD zeg dat ze de loads op hun structure bekeken en een globaal idee hadden wat ze doen op hun structure. B legt vervolgens uit hoe de structural department dit tot nu toe heeft gedaan. Diepgang is volgens HD tot nu toe voldoende. HD adviseert vooral naar de subsystemen te kijken, het is namelijk ook gewoon een deliverable.

\subsubsection{Rondvraag}
Er zijn verder geen vragen.

\subsubsection{Sluiting}
De volgende vergadering zal plaatsvinden op 03-06-2015 om 13:30 in Fellowship Instruction room 1.
\newline\newline
Dawud sluit de vergadering om 12:18.

\subsection{Status Meeting 10, 03-06-2015 13:30}
\subsubsection{Opening}
Locatie: Fellowship Instruction room 1\\
Aanwezig: HD, NR, DD, J, L, G, A, B, T, D, Su, Se\\
Verontschuldigd: - \\
Afwezig: - \\
Tijd: Dawud opent de vergadering om 13:35\\

\subsubsection{Agenda}
De agenda van de vergadering bestaat oorspronkelijk uit de volgende punten:
\begin{enumerate}
\item Opening
\item Agenda
\item Goedkeuring van de notulen
\item Mid-term grading
\item Status update huidig werk
\item Plan van aanpak toekomstig werk
\item Andere vragen
\item W.V.T.T.K.
\item Rondvraag
\item Sluiting
\end{enumerate}

Aan de agenda zijn verder geen punten toegevoegd.

\subsubsection{Goedkeuring van de notulen}
Opmerking over notulen van 27 mei: actiepunt over Joris Melkert was niet nodig, dit was verkeerd begrepen door Su (notulist).
Opmerking over notulen algemeen: De conclusie van discussies moet meer naar voren komen. Het is niet van belang om te notuleren wie wat precies gezegd heeft.
Verder geen op- of aanmerkingen.

\subsubsection{Mid-term grading}
Voorlopig groepscijfer 6. Met kanttekening dat dit cijfer een beetje vertekend is en dat de groep eigenlijk op een hoger cijfer zit, maar dit kan niet naar voren komen met het gegeven gradingsheet. Het groepscijfer is opgedeeld in drie onderdelen: Design, Process en Communication. De weging per onderdeel is niet gelijk, design weegt het zwaarst (ongeveer 40\%).
\newline\newline
%%%%%%%%
Ontwerp, communicatie en proces. Weegt niet allemaal even zwaar, design het zwaarst (40\%?).

Bij design te veel focus op tools, er is geen daadwerkelijk ontwerp. Dit wordt ook zo gegrade. 
Subsystemen kan verbeterd worden. Control system mist bijv. volledig. Sensitivity analysis en V\&V is dan weer goed. Bij V\&V kunnen afwijkingen beter beschreven worden.
Market en cost analysis kan iets uitgebreider (voor relatief makkelijk meer punten).
Requirements flowdown ontbreekt een beetje.

Proces is best aardig/goed. Use of (external) resources. Kennis die je niet hebt moet je van buiten halen. Betrek anderen bij ons werk.

Communication rond de goed. Communication with external staff nog steeds missend. Meetings zijn goed, terminologie, referenties zijn goed. Ability to answer staff questions is minder. We moeten meer met feedback doen. Trekken af en toe een muur op en lijken niet open voor commentaar.

Group component voor 40\% en individual op 60\%. Gemiddeld rond 6-6.5, er is nog veel ruimte om makkelijk omhoog te gaan. Als groep zijn de tutor en coaches tevreden.
Individual algemeen: tussen RV en G. Eén punt waar iedereen als individu kan winnen is het begrip van de tools (understanding of subject matter). Attitude is goed. Job performance/initiative/communication goed. 

Vraag B: Tot nu toe bij structures alleen load/ truss-analysis gedaan. Dit is enkel analysis, maar moeten we dan nog designen. HD: Ja in principe moet je wel designen zodat we een idee hebben hoe het eruit gaat zien.
Vraag L: Zijn er ook nog extreem negatieve punten? HD: Nee.

Conclusie: cijfer wat laag, maar er zijn ook verbeterpuntjes.
\subsubsection{Status update huidig werk}
D: We zijn bezig met de uitwerking van tools, moet vrijdag af zijn. Discussies gehad wat uit tools moeten komen en hoe we die interfacen met andere tools. We gaan even de alle departments langs.

G: Zelfde tool als bij MR, dus modified newtonian tool. Nu bezig met zoeken naar EEN optimale vorm. EEN optimale vorm, omdat het nog niet bekend is welke parameter je design optimaal maakt. Werkt in principe nu wel. Zal volgende week bekeken worden wat het beste is.
NR: Gaat dit automatisch of worden datasets overgedragen.
J: Legt uit hoe data overgedragen wordt tussen de tools. (q van orbit naar aero en qs dan van aero naar thermo).
HD: Zorg dat je weet wat de relatie is tussen je design parameters en performance parameters.
G: Die kennis heb je bijvoorbeeld nodig om gewichten aan je optimalisatietool te geven.
HD: Nu is er nog geen sprake van een optimaal punt. Het is beter om een initiële waarde te hebben. Zonder optimalistietool zou je moeten kunnen zeggen welke richting je op wil gaan om een bepaald iets te verbeteren.

T: Bank-control geïmplementeerd. 
D: Lift-vector aan het pointen, de manier waarop is niet gegeven. Die moet extern geregeld worden. Voordeel is minder afhankelijkheid van density zoals bij alpha-control. 
NR: Dus een rotationally asymmetric flow?
T: Dat idd, of sideslip. Dat dus geïmplementeerd, werkt beter. We sturen 
NR: Stuurt op een hoogte?
T: Ja hij stuurt op een hoogte m.b.v. alpha en sideslip-angle. 

Se: BEzig met augmented static stability. Problemen met het oplossen van 3 eqn.
T: Momenten hoeken etc. bekijken.
% Even vragen zo
NR: Hoe worden control-surfaces gedimensioneerd?
D: Dat zal daaruit moeten komen.

L: 1D-tool verbeterd. Eerst FTCS-scheme, nu Crank-Nicolson scheme (altijd stabiel). Eerst wilden we een 3D-tool. Dat is echter niet de prioriteit. Prioriteit is design. Design op stagnation point. Als eerste laag faalt, heb je of een andere laag nodig, of andere orbit. 1D-tool verbeteren, zodat we een goede tool hebben, ipv twee halve. Ook radiation gaan we bekijken.
HD: Ik denk dat TPS de missie gaat definieren.
L: Focus meer op design, dan op een mooie 3D tool.

A: Structures minder een tool. Alle elementen inflatables, centerbody, inflation system. Load analysis met een truss-structure. Die gebruiken om mass-model uit MR te verbeteren. Interne stressen niet gaan bekijken, meer systemen.
B: Inflatables vorm is al vastgezet. Straps etc. Voornaamste reden om de load te bekijken om te zeggen of iets feasible is of niet. 
G: Een minimum zal er wel zijn door de vorm.
NR: Wat doe je bij de lege voids?
B: Dat is allemaal geregeld.
A: Bij te weinig zul je misschien problemen krijgen.

HD: Is de diameter al bekend?
G: Nee, dat gaan we volgende week doen.
HD: Denk ook buiten de opgegeven 12m.

\subsubsection{Plan van aanpak toekomstig werk}
Gisteren (dinsdag 2 juni) een poging gewaagd om alle subsystemen die we gaan behandelen op een lijstje te zetten. Hierop staan welke subsystemen wij denken nodig te hebben, ook vanuit de deliverables gezien. Het is opgedeeld in delen over de decelerator, crew module en mission phase.
\newline
B vertelt dat we tot nu toe vooral bezig geweest met de tools voor de decelerator subsystems. Het ontwerp met deze tools zal uiteindelijk afdoende zijn voor het Final Report. Voor de subsystemen in de crew module is het plan om het volume, gewicht en eventueel locatie en power van de subsystems te bepalen/schatten om zo een completer beeld van het ontwerp te kunnen schetsen. We hadden al een hoofdstuk gewijd aan de mission phases in het Mid Term Report. We willen dit uitgebreider gaan doen, dus meer over de launch en meer over de interplanetary transfer. En dan vooral benadrukken wat voor invloed deze fases hebben op ons deel van de missie. (dit was ook een puntje van kritiek in de bespreking van het Mid Term Report).

\subsubsection{Andere vragen}
G vraagt of mogen aannemen dat we, zoals ook het plan is voor de Orion, met een transport capsule de interplanetary transfer doen. De transport capsule biedt dan een leefomgeving tijdens de langdurige transfer voor de astronauten. Dit verlaagt de eisen voor bijvoorbeeld lifesupport enorm. Nadelen zijn bijvoorbeeld een extra launch voor de transport capsule (naast het re-entry vehicle) en docking-requirements. HD zegt dat de groep daarover eerst een analyse moet doen en vervolgens met voorstellen moet komen. NR hint ook om te kijken naar de sterke punten van een aeroshell.
\newline
NR merkt vervolgens op dat dit soort vragen eigenlijk veel eerder hadden moeten komen. Als groep zijn we een beetje verkeerd om bezig. We zijn volgens HD iets te vroeg en te diep in de tools gedoken. Dit soort vragen zijn eigenlijk key drivers voor je design. Zo zal het veel uitmaken voor je design of je zo'n transport capsule mee krijgt of niet. Voorlopig mag er wel aangenomen worden dat er op Mars al spul aanwezig is, wel moeten de astronauten terug naar Aarde kunnen gaan.

\subsubsection{W.V.T.T.K.}

\textbf{1.} NR geeft aan vrijdag niet aanwezig te kunnen zijn bij de vergadering. Vrijdag komt Chris Mockel, die twee jaar geleden de DSE heeft gedaan. Chris Mockel is op dit moment bezig met zijn master en is vooral bezig met astrodynamica.

\subsubsection{Rondvraag}
Er zijn verder geen vragen.

\subsubsection{Sluiting}
De volgende vergadering zal plaatsvinden op 05-06-2015 om 11:00 in Fellowship Meeting room 2.
\newline\newline
Dawud sluit de vergadering om 15:00.

\subsection{Status Meeting 11, 05-06-2015 11:00}
\subsubsection{Opening}
Locatie: Fellowship Meeting room 2\\
Aanwezig: HD, DD, J, L, G, A, B, T, D, Su, Se, Chris Mockel (CM)\\
Verontschuldigd: NR \\
Afwezig: - \\
Tijd: Dawud opent de vergadering om 11:03\\

\subsubsection{Agenda}
De agenda van de vergadering bestaat oorspronkelijk uit de volgende punten:
\begin{enumerate}
\item Opening
\item Agenda
\item Goedkeuring van de notulen
\item Control system
\item Andere vragen
\item W.V.T.T.K.
\item Rondvraag
\item Sluiting
\end{enumerate}

In de agenda worden \textit{Bespreking met Chris Mockel} en \textit{Control system} samengevoegd.

\subsubsection{Goedkeuring van de notulen}
Er waren geen op- of aanmerkingen.

\subsubsection{Bespreking met Chris Mockel \& Control System}
Voordat we beginnen is het designproces samengevat voor Chris Mockel. Voor de Mid Term hebben we vijf concepten geanalyseerd en zijn we terug gegaan naar \'{e}\'{e}n concept: de stacked toroid. Voor de analyse is de groep opgedeeld in vijf departments: Aerodynamics, Thermodynamics, Structures, Astrodynamics en control. Per department volgt een uitleg. Chris is met name ge\"{i}nteresseerd in de orbit en vraagt of we aerobreaking of aerocapture doen om op Mars te landen. We maken dus gebruik van aerobreaking met een entryvelocity van 7$km\cdot s^{-1}$. Ook vraagt hij wat de peak heat load is. Deze ligt tussen 20-50 $W\cdot cm^{-2}$. Ook de diameter wordt behandeld, waarin naar voren komt dat de groep een onorthodoxe aanpak heeft wat betreft het designproces. Zo zijn eerst complexe tools gemaakt, voordat er benaderingswaardes waren. Deze waardes zouden met simpele berekeningen benaderd moeten worden. Normaal gesproken zou het dus andersom moeten. Bij de aerodynamics maakt CM een opmerking over de validity van Newtonian flow theory bij verschillende Mach-numbers. Ook vraagt hij in hoeverre de vorm te produceren valt, omdat het wordt gedefinieerd door de opblaasbare toroids. Verder vraagt CM of de temperatuur meegenomen wordt in de structural analysis. In principe niet, met als verklaring dat de TPS er voor zorgt dat de structural layers de temperatuur onder de 200 $^{\circ}C$ zal houden.
\newline\newline
Bij de control wordt er waarschijnlijk geen c.g. offset gebruikt om actief te controllen, enkel een initial offset om de $\alpha$ te trimmen. De lift vector wordt gecontroleerd met banking. Volgens CM is dit gevaarlijk door density variations in de atmosfeer van Mars. Ook duurt bank-reversal 30-40sec. Verder vraagt hij of een constant c.g. de beste optie is. Apollo moest namelijk in de ocean landen, onze missie in een range van 500m. Banken zal waarschijnlijk wel werken, maar is misschien niet accuraat genoeg. Hij adviseert ons de feasibility van nieuwe concepten aan te tonen en voor de feasibility van oude concepten te refereren naar papers. HD is benieuwd hoe de atmosfeer varieteit te combineren valt met S-descend (banking). We moeten laten zien dat het werkt als het nog niet eerder is gedaan. Chris Mockel wil graag vragen beantwoorden als we dat willen.
\newline\newline
Bank control heeft op dit moment de voorkeur boven alpha control, omdat bij alpha de hoogte sterk verandert wanneer de lift vector gecontroleerd moet worden. Er was nog even onduidelijkheid over waar het eindpunt is. Het klopt dat dit op 10km hoog en Mach 5 is. Het verschil is dat er een bepaald eindpunt is waar we uiteindelijk terecht moeten komen. Dit punt moeten we met een eventuele afwijking kunnen bereiken met de uitgezette trajectory. De maximale uitwijking mag dan 500m zijn. Uiteindelijk blijkt dat hetzelfde bedoeld werd, maar de woorden van Dawud waren niet goed gekozen. HD vraagt zich af of dit gaat lukken met banking. Nadeel van banking is dat je doel halen lastig wordt. Te snelle draaiing zorgt voor hoge g-krachten, wat dus nadelig is bij banking. Kijk of het feasible is voordat je verder gaat. Dit is iets waar we veel tijd in moeten gaan steken. Zorg voor eventuele back-up plans. Plan A is enkel bank control, plan B bank en alpha control en plan C dan enkel alpha control. Voor de volgende keer is het misschien een idee om wat resultaten laten zien van een trajectory/orbit. HD benadrukt dat er een tweede plan moet zijn, en dat er daadwerkelijk al mensen aan het tweede plan moeten werken.

\subsubsection{Andere vragen}
Volgens de planning is de deadline voor het final report op 30 juni. De vrijdag erna is de grading al, wat dus best snel op elkaar volgt. HD geeft aan het report het liefst voor de final review te hebben (25 juni). \underline{Deadline voor het Final Report wordt gezet op dinsdagavond 23 juni.} Eventuele kleine verbeteringen kunnen dan nog doorgevoerd worden voor 30 juni.

\subsubsection{W.V.T.T.K.}

\textbf{1.} In de DID voor de table of contents staat er een toc-depth van 3 levels. Dit lijkt nogal veel. DD geeft aan dat chapter, section, subsection goed genoeg is (dus toc-depth van 2 levels). HD zegt dat  wanneer er hierdoor problemen met de pagelimit ontstaan er waarschijnlijk ergens anders onnodig teveel pagina's geschreven zijn. 

\subsubsection{Rondvraag}
L vraagt aan CM wat hij precies bij zijn conferentie gaat doen en hoe hij dat gaat aanpakken. CM antwoordt dat hij zal presenteren wat zijn DSE-groep twee jaar geleden heeft gedaan. Hij zal ook vertellen over de design drivers, de gemaakte trade-offs etc. Voor ge\"{i}nteresseerden zegt hij dat de deadline van volgend jaar rond april is. 

\subsubsection{Sluiting}
De volgende vergadering zal plaatsvinden op 10-06-2015 om 11:00 in Fellowship Meeting room 2.
\newline\newline
Dawud sluit de vergadering om 12:01.

\subsection{Status Meeting 12, 10-06-2015 11:00}
\subsubsection{Opening}
Locatie: Fellowship Meeting room 2\\
Aanwezig: HD, NR, DD, J, L, G, A, B, T, D, Su, Se\\
Verontschuldigd: \\
Afwezig: CM \\
Tijd: Dawud opent de vergadering om 11:03\\

\subsubsection{Agenda}
De agenda van de vergadering bestaat oorspronkelijk uit de volgende punten:
\begin{enumerate}
\item Opening
\item Agenda
\item Goedkeuring van de notulen
\item Orbit \& Control
\item Status update
\item Andere vragen
\item W.V.T.T.K.
\item Rondvraag
\item Sluiting
\end{enumerate}

Aan de agenda worden verder geen punten toegevoegd.

\subsubsection{Goedkeuring van de notulen}
Banking was kort volgens NR, maar was niet besproken hoe die ge\"{i}mplementeerd is.
Er waren geen op- of aanmerkingen.

\subsubsection{Orbit \& Control}
Vorige week ook besproken. Vrijdagmiddag afspraak met Mooij. Transient analysis van bank vector wordt verwaarloosd, voor deze fase van design. Zijn we bij een bank angle een bocht aan het maken of slippen? Sliphoek vs. bank angle. G's in bochten is geen sprake van omdat we geen constante hoogte aanhouden. Richting van g verandert, magnitude niet. Verandert niet t.o.v. body frame, maar t.o.v. initial reference frame. Bank angle aanpassen is makkelijker door lagere moment of inertia. Alpha is veel stabieler dan bank (dus ook moeilijker te veranderen). Banking om vehicle minder diep de atmosfeer in te drukken. Alpha voordeel: drag te varieren. nadeel: overshoot in acceleraties. Non-minimum phase, eerst de ene kant voordat je de andere kant opgaat. Heel vatbaar voor atospheric varieteit. Je komt sneller in andere luchtlagen. Probleem door verschillende luchtlagen is dat er minder ruimte voor correctie is. Agility met alphacontrol, of rustiger met bank (duurt langer om in een bepaalde luchtlaag te komen). Tweede fase is ongeveer hetzelfde als eerste fase. Hoogteprofiel aanhouden en vervolgens naar beneden om naar het eindpunt te gaan. Gecontroleerd instabiel is acceptabel. Hoe snel en hoeveel kun je manuoevreren. Effect van beta-change is kleiner dan effect van alpha-change. Snel en veel reageren of minder en gedoseerd reageren. Eerste fase lijkt beta sowieso beter. Hoe stuur je alpha? Cg offset of locked bodyflaps. Waarom wordt alpha zo lastig? Je introduceert extra loads wanneer je dat niet wilt. Het is allebei haalbaar. We vonden beta haalbaarder dan alpha. beta is lichter te veranderen. Alpha control met offset wordt te groot. Mooij heeft nog een paper geschreven over combinatie bank en alpha. Alpha past drag lift, bank verandert alleen liftfactor, subtielere veranderig. Betacontrol niet handig voor laatste stuk, omdat je dan een grote drop aan het eind moet doen. Waarom IRVE alpha-control? Apollo enzo beta control want het maakte niet uit waar je land. Mooij's papers laten zien dat je met bank-control wel degelijk een accurate nominaal pad aanhoudt. Dat is in onze analyse nog niet te zien, omdat die papers een active control hebben. Paper gaat uit van afwijking op de grond, geen data voor afwijking op onze 10km. ESA's atmosfeer model is nauwkeuriger dan Mars-GRAM. Model heeft 10\% nauwkeurigheid. Pas op met beslissingen nemen op basis van papers van anderen. Het is veilig en al gedaan terugvallen werkt niet altijd, omdat het hier misschien niet goed genoeg gaat zijn. Kom met een oplossing die daadwerkelijk kan werken. Actieve pad herberekening met alpha en beta wordt lastig. Aanpassingsmogelijkheden geven limieten. Is nog niet eerder gedaan. Papers zijn dus ook ideeen etc. Wees creatief en niet bestaande dingen opleuken. 500m wordt lastig om aan te tonen. Lastig om precies Mach 5 en 10km te vinden. Je hoeft niet alles in perfect detail uitwerken.

\subsubsection{Status update}
Aero: Sensitivity gedaan hoe alles met elkaar schaalt. Geskewde vorm voor de L/D.

Structures: Forces geanalyseerd. Diktes komen overeen met IRVE. Structurally feasible voor grotere diameters. Voorkeur voor pointed shape voor bending stiffness. 40 graden hoek optimaal. Inflation system, en deployment system

Thermal: Deviations in heat flux sensitivity mass. Sensitivity met diameter bekeken. TPS-mass wordt los bekeken van structural mass Invloed van tijd wordt bekeken. Langs geweest bij Ferry Schrijer. Vragen over contact resistance. We maken nu een ghostlayer waarvan de conductivity wordt aangepast. Evt. lengtes aanpassen.


\subsubsection{Andere vragen}

\subsubsection{W.V.T.T.K.}

\textbf{1.}

\subsubsection{Rondvraag}
In hoeverre zijn er al dingen op Mars. Alles wat je op Mars nodig hebt en voor de weg terug is er. Er is geen Aardse society, dus niet te uitbundig maken.

NR: Kosten analyse, voor volgende week vrijdag.

Se: is klaar met zijn ontwerp vliegtuig.
\subsubsection{Sluiting}
De volgende vergadering zal plaatsvinden op 12-06-2015 om 11:00 in Fellowship Meeting room 2.
\newline\newline
Dawud sluit de vergadering om 12:26.

\subsection{Status Meeting 13, 12-06-2015 09:00}
\subsubsection{Opening}
Locatie: LR Meeting room 2 (locatie gewijzigd)\\
Aanwezig: HD, NR, DD, J, L, G, A, B, T, D, Su, Se\\
Verontschuldigd: \\
Afwezig: CM \\
Tijd: Dawud opent de vergadering om 09:09 (aanvangstijd gewijzigd)\\

\subsubsection{Agenda}
De agenda van de vergadering bestaat oorspronkelijk uit de volgende punten:
\begin{enumerate}
\item Opening
\item Agenda
\item Goedkeuring van de notulen
\item Status update
\item Andere vragen
\item W.V.T.T.K.
\item Rondvraag
\item Sluiting
\end{enumerate}

Aan de agenda worden verder geen punten toegevoegd.

\subsubsection{Goedkeuring van de notulen}
Er waren geen op- of aanmerkingen.

\subsubsection{Status update}
Afgelopen twee dagen zijn we bezig geweest met de subsystems in de crew module en de rest van de missie.
\newline
Voor de launch gaan we eerst naar LEO, waar evt. gedocked wordt. Van LEO naar een transfer orbit. Verder is er gekeken naar launch vibrations, diameter en maximale volume voor in de launcher. Dat wordt inderdaad vijf meter diameter en 10 meter hoog.
\newline
Bij de interplanetary flight is naar verschillende transfer orbits gekeken welke verschillende duur hebben en verschillende deltaV-budget vereisen.
\newline
G vertelt dat voor de heenweg geen habitatcapsule nodig is. Je hebt wel een earth-return vehicle nodig voor de terugweg naar Aarde. Dit betekent minimaal twee launches. De eerste voor de heenweg moet met SLS (Saturn V niet toereikend) om de juiste snelle transfer orbit te kunnen pakken. De tweede kan met een kleinere, omdat de earth-return vehicle niet per se supersnel bij Mars moet aankomen. HD geeft als tip alles zo op te schrijven zodat het zodat het geen extra vragen oproept. Misschien een beetje buiten de scope, maar er moet wel over nagedacht zijn.
\newline
We kunnen bij problemen niet terug naar Aarde. De entry-vehicle wordt in een parking orbit gezet om een goede landingsmogelijkheid te vinden. Eerst een aerocapture en dan ongeveer 100kg aan brandstof om een deltaV te krijgen voor de parking orbit. We weten nog niet precies welke thrusters we hiervoor willen gebruiken.
\newline
Voor de landing (of terminal descent) loont het gebruik van een parachute niet. Het is lichter om thruster fuel mee te nemen (200kg tegen 280kg). De aeroshell is gewoon naar beneden gericht met een kleine parachute (drogue) voor de stabiliteit. De aeroshell wordt afgeworpen. Ook hier is het nog niet duidelijk welke thrusters gebruikt gaan worden. Hier moeten we wel op gaan letten.
\newline
Voor de terugkeer is aangenomen dat er een raket op Mars de astronauten naar het earth-return vehicle, die al om Mars cirkelt.
\newline\newline
Verder zijn we bezig geweest met de subsystems van de crewmodule. We hebben zoveel mogelijk massa en volume toegekend aan ADCS, Power, Structures, C\&DH (communications and data handling), operational items en Thermal control. Veel is uit referenties gehaald. NR vraagt of er subsystems zijn met een te hoge massa. Voroalsnog niet, vandaag zal de packaging van alle subsystems in de cre module plaats vinden.
\newline\newline
Vervolg van de control-opties van woensdag 10 juni:
\newline
Control met vari\"{e}rende offset benodigt hele grote actuators (zeker bij enkel $\alpha$-control). Alleen de pompen in Apollo waren al 120kg/stuk en die konden minder massa verplaatsen. Dit zijn dan alleen de pompen en we moeten onder de 1000 kg blijven. CG offset benodigd bij bank is kleiner, omdat bank neutrally stable is, dus een minder groot moment is nodig. We willen wel een initial offset voor $\alpha$-trim. We hebben nog niet gekeken naar cg-shift door andere massa-verdeling. 
\newline
Bodyflaps werken niet buiten de atmosfeer (netzoals cgoffset). Je hebt dus sowieso thrusters nodig. Voordeel is dat ze niets kosten als ze gelocked zijn. Wel lastig te maken aan het einde van inflatable.
\newline
Thrusters werken buiten atmosfeer en is proven concepts. Nadeel is massaverbruik over tijd. Meestal hydrazine thrusters wat niet fijn is als het misgaat. Deze zitten echter ook op Orion, dus het kan wel gewoon.
\newline\newline
Er is in principe nog niet gekozen voor een control system, maar we neigen naar een skewed shape, fixed c.g. offset met thrusters.
\newline\newline
Er wordt even gepraat over Limit Cycle Oscillation, welke voor ongewenste vibraties kunnen zorgen. Is voor ons vrijwel niet te analyseren

\subsubsection{Andere vragen}
Er waren geen vragen ingediend.

\subsubsection{W.V.T.T.K.}
NR geeft aan sowieso woensdag er niet bij te zijn en vrijdag misschien niet bij kan zijn.

D groep geeft aan vrijdag (19 juni) liever geen vergadering te plannen. HD kon dan sowieso niet, dus komt goed uit. Deze meeting zal verplaatst worden naar maandag 22-06-2015. Liefst zelfde tijd (11:00-12:30)

Voorkant van het final report mag je zelf bepalen, de rest moet in zwart-wit leesbaar zijn.

\subsubsection{Rondvraag}
- HD vertelt dat er woensdag 17 juni een C\&S barbecue is - 

\subsubsection{Sluiting}
De volgende vergadering zal plaatsvinden op 17-06-2015 om 11:00 in Fellowship Meeting room 2.
\newline\newline
Dawud sluit de vergadering om 10:17.