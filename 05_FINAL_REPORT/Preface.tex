\section*{Preface}\label{cha:preface}

\begin{flushright}
	\today
\end{flushright}

Dear Reader,	
\\ [1cm]
A growing interest in manned spaceflight and human exploration of Mars requires a new solution. Conventional, rigid entry solutions require a significant decelerator mass to bring payload to ground. Inflatable concepts offer significant mass and packaging advantages and their application opens up broad venues for interplanetary human spaceflight. 

This Final Report centers around the analysis and design of a controllable inflatable aeroshell that is capable of bringing two crew members to Mars with a mere 10 \% decelerator mass fraction within 100 days. Conceptual and preliminary design have shown the feasibility of a mission with corresponding economical benefits and a reduced ecological footprint by a decreased required number of launches through increased payload-carrying capability.

The authors would like to acknowledge Dr. Ir. H.J. Damveld, Ir. D. Dolkens and Ir. N. Reurings for their guidance and support. [???? ? ? ? ? ? ? ? ? ? ? ? ? ? ? ?] [Herman, Dennis, Niels] + [Externe hulp]
%This Mid-Term Report is part of the deliverables for the \acrfull{mtr}. It describes the development, verification and validation of the computational tools produced to analyse several possible concepts for a controllable inflatable aeroshell designed for manned spaceflight. In addition to analysing these concepts a trade-off will be made between them to determine which is the most promising. This concept will then be further analysed in the coming period. The goal of this project is to develop an inflatable aeroshell system that can be used for atmospheric entry on Mars and is lighter than the current solutions to this problem.
\\ [1.5cm]
Design Synthesis Exercise Group 02
