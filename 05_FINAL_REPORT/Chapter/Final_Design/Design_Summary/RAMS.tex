\acrfull{rams} characteristics are established to address the safety critical functions, redundancy philosophy and expected reliability, availability and maintainability. The four aspects \gls{rams} are discussed hereafter.

\paragraph{Reliability}
Failure modes and the measures taken to increase their reliability are given in Table \ref{tab:relia}. It is key that the system is reliable and thereby the philosophy is that redundant equipment is warranted as long as it is within the mass budgets. To this end, for example inflation and propellant tanks are not redundantly equipped because of the significant mass increase effected thereby. 

On the basis of the failure modes specific to the inflatable design, it can be concluded that it is inherently more susceptible to mechanical failures than conventional rigid (re-)entry vehicles. While inflatables offers a significant decelerator mass decrease and achieve what is incapable of achieving with rigid solutions, reliability is penalized. Part of this increased failure probability is mitigated by increased safety margins in component selection and sizing for reliability.

\paragraph{Availability}
Availability is dominated by the launch window to Mars rather than production time for re-entry vehicles. The launch window occurs once every two years, in excess of production time, hence availability is limited. 

\paragraph{Maintainability}
Maintainability of the vehicles is limited to the on-board operations to be performed by the crew. A fail-safe and safe-life approach to design will reduce scheduled maintenance operations to zero within the 100-day time interval of the mission. Any maintenance operations will be incidental and repair limited to crew capability. To this end, the crew module shall allow for (limited) accessibility of critical parts, like the inflation system, such that crew members can take appropriate actions in case of incidental component failure.

\paragraph{Safety}
Safety is the consequence of reliability, by a highly limited maintainability dimension to the \gls{rams} analysis. As such, the system is roughly as safe as it is reliable: room for failure is limited to the redundancy applied in the system. Re-entry missions have demonstrated themselves inherently risky. The Space Shuttle is the most prominent example, with two out of five flights being a failure and the projected failure rate one out of a hundred flights\footnote{URL:\url{http://66.14.166.45/whitepapers/firewalls/ranum/Personal\%20Observations\%20on\%20the\%20Reliability\%20of\%20the\%20Space\%20Shuttle\%20-\%20Feynman.pdf}. Accessed:19-06-2015}. 