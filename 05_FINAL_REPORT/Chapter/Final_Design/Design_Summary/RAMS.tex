\gls{rams} characteristics are established to address the safety critical functions, redundancy philosophy and expected reliability, availability and maintainability. The four aspects \gls{rams} are discussed hereafter.

\paragraph{Reliability}
It is key that the system is reliable and thereby the philosophy is that redundant equipment is warranted as long as it is within the mass budgets. To this end, for example inflation and propellant tanks are not redundantly equipped because of the significant mass increase effected thereby. Failure modes of profound impact on the mission have been taken up in Table \ref{tab:fml}. These failure modes give an indication of design reliability and therefore serve as a basis for further design efforts to maximise concept reliability.

\begin{table}[H]
\centering
\caption{Overview of entry vehicle failure modes}
\label{tab:fml}
\begin{tabular}{|p{0.02\textwidth}|p{0.3\textwidth}|p{0.6\textwidth}|}
\hline
\multicolumn{1}{|c|}{{\bf \#}} & \multicolumn{1}{c|}{{\bf Failure mode}}                                                              & \multicolumn{1}{c|}{{\bf Effect of failure mode}}                                                                                                                                                                                              \\ \hline
01                             & Bladder puncture                                                                                     & \begin{tabular}[c]{@{}l@{}}Loss of structural rigidity by decreased\\ internal pressure, leading to loss of aerodynamic \\surface area and deceleration capability\end{tabular}                                                                \\ \hline
02                             & Tank leakage                                                                                         & \begin{tabular}[c]{@{}l@{}}Insufficient pressure in blowdown\\ high-pressure system. Leads to inability to inflate\\ toroids if pressure drops too much and thereby to \\loss of structural rigidity and deceleration capability\end{tabular} \\ \hline
03                             & Seam failure                                                                                         & \begin{tabular}[c]{@{}l@{}}De-attachment of inflatable flexible material\\ leads to an unpreserved aerodynamic shape\\and a loss of deceleration capability\end{tabular}                                                                     \\ \hline
04                             & Bladder burst                                                                                        & \begin{tabular}[c]{@{}l@{}}Breaks up structural shape,\\  leading to loss of aerodynamic surface area\\ and deceleration capability\end{tabular}                                                                                              \\ \hline
05                             & \begin{tabular}[c]{@{}l@{}}Faulty inflatable deployment\\ (strap band or valve failure)\end{tabular} & \begin{tabular}[c]{@{}l@{}}Inflatable cannot enter its deployed configuration, \\leading to loss of aerodynamic surface area\\ and deceleration capability\end{tabular}                                                                        \\ \hline
06                             & Sensor misreading                                                                                    & \begin{tabular}[c]{@{}l@{}}Introduces faulty attitude information leading to\\ a loss of control accuracy\end{tabular}                                                                                                                         \\ \hline
07                             & Severe environment conditions                                                                        & \begin{tabular}[c]{@{}l@{}}Induces deviations from nominal trajectory and\\ potentially significantly higher mechanical and\\ aero-thermal loading  in excess of (ultimate) sizing loads,\\ causing potential vehicle failure\end{tabular}                 \\ \hline
08                             & Computer failure                                                                                     & \begin{tabular}[c]{@{}l@{}}Switch to safe-mode computer; if both fail vehicle\\ command is lost\end{tabular}                                                                                                                                   \\ \hline
10                             & Apoareion boost thruster failure                                                                     & \begin{tabular}[c]{@{}l@{}}Inability to enter parking orbit, potentially entering\\ hazardous Martian weather conditions\end{tabular}                                                                                                          \\ \hline
11                             & Retro-propulsion thruster failure                                                                     & \begin{tabular}[c]{@{}l@{}}Loss of halting force in terminal descent phase\\ resulting in a hard landing\end{tabular}                                                                                                                          \\ \hline
%12 & Centerbody heat shield rejection & Potential collision with the inflatable aeroshell and puncture\\ \hline%
13                             & Solar panel failure                                                                                  & Loss of power-generating capability and inability to support on-board systems for a prolonged period                                                                                                                                           \\ \hline
14                             & Life support failure                                                                                 & Inability to support on-board crew, causing mission failure                                                                                                                                                                                    \\ \hline
\end{tabular}
\end{table}

On the basis of the failure modes specific to the inflatable design, it can be concluded that it is inherently more susceptible to mechanical failures than conventional rigid (re-)entry vehicles. While inflatables offers a significant decelerator mass decrease and achieve what is incapable of achieving with rigid solutions, reliability is penalised. Part of this increased failure probability is mitigated by increased safety margins in component selection and sizing for reliability.

\paragraph{Availability}
Availability is dominated by the launch window to Mars rather than production time for re-entry vehicles. The launch window occurs once every two years, in excess of production time, hence availability is limited. 

\paragraph{Maintainability}
Maintainability of the vehicles is limited to the on-board operations to be performed by the crew. A fail-safe and safe-life approach to design will reduce scheduled maintenance operations to zero within the 100-day time interval of the mission. Any maintenance operations will be incidental and repair limited to crew capability. To this end, the crew module shall allow for (limited) accessibility of critical parts, like the inflation system, such that crew members can take appropriate actions in case of incidental component failure.

\paragraph{Safety}
Safety is the consequence of reliability, by a highly limited maintainability dimension to the \gls{rams} analysis. As such, the system is roughly as safe as it is reliable: room for failure is limited to the redundancy applied in the system. Re-entry missions have demonstrated themselves inherently risky. The Space Shuttle is the most prominent example, with two out of five flights being a failure and the projected failure rate one out of a hundred flights\footnote{URL: \url{http://66.14.166.45/whitepapers/firewalls/ranum/Personal\%20Observations\%20on\%20the\%20Reliability\%20of\%20the\%20Space\%20Shuttle\%20-\%20Feynman.pdf}. Accessed:19-06-2015}. 