The mission starts at launch from Earth. A $10 000 kg$ crew module carrying two crew members will be injected into a $89$ day high energy transfer orbit to Mars, where it will arrive with a velocity of $7 kms^{-1}$. Before entering the Martian atmosphere, a Nitrogen inflation system will deploy a stacked toroid inflatable heat shield with a mass of $1000 kg$. This heat shield will be supported by ten toroids made out Zylon fibres and will feature a \gls{tps} consisting of a silicon carbide heat barrier protecting a pyrogel insulator. The vehicle will enter the Martian atmosphere twice. The first entry into the atmosphere will decelerate the vehicle to $4.53 kms^{-1}$ to enter an elliptic orbit around Mars. At the apoareion of this orbit, the vehicle will perform an orbit raising burn to place itself in a parking orbit around Mars. Once final landing checks have been completed in orbit, the vehicle will perform a de-orbit burn and enter the atmosphere for landing. During both atmospheric manoeuvres, the vehicle will control its trajectory using bank angle adjustments. These adjustments allow the vehicle to ensure the terminal descent stage starts within $500m$ of its intended location. After landing using a retro propulsion system and carrying out the mission on Mars, the crew will return to Mars orbit using a pre-placed \gls{mav} and rendezvous with an \gls{erv} already waiting in orbit to take the crew back to Earth. The mission and design parameters are summarized in Tables \ref{tab:MissionPar} through \ref{tab:PropMass}.


\begin{table}
	\centering
	\caption{Global Mission parameters}
	\label{tab:MissionPar}
	\begin{tabular}{|l|l|} \hline
		Mission Duration				             	& 	3.75 Years						\\ \hline 
		Total Launches       							&	2  		   	  					\\ \hline
		Crew size				 						&	2     	  						\\ \hline
		Time on Mars				              	   	&  	2 Years    						\\ \hline
		Estimated mission cost (including development) 	&  	44 Billion U.S Dollars			\\ \hline
	\end{tabular}
\end{table}

\begin{table}
	\centering
	\caption{Hypersonic decelerator parameters}
	\label{tab:DeceleratorPar}
	\begin{tabular}{|l|l|} \hline
		\textbf{Orbital characteristics}             	& 			\\ \hline \hline
		Aerocapture entry velocity       				&	  		\\ \hline
		Aerocapture exit velocity				 		&     	  	\\ \hline
		Aerocapture bank reversals				        &  	   		\\ \hline
		Aerobraking entry velocity					 	&  			\\ \hline
		Aerobraking exit velocity					 	&  			\\ \hline
		Aerobraking bank reversals					 	&  			\\ \hline
		Parking orbit period						 	&  			\\ \hline
		Parking orbit apoareion			 				&  			\\ \hline
		Parking orbit periareion			 			&  			\\ \hline
		\textbf{Aerodynamic Characteristics}			&			\\ \hline \hline
		Trim angle of attack				 			&  			\\ \hline
		Trim \gls{cg} offset							&			\\ \hline
		Lift to Drag ratio at trim			 			&  			\\ \hline
		Drag coefficient at trim			 			&  			\\ \hline
		Moment derivative w.r.t. $\gls{sym:alpha}$		&  			\\ \hline
		Aeroshell Outer diameter						&  			\\ \hline
		Aeroshell Height								&  			\\ \hline
		Aeroshell lengthwise offset						&  			\\ \hline
		\textbf{Structural characteristics}				&			\\ \hline
		Number of toroids					 			&  			\\ \hline \hline
		Toroid material						 			&  			\\ \hline
		Toroid wall thickness (excluding \gls{tps})		&			\\ \hline
		Toroid wall thickness (including \gls{tps})		&			\\ \hline
		\textbf{Thermal protection system characteristics}	&		\\ \hline \hline
		Heat barrier material							&			\\ \hline
		Heat barrier thickness							&			\\ \hline
		Insulator material								&			\\ \hline
		Insulator thickness								&			\\ \hline
			
		
	\end{tabular}
\end{table}

\begin{table}[ht]
	\centering
	\caption{Hypersonic decelerator mass breakdown}
	\label{tab:DeceleratorMass}
	\begin{tabular}{|l|l|} \hline
		\textbf {Hypersonic Decelerator}             & \textbf{Mass $\mathbf{[kg]}$ } \\ \hline \hline
		Structure          &		 275       \\ \hline
		Thermal Protection System &		  270      \\ \hline
		Control System            		   &  212      \\ \hline \hline
		Total excluding contingency              	   &  757     \\ \hline
		\textbf {Total including contingency}                 &  957      \\ \hline
	\end{tabular}
\end{table}

\begin{table}[ht]
	\centering
	\caption{Crew module mass breakdown}
	\label{tab:CrewModuleMass}
	\begin{tabular}{|l|l|} \hline
		\textbf {Crew module}             & \textbf{Mass $\mathbf{[kg]}$ } \\ \hline \hline
		Power        &		 280       \\ \hline
		\gls{adcs} &		  225      \\ \hline
		Thermal control & 600\\ \hline
		Structure & 1300\\ \hline
		Operational items & 3140\\ \hline
		Command \& Data handling & 585 \\ \hline
		Crew & 160 \\ \hline
		Terminal descent system           		   &  1500      \\ \hline \hline
		Total excluding contingency              	   &  7790     \\ \hline
		\textbf {Total including contingency}                 &  9590      \\ \hline
	\end{tabular}
\end{table}

\begin{table}[ht]
	\centering
	\caption{Propellant mass breakdown}
	\label{tab:PropMass}
	\begin{tabular}{|l|l|} \hline
		\textbf {Manoeuvre}             & \textbf{Mass $\mathbf{[kg]}$ } \\ \hline \hline
		Momentum unloading       &		 20       \\ \hline
		Orbit clean up &		  33      \\ \hline
		Atmospheric control           		   &  54      \\ \hline 
		Parking orbit/de-orbit            	   & 54    \\ \hline
		Landing            	   &  930     \\ \hline \hline
		\textbf {Total}                 &  1091      \\ \hline
	\end{tabular}
\end{table}
