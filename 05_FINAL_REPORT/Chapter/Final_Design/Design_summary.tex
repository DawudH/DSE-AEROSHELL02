\subsubsection{Requirement compliance matrix} \label{sec:ComMat}

Table \ref{tab:compm} and \ref{tab:compv} present the compliance matrix for the mission and vehicle requirements as defined top level. It can be noted that all requirements are met. For some of the requirements however no explicit values can be named. Nevertheless all required can be argued to be met. This argumentations is provided in the paragraphs below:

\begin{table}[h]
\centering
	\caption{Mission requirements compliance matrix} 
	\label{tab:compm}
\begin{tabular}{|p{0.12\textwidth}|p{0.65\textwidth}|c|}
    \hline
    ID          & Description   &                                                                                    \\ \hline \hline
    CIA-M01& The re-entry vehicle shall decelerate from a velocity of 7 $[km\cdot s ^{-1}]$ to Mach 5 $[-]$   & \cmark \\ \hline
    CIA-M02 & The re-entry vehicle shall not exert an acceleration greater than 29.4 $[m \cdot s^{2}]$ on any crew member for the duration of the mission	& \cmark \footnote{Under non-nominal trajectories temporarily higher loads may be experienced}		\\ \hline
    	CIA-M03 & The re-entry vehicle shall attain its final velocity at an altitude of 15 000 [m] \gls{mola}  & \cmark \\ \hline
    	CIA-M04 & The re-entry vehicle shall reach its final position with a precision of 500 [m]  & \cmark \\ \hline
    	CIA-M05 & The re-entry vehicle shall attain its final velocity within 10 days of mission start & \cmark \\ \hline

    \end{tabular}
\end{table}

\begin{table}[h]
\centering
	\caption{Re-entry vehicle requirements compliance matrix} 
	\label{tab:compv}
	\begin{tabular}{|p{0.12\textwidth}|p{0.65\textwidth}|c|c|}
	    \hline
	    ID          & Description   & Value &                                                                                           \\ \hline \hline
	CIA-R01 & The re-entry vehicle shall have an undeployed diameter smaller than 5 [m]                   & 4.5-5.0 [$m$]  & \cmark     				            \\ \hline
	CIA-R02 & The re-entry vehicle shall have a deployed diameter smaller than 12 [m]                     & 12 [$m$] &  \cmark 				            \\ \hline	
	CIA-R03 & The re-entry vehicle shall have a mass of 10000 [kg] at the start of the re-entry           & 10 000 [$kg$] &  \cmark          				            \\ \hline
	CIA-R04 & The hypersonic decelerator shall have a mass fraction of no greater than 10\% of the vehicle mass	& 946 [$kg$] & \cmark \\ \hline 
	CIA-R05 &  The re-entry vehicle shall adhere to the \gls{cospar} regulations  & - & \cmark \\ \hline
	CIA-R06 &  The re-entry vehicle shall have control system accuracy of at least $5\cdot 10^{-4}$ & - & \cmark \\ \hline
    \end{tabular}
\end{table}


\paragraph{Mission requirements}

\begin{itemize}
\item[CIA-M01]	The re-entry vehicle has been sized for a entry velocity of 7[$km \cdot s^{-1}$] and final Mach number of 5. No adjustments were required to these values to meet the other requirements and as such these values has been adhered to. 
\item[CIA-M02]	The trajectories have been sized for peak accelerations of 29.4 $[m \cdot s^{2}]$. Within the Ma
\item[CIA-M03]
\item[CIA-M04]
\item[CIA-M05]

\end{itemize}

\paragraph{Re-entry vehicle requirements}

\begin{itemize}
\item[CIA-R01] Special care has been taken that the deployed diameter remains below the 5[$m$] limit. As such the undeployed outer diameter was constraint to 4.5[$m$]. The sole exception herupon is the inflatable structure with the accompanying hold down and release system. This will add slightly in diameter but is merely constraint by the how tight the inflatable is folded. This should fit easily within the 0.25[m] remaining margin on either side considering the thinness of the inflatable structure. 
\item[CIA-R02]
\item[CIA-R03]
\item[CIA-R04]
\item[CIA-R05]
\item[CIA-R06]
\end{itemize}










