The inflatable consists of ten toroids, stacked aside and on top of one another. The asymmetric shape obtained by aerodynamic optimization is attained by arranging the toroids at an angle with respect to one another. The result is an assembly of circular inflatables, placed at differing radial distances with respect to the centerbody. While the structural performance of the inflatable is altered, an asymmetric configuration is achieved by stitching the toroids and varying the radial length of the straps over the sphere cone circumference. Structural performance is altered in the sense that the asymmetry of the configuration implies additional concerns for aeroelasticity phenomena, such as limit cycle oscillations, for example. These phenomena, however, are highly unpredictable and warrant additional wind tunnel and flight testing in any case. 

Stitching of the fabrics making up the toroids is used for the joints of the inflatable, on one hand to join the toroids to each other and on the other hand to join the toroids to the radial straps. This is a method excellently suited, applied, tested and proven in the \gls{irve} missions \cite{Lindell2006,Hughes2011,Dillman2012}. Joints are thereby proven high-strength and suitable for space application and a stacked-toroid configuration.




%Structural integrity is provided by PBO Zylon AS, capable of retaining its strength at high temperature and able to withstand the required loads\footnote{URL:\url{http://www.toyobo-global.com/seihin/kc/pbo/zylon-p/bussei-p/technical.pdf}. Accessed: 15-06-2015}}. It has high specific properties, leading to a low structural mass, as follows from Figure \ref{fig:mat}. It performs only slightly worse than Spectra 2000 in the parametric mass model, but differences are slight (below 5 $\%$) and 



Structural stiffness of the toroids is provided by PBO Zylon