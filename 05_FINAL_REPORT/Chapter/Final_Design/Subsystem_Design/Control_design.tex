\paragraph{Control system selection}
Based on the arguments for and against certain control system concepts given in section \ref{subsec:controltool} a selection of suitable control system solutions was made. Figure \ref{fig:cgoffset} shows the \gls{cg} offset required along the Z-axis to trim the spacecraft at certain angles of attack for various \gls{cg} locations on the X-axis. 
\begin{figure}[h]
	\centering
	\includegraphics[width=0.95\textwidth]{./Figure/control/moment}
	\caption{\gls{cg} offset along Z-axis required for a trimmed condition at various angles of attack}
	\label{fig:cgoffset}
\end{figure}
From figure \ref{fig:cgoffset} it can be seen that the required \gls{cg} Z-offset grows as the X-offset grows. For an angle of attack change of $2$ degrees corresponding to an X-\gls{cg} located at $-5$ meters a \gls{cg} shift of $0.2$ meters is required. Angle of attack-based trajectory control was found to require trimmed \gls{sym:alpha}-shifts of $5$ degrees that have to be adjusted with a rate of $1$ $deg \cdot s^{-1}$. To pull this off would require the actuation system to produce a \gls{cg} displacement of $0.5$ meters with a required rate of $0.1$ $m \cdot s^{-1}$. This would require excessively heavy actuators that would also have to be able to operate under 3-g loads. Not only has this never been done before in space, the reliability of such a system would be questionable. Based on these arguments a decision was made against active \gls{cg} control.\\
Following the decision to discontinue the consideration of active \gls{cg} control a selection had to be made between the other two control system design options: Body flaps and thrusters. Regarding body flaps some of the same arguments can be made as were used against active \gls{cg} control. Body flaps can require excessively large actuators that are very heavy. In addition to this the dynamic behaviour of the inflatable structure is difficult to compute and can be very unpredictable. Furthermore the use of body flaps on inflatable structures has a very low \acrlong{trl}, which poses an additional development risk. Based on these and other downsides pertaining to aerodynamic control surfaces mentioned in section \ref{subsec:controltool} it was decided to employ thrusters as control system.

\paragraph{Control system components}

The components of the control system can  globally be subdivided into two components. The control system mass and its corresponding accuracy.

Mass estimates are based on the required propellant mass, thruster mass and fuel tank mass. The propellant mass can be subdivided into a further two categories: the control within the atmosphere and the control outside of the atmosphere. General equation were previously discussed within section \ref{subsec:controltool}. An overview of the mass components and their respective weights is provided in Table \ref{tab:controlmassbreakdown}.

\subparagraph{Control within the atmosphere} 

Within the Martian atmosphere control is performed on the basis of banking. A control system featuring a single bank control reversal manoeuvre is always able to derive at the destination with a average accuracy of 1.009 [$km$] at Mars\cite{Lu2007}. The control system accuracy can further be significantly reduced by using multiple bank reversals. Reductions are primarily found in the observed dispersions.

Accuracies using bank control where obtained using dispersions with a \gls{sym:CL} of $\pm 0.03 $, \gls{sym:CD}  of $\pm 0.06 $ and mass and atmospheric dispersion of 5\% and 30\% respectively \cite{Lu2007}. Accuracies of up to 10[$m$] can be achieved if the staging and final descent are included \cite{Davis2010}. 

It is argued that with an increasing amount of bank reversals, complemented with additional control measures higher accuracies can be obtained. The trajectories are budgeted for a total of twelve bank reversals, six for both the initial aero-capture and the final \gls{edl}. Six bank reversals are typical values for single orbit \cite{Lu2007, Cianciolo2010}. A very qualitative definition of six bank reversals as defined within the control system analysis is displayed in Figure \ref{fig:bankdef}

\begin{figure}[h]
	\centering
	\includegraphics[width=0.8\textwidth]{./Figure/control/bankdef.eps}
	\caption{Qualitative figure displaying six bank reversals}
	\label{fig:bankdef}
\end{figure}

The mass estimates of the propellant within the Martian atmosphere are based on peak rotational rates of 20 [$deg\cdot s^{-1}$] and 5 [$deg \cdot s^{-2}$] as used by Davis et al. \cite{Davis2010}.

The inertial moments are based on a homogeneous mass distribution and a simplified geometric shape. The crew module is assumed to be a hollow cylinder with the structural components attached to the in and outside of this cylinder. The inflatable structure is assumed to be of a circular disk shape, again with a homogeneous mass distribution. Within this shape the mass is assumed to be primarily situated on the outside of the spacecraft such that conservative mass estimates will be achieved.

\subparagraph{Control outside the atmosphere}
Control outside the Martian atmosphere is required fro two purposes. Clean-up corrections and orbit (de)-raising. The latter allows for a controlled entry time into the Martian atmosphere for the final \gls{edl} operations. The former makes sure that the desired orbit can be reached after the aerobraking.

For the purpose of orbit raising it is desired that the consequential orbit no longer covers the Martian atmosphere and that the orbital period fits within a Martian SOL. For practical purposes and considering the relatively short period in space (i.e days) after the initial aero braking the pericenter limit is set at 200 [$km$]. 

Clean-up corrections are estimated on the basis of results presented by Cianciolo et al. \cite{Cianciolo2010}. The most representative shapes are the 23m \gls{hiad} and in lesser extend the rigid aeroshell. On the basis of the former the clean-up velocity are estimated to be 10.47[$m\cdot s^{-1}$] ($3\sigma$) \cite[p.37]{Cianciolo2010}. Not that Cianciolo et al. include the orbit raising within the clean-up estimates whereas this report considers them as two different entities.


\subparagraph{Thrusters}
Inertial moments combined with rotational rates deliver the required control moments via Equation \ref{eq:mcontrol}. For the most efficient performance the thruster are placed on the outside of the crew module. Although thrusters placed on the outside of the inflatable are able to generate higher torques, multiple disadvantages hinder this placement:

\begin{itemize}
\item Thruster placed on the inflatable will difficult the deployment
\item Deformation of the inflatable, and thus the thrusters performance is difficult to predict.
\item Placement of thrusters on the inflatable may induce disadvantageous vibrations or aeroelastic effects.  
\end{itemize} 

Further details on the placement are also discussed in Chapter \ref{sec:crewpackaging}.

Thruster performance requirements are primarily based on the bank control speed. Apocenter velocity increments may be slower but are however greater in magnitude. The thrusters used for creating the bank control moments require a peak thrust of around 900 [$N$]. Torque is provided by multiple thrusters such that partial operations may continue if a single thrusters fails.
 Indicative values are for a capable thruster are for example given by the  MONARC-445 hydrazine thruster\footnote{URL: \url{http://www.moog.com/literature/Space\_Defense/Spacecraft/Propulsion/Monopropellant\_Thrusters\_Rev\_0613.pdf} Accessed 15 June 2015}. The MONARC-445 thruster delivers a nominal thrust of 445[$N$] at a weight of 1.6[$kg$].  A eightfold of these thrusters allows for control around the roll axis and moreover provides partial control in the case of failure of one such thruster.
 
Specific preference lies in the use of hydrazine as propellant such that it is interchangeable with the remaining propellent requiring systems. The use of a single propellant allows for lower fuel fractions as propellants margins required for the different mission phases can be combined.

Additionally thrusters for the velocity increments in the apocenter are required. Again hydrazine thrusters are considered for interchangeability throughout the various mission stages. This is however combined with a second propellent as bi-propellant thrusters yield significant performance increases (in terms of \gls{sym:Isp})\cite{Wertz2011}. 

A thruster suitable for such a purpose is the Apogee kick engine by Japan IHI at a weight of 15.7 [$kg$] and a specific impulse of 321.4 [$s$]. As secondary propellant NTO is required \cite[p.538]{Wertz2011}. This thruster can only be used if placed aft of the re-entry vehicle, since the front is covered by the \gls{hiad} additional pointing by the crewmodule ADCS system may be required. To ensure a sufficient control system reliability two of these thrusters are considered such that no \acrfull{spf} can occur. 

\begin{table}[H]
\caption{Overview of thruster properties}
\label{tab:thrusters}
\hspace{-5mm}
\begin{tabular}{|p{0.15\textwidth}|p{0.15\textwidth}|l|l|l|p{0.16\textwidth}|p{0.11\textwidth}|l|} \hline 
\textbf{Engine    }          &\textbf{ Manufacturer }         & \textbf{Qt.} &\textbf{ Mass }      & \textbf{Length } & \textbf{Propellants }  & \textbf{Nominal Thrust} & \textbf{\gls{sym:Isp}} \\ \hline \hline
MONARC 445          & MOOG                  & 8        & 1.6[$kg$]  & 0.41[m] & Hydrazine     & 445[N]         & 321.4[s]      \\ \hline
Apogee kick engine & Japan IHI company ltd & 2        & 15.7[$kg$] & 1.03[m] & NTO/~~~~~ Hydrazine & 1700[N]        & 235[$s$]     \\ \hline
\end{tabular}
\end{table}


\subparagraph{Propellant tank}

One of the main arguments for the use of Hydrazine as the primary propellant is the ability to combine the propellent budgets for multiple mission phases as previously mentioned. This allows for a lower control system mass fractions as for an equal control system reliability. Nevertheless a mass estimate for the propellent tank is provided to yield a fair mass estimate for the \gls{hiad} design. 

The tank mass is estimated via the empirical Equation \ref{eq:tankmass} \cite[p. 543]{Wertz2011}.

\begin{equation}
\label{eq:tankmass} 
\gls{sym:m}_{Tank}=2.7086 \cdot 10^{-8} \cdot V^3 -6.1703*10^{-5} \cdot V^2 +6.66290 \cdot 10^{-2}  \cdot V +1.3192;
\end{equation}

\subparagraph{Component mass overview}

Table \ref{tab:controlmassbreakdown} provides an overview of the individual mass components discussed above. Mass estimates exclude the final contingency factor of 20 \% applicable for all the \gls{hiad} components.

\begin{table}[h]
\centering
\caption{Control system mass components}
\label{tab:controlmassbreakdown}
\begin{tabular}{|l|c|} \hline
\textbf{Component}                 & \textbf{Mass[$kg$]} \\ \hline
Bank Control Propellant      &     54       \\ \hline
Clean-up Propellant          &     33       \\ \hline
Orbit (de)raising Propellant &     54       \\ \hline
Fuel tank              		 &     11      \\ \hline
Thrusters                	 &     44     \\ \hline \hline
Total                        &     196      \\ \hline
\end{tabular}
\end{table}

\paragraph{Control system method}

?? Mogelijk iets over predictor corrector schemes




