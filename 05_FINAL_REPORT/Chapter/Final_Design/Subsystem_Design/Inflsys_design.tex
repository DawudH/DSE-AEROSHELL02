The inflation gas is key to providing the structural stiffness for the inflatable: it is required to bring all members into tension to prevent skin wrinkling under compressive loading. To this end, the aeroshell will require an inflation system that is reliable, lightweight and fitting within mass and volume constraints. This subsection details the selection of an inflation gas and design of the inflation system upon which aeroshell deceleration capability hinges.

\subsubsection{Gas generator}
Inflation systems can be categorized as tanked-gas systems, phase-change systems and chemical gas-generation systems \cite{Jenkins2001}. These systems have each been considered for their respective advantages, yielding a tanked nitrogen inflation system as outcome. 

%\paragraph{Phase-change system}
Phase-change systems have the potential to provide significant weight reductions. The most promising option is a liquid hydrogen inflation system, while other phase-change systems involve subliming powders, although these are uncapable of achieving high pressures \cite{Freeland1998}.  On the basis of system mass fractions investigated by Brown et al. \cite{Brown2009} and the mass estimation tool detailed in section \ref{subsec:structool}, a structural mass reduction of nearly 20 [$kg$] is deemed feasible with a cryogenic liquid hydrogen inflation system following from the mass estimation tool formulated in section \ref{subsec:structool}. 

This mass reduction comes at the expense of reliability, however. These systems involve a phase-change process, inherently unpredictable and thereby accompanied with reduced inflation system reliability \cite{Jenkins2001}. In addition, cryogenic storage requires profound thermal control to keep it below its required temperature. While this poses a challenge for orbiting satellites, it is even more so an issue in the heated re-entry environment of the inflatable aeroshell. Reliability is further lessened by the absence of succesful efforts in the past to accommodate a phase-change inflation system in spaceflight, let alone a high-pressure application like the aeroshell at hand. As reliability is key for transporting human payload, phase-change systems are deemed ill-suited. Moreover, a liquid hydrogen inflation system poses issues for safety when operating in the Earth atmosphere, in which flammability risk is present by the dual presence of hydrogen and oxygen in a heated environment.

%\paragraph{Chemical gas-generation system
Chemical gas-generation systems similarly feature a higher level of complexity and thereby lower level of reliability than tanked-gas systems \cite{Jenkins2001}. Moreover, while weight reductions are deemed feasible, these involve the use of hydrazine \cite{Jenkins2001, Freeland1998}. Hydrazine poses issues with respect to cost and handling, but most importantly with respect to sustainability. As the decelerator will make contact with a hard surface, leakage of hydrazine into the Martian atmosphere and pollution of the landing site by its toxic nature poses a risk. This risk would violate \gls{cospar} regulations and moreover limit the sustainable dimension of the mission.

Tanked-gas systems are the preferred choice, featuring a significantly higher level of reliability and past application. Most notably, these have seen application in the \gls{irve} missions in the form of a nitrogen blow-down system \cite{Smith2010}. Blow-down systems offer controllable gas flow at low development and hardware cost \cite{Freeland1998}. Moreover, these are excellently suited for high-pressure applications in inflatable structures \cite{Jenkins2001}.








