The design of the entry vehicle requires an analysis of the aerodynamic properties of the vehicle. Although high fidelity solutions which describe the entire flow field around the vehicle exist, these are prohibitively expensive in both runtime and computational resources for the design study at hand. A low fidelity tool has been developed to allow for rapid design iterations. 

\paragraph{Input and Output}
For a given external shape, the aerodynamic analysis tool provides aerodynamic lift, drag and moment coefficients for ranges of angles of attack and angles of sideslip. This is used in the trajectory analysis and the stability \& control analysis of the entry vehicle. It also calculates the heat flux in the stagnation point for a given flight condition and vehicle shape. The heat flux is required for the analysis of the \gls{tps}. 

\paragraph{Analysis method}
The aerodynamic analysis is based on modified Newtonian flow theory. This theory relates the pressure coefficient on a given surface \gls{sym:CP} with the incidence angle \gls{sym:chi} this surface has with respect to the freestream. The equation for pressure coefficient is given in Equation \ref{eq:modnewtoniancp}, while the maximum pressure coefficient can be calculated using Equation \ref{eq:cpmaxfinal}  \cite{AndersonJr.2006}.

\begin{multicols}{2}
	\begin{equation}
		\gls{sym:CP}=\gls{sym:CP}_{,max}sin^{2}(\gls{sym:chi})
		\label{eq:modnewtoniancp}
	\end{equation} \break
	\begin{equation}
		\gls{sym:CP}_{,max} = \frac{2}{\gls{sym:kappa} \gls{sym:M}_{\infty}^{2}}\left(\frac{\gls{sym:p}_{O_{2}}}{\gls{sym:p}_{\infty}}-1\right)
		\label{eq:cpmaxfinal}
	\end{equation}
\end{multicols}

These pressure coefficients can then be integrated to find the force and moment coefficients acting on the vehicle. The local change in static pressure due to the aerodynamic effects can be found by multiplying \gls{sym:CP} by the dynamic pressure $\gls{sym:q}=\frac{1}{2}\rho_{\infty}V_{\infty}^{2}$. This method provides reasonable accuracy in determining the pressure coefficient distribution over blunt bodies for a low computational cost. It is therefore well suited for initial design studies such as the one performed in this report \cite{AndersonJr.2006}.

The heat flux in the stagnation point is calculated using the method developed by Tauber et al. \cite{Tauber1986}. Equation \ref{eq:modnewtonianqw} gives the heat flux in the stagnation point. This equation uses the ratio between the wall temperature and the temperature in the stagnation point in the flow, which can be calculated using Equation \ref{eq:stagnationtemperature}\cite{AndersonJr.2006}.


%\begin{multicols}{2}
	\begin{equation} \label{eq:modnewtonianqw}
		\gls{sym:qdot}_{s} = 1.83 \times 10^{-8} \gls{sym:rho}_{\infty}^{0.5} \gls{sym:V}_{\infty}^3  \gls{sym:rcurvature}^{-0.5} \left(1-\frac{\gls{sym:T}_{w}}{\gls{sym:T}_0}\right)
	\end{equation} \break
	\begin{equation}
		\gls{sym:T}_0 = \gls{sym:T}_\infty \frac{\gls{sym:kappa}-1}{2}\gls{sym:M}_\infty
		\label{eq:stagnationtemperature}
	\end{equation}
%\end{multicols}
	


\paragraph{Limitations}
The modified Newtonian flow method is more accurate for high incidence angles with respect to the flow \cite{AndersonJr.2006}. As described in Chapter \ref{cha:conceptselection}, the body to be analysed is a blunt body, which limits the impact of this loss of accuracy since the majority of the body is at a high incidence angle to the flow. The method will not produce accurate results below a Mach number of 5, since at lower Mach numbers the forces on the entry vehicle will no longer be dominated by pressure drag. This will invalidate the modified Newtonian theory \cite{AndersonJr.2006}.
Since the part of the mission that is analysed in-depth in this report ends at a Mach number of 5, the analysis will not be influenced.

\paragraph{Optimisation} \label{par:Optimisation}
An optimisation algorithm is implemented that allows for a single or multiple objective shape optimisation. To this end, the aerodynamic shape is parametrised to allow optimisation using genetic algorithms as implemented in MATLAB. This parametrisation is done by choosing the coefficients of a polynomial such that it represents the external shape of the \gls{cia}. This polynomial is then revolved around an axis to obtain the 3D shape of the \gls{cia}. Furthermore, the height and skewness are optimisation parameters as well. The genetic algorithm searches for a minimum of a certain function, which can be chosen to be an aerodynamic performance parameter such as the drag or moment. It tries different combinations of coefficients of the polynomial, assesses their performance according to the objective and combines the best performing specimens into even better specimens. Furthermore constraints can be given, such as a requirement on lift-to-drag ratio or static stability. Optimisation can then be used to efficiently search the multi-dimensional design space for the global optimum, given constraints and one or multiple objectives such as a maximum drag or minimum heat flux.

\paragraph{Concluding remarks}
The aerodynamic analysis is capable of calculating the pressure distribution on the surface, the lift, drag and moment coefficients of an arbitrary body as well as their derivatives with respect to angle of attack and sideslip. It is also capable of calculating the heat flux in the stagnation point. Verification and validation have been performed to ensure the consistency and accuracy of the method. Details on this can be found in Appendix \ref{sec:VandVaero}.