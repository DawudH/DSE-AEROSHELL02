\paragraph{Input and output}
As input the tool requires the entry velocity, flight path angle at the boundary of the atmosphere, an aerodynamic model (\gls{sym:CL} and \gls{sym:CD} as a function of \gls{sym:alpha}), an \gls{sym:alpha} profile (changes in the angle of attack during the aerocapture and re-entry), and a \gls{sym:mu} profile (changes in the bank angle during the aerocapture and re-entry).

As output the trajectory tool can generate important parameters at each moment in time. The most important parameters are: acceleration $\left(\gls{sym:acc}\right)$, dynamic pressure $\left(\gls{sym:q}_{\infty}\right)$, velocity $\left(\gls{sym:Vv}\right)$, Mach number $\left(\gls{sym:M}\right)$, atmospheric temperature $\left(\gls{sym:T}_{\infty}\right)$, atmospheric density $\left(\gls{sym:rho}_{\infty}\right)$.


\paragraph{Assumptions}
 \label{sec:astroassumption}
 Some of the assumptions have a big impact, these are the primary assumptions. There are, however, also some assumptions that have a negligible effect on the results. These are the secondary assumptions.
 
 \subparagraph{primary assumptions}
 \begin{itemize}
 \item All atmospheric properties only vary with the height above \gls{mola} and not with longitude, latitude or time. 
 \item All trajectories are assumed to only occur in the equatorial plane. This means that the latitude is always $0 \left[deg\right]$. Changing the latitude will have a big impact on the relative speed of the Martian atmosphere.
 \item The gravitational pull is assumed to only vary with the height above \gls{mola}. The gravitational field of Mars is however not uniform over longitude and latitude, this will induce errors in the trajectory as gravity is one of the major forces in the analysis.
 \end{itemize}

 \subparagraph{secondary assumptions}
 \begin{itemize}
 \item The spacecraft is assumed to only feel a gravitational pull from Mars. It is thus assumed that there is no gravitational pull from the sun, any other planet or the Martian moons.
 \item The atmosphere stops at a height of 400 $\left[km\right]$. At this point the atmosphere is negligibly thin, expanding the atmospheric model would not contribute to the results.
 \item The effect of other disturbances i.e. solar radiation is neglected.
 \end{itemize}

\paragraph{Analysis method}


\paragraph{Limitations}


\paragraph{Concluding remarks}


