In this section the methods used for determining appropriate control systems are explained. These systems should be able to keep the spacecraft on the trajectory as defined by the astrodynamics tool in section \ref{subsec:orbittool}. First the assumptions used and their effects on the accuracy of the analysis are explained in section \ref{control:assumptions}. Than, in section \ref{control:trim}, a point is determined in which the center of gravity should lie in order for the spacecraft to be trimmed at an angle of attack of *** as determined in section  \ref{subsec:orbittool}. The stability of the spacecraft is anaysed in section \ref{control:stab}. In section \ref{control:system} the available control systems that can be used to follow the chosen angle of attack and bank angle profiles are weighed off.

\paragraph{Assumptions}
\label{control:assumptions}
**Primary and Secondary assumptions**

\paragraph{Trim point}
\label{control:trim}
**Moment equilibrium figures/equations**
**CG-location plot(s)  with conclusion on CG for AoA~20 and sideslip angle=0**

\paragraph{Stability}
\label{control:stab}
**From E.Mooij**

\paragraph{Available control systems}

**Intro**

\subparagraph{\acrfull{cg} offset}

**Not nessesary for AoA, not feasible for sideslip**

\subparagraph{Thrusters}

**Sebstiaan**

\subparagraph{Aerodynamic surfaces}

**Guido**



