\paragraph{Inflatable structural mass}

Based on the mass estimation model outlined in subsection \ref{subsec:structool}, the effect of changing design parameters on inflatable structural mass is investigated hereafter. To this end, the following design parameters have been investigated: centerbody and inflated diameter, half-cone angle, the number of toroids and aerodynamic loading.

\begin{figure}[h]
	\centering

	\begin{subfigure}[b]{0.49\textwidth}
		\includegraphics[width=0.96\textwidth]{./Figure/Structure/diameters_test.eps}
		\caption{Mass versus centerbody and deployed diameter}
		\label{fig:diameters_strucmass}
	\end{subfigure}
	\begin{subfigure}[b]{0.49\textwidth}
		\includegraphics[width=0.96\textwidth]{./Figure/Structure/halfcone_test.eps}
		\caption{Mass versus half-cone angle and peak dynamic pressure}
		\label{fig:halfcone_strucmass}
	\end{subfigure}
	\begin{subfigure}[b]{0.49\textwidth}
		\includegraphics[width=0.96\textwidth]{./Figure/Structure/pressure_test.eps}
		\caption{Mass versus deployed diameter and peak dynamic pressure}
		\label{fig:press_strucmass}
	\end{subfigure}
	\begin{subfigure}[b]{0.49\textwidth}
		\includegraphics[width=0.96\textwidth]{./Figure/Structure/inflation_test.eps}
		\caption{Inflation pressure versus number of toroids and deployed diameter}
		\label{fig:inflpress_strucmass}
	\end{subfigure}
\caption{Inflatable structural mass and inflation pressure as a function of design parameters}
\end{figure}
Firstly, from Figure \ref{fig:diameters_strucmass} it follows that mass decreases with an increasing centerbody diameter given a deployed diameter. This is due to the fact that an increasing centerbody diameter increases the areal contribution of the centerbody: the inflatable requires less structural mass by decreased aerodynamic loading thereof, as aerodynamic pressure works over an area. In turn, this suggests that the centerbody becomes heavier, which is not the case as the centerbody is typically sized for launch rather than (re-)entry loads \cite{Lindell2006}. It can therefore be concluded that maximizing centerbody diameter is beneficial for structural mass. 

Secondly, from Figure \ref{fig:press_strucmass} it follows that increasing dynamic pressure effects an increase in structural mass of the inflatable. This is the result of an increased aerodynamic loading and therefore structural taxation of the inflatable. To withstand this loading, extra structural mass is required. Moreover, for a given peak dynamic pressure an increase in deployed diameter effects an increase in structural mass. Primary cause hereof is the fact that pressure works over a surface area and an increase in area thereby increases the loading. This is further amplified by an increase in bending moments by the larger distance from tip to root.

[MASS VERSUS DYN]

From Figure \ref{fig:halfcone_strucmass} it may be observed that the half-cone angle significantly affects inflatable structural mass: in general smaller half-cone angles are preferable. Increasing half-cone angle beyond an optimum region at approximately 45 degrees strongly increases structural mass; decreasing it below this region similarly increases structural mass, but less strongly. Moreover, as aerodynamic loading is increased the optimum region shifts and smaller half-cone angles are preferable. This is due to the fact that decreasing the half-cone angle increases bending stiffness by an increased moment of inertia in the bending plane. This increased bending stiffness is further amplified by the three-dimensionality of the sphere cone and carries over to more effective use of material in bending, requiring less mass to resist the bending moment by aerodynamic loading. For a given deployed diameter, however, decreasing the half-cone angle increases the effective inflatable length. This addition of material is to be traded off against the increased bending material. At low dynamic pressures, increased bending stiffness is less warranted than at higher pressures, at which bending loads increase and bending stiffness is increasingly more warranted.

In Figure \ref{fig:inflpress_strucmass} inflation gas pressure is observed to increase for an increasing number of toroids and to decrease with an increasing deployed diameter. Both an increase in the number of toroids and a decrease in deployed diameter decrease toroid radii, effecting an increase in the working area of the inflation pressure. Due to the proportionality of the running load induced by inflation pressure via Equation \ref{eq:inflationpressure} with toroid radius, a larger inflation pressure is required to induce the same running load with a smaller radius. This running load is based on the consideration that the work done by inflation gas and external forces in axial direction are equal \cite{Brown2009}, independent of the number of toroids. It is similarly independent of the deployed diameter, since both inflation and aerodynamic pressure have the same working area in axial direction.

Material selection has a significant effect on inflatable structural mass, as illustrated by Figure \ref{fig:mat}. Materials with a higher specific strength perform better in terms of structural mass. Flexible material is fully loaded in tension by the inflation pressure is required not to fail under tension, dictated by ultimate strength. To this end, a certain thickness with a corresponding mass is required. Mass performance is directly linked to specific strength and this is confirmed by Figure \ref{fig:mat}. Aramid fibers Kevlar and Technora have the lowest specific strenghts, approximately 2 [$MNm/kg$]. A notably higher specific strength of 3.44 and 3.77 [$MNm/kg$] is attained by Spectra 2000 and PBO Zylon respectively. This confirms the choice for PBO Zylon for its weight advantages over Kevlar in \gls{irve}-3 \cite{Dillman2012a}. Spectra 2000 is capable of achieving a lower mass than PBO Zylon despite its lower specific strength, due to its low density. At low loads, a certain minimum thickness limits the mass-reducing capability. Below this minimum thickness, performance is dictated by density rather than specific strength. Due to the significantly lower density of Spectra 2000, 970 [$kg/m^{3}$], versus that of PBO Zylon, 1540 [$kg/m^{3}$], Spectra 2000 offers weight advantages at low dynamic pressures. All materials are selected based on their operating temperature since these are required to operate in an environment with significant thermal loading. A summary of material properties is given in the Mid-Term Report \cite[p.64]{Balasooriyan2015b}.

%The thickness required increases with decreasing material ultimate strength, which explains the differences in structural mass observed between different materials at low dynamic pressures. These differences carry through as loading is increased, where materials with a higher specific strength require less mass to withstand the loading. The material used in \gls{irve}-3, PBO Zylon, is observed to offer notable weight advantages over heritage aramid fibers, such as Kevlar. Weight reduction beyond this level is possible by using Technora and Spectra 2000. All materials are selected based on their operating temperature since these are required to operate in an environment with significant thermal loading. A summary of material properties is given in the Mid-Term Report \cite{Balasooriyan2015b}.
\begin{figure}[h]
	\centering
	\includegraphics[width=0.6\textwidth]{./Figure/Structure/material_test.eps}
	\caption{Structural mass for potential materials}
	\label{fig:mat}
\end{figure}





