To investigate the influence of lay-up materials, heat flux variations, vehicle diameter and the trajectory approach on the \gls{tps} a sensitivity analysis is performed. First materials are selected to form multiple lay-ups. The different lay-ups are then optimised for different loading conditions. First the influence of heat flux variations on areal density is investigated. Subsequently, variations in mass due to changing diameters are tested. Lastly, the lay-ups are tested for different trajectories, either with a direct trajectory or with a parking orbit between aerocapture and landing. This investigation is done by using the tool described in Section \ref{subsec:thermaltool} and successfully validated in Appendix \ref{sec:VandVthermo}.

\paragraph{\gls{tps} materials}
Table \ref{tab:tpsmatprop} shows the materials that are used in the inflatable heat shield. A more extensive list of possible \gls{tps} materials and their properties can be found in Appendix \ref{sec:Thermoprop}. These materials have been proposed during the design of multiple inflatable decelerator concepts, such as \gls{irve} and \gls{thor} \cite{Hughes2005}. For each material the thermal conductivity, the density, the specific heat, the maximum operative temperature and if applicable, the emissivity are given. The latter is only applicable to the upper thermal protection layers, because these layers, or heat barriers, will radiate heat into the surroundings.
\newline\newline
A selection is made for the most promising thermal protection and insulation layers. These are Nextel BF-20 and Nicalon for the heat barrier layers. Nextel is a material already used by \gls{nasa} in \gls{irve}. Nicalon is a heavier alternative made up of continuous fibers of silicon carbide (SiC) that can withstand higher temperatures than Nextel up to $2073 \left[K\right]$. Also the emissivity of Nicalon is much higher that Nextel, which allows for more radiation. For the insulation layers these are Pyrogel 3350 and Pyrogel 6650. With those materials three lay-ups are created such that a comparison can be made between the heat barriers and insulators. A schematic view of the layers is shown in Figure \ref{fig:layersensthermal}. Lay-ups 1 and 2 can be used to compare the performance of Pyrogel 3350 and 6650. Lay-ups 2 and 3 serve as comparison for the Nextel BF-20 and Nicalon. These lay-ups will be tested for different heat fluxes as well as different diameters.

\begin{table}[ht]
	\caption {Flexible \acrlong{tps} material properties \cite{Corso2009,Corso2011,DuPont2011,Smith2011,Nye,Zinkle1998}}
	\centering
	\begin{tabular}{|l|l|l|l|l|l|l|}
		\hline
	        \textbf{Material}         & \textbf{ $\mathbf{k}$ $\mathbf{\left[\frac{W}{m\cdot K}\right]} $} & \textbf{ $\mathbf{ \rho }$ $\mathbf{ \left[ \frac{kg}{m^3} \right] }$} & \textbf{  $\mathbf{ c_{p} }$ $\mathbf{ \left[ \frac{J}{kg \cdot K} \right] }$ }& \textbf{ $\mathbf{ T_{max} }$ $\mathbf{ [ K ] }$} &\textbf{ $\mathbf{ \varepsilon }$ $\mathbf{ [ - ] }$} & \textbf{Function} \\[1.6ex]   \hline \hline
		Hi-Nicalon			& 2.4			& 2900	& 1200	& 2073	& 0.93	& rad. \& barrier	\\ \hline
		Nextel BF20			& 0.146			& 1362	& 1130	& 1643	& 0.443	& rad. \& barrier	\\ \hline
		Pyrogel 6650		& 0.030			& 110	& 1046	& 923	& -		& insulator			\\ \hline
		Pyrogel 3350		& 0.0248		& 170	& 1046	& 1373	& -		& insulator			\\ \hline
		Kapton				& 0.12			& 1468	& 1022	& 673	& -		& gas barrier		\\ \hline
		Kevlar				& 0.04			& 1440	& 1420	& 443	& -		& structural		\\ \hline
		PBO Zylon			& 20			& 1540	& 900	& 673	& -		& structural		\\ \hline

	\end{tabular}
	\label{tab:tpsmatprop}
\vspace{-4mm}
\end{table}

\begin{figure}[h]
	\centering
	\includegraphics[width=\textwidth]{./Figure/Thermal/layersensthermal.pdf}
	\caption{Tested lay-ups for the sensitivity analysis}
	\label{fig:layersensthermal}
\end{figure}

\paragraph{Effect of heat flux}
In order to analyse areal density performance of lay-ups and changes due to varying atmospheric conditions a heat flux sensitivity is performed. To achieve this, ratios of the heat flux of a possible trajectory are used. The trajectory is found using a diameter of $12 \left[ m \right]$. The results are shown in Figure \ref{fig:sensitivityq}. The horizontal axis shows the heat flux ratio and the areal density is shown on the vertical axis. 

\begin{figure}[h]
	\centering
	\includegraphics{./Figure/Thermal/Sensitivityq.pdf}
	\caption{Heat flux sensitivity for the three selected lay-ups}
	\label{fig:sensitivityq}
\end{figure}


As expected, the mass of the \gls{tps} increases with increased loading. Secondly and most important, the relative performance of the lay-ups can be observed. Lay-up 1 is clearly the lightest solution, followed by lay-up 3 and 2. Although lay-up 1 performs better in terms of its mass, the amount of loading it can bear is limited. If small changes in atmospheric properties occur during the \gls{edl} phase, for instance due to Martian storms, the \gls{tps} may succumb under the increasing loads. Therefore it is wise to choose Nicalon for further design. Lastly, if lay-up 1 and 3 are analysed relative to each other, it is clear that Pyrogel 6650 performs much better than the 3350 variant. Therefore, for further design it is more favourable to use Pyrogel 6650 as an insulator. The drawback of Pyrogel 6650 is that it has a lower maximum use temperature. This is solved by using a good heat barrier such as Nicalon.

\paragraph{Effect of diameter}
The three lay-ups are put to the test for different diameters. Aerodynamic analysis has provided heat fluxes for trajectories with corresponding diameters of $6$, $9$, $12$, $15$ and $18 \left[ m \right]$. As a side note, because the aerodynamic shape is different from the one in the previous paragraph, Figures \ref{fig:sensitivityq} and \ref{fig:sensitivityA} cannot be directly compared. An increase in heat flux caused an increase in the maximum temperature, surpassing the Nextel operative temperature limit which made it impossible for lay-ups 1 and 3 to fly trajectories at diameters of $12 \left[ m \right]$. Optimising the thickness of the lay-ups for these heat flux result in Figure \ref{fig:sensitivityA}. The solid lines indicate the nominal trajectory, with a parking orbit after aerocapture. For both graphs, the horizontal axis shows the relevant diameters. The plot on the left shows the areal density on the vertical axis and the right plot shows the total mass of the frontal \gls{tps} on this axis.

\begin{figure}[h]
	\centering
	\includegraphics{./Figure/Thermal/SensitivityA.pdf}
	\caption[Areal sensitivity for the three selected lay-ups]{Areal sensitivity for the three selected lay-ups, both for a direct trajectory and a usual trajectory with a parking orbit after aerocapture. Left plot shows areal density, whereas the right plot shows the total mass.}
	\label{fig:sensitivityA}
\end{figure}

For increasing diameters, larger radii of curvature can be obtained, resulting in a direct decrease of incoming heat flux. Also, due to the increasing diameters which causes an increase in \gls{sym:CD} and a reduction in ballistic coefficient, the vehicle can decelerate by the same amount at lower dynamic pressures. Therefore, the vehicle can stay higher in the atmosphere and fly in thinner air with the same velocity, decreasing heat development and incoming heat flux.\\

This effect can clearly be seen in the left figure, where the areal density decreases for increasing diameters. Obviously more material must be used to create larger \gls{tps}, which mostly results in a total mass increase for larger diameters. This can be seen in the right figure. The only exception is lay-up 2, the lay-up that is able to cope with the larger incoming heat flux at lower diameters. An optimum of its thermal performance is found at $9 \left[ m \right]$ where the frontal \gls{tps} mass reduces to approximately $150 \left[ kg \right]$. In addition, the relative mass performance of the different lay-ups is comparable to the performance in the previous paragraph.

\paragraph{Effect of time}
Whenever the vehicle is changing its descend rate, the total dissipated energy is still the same. However, the energy rate profile will have a different distribution over time, changing the temperature throughout the \gls{tps}. Steeper descends require a thicker heat barrier, limiting the heat flow to the rest of the shell, such that operational temperature of the insulator is not exceeded. A more gradual descend increases the time spend in the atmosphere and therefore increases the heat stored in the heat shield. This puts limits on the insulators minimum thickness, to block the heat flow to the structural layers and the rest of the vehicle. Therefore, the effect of descent time is analysed. The results are also shown in Figure \ref{fig:sensitivityA}. An alteration in time is visible by considering two types of viable trajectories, a direct trajectory and one with an orbit after aerocapture. From the figure it can be seen that the direct trajectory is the limiting one.