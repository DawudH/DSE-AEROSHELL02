This section will investigate the influence of the shape of the re-entry vehicle on several important design parameters, as well as the sensitivity of these parameters to changes in vehicle shape. It will start by identifying the most important design parameters. This will be followed by a section discussing which shapes will generate a local optimum for a single design parameter. The knowledge of these single parameter optima will be used in the final section to discuss the total set of design parameters for various vehicle shapes. 


\paragraph{Important parameters}
 The following parameters were determined to have a significant influence on te performance of the vehicle.


\begin{itemize}
	\item{Lift. As detailed in section \ref{subsec:controlsens}!!CHECK REFERENCE!!, the vehicle requires a lift vector to provide flight path control. A larger lift vector provides an increase in flight path control.}
	\item{Drag. The vehicle decelerates purely on atmospheric drag. An increase in drag will decrease the required time for re-entry and provides greater flexibility in terms of the path through the atmosphere. }
	\item{Lift to Drag ratio. The lift to drag is an indication of the freedom in the selection of the orbital trajectory. A higher lift to drag ratio will provide greater flexibility.}
	\item{\gls{sym:cm-alpha}. The derivative with respect to angle of attack of the moment coefficient is a measure of the stability of the vehicle. }
	\item{\gls{cg} offset}. The \gls{cg} offset at a given angle of attack required to cancel the moment generated by the vehicle at that angle of attack. It is a measure of the control effort required to trim the vehicle. 
	\item{Heat flux. For a given flight condition, the heat flux in the stagnation point depends only on the vehicle geometry.  }
\end{itemize}


\paragraph{Single parameter optima} \label{sec:aerooptima}
To illustrate the characteristics of different aerodynamic shapes, an optimisation has been performed towards certain aerodynamic coefficients. These shapes serve to enlarge understanding of how certain shape aspects correspond to certain aerodynamic properties. For the following parameters has been optimised:
\begin{itemize}
	\item Drag coefficient \gls{sym:CD}: The maximum drag should be attained by a flat plate at a zero angle of attack. This is also the result of the optimisation towards a maximal drag.
	\item Lift coefficient \gls{sym:CL}: As per the analysis in Paragraph \ref{sec:aeroparams}, the maximum lift coefficient is achieved by a flat plate at an angle of attack of $35^\circ$. This is confirmed by the optimisation algorithm, which produces the same flat plate as for maximum drag, but at an angle of attack.
	\item Lift over Drag $\frac{\gls{sym:L}}{\gls{sym:D}}$: The maximum Lift over Drag ratio is also found for a flat plate at an angle of attack as hight as possible. This result was achieved at an angle of attack of $40^\circ$, which is limited to keep the design in the range where the shockwaves don't hit the payload module.
	\item Static stability \gls{sym:cm-alpha}: For this parameter, it is n
\end{itemize}

\paragraph{Various shapes} \label{sec:aeroshapes}
Several large groups of varying shapes can be identified. Every aeroshell shape can be categorized in one of these groups.  Each group of shapes has advantages and disadvantages. The relative performance of each group can be qualitatively assessed by looking at the variations of the shape with respect to the optimal shapes for the various parameters. The effect of asymmetric cross-sections will be ignored in this assessment, and will be investigated separately. In figure \ref{fig:aeroshapes} representative cross sections of the groups can be seen. Group A represents simple concave surfaces. Group B has a concave centre section with a flat ring around it. Group C has an approximately flat central section, with steep edges around the outer radius. Group D represents the half cone shapes, with relatively straight sides and a blunt nose. 

As was discussed in section \ref{sec:aerooptima}, a flat plate will generate the most lift and the most drag, albeit at different angles of attack. Since Group C closely mimics a flat plate in the majority of its cross-section, it will have the best lift and drag performance. Group B also has a significant flat section and will therefore also have good performance in terms of lift and drag. Groups A and D will both have significant portions of their cross-sections at sub-optimal incidence angles for maximum lift or drag, and will therefore have lower lift and drag performance.



\paragraph{Asymmetry}























