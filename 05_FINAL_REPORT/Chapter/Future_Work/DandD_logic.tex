Key steps to be taken for manned missions to Mars for the proposed design are:
\begin{itemize}
\item Crew module design and decelerator detailed design
\item Ground and unmanned flight testing to further design and component \acrfull{trl}
\item Production and integration
\item Crew preparation and training
\item Establishing infrastructure on Mars
\end{itemize}

\subsubsection{Future design activities}
The crew module is to be designed. This involves the subsystems defined in Chapter \ref{ch:crewmod}, the crew cabin lay-out and the packaging of the subsystems as outlined in Section \ref{sec:crewpackaging}. Moreover, the decelerator requires further detailed design to fully establish its configuration and ready it for production and integration. 

\subsubsection{Testing activities} \label{sec:TestAct}
Table \ref{tab:tests} gives an overview of proposed testing activities. In addition, it outlines the articles on which these are performed and the purpose of the tests.
\begin{table}[h]
%\centering
\caption{An overview of proposed testing activities}
\label{tab:tests}
\hspace{-5mm}
\begin{tabular}{|p{0.155\textwidth}|p{0.24\textwidth}|p{0.55\textwidth}|}
\hline
\multicolumn{1}{|c|}{{\bf Testing activity}} & \multicolumn{1}{c|}{{\bf Performed on}}                                                                                 & \multicolumn{1}{c|}{{\bf Purpose}}                                                                                                                                                                                                \\ \hline \hline
Wind tunnel testing                          & Scaled decelerator wind tunnel model                                                                                                & \begin{tabular}[c]{@{}l@{}}1) Estimate aerodynamic properties\\ 2) Investigate effect of structure flexibility\\ 3) Investigate aerodynamic phenomena \\(e.g. aeroelasticity)\end{tabular}                                          \\ \hline
Aerothermal testing                          & \begin{tabular}[c]{@{}l@{}}- TPS lay-up samples\\ - Decelerator assembly \\ - Crew module\end{tabular}                                   & \begin{tabular}[c]{@{}l@{}}1) Demonstrate heat-carrying capability \\ and temperature\\ 2) Internal heat transfer \\ (e.g. to structural layers and inflation gas)\end{tabular}                                                         \\ \hline
Structural testing                           & \begin{tabular}[c]{@{}l@{}}- PBO Zylon samples\\ - Decelerator assembly\\ - Crew module\end{tabular}                    & \begin{tabular}[c]{@{}l@{}}1) Demonstrate load-carrying capability\\ 2) Investigate decelerator deflection\\ 3) Estimate effect of temperature on \\mechanical properties\\ 4) Determine effect of (launch) vibrations\end{tabular} \\ \hline
Deployment system testing                    & \begin{tabular}[c]{@{}l@{}}- Strap-band assembly\\ - Centerbody release\\ - Decelerator assembly\end{tabular} & Investigate reliability of deployment                                                                                                                                                                                             \\ \hline
End-to-End information system testing        & Avionics (CD\&H, ADCS and telecommunications)                                                                           & Ascertain compatibility of data handling systems                                                                                                                                                                                  \\ \hline
Flight testing (Earth)                       & Prototype scaled-down model (unmanned)                                                                                           & \begin{tabular}[c]{@{}l@{}}1) Determine control system performance\\ 2) Determine scaled-down vehicle performance\\ 3) Validate analysis models\end{tabular}                                                                      \\ \hline
Flight testing (Earth)                       & Prototype full-scale model (unmanned)                                                                                            & \begin{tabular}[c]{@{}l@{}}1) Validate scalability of design\\ 2) Determine integrated vehicle performance\end{tabular}                                                                                                           \\ \hline
Mission scenario testing (simulation)        & Avionics (CD\&H, ADCS and telecommunications)                                                                         & Demonstrate that flight hardware and software can execute the mission in terms of data flow with no time constraints                                                                                                              \\ \hline
Operations readiness testing (simulation)    & Avionics (CD\&H, ADCS and telecommunications)                                                                           & Demonstrate that flight hardware and software can execute the mission in terms of data flow with real timeline                                                                                                                    \\ \hline
Acceptance testing (Mars)                    & Flight full-scale model (unmanned)                                                                                      & Demonstrate system performance under limit loads                                                                                                                                                                                  \\ \hline
Pilot training (simulation) & Crew members & Investigate man-machine interaction during interplanetary flight and entry \\ \hline
\end{tabular}
\end{table}

\subsubsection{Production and integration}
Production and integration of the vehicle commences by a definition and analysis of the most cost-effective manufacturing methods and the most reliable and cost-effective joining methods. Hereafter, production and integration proceed in dedicated facilities with a dedicated work crew to take full advantage of crew experience and learning effect. In view of sustainability, non-value-adding activities are to be minimized in conformance with the lean manufacturing principle. 

\subsubsection{Crew preparation}
Crew members are to be trained and prepared for the 89-day journey and ensuing entry, during which they are exposed to high g-loads. Selection, training and preparation of crew members shall include their physical fitness, capability to perform required on-board activities and mental state for their isolatory condition.

\subsubsection{Establishment of a Martian infrastructure}
It is proposed that the entry vehicle is first flown unmanned, in the acceptance testing, to Mars to carry cargo required to establish an infrastructure. In addition, an infrastructure shall be laid out on Mars by previous missions. To this end, the required facilities on Mars are to be inventoried, packaged and sent as cargo on these missions.



