While the inflatable aeroshell offers prominent weight and packaging advantages with respect to conventional rigid solutions, it is inherently more unreliable. The failure modes in Table \ref{tab:fml} and the risk map in Table \ref{tab:riskmap} indicate that risk mitigating actions are to be taken in:
\begin{itemize}
\item Deployment
\item Inflation
\item Terminal descent
\item Nicalon\textsuperscript{TM} application in \gls{tps}
\item Asymmetrically stacked toroids
\end{itemize}
Prominent design recommendations are therefore an increase in reliability by addressing these issues. For the deployment and terminal descent phases, it is recommended that a trade-off for available methods is performed to yield the most reliable method within mass constraints. For the inflation system, it is recommended that in design of the blow-down system reliability is key. For the application of Nicalon\textsuperscript{TM}, extensive testing is required to ascertain its suitability for application in the \gls{cia}. Finally the structural and aero-elastic effects of stacking the toroids asymmetrical needs to be investigated and tested as this can prove to be a high risk factor.