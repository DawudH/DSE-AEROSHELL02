System verification and validation will need to be carried out as the project progresses. An outline of future verification and validation procedures is given in this section. This outline can be used to develop the verification and validation procedures as the project progresses. 

\subsubsection{Requirement verification}
\label{sec:ReqVer}
Although compliance to all top level requirements has been shown in Section \ref{sec:ComMat}, further verification will be needed as the design progresses and higher fidelity analysis have been performed. 

\paragraph{Mission requirements}
The mission requirements can be verified by analysis. A high fidelity model of the re entry must be created. This model will require validated aerodynamic, thermodynamic and inertial properties of the final design. This data can be obtained using a mix of computational models and physical tests. It will also require the control logic that will be used during the re entry to be implemented in the trajectory model. This high fidelity model is then used to demonstrate that the proposed design is capable of fulfilling all mission requirements under all reasonable circumstances. 

\paragraph{Entry vehicle requirements}
Entry vehicle requirements can be verified by inspection. Design documentation will provide all the relevant dimensions, procedures and masses to be able to prove that all entry vehicle requirements are met. Despite this, the total vehicle mass should also be verified using the final product to ensure full compliance to launch constraints. It must also be verified that all \gls{cospar} adherence procedures have actually been followed throughout the production of the vehicle. 

\subsubsection{Product validation}
Product validation will be performed by physical testing of part scale and full scale models. These tests have already been mentioned in Section \ref{sec:TestAct}. The tests relating to the complete, integrated product will be expanded on in this section.  

\paragraph{Deployment tests}
It must be demonstrated that the inflatable will deploy under representative conditions. Several critical tests must be passed before the system can be cleared for flight testing and eventually operational status. The first test of the deployment system must demonstrate that the deployment can be achieved without damaging the spacecraft or the inflatable. This is followed by deployment tests under vacuum conditions. The vacuum tests will also be used to validate the expected loss in pressure over time. Several tests will be needed to validate the performance of the deployment system under adverse conditions or malfunctions such as pyro-cutter misfires and incorrect stowage. The final deployment tests will take place during early flight testing. These will validate the ability of the inflatable to deploy under zero-$g$ conditions. 

\paragraph{Scaled flight testing}
After the performance of the deployment mechanism has been validated, scaled flight testing will take place. These tests will focus on the  performance of the control system and should prove that the control systems are capable of accurate trajectory control. They will also be used for further refinement and validation of the aerodynamic, thermodynamic and flight control models.  The scaled tests will use sounding rockets for suborbital test flights.

\paragraph{Full system flight testing}
The final validation tests will consist of three stages. After these tests, the performance of the system will have been completely validated and the system will be ready for human missions to Mars. The first stage consists of orbital re-entries of the full scale system. This will prove the system is capable of accurately entering the atmosphere of a  planet from orbit for re-entry. The second stage will send the full system on a trip around the moon, and re-enter the Earth's atmosphere using a mission profile comparable to what will be used on Mars (i.e. aerocapture followed by aerobraking). The final test of the system will be to land the cargo required for the mission on Mars. This final landing will prove that the system is ready for operational use. 




