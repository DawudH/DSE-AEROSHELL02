\section{Conclusion}

Scientific and commercial interest in extraterrestrial human exploration and habitation call for a feasible and efficient solution to entry. An inflatable aeroshell offers significantly lower mass and higher packaging efficiency than conventional, rigid solutions. Whereas rigid decelerator mass is estimated at over 3000 $ \left[ kg \right] $, preliminary design has yielded a guidable inflatable stacked toroid decelerator of a mere 1000 $ \left[ kg \right] $, capable of bringing two crew members in a 9000 $ \left[ kg \right] $ capsule to Mars.

Aerodynamic deceleration is performed by two passes through the atmosphere: aerocapture, intermitted by a parking orbit, followed by entry. This sequence, taking place in 1 Mars day, and decelerates the vehicle from a 7 $ \left[ km \cdot s^{-1} \right] $ upon entry of the atmosphere to Mach 5 at 15 $ \left[ km \right] $ altitude while keeping crew member loading under 3\gls{con:ge}. Trajectory adherence and control is provided by bank control, effected by reaction control thrusters and control system estimated at 212 $ \left[ kg \right] $.

Key feature of the aeroshell design is a skewed shape. The asymmetry follows from aerodynamic optimization and yields higher lift-generating capability at lower angles of attack to firstly achieve more lift and secondly require smaller angles of attack to keep the crew module from being impinged by the flow. Aerodynamic performance is characterized by a 0.35 lift-to-drag ratio and a 22.5 $ \left[ deg \right] $ trim angle of attack.

The asymmetry is adopted by the structural shape through stitching of ten inflatable toroids at a variable half-cone angle with respect to one another. Structural rigidity under an ultimate aerodynamic pressure of 3500 $ \left[ Pa \right] $ is ensured by the use of a nitrogen blow-down system that inflates five bladder volumes at 169 $ \left[ kPa \right] $, which keeps the flexible bladder material in tension to prevent compressive wrinkling. Resulting loads are carried by woven Zylon fibres of 0.125 $ \left[ mm \right] $ thickness at a 95 $ \left[ kg \right] $ mass. At a minimum half-cone angle,  structural mass is estimated at 300 $ \left[ kg \right] $. 

The \acrlong{tps} is exposed to a peak heat flux of 21 $ \left[ W \cdot cm^{-2} \right] $ and a peak temperature of 1376 $ \left[ K \right] $ during aerocapture. This thermal loading is withstood by a multi-material lay-up 256 $ \left[ kg \right] $ consisting of a state-of-the-art Nicalon barrier of 0.51 $ \left[ mm \right] $  thickness and Pyrogel 6650 insulator of 2.4 $ \left[ mm \right] $  thickness, complemented by dual 25 $ \left[ \mu m \right] $  Kapton gas barriers. 

Compatibility of the aeroshell with a manned Mars mission is ascertained by preliminary crew module and mission design. The crew module accommodates two crew members for a 89-day journey to Mars and its mass is estimated at 9000 $ \left[ kg \right] $. Return from Mars requires an additional launch prior to crew module launch, during which the \acrlong{mav} and an \acrlong{erv} are brought onto Mars and in an orbit around Mars respectively. Mission cost including development is estimated at 44 billion US dollars.

Recommendations are a propagation of design on decelerator and crew module, testing activities and crew and mission preparation thereafter. Key driver for further design is concept reliability. Deployment, inflation and terminal descent are critical mission phases and inherently unreliable for an inflatable aeroshell design. These therefore require particular attention in future design.


