Increasing awareness with respect to sustainable development makes sustainability an important consideration within the design of the \gls{cia}. Masud et al. define development as being sustainable  ``by ensuring the needs of the present demands without compromising any power or ability of future generations to meet their own needs'' \cite[p.85]{Masud2011}. 

Within the scope of the mission sustainability is considered where possible. It must however be considered that the production series length is small and less emphasis is given to sustainable development when compared to (for example) a commercial passenger jet. As such the overall environmental impact of the \gls{cia} is negligible and the sustainability of the concepts discussed in this report is not taken into account as a strong design driver. Nevertheless important consideration with regards to sustainability may be taken. 

Decelerator structural mass reductions directly allow for increases in useful payload or, allow for the use of smaller launchers for the same mission. By doing so the environmental footprint of each launch may be reduced with respect to comparable missions. A conventional rigid solution was investigated in the concept selection phase \cite{Balasooriyan2015b}. Preliminary mass estimates were over a factor three larger than the design presented within this final report. Choosing such a conventional concept would not only violate the mission requirements but would also incur additional emissions during the initial launch.

Sustainability is also taken into account outside of the Earth's atmosphere. Special care will be given to prevent accidental contamination of other orbital bodies with organic lifeforms and other contaminants. For this purpose no parts of the decelerator structure are separated during the descent towards the surface of Mars. This is in line with article IX of the Outer Space Treaty of 1967 \cite{UnitedNations2008}, enforced by the \gls{cospar}. Moreover the materials used in the design are considered where possible. One such example is the use of nitrogen as the inflation gas as further detailed in Section \ref{subsec:inflsys}. Less sustainable inflation gasses could be considered, such as for example hydrazine, which could achieve marginal mass reductions. From a sustainability point of view such an option was not preferred.

Looking at the full impact of an interplanetary mission of such a scale, environmental impact can never be prevented. However, in line with aforementioned definition of sustainability, the design presented in this report will be able to deliver for present demands while simultaneously working towards a design with less impact on the design than current technologies.

%\section{Approach with respect to sustainable development}
%\label{ch:sustain}
%
%With a increasing awareness with respect to sustainable development it is important to consider mission sustainability. This chapter discusses the approach with respect to sustainable development of the \gls{hiad} design concepts at hand. Masud et al. define development as being sustainable "by ensuring the needs of the present demands without compromising any power or ability of future generations to meet their own needs" \cite[p.85]{Masud2011}.
%
%Since the production series length of the controllable inflatable aeroshell is limited to very low numbers less emphasis is given to sustainable development when compared to (for example) a commercial passenger jet. As such the aeroshell's overall environmental impact is negligible and the sustainability of the concepts discussed in this report is not taken into account as a strong design driver. That is not to say sustainability is completely disregarded during product development. If for a certain design manufacturing methods are required that are very polluting these will be avoided and exchanged for less environmentally unfriendly, 'greener' methods. 
%Not only sustainability on Earth is taken into account, but also the impact of a space mission to another orbital body on the environment of said body is considered. Special care will be given to prevent accidental contamination of other orbital bodies with organic lifeforms and other contaminants. This is in line with article IX of the Outer Space Treaty of 1967 \cite{UnitedNations2008}, enforced by the Committee on Space Research (COSPAR). In addition to preventing forward interplanetary contamination care will be taken to minimise the amount of space debris left behind in an orbit around Earth and Mars. 
%
%The development of a controllable inflatable aeroshell can however improve the sustainability of both planetary and interplanetary spaceflight. Since implementing an inflatable aeroshell contributes to reducing the total mass of a spacecraft this can reduce the required launcher mass and thereby the launch emissions. The differences with respect to sustainability between the different concepts to be analysed is very small and is thus not used as a separate element in the trade-off process.
%
%Going back to the definition of sustainable development presented at the beginning of this chapter one can see that the measures taken during the design are indeed in line with sustainable development.
%


