Cost can be split up into two sections: Development cost and production cost. Whereas the development-related costs consist of non-recurring expenses, the production cost is dependent on the number of mission to be carried out. Reference \cite{Wertz2011} will be used to determine the costs associated with these components. Since these are determined in constant 2010 US dollars a factor accounting for inflation is used. This factor was found by looking at the \gls{cpi} ratio between April of 2010 and 2015. From Reference \cite{Crawford2015} this factor was found to be $1.075$, corresponding to an inflation of $7.5\%$.

\subsubsection{Development costs}
In contrast to the total mission the development costs consist of those incurred by aeroshell and capsule development. From Reference \cite{Wertz2011} the development cost per kilogram of mass can be found for various vehicles. These values, together with the total development cost are shown in table \ref{tab:devcosts}.

\begin{table}
	\centering
	\caption{Development costs in 2015 US dollars}
	\begin{tabular}{|c|c|c|}
		\hline
		\textbf{Cost component} & \textbf{Development cost per kg} & \textbf{Total development cost} \\ \hline \hline
		Aeroshell & & \\
		Crew capsule & & \\
	\end{tabular}
	\label{tab:devcosts}
\end{table}