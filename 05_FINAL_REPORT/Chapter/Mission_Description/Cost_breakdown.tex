Cost can be split up into two sections: Development cost and production cost. Whereas the development-related costs consist of non-recurring expenses, the production cost is dependent on the number of mission to be carried out. Reference \cite{Wertz2011} will be used to determine the costs associated with these components. Since these are determined in constant 2010 US dollars a factor accounting for inflation is used. This factor was found by looking at the \gls{cpi} ratio between April of 2010 and 2015. From Reference \cite{Crawford2015} this factor was found to be $1.075$, corresponding to an inflation of $7.5\%$.\\
For the development and production cost a \gls{hiad} propellant mass of $153$ kilograms was used, conform to the final design presented in Chapter \ref{cha:finaldesign}. A total \gls{hiad} mass (including propellant) of $1000$ kilograms was assumed, in order to take into account the maximum allowable \gls{hiad} mass growth. For the crew capsule a total mass of $9000$ kilograms was assumed from which $700$ kilograms of propellant mass was subtracted. This propellant mass will be covered in further detail in chapter \ref{ch:crewmod}. In addition to the propellant mass the crew mass was also subtracted in order to arrive at the spacecraft dry mass.

\subsubsection{Development costs}
In contrast to the total mission the development costs consist of those incurred by aeroshell and capsule development. From Reference \cite{Wertz2011} the development cost per kilogram of dry mass can be found for various vehicles. These values, together with the total development cost are shown in table \ref{tab:devcosts}.

\begin{table}[h]
	\centering
	\caption{Development costs in 2015 US dollars}
	\begin{tabular}{|c|p{5cm}|p{5cm}|}
		\hline
		\textbf{Cost component} & \textbf{Development cost per kg dry weight $\mathbf{[2015}$ $\mathbf{US\$]}$} & \textbf{Total development cost $\mathbf{[2015}$ $\mathbf{US\$]}$} \\ \hline \hline
		Aeroshell & 2 569 000 & 2 176 154 000 \\
		Crew capsule & 1 255 000 & 10 215 700 000 \\ \hline
		\textbf{Total} & - & 12 391 854 000 \\
		\hline
	\end{tabular}
	\label{tab:devcosts}
\end{table}

\subsubsection{Production costs}
The production costs were determined in similar fashion to the development costs presented in the previous section. Table \ref{tab:productioncosts} shows the corresponding productions costs per kilogram of dry spacecraft mass and for the complete respective component.
\begin{table}[h]
	\centering
	\caption{Production costs in 2015 US dollars}
	\begin{tabular}{|c|p{5cm}|p{5cm}|}
		\hline
		\textbf{Cost component} & \textbf{Production cost per kg dry weight $\mathbf{[2015}$ $\mathbf{US\$]}$} & \textbf{Total production cost $\mathbf{[2015}$ $\mathbf{US\$]}$} \\
		\hline \hline
		Aeroshell & 341 000 & 288 827 000 \\
		Crew capsule & 173 000 & 1 408 220 000 \\ \hline
		\textbf{Total} & - & 1 697 047 000 \\
		\hline
	\end{tabular}
	\label{tab:productioncosts}
\end{table}

\subsubsection{Mission costs}
In addition to the costs incurred by the spacecraft itself the overall mission architecture requires the use of additional resources such as launch vehicles. 