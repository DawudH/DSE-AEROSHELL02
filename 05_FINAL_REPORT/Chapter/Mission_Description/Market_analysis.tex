Three dimensions are used to define the market for the product: function, technology and customer. The purpose of the market analysis is a minimisation of risk of selecting incompatible function and technology for a selected set of customers. A \gls{swot} analysis gives an overview of product characteristics.

\subsubsection{Customer base}
Prospective customers are scientific or governmental agencies on one hand and private ventures on the other hand. Leading player in the former is \gls{nasa}, by order of the \gls{us} government. The \gls{us} are in pursuit of human exploration of Mars in the 2030's\footnote{URL: \url{https://www.nasa.gov/content/nasas-journey-to-mars}.  Accessed 28 April 2015}, formulated in the National Space Policy issued in 2010\footnote{URL: \url{https://www.whitehouse.gov/sites/default/files/national\_space\_policy\_6-28-10.pdf}. Accessed 28 April 2015}. On the basis thereof, it has been formulated as a goal in the \gls{nasa} Authorization Act of 2014\footnote{URL: \url{http://science.house.gov/sites/republicans.science.house.gov/files/documents/HR\%204412.pdf}. Accessed 28 April 2015}.

The \gls{us} government is a key player due to the significant budget allocated to planetary science and Mars exploration. Forecasts dating from the Fiscal Year 2013 budget estimates \cite{NASA2014a} are taken up in Table \ref{tab:cashbudgets}. The second row reflects the \gls{us} government's dedication to extraterrestrial exploration. The third row shows increasing budgets allocated to Mars exploration, reflecting the continuing interest and dedication to Mars exploration.

\begin{table}[h]
\centering
\caption{NASA budget forecasts}
\label{tab:cashbudgets}
\begin{tabular}{|l|l|l|l|}
\hline
{\bf Fiscal year}                    & 2015     & 2016    & 2017    \\ \hline
{\bf Planetary science {[}mln \${]}} & 1 102.0 & 1 119.4 & 1 198.8 \\ \hline
{\bf Mars exploration {[}mln \${]}}  & 188.7   & 266.9   & 503.1   \\ \hline
\end{tabular}
\end{table}

The interest expressed by the \gls{us} in human exploration of Mars is shared by a number of private ventures, most notably Mars One, the Inspiration Mars Foundation and SpaceX. The former two are non-profit organisations, while SpaceX is a commercial venture. Mars One has expressed its goal as the permanent human settlement on Mars with planned departure of the first non-human payload in 2020 and the first human payload in 2026\footnote{URL: \url{http://www.mars-one.com/}. Accessed 28 April 2015}. The Inspiration Mars Foundation, in cooperation with \gls{nasa}, seeks to transport two humans, a male and female, to Mars for planned launch in 2021\footnote{URL: \url{http://spacenews.com/39714inspiration-mars-sets-sights-on-venusmars-flyby-in-2021/}. Accessed 28 April 2015}. SpaceX is a privately funded venture currently working in close cooperation with \gls{nasa} to provide launchers for manned missions to Mars\footnote{URL: \url{http://www.spacex.com/falcon9}. Accessed 28 April 2015}.

These planned missions illustrate the commercial interest in human spaceflight to Mars. Commercial interest in controllable inflatable aeroshells is directly coupled to this by the fact that these provide a cost-effective means of entry and re-entry. Along with this commercial interest, ongoing investigations by \gls{nasa} provide an indication of scientific interest in this field of study. In the end, all interest is fuelled by human curiosity and the desire to explore and habitate extraterrestrial environments. These environments are expected to expand beyond Mars and therefore interest in (re-)entry vehicles is expected to remain.

%Direct customers are thus governmental agencies on one hand, primarily \gls{nasa}, and commercial providers on the other hand. 

\subsubsection{Function}
Primary prospects for the use of a controllable inflatable aeroshell are the following:
\begin{itemize}
\item Perform entry for manned spaceflight on Mars;
\item Serve as a basis for design extrapolation to perform manned (re-)entry at other sites, for example Earth;
\item Serve as a basis for design extrapolation to perform (re-)entry of unmanned spaceflight;
\item Further the technology development and application of inflatable technologies in spaceflight.
\end{itemize}
A direct function or use is the first item: the controllable inflatable aeroshell provides aerodynamic deceleration for (safe) transportation of human payload in a cost-effective manner. While the aeroshell is designed for entry on Mars, the design can be extrapolated to perform entry or re-entry on a number of sites, for one Earth.

\subsubsection{Technology}
The controllable inflatable aeroshell will demonstrate predominantly the following technologies:
\begin{itemize}
%\item Manned aerocapture and \gls{edl} on Mars
\item An asymmetric stacked toroid structure
\item A large-scale inflatable and inflation system
\item Bank control for Mars targeted aerocapture and landing
\item A multi-layer flexible and foldable thermo-structural design using state-of-the-art PBO Zylon fibres and HI-NICALON
\end{itemize}
These technologies are firstly of key importance for commercial interest. An inflatable structure in itself has significant advantages over conventional rigid solutions, but in particular the asymmetric shape and the use of state-of-the-art materials provide means by which to increase the cost-effectiveness of (re-)entry solutions. Secondly, the demonstration of these technologies will further their stage of development and gain additional knowledge in the use of controllable inflatable aeroshell technology for (re-)entry.

%The functionalities provided by the aeroshell, thus the deceleration provided during (re-)entry, are effected by an inflatable aeroshell as primary technology used. This may be further subdivided into an aerodynamic shape, a control system, a \gls{tps} and a supporting structure (including deployment mechanisms). 

\subsubsection{SWOT analysis}
Identification of the primary characteristics, in terms of a \gls{swot} analysis\footnote{URL: \url{http://www.usfca.edu/fac_staff/weihrichh/docs/tows.pdf}. Accessed: 19-06-2015}, of the proposed controllable inflatable aeroshell yields Table \ref{tab:swot}.

\begin{table}
\centering
\caption{Design high-level SWOT analysis}
\label{tab:swot}
\begin{tabular}{|l|l|}
\hline
{\bf Strengths}                                                                                         & {\bf Weaknesses}                                                                                     \\ \hline
\begin{tabular}[c]{@{}l@{}}+ \textless10 \% decelerator mass fraction\\ + Compact solution\end{tabular} & \begin{tabular}[c]{@{}l@{}}- Development risk\\ - Deployment risk\end{tabular}                       \\ \hline
{\bf Opportunities}                                                                                     & {\bf Threats}                                                                                        \\ \hline
\begin{tabular}[c]{@{}l@{}}+ Growing demand\\ + Breakthrough technology\end{tabular}                    & \begin{tabular}[c]{@{}l@{}}- Catastrophic failure manned mission\\ - Competing concepts\end{tabular} \\ \hline
\end{tabular}
\end{table}

Strengths and weaknesses are internal to the design, while opportunities and threats are external factors. The cost at which the significant mass decrease and packaging efficiency increase (with respect to conventional rigid solutions) comes is reflected by an increased development risk and deployment risk. The former is the result of the novelty of inflatable decelerators; the latter inherent to the use of an inflation and deployment system. While the design retains a development risk, being a relatively new concept, this weakness can be mitigated by proper verification activities. Such activities do, however, incur additional time and costs to the design process. As such, risk remains inherent to the design. 