In this section the mission requirements for the mission as described in section \ref{sec:missionscope} are outlined and their origin is explained. A full list of requirements as defined top level can be found in Table \ref{tab:misreq} and \ref{tab:vehreq}.

The mission starts at the boundary of the atmosphere of Mars. Here the velocity of the entry vehicle is $7 \left[km \cdot s^{-1} \right]$. This requirement is imposed by the transfer trajectory that is taken from Earth to Mars. This trajectory should take as short as possible in order to both shorten the entire mission duration and decrease the physical taxation on the crew. The interplanetary transfer time corresponding to an entry velocity of $7 \left[km \cdot s^{-1} \right]$ is $89$ days. 

The mission ends at a speed of $M=5$ at $10 \left[km\right]$ altitude. At this point a terminal descent system takes over. The predetermined point at which the mission ends shall be reached with a precision of $500 \left[m\right]$. This requirement is imposed by the distance the final landing position can be from the provision. When the landing position lies too far from the provision a lot of time will be lost moving the crew or the astronauts might not even be able to reach the provision.

While decelerating in the atmosphere the maximum deceleration shall not exceed $3\gls{sym:g}_{earth}$. This requirement is imposed because of the limited capability of the crew to carry high deceleration loads.

The entry vehicle should attain its final velocity within 10 earth days. This requirement is, just like the interplanetary transfer time, imposed both to shorten the mission duration and decrease the physical taxation on the crew. An additional reason for this time constraint is to limit the cost for and strain on the ground control crews that will be active continuously during the mission.


\begin{table}[h]
	\caption{Overview of mission requirements}
	\label{tab:misreq} 
	\begin{tabular}{|p{0.12\textwidth}|p{0.88\textwidth}|}
    \hline
    ID          & Description                                                                                                      \\ \hline \hline
    CIA-M01& The re-entry vehicle shall decelerate from a velocity of 7 $[km\cdot s ^{-1}]$ to Mach 5 $[-]$  \\ \hline
    CIA-M02 & The re-entry vehicle shall not exert an acceleration greater than 29.4 $[m \cdot s^{2}]$ on any crew member for the duration of the mission			\\ \hline
    	CIA-M03 & The re-entry vehicle shall attain its final velocity at an altitude of 15 000 [m] \gls{mola} \\ \hline
    	CIA-M04 & The re-entry vehicle shall reach its final position with a precision of 500 [m]\\ \hline
    	CIA-M05 & The re-entry vehicle shall attain its final velocity within 10 days of mission start \\ \hline
    \end{tabular}
\end{table}

\begin{table}[h]
	\caption{Overview of re-entry vehicle requirements} 
	\label{tab:vehreq}
	\begin{tabular}{|p{0.12\textwidth}|p{0.88\textwidth}|}
	    \hline
	    ID          & Description                                                                                                      \\ \hline \hline
	CIA-R01 & The re-entry vehicle shall have an undeployed diameter smaller than 5 [m]                         				            \\ \hline
	CIA-R02 & The re-entry vehicle shall have a deployed diameter smaller than 12 [m]                         				            \\ \hline	
	CIA-R03 & The re-entry vehicle shall have a mass of 10 000 [kg] at the start of the re-entry                       				            \\ \hline
	CIA-R04 & The hypersonic decelerator shall have a mass fraction of no greater than 10\% of the vehicle mass  \\ \hline
	CIA-R05 &  The re-entry vehicle shall adhere to the \gls{cospar} regulations \\ \hline
	CIA-R06 &  The re-entry vehicle shall have control system accuracy of at least $5\cdot 10^{-4}$  \\ \hline
    \end{tabular}
\end{table}




