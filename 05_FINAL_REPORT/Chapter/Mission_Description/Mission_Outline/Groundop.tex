It is important that the ground segment is taken into a consideration at this stage to assess mission feasibility and to provide an early impression of the required ground facilities. The ground segment is an essential mission feature to facilitate communication flow between Earth and spacecraft and thereby to monitor mission progress and crew member status as well as take corrective actions if need be and circumstances allow.

To this end, the ground segment consists of a missions operations centre and a communications network. This set-up is similar to \gls{esa} ground operations for deep space missions Rosetta and Venus Express\footnote{URL: \url{http://www.esa.int/esapub/bulletin/bulletin124/bul124e_warhout.pdf}. Accessed: 10-06-2015}  \cite{Warhaut2007}. An alternative would be a decentralized structure, in which control centres are not included in the missions operations centre but linked separately to it. 

\paragraph{Operations centre}
The operations centre is manned continually with the purpose of monitoring and controlling mission progress \cite{Warhaut2007}. It is the ground system element that is in direct contact with the spacecraft by the link established through the ground stations for uplink and downlink \cite[p.879]{Wertz2011}. Downlink data are analyzed and formatted, partially sent through to the end-receivers of scientific information and partially used for mission health monitoring and control. The nature of these end-receivers of scientific information depends on the payload activities conducted on Mars. 

Examples of such an operations centre are the California Institute of Technology's Jet Propulsion Laboratory, responsible for NASA's \gls{dsn}, or the \gls{esoc}, responsible for \gls{esa} deep space missions. The former has been used for one for the manned Apollo missions to the moon, the latter for Rosetta and Apollo missions \cite{Wertz2011,Warhaut2007}. Both of these operations centres would be suitable for the mission at hand, mainly due to their succesful operation in past deep space and manned missions. 

\paragraph{Communications network}
A communication link is established between the ground segment and the spacecraft. Key feature is its ability for communication between Mars and Earth, over which free space losses are highly significant \cite{Wertz2011}. While manned missions to Mars have not been flown, a good reference point is a previous unmanned Mars mission, such as the Mars Rover, as both face similar communication requirements. The Mars Rover was reliant on the \gls{dsn}\footnote{URL: \url{http://mars.nasa.gov/mer/mission/communications.html}. Accessed: 10-06-2015} for its communications on X-band. 

The \gls{dsn} uses three complexes separated by 120 degrees of longitude to provide continual coverage with a rotating Earth. Sensitive 70 [$m$] diameter antennas are used for maximum sensitivity and complemented by a number of 34 [$m$] diameter antennas. \cite{Wertz2011} These antennas would be suitable for the mission at hand by their intended and proven purpose of providing communication in deep space and therefore to and from Mars. While the technology is thereby sufficient, continuous maintenance of and improvements on the \gls{dsn} will ensure proper functioning and network availability over the next decades. An alternative would be ESA's ESTRACK, consisting of 10 ESA operated ground stations for communicaton support. These do not allow for Ka-band transmission, however \cite{Wertz2011}.

Bandwidths are required to allow for sufficient signal strength upon reception and additionally follow from the required bit rate. Current standard for deep space missions are S-band, in a frequency range of 2.0-2.3 [$GHz$], and X-band, in a frequency range of 8.45 to 8.5 [$GHz$] \cite{Wertz2011}.

An advancing trend is the use of Ka-band for deep space communication downlink, in a frequency range of 25.5-32.3 [$GHz$]. Ka-band is able to provide more data volume in less \gls{dsn} tracking time, while continuing automation for \gls{dsn} ground systems will further increase antenna availability through a reduction of required calibration time\cite{Edwards1999}. 

Following requirements on NASA's \gls{dsn} S-band will be available for both up- and downlink, while Ka-band will be available for high-data-rate science returns \cite{Labelle2012}. The crew module itself will not necessitate Ka-band for the purpose of science returns, but transmission of detailed system state measured by sensors for the purpose of monitoring will benefit from the use of a Ka-band for downlink. For the purpose of uplink, limited communication flow is present and S-band suffices.%Uplink will merely require communication with on-board payload, since the spacecraft is predominantyl autonomous by the excessive transfer time of ground-based commands and thereby inability to otherwise react

As such, Ka-band is used for downlink telecommunication for its high data link capability, while S-band is used for uplink. Both are supported by the \gls{dsn}.
