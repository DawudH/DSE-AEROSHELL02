Launch serves to bring the \gls{hiad} and mission-required vehicles in a transfer orbit towards Mars. From the top-level requirements as summarised in Section \ref{sec:missionreq} an entry velocity of 7 [$km\cdot s^{-1}$] is desired which is an implicit requirement on the total mission duration.
Based on this requirement specific launch operations can be considered and additional mission requirements can be considered. Important factors are payload size, loads and required velocity increments. Launch is important to consider in the \gls{hiad} design as a lot of the system requirements can be traced down to launch.

In order to reach the Martian atmosphere with the desired approach velocity a total velocity increment of about 19.6 [$km\cdot s^{-1}$] is required. This velocity increment includes the escape velocity of the Earth to its \gls{soi} and an additional velocity increment to reach the Martian atmosphere with the required approach velocity.

The velocity increments are typically divided into two parts: a first velocity increment into \gls{leo} and a second velocity increment into the transfer orbit. Within the \gls{leo} separate payload modules the \gls{hiad} may be joined \cite{George2009}. The period in \gls{leo} also allows for more precisely controlled arrival conditions at Mars as, to a certain extent, the launch is omitted from the timing sequence. Moreover, it widens the launch window.

%The velocity increments up to \gls{leo} are typically subdivided into multiple integrated stages or launchers for optimal efficiency as no single launcher can deliver the total required velocity increment. These stages are separated after depletion of the propellant.

Important considerations concerning the launch are the encountered vibrations and loads as well as the total mass required to bring into the interplanetary transfer orbit. Launch vibrations should be considered as the natural frequencies of the subsystems should remain above the launch-induced vibrations. This should, for one, be considered for the inflatable part of the decelerator.

Launch loads are typically in the order of 2.8-4.3 \gls{con:ge} in longitudinal and 0.9-3 \gls{con:ge} in lateral direction \cite{Wertz2011}, which is above the maximum allowed top level deceleration of 3g into the Martian atmosphere. For this reason launch loads are an important factor for the structural sizing of the aeroshell and accompanying elements. 

A launcher currently being developed for missions to Mars is the \acrfull{sls} developed by \gls{nasa}. The \gls{sls} features multiple stages and allows for a 5 [$m$] diameter in line with the top level mission requirements. The \gls{sls} features multiple stages able to deliver the required velocity increments. Its design is tailored to the Orion spacecraft which is being developed to, in the future, go to Mars. For the modules featuring a 5 [$m$] diameter payload the volume is constrained to 225 [$m^3$] \cite{NASA2014}.

%The launch phase is concluded with the start of the interplanetary phase after the final velocity increment in \gls{leo}. 

