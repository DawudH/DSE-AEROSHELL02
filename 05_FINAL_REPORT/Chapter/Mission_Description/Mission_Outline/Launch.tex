Mission launch is the first phase of the mission. Launch serves to bring the \gls{hiad} and accompanying mission elements in a transfer orbit towards Mars. From the top level requirements as summarised in \ref{sec:missionreq} an approach speed of 7 [$km\cdot s^{-1}$] is desired which is an implicit requirement on the total mission duration.
Based on this requirement specific launch operations can be considered and additional mission requirements can be considered. Important factors include payload size, loads and required velocity increments. Launch is important to consider also in the \gls{hiad} design as a lot of the requirements can be traced down to mission launch.

The launch mission phase initiates with the take-off from earth. In order to reach the Martian atmosphere with the desired approach speed a total velocity increment of about 19.6 [$km\cdot s^{-1}$] is required. This velocity increment comes forth of the escape velocity of the Earth to its sphere of influence and an additional velocity increment to reach the Martian atmosphere with the required approach velocity.

The velocity increments are typically subdivided into two parts: a first velocity increment into \gls{leo} and a second velocity increment into the transfer orbit. Within the \gls{leo} separate payload modules such as the \gls{hiad} and a possible habitation module may be joined \cite{George2009}. The period in \gls{leo} also allows for more precisely controlled arrival conditions at Mars as, to a certain extend, the launch is omitted from the timing sequence. 

The velocity increments up to \gls{leo} are typically subdivided into multiple integrated stages or launchers for optimal efficiency as no single launcher can deliver the total required velocity increment. These stages are separated after depletion of the propellant.

Important considerations concerning the launch are the encountered vibrations and loads as well as the total mass required to bring into the transfer orbit. Launch vibration should be considered as the natural frequencies of the subsystem should remain above the launch-induced vibrations. Moreover the vibrations require all components to be properly stowed. This should for example be considered for the inflatable part of the decelerator, which is stowed during launch.

Launch loads are typically in the order of 2.8-4.3 g's in longitudinal and 0.9-3 g's in lateral direction \cite{Wertz2011}, which is above the maximum allowed top level deceleration of 3g into the Martian atmosphere. For this reason launch loads are an important factor for the structural sizing of the aeroshell and accompanying elements. 

A launcher currently being developed for missions to Mars is the \acrfull{sls} developed by \gls{nasa}. The \gls{sls} features multiple stages and allows for a 5 [$m$] diameter in line with the top level mission requirements. The \gls{sls} features multiple stages able to deliver the required velocity increments. Its design takes close account of the Orion spacecraft which is being developed to, in the future, go to Mars. For the modules featuring a 5 [$m$] diameter payload the volume using this launcher is constraint to 225 [$m^3$] \cite{NASA2014}.

The launch phase is concluded with the start of the interplanetary phase after the final velocity increment in \gls{leo}. 

