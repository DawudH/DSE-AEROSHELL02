The third phase of the mission is the arrival at Mars and the deceleration to a velocity of $M=5$ at $10 \left[km\right]$ height above the surface of Mars with an accuracy of $500 \left[m\right]$ in all directions. This deceleration is split into an initial aerocapture, a parking orbit and a final entry and should not take longer than 10 earth days in total. This is the phase of the mission where our design focussed on most. In this phase of the mission the aeroshell is used to decelerate the capsule and protect it against the thermal loads imposed by the deceleration.

%Upon arrival on Mars, first the interplanetary habitat is separated from the entry capsule. The habitat will be put on a trajectory which causes it to burn up in the vicinity of the sun.
Upon arrival on Mars, the first thing that happens, just before the spacecraft enters the atmosphere, is that the aeroshell is deployed and inflated. %The aeroshell will be continuously kept at pressure during the rest of the mission phase.
The spacecraft is now transformed into a entry vehicle and is ready to start the aerocapture.

The entry vehicle will enter the atmosphere for the first time. This first pass through the atmosphere is called aerocapture. The entry vehicle will fly through the atmoshere following a pre-calculated path using active bank control. A real-time control computer will manage the active control to account for unexpected differences in aerodynamic properties. The goal of this control computer is to keep the kinetic energy that is lost during the aercapture equal to what is pre-calculated. This loss in kinetic energy determines the characteristics of the trajectory which the spacecraft will follow once it leaves the atmosphere. %Diminishing too little energy causes the trajectory to be more eliptic or even hyperbolic. Diminishing too much energy causes the the trajectory to be less eliptic, or it can even cause the entry vehicle to not even go out of the atmosphere anymore. These trajectory characteristics change the orbit period and the fuel fraction needed to change 

After the aerocapture the spacecraft goes into an eliptic kepler orbit. While the spacecraft is on its way to the apocenter of the orbit its attitude gets changed so that the trusters point in the right direction to give the spacecraft a velocity change. In the apocenter the spacecraft gets a velocity change which will get it in an eliptic Mars-synchronous orbit which will not pass through the atmosphere. From this parking orbit the atmosphere can be observed and a plan can be made for the entry into the atmosphere in order to get to the intended landing location. The observations made of the atmosphere will help determine a suitable moment to do the final entry and will give us information that can be used to predict the final entry trajectory more accurately.

Once the decision has been made to do the final entry the entry vehicle is given a second boost to decelerate it just enough to get the entry vehicle onto the wanted trajectory. Here, just as during the first pass through the atmosphere, the spacecraft is controlled using active bank control managed by a real-time control computer. When the entry vehicle is aproaching the intended final location at $10 \left[km\right]$ height a change in angle of attack is used to dive to that location. Having control over the time of initiation of the dive gives us an additional safety on landing within  $500 \left[m\right]$ from the target.