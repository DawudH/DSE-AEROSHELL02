The third phase of the mission is the arrival at Mars and the deceleration to a velocity of $M=5$ at $15 \left[km\right]$ height above the surface of Mars with an accuracy of $500 \left[m\right]$ in each direction. This deceleration is split into an initial aerocapture, a parking orbit and a final entry. The combined sum of these components should not take longer than 10 Earth days. In this phase of the mission the \gls{hiad} is used to decelerate the capsule and protect it against the thermal loads imposed by the deceleration. In addition, taxation of human crew members requires loads not to exeed 3\gls{con:ge}.

%Upon arrival on Mars, first the interplanetary habitat is separated from the entry capsule. The habitat will be put on a trajectory which causes it to burn up in the vicinity of the sun.
Upon arrival on Mars the first thing that happens, just before the spacecraft enters the atmosphere, is the deployment and inflation of the \gls{hiad}. %The aeroshell will be continuously kept at pressure during the rest of the mission phase.
%The spacecraft is now transformed into a entry vehicle and is ready to start the aerocapture.

The entry vehicle then enters the atmosphere for the first time. This first pass through the atmosphere is called aerocapture. The entry vehicle will fly through the atmosphere following a pre-determined path using active bank control. A real-time controller will manage the active control systems to account for unexpected differences in aerodynamic properties. The goal of this controller is to keep the kinetic energy lost during the aerocapture equal to what is pre-calculated. This loss of kinetic energy determines the characteristics of the trajectory which the spacecraft will follow once it leaves the atmosphere. %Diminishing too little energy causes the trajectory to be more eliptic or even hyperbolic. Diminishing too much energy causes the the trajectory to be less eliptic, or it can even cause the entry vehicle to not even go out of the atmosphere anymore. These trajectory characteristics change the orbit period and the fuel fraction needed to change 

After the aerocapture the spacecraft goes into an elliptic Kepler orbit. When the spacecraft is headed to the apocentre of the orbit it changes attitude so that the thrusters point in the along-path direction to give the spacecraft a velocity change. While in the apocentre the spacecraft gets a $\Delta\gls{sym:V}$ to raise the pericentre altitude of the Mars-centred orbit to a parking orbit at 200 [$km$] height.
%In the apocentre the spacecraft gets a velocity change which will get it in an elliptic Mars-synchronous orbit which will not pass through the atmosphere. 

From this parking orbit the atmospheric conditions can be observed and a plan can be made for the entry into the atmosphere in order to get to the intended landing location. The observations made of the atmosphere will help determine a suitable moment to do the final entry and will give information that can be used to predict the final entry trajectory more accurately. For example, in case of a dust storm, characteristic of Mars, \acrfull{edl} can be delayed until it has passed.

Once the decision has been made to conduct the final entry the spacecraft is given a second boost to decelerate it just enough to get the entry vehicle into the desired trajectory. Here, just as during the first pass through the atmosphere, the spacecraft is controlled using active bank control managed by a real-time controller. %When the entry vehicle is approaching the intended final location at $10 \left[km\right]$ height a change in angle of attack is used to dive to that location. Having control over the time of initiation of the dive gives us an additional safety on landing within  $500 \left[m\right]$ from the target.