The return from Mars will require several systems to already be in place by the time the crew arrives. A \gls{mav} is required to lift the crew back into an orbit around Mars. An \gls{erv} is required to take the crew back from Mars to Earth. To reduce the risk of stranding the crew on Mars without any option to return to Earth, these vehicles should be in place before the crew commences the aerocapture. 

The \gls{mav} and \gls{erv} will need to be part of the cargo sent to Mars ahead of the manned mission. The amount of thrust and propellant required to lift off from the Martian surface makes it infeasible to combine the ascent and descent stages of the mission. The \gls{mav} therefore needs to be prepositioned on the surface of Mars, along with the habitat and supplies required for the stay on Mars. The \gls{erv} requires a sizeable habitation module for the return to Earth. Due to the mass associated with this size, it should be placed in orbit around Mars while waiting for the return trip rather than launched from the Martian surface as part of the \gls{mav} \cite{Hoffman1997a}.