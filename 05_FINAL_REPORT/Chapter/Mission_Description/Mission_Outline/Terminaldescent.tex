The terminal descent of the spacecraft is the part of the mission between the end of aerocapture and landing on the surface of Mars. The velocity is to be brought back to zero at an altitude of zero, from an initial velocity at the end of the aerocapture phase of the mission. For the terminal descent, several design options are available to decrease the velocity. In this section, firstly, the mission characteristics are discussed. After that, the design options are summarized after which a choice is made based on feasibility and performance.

\paragraph{Terminal descent characteristics}
The start of this part of the mission is given by the end of the aerocapture part, of which the requirements dictate a Mach number of $5$ at a height of $15$ [km], as given in Section \ref{sec:missionreq}. This means the aerodynamic flow regime changes from hypersonic to supersonic, and finally to subsonic during the terminal descent. The speed of sound in the lowest fifteen kilometres of Mars is approximately $220$ $[m\cdot s^{-1}]$, which means the velocity of the spacecraft is $1,100$ $[m\cdot s^{-1}]$ at the beginning of terminal descent. The velocity only marginally increases due to the gravity influence of Mars: the velocity with no deceleration would be 3 percent higher on the surface of Mars than at an altitude of $10$ $[km]$, assuming no deceleration due to drag or thrust. Finally, the flight path angle is not predetermined by the orbit, so it can be changed to fit the needs of the terminal descent phase.

\paragraph{Design options}
Terminal descent can be split up in two parts: the supersonic and subsonic flight part and final touchdown. For both parts, different options are available.

The first option for the flight is to use retro-propulsion for every part of the descent. The fuel mass would be about $23.3\%$ of the total spacecraft mass if no aerodynamic effects were taken into account. However, the aeroshell has a large frontal area which adds a significant amount of drag. Also, in numerical simulations and wind tunnel tests the interaction between retro-propulsion and the aeroshell were found to result in a mass fraction that is approximately twice as small as would be expected when considering the thrust and drag forces to act independently of each other \cite{Korzun2009}. Since a blunt body is unstable at transonic and supersonic speeds, a small drogue parachute is needed to stabilise the spacecraft.

The fuel mass is estimated assuming a constant deceleration of 3 [g]. This condition in combination with the initial conditions of the terminal descent, leads to a specified flight path angle of 38 [deg] and a velocity at each height. Using this velocity, the drag was calculated assuming the same drag coefficient throughout the supersonic regime. This analysis is known to be incorrect to a certain level, but since it is preliminary analysis this is taken for granted. The drag at every height leads to a deceleration lower than 3 [g], and thrust is delivered at a level such that this deceleration is achieved, incorporating the gravitational force. The drag, thrust and total required force are plotted in Figure \ref{fig:TDforce}. This requires the rocket engines to be sized such that, a total thrust of 312 [kN] is achieved. To this end, 3 RL 10A-4 rocket engines are placed at the front of the centerbody. The mass of these rockets is 504 [kg] combined. The thruster fuel flow is calculated using the specific impulse of the engine, 451 [s], and integrated over time to find the total fuel, estimated to be 680 [kg] \cite{Wertz2011}. The tank mass is estimated using an empirical relation, at 45 [kg] and assuming a density of 1 [kg/l] for the fuel \cite{Wertz2011}.

\begin{figure}
	\centering
	\setlength\figureheight{0.4\textwidth} 
	\setlength\figurewidth{0.7\textwidth}
	% This file was created by matlab2tikz.
% Minimal pgfplots version: 1.3
%
\definecolor{mycolor1}{rgb}{0.00000,0.44700,0.74100}%
\definecolor{mycolor2}{rgb}{0.85000,0.32500,0.09800}%
\definecolor{mycolor3}{rgb}{0.92900,0.69400,0.12500}%
%
\begin{tikzpicture}

\begin{axis}[%
width=0.95092\figurewidth,
height=\figureheight,
at={(0\figurewidth,0\figureheight)},
scale only axis,
xmin=0,
xmax=36.2,
xlabel={$t [s]$},
xmajorgrids,
ymin=0,
ymax=327.127499399344,
ylabel={$F [kN]$},
ymajorgrids,
axis x line*=bottom,
axis y line*=left,
legend style={at={(0.97,0.5)},anchor=east,legend cell align=left,align=left,draw=white!15!black}
]
\addplot [color=mycolor1,solid,mark=o,mark options={solid}]
  table[row sep=crcr]{%
0	311.243849696306\\
3	311.2690765018\\
6	311.294330905114\\
9	311.31961294652\\
12	311.344922666362\\
15	311.370260105056\\
18	311.395625303094\\
21	311.421018301041\\
24	311.446439139536\\
27	311.471887859293\\
30	311.4973645011\\
33	311.52286910582\\
36	311.548401714389\\
};
\addlegendentry{Required force};

\addplot [color=mycolor2,solid,mark=diamond,mark options={solid}]
  table[row sep=crcr]{%
0	291.174450058233\\
3	277.302732528785\\
6	258.852292042653\\
9	236.781994493662\\
12	211.655388803588\\
15	182.707467963119\\
18	150.782315705625\\
21	118.023689876921\\
24	85.8538439967556\\
27	55.3620239309924\\
30	28.4892892980397\\
33	8.59563822808\\
36	0.0416064994811875\\
};
\addlegendentry{Drag};

\addplot [color=mycolor3,solid,mark=square,mark options={solid}]
  table[row sep=crcr]{%
0	20.0693996380731\\
3	33.9663439730147\\
6	52.442038862461\\
9	74.5376184528585\\
12	99.6895338627736\\
15	128.662792141937\\
18	160.613309597469\\
21	193.397328424119\\
24	225.59259514278\\
27	256.109863928301\\
30	283.00807520306\\
33	302.927230877739\\
36	311.506795214908\\
};
\addlegendentry{Thrust};

\end{axis}
\end{tikzpicture}%
	\caption{Thrust, drag and required force for 3 [g] deceleration starting from 15 [km] height at $M=5$}
	\label{fig:TDforce}
\end{figure}


The other option is to use a large parachute to decelerate. Since a parachute's performance decreases quadratically with lower velocities, the final landing still requires thrusters to bring the velocity down to an acceptable value for landing \cite{Braun2007}. The difference in fuel mass was estimated by using 

The final touchdown can happen by carefully manoeuvring the spacecraft with thrusters to land on legs. The other option is to land using airbags, as was performed by for example the Mars Pathfinder. However, this induces high peak accelerations during the landing and introduces uncertainties in landing location since the airbag bounces before coming to a stop.

\paragraph{Terminal descent design}
The final design of the terminal descent stage of the mission is chosen te be a retro-propulsion deceleration. Using just the large area of the aeroshell and thrusters leads to a propulsion system that is $200$ $[kg]$ heavier than a system with parachute. This was calculated by assuming the aerodynamic drag and thrust to add up to a constant deceleration force, meaning the thrust would attain higher levels at smaller velocities where the drag force is smaller. Having the thrust allowed for an estimation of the fuel used, assuming a specific fuel consumption of $0.225$ $[kg \cdot kN s^{-1}]$\footnote{\url{http://en.wikipedia.org/wiki/Specific_impulse}. Accessed: 11-06-2015}. The parachute has an estimated mass of $280$ $[kg]$, as calculated using an empirical relation. Thus, using just a thruster is lighter than the thruster and parachute combination.

For the landing phase, it is necessary to jettison the inflatable heat shield from the spacecraft to allow landing legs to deploy from the side of the vehicle.