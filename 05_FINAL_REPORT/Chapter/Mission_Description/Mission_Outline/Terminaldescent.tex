The terminal descent of the spacecraft is the part of the mission between the end of aerocapture and landing on the surface of Mars. The velocity is to be brought back to zero at an altitude of zero, from an initial velocity at the end of the aerocapture phase of the mission. For the terminal descent, several design options are available to decrease the velocity. In this section, firstly, the mission characteristics are discussed. After that, the design options are summarized after which a choice is made based on feasibility and performance.

\paragraph{Terminal descent characteristics}
The start of this part of the mission is given by the end of the aerocapture part, of which the requirements dictate a Mach number of 5 at a height of $10$ [km], as given in Section \label{sec:missionreq}. This means the aerodynamic flow regime changes from hypersonic to supersonic, and finally to subsonic. The speed of sound in the lowest ten kilometres of Mars is approximately 220 [m/s], which means the velocity of the spacecraft is 1100 [m/s] at the beginning of terminal descent. The velocity only marginally increases due to the gravity influence of Mars: the velocity with no deceleration would be 3 percent higher on the surface of Mars than at an altitude of 10 [km].

\paragraph{Design options}
Terminal descent can be split up in three parts: the supersonic flight part, the subsonic flight part and final touchdown. For each part, different options are available.

The first option is to use retropropulsion for every part of the descent. The fuel mass would be about 23.3$\%$ of the total spacecraft mass, although in numerical simulations the interaction between retropropulsion and the aeroshell results in a mass fraction that is approximately twice as small \cite{Korzun2009a}.

The other option is to use a parachute to decelerate very quickly on a nearly constant altitude. Since a parachute's performance decreases quadratically with lower velocities, the final landing still requires thrusters to bring the velocity down to an acceptable value for landing. \cite{Braun2007}.