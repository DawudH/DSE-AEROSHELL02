Terminal descent of the spacecraft commences at 15 [$km$] altitude and is concluded by landing on the surface of Mars. The velocity is to be brought back to zero at an altitude of zero, from an initial velocity of Mach 5 at 15 [$km$] altitude. For the terminal descent, several design options are available to decrease the velocity.% In this section, firstly, the mission characteristics are discussed. After that, the design options are summarised after which a choice is made based on feasibility and performance.

%\paragraph{Terminal descent characteristics}
The start of this mission section is given by the end of the aerocapture segment, of which the requirements dictate a Mach number of 5 at an altitude of 15 $[km]$, see Section \ref{sec:missionreq}. This means the aerodynamic flow regime changes from hypersonic to supersonic, and finally to subsonic during the terminal descent. The speed of sound in the lowest fifteen kilometres of the Martian atmosphere is approximately 220 $[m\cdot s^{-1}]$, which means the velocity of the spacecraft is 1100 $[m\cdot s^{-1}]$ at the beginning of terminal descent. The flight path angle follows from aerocapture and entry as approximately 20 [$deg$].%The velocity only marginally increases due to the gravitational pull of Mars: the velocity with no deceleration would be 3 percent higher on the surface of Mars than at an altitude of 15 $[km]$, assuming no deceleration due to drag or thrust. Finally, the flight path angle is not pre-determined by the orbit, so it can be changed to fit the needs of the terminal descent phase.

Terminal descent can be split up in two parts: the supersonic \& subsonic flight segment and final touchdown. For both parts, different design options are available.

The first option for the flight is to use retro-propulsion for every part of the descent. The fuel mass would be about $23.3\%$ of the total spacecraft mass if no aerodynamic effects are taken into account. However, the aeroshell has a large frontal area which produces a significant amount of drag. Also, in numerical simulations and wind tunnel tests the interaction between retro-propulsion and the aeroshell were found to result in a mass fraction that is approximately half as big as would be expected when considering the thrust and drag forces to act independently of each other \cite{Korzun2009}. Since a blunt body is unstable at transonic and supersonic speeds, a small drogue parachute is needed to stabilise the spacecraft. Scaling a mass estimate for an inflatable aeroshell from NASA, this stabilisation drogue parachute is approximately 20 $[kg]$.

The fuel mass is estimated assuming a constant deceleration of 3\gls{con:ge}. This condition in combination with the initial conditions of the terminal descent leads to a specified flight path angle (equal to 38 $[deg]$) and velocity at each height. Using this velocity the drag was calculated assuming the same drag coefficient throughout the whole supersonic regime. This analysis is known to be incorrect to a certain degree, but since this is a preliminary analysis this is taken for granted. The drag at every height leads to a deceleration lower than 3\gls{con:ge}, and thrust is delivered at a level such that this deceleration is achieved, incorporating the gravitational force. 

The resultant drag, thrust and total required force are shown in Figure \ref{fig:TDforce}. This requires the rocket engines to be sized such that a total thrust of 312 $[kN]$ can be achieved. To this end 3 RL-10A-4 rocket engines are placed at the front of the centre body. The combined mass of these rockets is 504 $[kg]$. The thruster fuel flow is calculated using the specific impulse of the engine, equal to 451 $[s]$, and integrated over time to find the total fuel mass, estimated to be 680 $[kg]$ \cite[p.538]{Wertz2011}. Propellant tank mass is estimated to be 45 $[kg]$ using an empirical relation, assuming a density of 1 $[kg\cdot dm^{-3}]$ for the fuel \cite[p.543]{Wertz2011}.

\begin{figure}[h]
	\centering
	\setlength\figureheight{0.4\textwidth} 
	\setlength\figurewidth{0.7\textwidth}
	% This file was created by matlab2tikz.
% Minimal pgfplots version: 1.3
%
\definecolor{mycolor1}{rgb}{0.00000,0.44700,0.74100}%
\definecolor{mycolor2}{rgb}{0.85000,0.32500,0.09800}%
\definecolor{mycolor3}{rgb}{0.92900,0.69400,0.12500}%
%
\begin{tikzpicture}

\begin{axis}[%
width=0.95092\figurewidth,
height=\figureheight,
at={(0\figurewidth,0\figureheight)},
scale only axis,
xmin=0,
xmax=36.2,
xlabel={$t [s]$},
xmajorgrids,
ymin=0,
ymax=327.127499399344,
ylabel={$F [kN]$},
ymajorgrids,
axis x line*=bottom,
axis y line*=left,
legend style={at={(0.97,0.5)},anchor=east,legend cell align=left,align=left,draw=white!15!black}
]
\addplot [color=mycolor1,solid,mark=o,mark options={solid}]
  table[row sep=crcr]{%
0	311.243849696306\\
3	311.2690765018\\
6	311.294330905114\\
9	311.31961294652\\
12	311.344922666362\\
15	311.370260105056\\
18	311.395625303094\\
21	311.421018301041\\
24	311.446439139536\\
27	311.471887859293\\
30	311.4973645011\\
33	311.52286910582\\
36	311.548401714389\\
};
\addlegendentry{Required force};

\addplot [color=mycolor2,solid,mark=diamond,mark options={solid}]
  table[row sep=crcr]{%
0	291.174450058233\\
3	277.302732528785\\
6	258.852292042653\\
9	236.781994493662\\
12	211.655388803588\\
15	182.707467963119\\
18	150.782315705625\\
21	118.023689876921\\
24	85.8538439967556\\
27	55.3620239309924\\
30	28.4892892980397\\
33	8.59563822808\\
36	0.0416064994811875\\
};
\addlegendentry{Drag};

\addplot [color=mycolor3,solid,mark=square,mark options={solid}]
  table[row sep=crcr]{%
0	20.0693996380731\\
3	33.9663439730147\\
6	52.442038862461\\
9	74.5376184528585\\
12	99.6895338627736\\
15	128.662792141937\\
18	160.613309597469\\
21	193.397328424119\\
24	225.59259514278\\
27	256.109863928301\\
30	283.00807520306\\
33	302.927230877739\\
36	311.506795214908\\
};
\addlegendentry{Thrust};

\end{axis}
\end{tikzpicture}%
	\caption{Thrust, drag and required force for 3\gls{con:ge} deceleration starting from 15 $[km]$ altitude at $M=5$}
	\label{fig:TDforce}
\end{figure}


The other option is to use a large parachute to decelerate. Since a parachute's performance decreases quadratically with lower velocities, the final landing still requires thrusters to bring the velocity down to an acceptable value for landing \cite{Braun2007}. The difference in fuel mass was estimated by using a parachute with a diameter of 30 $[m]$ and a drag coefficient of 0.3, deployed at the moment in time where the added drag of the parachute would make the total acceleration 3\gls{con:ge}. For these conventional figures, the fuel mass loss was approximately 200 $[kg]$, while the added mass of a parachute is approximately 280 $[kg]$ per an empiric relation \cite{Anderson1969}. The absence of mass reduction for adding a parachute, added to the fact that the atmospheric density on Mars offers unacceptable parachute deployment \cite{Korzun2009}, leads to the conclusion that a parachute is not beneficial for the final descent.

Final touchdown can happen by carefully manoeuvring the spacecraft with thrusters to land on legs. These were estimated to have a mass of 200 $[kg]$, as estimated using a structural sizing for a smaller spacecraft to be landing on Mars.\footnote{\url{http://www.nasa.gov/pdf/458812main_FTD_AerocaptureEntryDescentAndLanding.pdf}. Accessed: 18-06-2015} The other option is to land using airbags, as was performed by for example the Mars Pathfinder. However, this induces high peak accelerations during the landing and introduces uncertainties in landing location since the airbag bounces before coming to a halt.
