Whereas Section \ref{sec:missionoutline} covers the entire mission from launch to return on Earth the focus of this report lies with the aerocapture into a parking orbit around Mars, together with the subsequent aerobraking. To achieve this a \gls{cia} has been designed, based on the requirements covered in Section \ref{sec:missionreq}. Even though the mission of the \gls{cia} is concerned with the entry procedure described in Section \ref{sec:entry} the other mission elements also carry an effect on its design. From the launch mentioned in Section \ref{sec:launch} follows that the \gls{cia} must be able to withstand the launch loads and vibrations. 

Following launch, Earth orbit and subsequent acceleration into a heliocentric orbit the interplanetary flight phase of the mission takes place, as described in Section \ref{sec:interplanetary}. From this mission phase comes the requirement for the deceleration capability of the \gls{cia}, a shorter interplanetary transfer time results in a higher velocity with respect to Mars. This is further covered in Section \ref{sec:missionreq}. 

When the capsule carrying the crew arrives at Mars with its accompanying modules and systems required for interplanetary transfer the actual mission of the \gls{cia} takes place. It is this mission phase that forms the scope of this report and is where the \gls{cia} performs its function. The aerocapture, parking orbit and subsequent entry and terminal descent procedure can take up to ten days altogether. After the terminal descent \& entry procedure has been initiated the aeroshell will be retained to aid in the final descent and deceleration until touchdown.

As such, the considerations in this chapter on the entire mission and the crew capsule design in Chapter \ref{ch:crewmod} are conceptual suggestions for further design efforts. To this end, their main purpose is to investigate the compatibility of the \gls{cia}, crew module and mission. Other design options for the crew module remain possible.
