In order to successfully operate the mission during the Earth to Mars transfer including the \gls{edl} phase, several components require electrical power supply and management. Although multiple  energy sources exist, photovoltaic energy is already a known and widely applied technology. Also, during the interplanetary transfer, sunlight will almost always be available. Therefore, photovoltaic energy will be used as the primary energy source.\\
 
Before performing an aerobrake however, solar panels must obviously be retracted. Hence, during the \gls{edl} phase the vehicle will run on batteries or regenerative fuel cells. When the vehicle goes into an orbit around Mars, the solar arrays can be re-deployed in order to recharge the batteries. This will reduce the energy demand as well as battery mass.\\

Several elements require a constant power supply. Among these elements are the life support system, thermal contact system, a galley, airlock, communications, personal quarters, command centre, health maintenance facility, data management system, audio \& video facilities, a science lab, hygiene, vehicle control and the propulsion system. \gls{nasa} has made an estimate for a $30$ $kWe$ power system for a crew of six with a corresponding mass of 500 kilograms \cite{Hoffman1997a}. Yet, the current mission is only one of three members. By linearly scaling down all crew-dependent elements, the power required can affectively be reduced to $16.7$ $kWe$. Assuming that the power need scales linearly with the total mass of the power subsystem, this will result in a power subsystem mass of approximately $280$ kilograms.\\