The structure of the crew module serves the important function of connection all the individual subsystems of the crewmodule and moreover connects with the \gls{hiad}. The main scope of the design lies within this \gls{hiad}. New advances from for example the currently being developed and Orion mission are therefore not considered.

A schematic layout as for example also used in the Orion spacecraft\footnote{URL: \url{http://www.spaceflight101.com/orion-spacecraft-overview.html}, Accessed 11 June 2015 }, features a Aluminiumgrid structure. This structure encloses the pressurized volume inhabited by the astronauts. The grid structure allows for easy attachment of the individual subsystems. Subsystems which require pressurization, typically those involving the astronauts can be placed within in this shell, whereas the systems that do not require pressurization are placed on the outside of this shell. A more detailed analysis on where each of these individual subsystems are placed is discussed in Section \ref{sec:crewpackaging}.


A full estimate of the structural mass is only to be provided in later design phases. More detailed structural estimates are typically provided by detailed \gls{cad} models and \gls{fem} models \cite{Wertz2011}. A rough estimate can be provided on the basis of previous reference missions. \glspl{nasa} Orion mission is again of primary interest as it also features astronauts. Some differences with respect to the structural elements can however be noted:

\begin{itemize}
\item The heat shield is incorporated within the structure
\item A additional backshell structure is provided
\end{itemize}


In the design at hand the heatshield structure is incorporated in the \gls{hiad}. The crew module is merely connected to this \gls{hiad} of which the latter is designed in more detail in the remaining contents of this report. Due to the implementations of the \gls{hiad} a additional back shell structure is also no longer required. The back shell normally functions a protection against thermal loading which moves sideways along the body. Using a large frontal inflatable this is prevented denoting one of the advantages of using a inflatable structure \cite{Hughes2005}.

For this reason the heatshield carrier structure of around 1500 [$kg$] \cite{Ainsworth2014} is not taken into account into the mass estimate of the structure. A total structural mass for manned re-entry vehicles lies at around 30 \%. The manned Apollo mission featured a 31\% structural mass fraction\footnote{URL: \url{http://braeunig.us/space/specs/apollo.htm}, Accessed 11 June 2015}. Equating with this value, combined with 9000[$kg$] crewmodule mass yields mass of around 1300[$kg$] excluding the heat shield structure. This is in line with values suggest by Wertz et al. \cite{Wertz2011}. Taking into account a 30 \% mass contingency factor yields final structural mass estimate of around 1700[$kg$]




