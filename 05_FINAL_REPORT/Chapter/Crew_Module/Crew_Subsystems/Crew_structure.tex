The structure of the crew module serves the important function of connection all the individual subsystems of the crew module and moreover connects with the \gls{hiad}. The main scope of the design described in this report lies within this \gls{hiad}. New advances from for example the currently being developed and Orion mission are therefore not considered. The Orion capsule can already be considered state of the art and is in a large amount representative for the crew module design.

A schematic layout as for example also used in the Orion spacecraft\footnote{URL: \url{http://www.spaceflight101.com/orion-spacecraft-overview.html}, Accessed 11 June 2015 }, features a Aluminium grid structure. This structure encloses the pressurised volume inhabited by the astronauts. The grid structure allows for easy attachment of the individual subsystems. Subsystems which require pressurisation, typically those involving the astronauts can be placed within in this shell, whereas the systems that do not require pressurisation are placed on the outside of this shell. A more detailed analysis on where each of these individual subsystems are placed is discussed in Section \ref{sec:crewpackaging}.


A full estimate of the structural mass is only to be provided in later design phases. More detailed structural estimates are typically provided by detailed \gls{cad} models and \gls{fem} models \cite{Wertz2011}. A rough estimate can be provided on the basis of previous reference missions. \glspl{nasa} Orion mission is again of primary interest as it also features astronauts. Some differences with respect to the structural elements thereof can however be noted:

\begin{itemize}
\item Orion incorporates an integrated heat shield and structure
\item Orion features a backshell, not required for the inflatable aeroshell by its larger deployed diameter
\end{itemize}

In the design at hand the heat shield structure is incorporated in the \gls{hiad}. The crew module is merely connected to this \gls{hiad} of which the latter is designed in more detail in the remaining chapters of this report. Due to the implementations of the \gls{hiad} a additional backshell structure is also no longer required. The back shell normally functions a protection against thermal loading which moves sideways along the body. Using the large frontal of the inflatable this is prevented, denoting one of the advantages of using a inflatable structure \cite{Hughes2005}. The crew module structure should however also be sized considering , and may as such not be too tall and may feature a tapered end such that the crew module is not exposed to thermal loading passing the \gls{hiad}.

For this reason the heat shield carrier structure of around 1500 $[kg]$ \cite{Ainsworth2014} is not taken into account into the mass estimate of the structure. A total structural mass for manned re-entry vehicles lies at around 30 \%. The manned Apollo mission featured a 31\% structural mass fraction \footnote{URL: \url{http://braeunig.us/space/specs/apollo.htm}, Accessed 11 June 2015} including a heat shield structure. Extrapolating this value, with a 9000 $[kg]$ crew module mass yields a structural mass of 1300 $[kg]$, excluding the heat shield structure. This is in line with values suggested by Wertz et al. \cite{Wertz2011}. Taking into account a 30 \% mass contingency factor yields a final structural mass estimate of around 1700 $[kg]$. A similar mission featuring a descent towards Mars from 7 [$km \cdot s^{-1}$] has a structural mass of 517 $[kg]$ on a dry mass of 2863 $[kg]$ including contingency factors\footnote{URL: \url{http://www.nasa.gov/pdf/458812main\_FTD\_AerocaptureEntryDescentAndLanding.pdf} , Accessed 11 June 2015 }. Scaling this value yields a similar mass estimate of around 1800 [$kg$]. 

The connection between the crew module and the \gls{hiad} is taken into account in the capsule structural mass estimate of 1300 $[kg]$. 