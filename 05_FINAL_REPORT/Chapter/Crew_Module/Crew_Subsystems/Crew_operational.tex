In this section the operational items are sized. This can be summarised as the mass needed by the astronauts to stay alive in the crew module during the mission. For this purpose the paper by Tito et al \cite{tito2013} has been used. In this paper the operational items are called \textcolor{red}{GLS} Environmental Control and Life Support System (ECLSS). First the method for estimation is described with its assumptions. Followed by the results of the estimation.

\paragraph{Estimation method}
\label{par:operationalest}
The mass of the \textcolor{red}{ECLSS (GLS)} is primarily driven by the crew size and mission length. The \textcolor{red}{ECLSS (GLS)} is divided into subsystems: Air Management, Thermal and Humidity Management, Water Management, Waste Management, Human Accommodation, Food Preparation and Storage. Each of these subsystems can be subdivided into components. Examples of these are a water heater or packed food in the Food Preparation and Storage. It is evident that some components scale with the keydrivers and others do not. For example, adding a crew member does not necessitate an extra water heater, but it does require extra packed food. 


Taking this into account the mass has been divided into two components. A basic system mass which scales with crew size and the consumable mass that scales with crew size and mission length.





afasasfasf
\paragraph{Results}



