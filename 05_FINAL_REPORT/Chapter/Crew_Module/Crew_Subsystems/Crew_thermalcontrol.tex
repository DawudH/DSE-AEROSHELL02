Where the \acrfull{tps} is used to protect the crew module from excessive heating during re-entry, the \acrfull{tcs} is used to keep other subsystems in the crew module within their operating temperature limits. It is assumed that re-entry phase does not impose extra requirements on the \gls{tcs} as it is completely covered by the \gls{tps}. Note that this only holds when the angle of attack ($\alpha$) is low enough such that the crew module stays out of the wake. Furthermore, the paper by Tito used in the mass estimation for the operational items already assigns a mass for the thermal control within the living compartments of the crew module \cite{Tito2013}. The paper calculates a mass of $479$ $[kg]$ for a crew module suitable for the life support of two astronauts. Therefore this part only focusses on the thermal control of components that are not placed within the living compartments.

Examples of these components that need to operate at the edge or outside of the crew module are star trackers from the \textcolor{red}{ADCS} or the antennas from the telecommunications. To provide typical temperature limits, Table 22-9 from Reference \cite{Wertz2011} is provided in Table \ref{tab:cmtherm}. In here there is a distinction between operational and surviving temperatures. From this table it is evident that components that operate outside the spacecraft can typically handle a wider temperature range than components that operate within the crew module.

\begin{table}[h]
	\centering
	\caption{Typical temperature requirements for different components}
	\begin{tabular}{|l|ll|}
		\hline
		\textbf{Equipment} & \textbf{Operational} & \textbf{Survival}\\ \hline \hline
		Avionics Baseplates & -20 to 60 & -40 to 75 \\
		Batteries & 10 to 30 & 0 to 40 \\
		Hydrazine Fuel & 15 to 40 & 5 to 50 \\
		Solar Arrays & -150 to 110 & -200 to 130 \\
		Antennas & -100 to 100 & -120 to 120 \\
		Reaction Wheels & -10 to 40 & -20 to 50 \\
		\hline
	\end{tabular}
	\label{tab:cmtherm}
\end{table}

In order to keep the subsystems within their operative temperature range the \gls{tcs} uses different tools and techniques. According to Karam the most commonly used are coatings, insulators and isolators, heaters, louvers and heat pipes \cite{Karam1998}. \textcolor{red}{Explain more about these tools and techniques...}
According to SMAD the \gls{tcs}-mass ranges from 3\% to 10\% with an average of 6\% for the dry mass of a interplanetary spacecraft. Note that these spacecraft are not aimed to carry astronauts, therefore it is assumed that this 6\% adds on top of the 479 $[kg]$ calculated by Tito. This would add an extra 600 $[kg]$ dedicated to the \gls{tcs}.
