Concepts have been evaluated on the basis of the following four criteria: decelerator mass, deceleration time, stability and development risk. These are discussed hereafter.

\subsubsection{Decelerator mass}
To take full advantage of launcher capability, the total vehicle mass is kept at its maximum. An increase in decelerator mass then leads to a decrease in payload mass, so it is essential that decelerator mass is kept to a minimum. To this end, the three primary components making up decelerator mass were evaluated for each concept: \gls{tps} mass, structural mass and control system mass. Their weighted average was computed to yield a total mass, taking into account their respective significance. The weight factors were determined from a comparable re-entry vehicle, namely \gls{irve} \cite{Hughes2005}.

The relative structural mass was determined using the structural mass estimation tool described in section \ref{subsec:structool} and in more detail in the Mid-Term Report \cite[p.47-66]{Balasooriyan2015b}. Relative \gls{tps} mass is reflected by the estimated peak heat flux, a first-order estimation of the thermal energy to be dissipated and a key design driver for the \gls{tps}. Relative control system mass is reflected by the control moment required to be effected by the control system, in the form of the moment coefficient following from the aerodynamic analysis using modified Newtonian flow theory, as described in section \ref{subsec:aerotool}. This was characterised by the lift-to-drag ratio, to account for the difference in lifting capability between concepts. Lower peak heat flux and low moment coefficients are favourable in terms of mass. 

\subsubsection{Deceleration time}
Minimising deceleration time is favourable for minimising ground operations expenses, since ground control is required to be fully active at the time of entry, which is the most critical mission phase. Furthermore, taxation of crew members is then alleviated. The time spent in the atmosphere is reflected by vehicle lift-to-drag ratio. For a given \gls{sym:CD}\gls{sym:A}, the maximum deceleration can be chosen by varying the lowest part of the orbit: the density in lower parts of the atmosphere is higher, which then compensates for a low drag coefficient to produce the same force as a spacecraft with a high drag coefficient at a higher altitude with lower density. Because of the large variation of density in the atmosphere, it is possible to find a trajectory for any \gls{sym:CD}\gls{sym:A}. Thus, the drag coefficient itself is not a key driver for the design. However, the spacecraft can influence its deceleration time in the atmosphere by producing lift: if the spacecraft were to fly out of the atmosphere, a downward pointing lift would divert it's trajectory more through the atmosphere. The ability of the spacecraft to influence this trajectory through the atmosphere is characterised by the amount of lift that can be produced, with respect to the amount of drag produced at the same \gls{sym:alpha}. The dependence on drag is due to the fact that two spacecraft with the same lift-to-drag ratio but a different \gls{sym:CD}\gls{sym:A}, will just have the lowest part of the trajectory at a different altitude, where the total lift and drag force will be the same for both spacecraft. Therefore, the deceleration time is characterised bythe  lift-to-drag ratio. 

Lift and drag coefficients follow from the aerodynamic analysis tool, described in detail in the Mid-Term Report \cite[p.34-46]{Balasooriyan2015b}.

\subsubsection{Stability}
Vehicle stability is preferable, since a stable vehicle will react to disturbances with a restoring moment to revert to its original equilibrium condition without requiring control system activity. Not only does this reduce required control system activity, thereby limiting the system mass, but in addition the vehicle is more robust and less susceptible to perturbations. Stability is reflected by the static stability coefficient of concepts, following from aerodynamic analysis.

\subsubsection{Development risk}
It is key that concepts are evaluated for their development risk, an indication of schedule and cost risk. A concept with a high development risk will require extensive investigation to fully explore its capabilities and mitigate risks by technical uncertainty associated with such an underdeveloped concept. These investigation efforts incur additional cost and schedule risk. Development risk of concepts is evaluated by their \gls{trl}, denoting the current state of testing and application.



