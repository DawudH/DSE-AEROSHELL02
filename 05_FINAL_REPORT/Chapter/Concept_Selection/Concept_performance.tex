In terms of decelerator mass, the isotensoid was estimated as the lightest concept, followed upon by the stacked toroid, illustrated by Table \ref{tab:cmass}. Tension cone and trailing ballute were notably heavier, primarily due to a higher structural mass. From this mass analysis, mass benefits of an inflatable versus rigid concepts were clearly identifiable. On the basis of reference missions and scaling of estimated mechanical and thermal loading, decelerator thermo-structural mass was estimated at nearly 3000 $[kg]$. Key contributor was the backshell weighing well over 1400 $[kg]$ required for a rigid concept, as opposed to the inflatable concepts. This mass was far in excess of the 1000 $[kg]$ limit imposed on maximum decelerator mass.
\begin{table}[ht]
\centering
\caption[Concept mass comparison]{Concept mass comparison (expressed as percentage of stacked toroid mass)}\label{tab:cmass}
\begin{tabular}{|p{0.2\textwidth}|p{0.2\textwidth}|p{0.2\textwidth}|p{0.2\textwidth}||p{0.079\textwidth}|}

\hline
                          & \textbf{Structural mass (20\%)} & \textbf{Thermal mass (50\%)} & \textbf{Control system mass (15\%)} & \textbf{Total mass} \\ \hline
\textbf{Stacked toroid}   &  100                                 & 100                          & 100                                      &\cellcolor{green!70}  100                           \\ \hline
\textbf{Tension cone}     &  168                               & 100                               &  100                                     &\cellcolor{yellow!70} 116                                 \\ \hline
\textbf{Trailing ballute} &  221                                 & 84                               & 67                                      &\cellcolor{yellow!70} 113 \\ \hline
\textbf{Isotensoid}       &  110                                 & 76                               & 96                                      &\cellcolor{green!70} 88 \\ \hline \hline
\textbf{Rigid}            &  \multicolumn{4}{|p{0.762\textwidth}|}{\cellcolor{red!60} ~~~~~~~~~~~Estimated 3000 $[kg]$: Far in excess of 1000 $[kg]$ limit}    \\ \hline
\end{tabular}
\end{table}

In terms of deceleration time, lift-to-drag ratio and thereby deceleration performance of the rigid concept was highest, that of the isotensoid notably lowest and those of the other three inflatable concepts in between and comparable. In terms of concept static stability, the isotensoid again performed notably worst, being unstable, whereas the rigid concept proved neutrally stable and the other three inflatables stable. These results are illustrated by Tables \ref{tab:decel} and \ref{tab:stab}.
\begin{table}[ht]
\caption{Review of concept deceleration time}
\hspace{-10mm}
\begin{tabular}{|p{2.5cm}|c|c|c|c|c|}
\hline
\textbf{}                          & \textbf{Stacked toroid} & \textbf{Tension cone} & \textbf{Trailing ballute} & \textbf{Isotensoid} & \textbf{Rigid} \\ \hline
\textbf{Lift-to-drag ratio} & \cellcolor{yellow!75} -0.176  &\cellcolor{yellow!75} -0.176   &\cellcolor{yellow!75} -0.210 & \cellcolor{red!60} -0.072 &\cellcolor{green!70} -0.311              \\ \hline
\textbf{Deceleration performance} &\cellcolor{yellow!75} Adequate &\cellcolor{yellow!75}  Adequate  &\cellcolor{yellow!75} Adequate & \cellcolor{red!60}     Poor       &\cellcolor{green!70} Excellent                 \\ \hline
\end{tabular}
\label{tab:decel}
\end{table}

\begin{table}[ht]
\caption{Review of concept stability}
\hspace{-10mm}
\begin{tabular}{|p{2.5cm}|c|c|c|c|p{2cm}|}
\hline
\textbf{}                          & \textbf{Stacked toroid} & \textbf{Tension cone} & \textbf{Trailing ballute} & \textbf{Isotensoid} & \textbf{Rigid} \\ \hline
\textbf{Static stability} &\cellcolor{green!70} Stable  &\cellcolor{green!70}  Stable   &\cellcolor{green!70} Stable & \cellcolor{red!60}   Unstable          &\cellcolor{yellow!75} Neutrally stable                 \\ \hline
\end{tabular}
\label{tab:stab}
\end{table}
Technology readiness, reflected by the \glspl{trl} in Table \ref{tab:gls_rev}, is highest for the conventionally tested and flown rigid concept. The stacked toroid concept was flown in a multitude of NASA's \gls{irve} missions and prototypes thereof have thus been tested in a relevant environment. The other three inflatables have received notably less attention, having solely undergone wind tunnel and laboratory testing. In addition, the difficulty of controlling a trailing ballute using conventional methods necessitates the use of morphing. As morphing is a relatively underdeveloped concept and has only been formulated in theory for trailing ballute configurations, the \gls{trl} of the trailing ballute reflects this by the lowest \gls{trl} of all concepts.
\begin{table}[ht]
\centering
\caption{Review of concept development risk}
\begin{tabular}{|l|l|l|l|l|l|}
\hline
\textbf{Concept {[}-{]}} & \textbf{Stacked toroid} & \textbf{Tension cone} & \textbf{Trailing ballute} & \textbf{Isotensoid} & \textbf{Rigid} \\ \hline \hline
\textbf{TRL {[}-{]}}     &\cellcolor{green!70} 7  &\cellcolor{yellow!75}  4   &\cellcolor{red!60} 2 & \cellcolor{yellow!75}      4          &\cellcolor{green!70} 9     \\ \hline
\end{tabular}
\label{tab:gls_rev}
\end{table}






