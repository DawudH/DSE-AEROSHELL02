It is essential that mass contributions of centerbody, inflatable structure and connections are estimated in order to verify that decelerator mass is below the 1000 [kg] limit imposed. Moreover, the parametric mass model for the inflatable structure allows identification of design venues to minimize structural mass of the decelerator, such that decelerator mass can be minimized and thereby payload mass maximized. 

\subsubsection{Mass estimation method}
Mass estimation for the inflatable structure is performed on the basis of a parametric mass model proposed by Samareh \cite{Samareh2011}. The mass model is based on a number of stress equations and cone deformation to compute:
\begin{itemize}
\item Flexible material mass
\item Inflation gas mass
\item Inflation system mass
\end{itemize}
on the basis of the following input
\begin{itemize}
\item Geometry parameters (centerbody and deployed diameter, half-cone angle, toroid dimensions)
\item The number of toroids
\item External loading
\item Vehicle drag coefficient
\item Inflation gas properties
\item Material properties
\end{itemize}
Moreover, the model calculates inflation gas pressure based on the premise that work done by inflation gas and aerodynamic pressure should be equal, in line with relations established by Brown \cite{Brown2009}. The model is based on the assumption that the inflatable is an axisymmetric sphere cone of constant half-cone angle, thereby treating a simplified model of the shape determined by aerodynamic optimization in Chapter \ref{ch:XXX}. 

Mass estimation for the centerbody structure is performed on the basis of the basic sizing performed in section \label{sec:struc_Centerbody} as the amount of material required to withstand buckling and yielding failure criteria at the ultimate load. Connections are taken into account in the form of a mass margin. [ADD MASS MARGIN] 

\subsubsection{Mass estimation results}
Mass estimation for the inflatable structure using the parameters previously defined in section \label{sec:struc_InflatableLoad} yields the masses given in Table \ref{tab:mass_est}. 

From these masses, ...

For the centerbody, a thickness of x [mm] was determined with

[CONNECTORS]

[CONCLUSION TOTAL MASS]

