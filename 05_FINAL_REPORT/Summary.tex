\section*{Summary}\label{cha:summary}
Inflatable aeroshell concepts hold the key to what is currently unattainable for interplanetary human spaceflight. In the wake of current \acrshort{nasa} investigations on the feasibility of inflatable decelerators for hypersonic guidable entry, this study focuses on the preliminary design of a Mars entry vehicle with a payload mass of at least 9000 $\left[kg\right]$ using an inflatable aerodynamic decelerator of at most 1000 $\left[kg\right]$. Such a solution provides a large economical advantage over conventional solutions by maximizing payload-carrying capability through a light-weight device less burdened by launcher size considerations.
\newline
\newline
To aid in the design of such a vehicle several tools have been developed with the following purposes:

\begin{itemize}
\item A tool for parametric structural mass modeling
\item A modified Newtonian flow aerodynamic tool for the characterisation of aerodynamic and aerothermal behaviour and shape optimisation
\item A thermal model for \acrfull{tps} sizing and analysis
\item An astrodynamic tool with an implemented control system for trajectory control
\end{itemize}
\vspace{1mm}
This results in a vehicle that has an undeployed diameter of 5 $ \left[ m \right] $ and a deployed diameter of 12 $ \left[ m \right] $. It is designed for an entry velocity of 7 $ \left[ km \cdot s^{-1} \right] $ and a final velocity of Mach 5 at  15 $ \left[ km \right] $ altitude within a horizontal precision range of 500 $ \left[ m \right] $. To this extent the 10 000 $ \left[ kg \right] $ vehicle has a aerodynamic decelerator mass of 928 $ \left[ kg \right] $ including contingency. Due to the human payload the mission is sized for an acceleration under 3 \gls{con:ge}. Furthermore, the maximum mission duration is 10 days, where two periods of aerodynamic deceleration exist. Both phases, aerocapture and entry, will spend up to 800 $ \left[ s \right] $ in the Martian atmosphere. The vehicle adheres to \acrshort{cospar} regulations and has a control system accuracy of $ 5 \cdot 10^{-4} $.
\newline
\newline
A key feature of the aeroshell design is a asymmetric, skewed shape. The asymmetry follows from aerodynamic optimisation and yields higher lift-generating capability at lower angles of attack to firstly achieve more lift and secondly require smaller angles of attack to keep the crew module from being exposed by the flow. Aerodynamic performance is characterised by a 0.35 lift-to-drag ratio and a 22.5 $\left[deg\right]$ trim angle of attack.
\newline
\newline
The asymmetry is adopted by the structural shape through stitching of ten inflatable toroids at a variable half-cone angle with respect to one another. Structural rigidity under an ultimate aerodynamic pressure of 3500 $\left[Pa\right]$ is ensured by the use of a nitrogen blow-down system that inflates five bladder volumes at 169 $\left[kPa\right]$, which keeps the flexible bladder material in tension to prevent compressive wrinkling. Resulting loads are carried by woven Zylon fibres of 0.125 $\left[mm\right]$ thickness at a 95 $\left[kg\right]$ mass. At a minimum half-cone angle, the structural mass is estimated at 300 $\left[kg\right]$. 
\newline
\newline
The \acrlong{tps} is exposed to a peak heat flux of 21 $\left[W \cdot cm^{-2}\right]$ and a peak temperature of 1376 $\left[K\right]$ during aerocapture. This thermal loading is withstood by a multi-material lay-up 256 $\left[kg\right]$ consisting of a state-of-the-art Nicalon barrier of 0.51 $\left[mm\right]$ thickness and Pyrogel 6650 insulator of 2.4 $\left[mm\right]$ thickness, complemented by dual 25 $\left[\mu m\right]$ Kapton gas barriers. 
\newline
\newline
Compatibility of the aeroshell with a manned Mars mission is ensured by preliminary crew module and mission design. The crew module accommodates two crew members for a 100-day interplanetary mission and its mass is estimated at 9000 $\left[kg\right]$. Return from Mars requires two launches prior to crew module launch, which respectively bring the \acrlong{mav} onto Mars and an \acrlong{erv} in an orbit around Mars. Mission cost is estimated at 44 billion US dollars.
\newline
\newline
Recommendations are a propagation of design on decelerator and crew module, testing activities, and crew and mission preparation thereafter. Key driver for further design is concept reliability. Deployment, inflation and terminal descent are critical mission phases and inherently unreliable for an inflatable aeroshell design. These therefore require particular attention in future design.
