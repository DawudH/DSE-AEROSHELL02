
\section{Guido van Koppenhagen}
At the start of the fourth period, work began on the design synthesis exercise.  It started with a week of project planning, which was followed by an initial analysis of the project and the requirements associated with a controllable inflatable aeroshell.  During this time I worked on the workflow of the project, the requirement analysis and on the literature review of re-entry aerodynamics.  

After the initial project start-up and analysis, I focused on the aerodynamic analysis of the various concepts. Specifically, I focussed the section of code which creates the geometry and discretises it for use by the Newtonian flow solver. 

I am happy with the way the group works together and am pleased with the work produced so far. Discussion within the group is productive and generally has participation from the entire group.  I do not believe there is a lot to improve in terms of the group. One potential pitfall for this group is the universal preference for technical work, which could lead to neglected managerial tasks and below average performance on non technical aspects of the work to be done. 

The facilities and support provided by the university are generally good. One potential improvement in this area is communication to the students (specifically from the managing DSE staff), as deadlines and methods of delivery are not always clearly specified and conflicting information on various deliverables can be found.  

   
