\section{Lucas Mathijssen}
During the span of the ongoing DSE there are several tasks that I have performed so far. Those can be split up in two main directions. My main (technical) task was being part of the thermodynamics department, whereas the secondary task was being the secretary.\\
In the thermodynamics department I first obtained general knowledge by reading relevant papers, websites, books, etc. Afterwards I focussed more on the thermal protection system. There I obtained knowledge on the aeroshell thermal protection systems. I focussed on the general layup, materials used and corresponding material properties. Furthermore, I also got into the technical details. This included updating knowledge on the corresponding equations and technical terms. Afterwards, Suthes and I started working on a tool that could calculate the temperature difference over input lay-ups. Unfortunately, there was an error in this program that could not be solved. Therefore, Suthes and I made a new plan, where the heat flux was integrated to a heat load. I worked on the programming of this and also on generating the relevant plot. \\
For all these tasks, there were corresponding tasks like meeting, writing down the results in the reports and giving a presentation. \\
As a secretary I made minutes of the status meetings with the group and tutors. This task required writing during the meetings and processing the text afterwards. To do so in a consistent way, I set up a minute format. Furthermore, I tried to keep track of the action items. Those tasks are the tasks that are not clearly the responsibility of a department or person.  Moreover, I kept track of the tasks of all the group members after each ‘end of day’ meeting. I put all previously mentioned items together in one file, the logbook. For this book, I another made a format. As a last secretary tasks, I was responsible for all internal and external communication. This task included keeping the mailbox ordered and being the person through which all mail communication flows. It also included reserving meeting rooms and communicating this information to team members and tutors.\\
I think I could have performed better as a secretary by keeping the logbook more up-to-date. However, this would have reduced the available technical time. Over all I am satisfied with my performance, and I am looking foreward to working on a more detailed design.\\
My impression of the team is good. The team is working hard to deliver assignments on time and we have obtained a lot of knowledge on a short time scale. What is very typical, however, is that the team is having a lot of discussions about all sorts of technical details. The efficiency of the group can be improved by keeping some of these discussions in sub groups. On the other hand in conseptual work the team needs to be awair of a lot of discissions.\\
The organisation of the DSE is in my view poor. There are differences in documents provided by the tutors or by the OSSA's, reserving meeting rooms is inconsistant and mail contact via the OSSA's is seldomly fluent. There could be great improvement in the organisation.\\
Help from the tutors has been helpfull. There is a clear distinction between the role of the customer and the tutor. Also, the tips and knowledge shared by the tutors is given only if needed and it usually is a push in the right direction.\\