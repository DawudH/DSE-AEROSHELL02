\section{Bjorn van Dongen}
This chapter details the personal contributions of Bj{\"o}rn van Dongen as performed in the Design Synthesis Exercise (DSE) up to and including the Mid-Term Review (MTR).

As structural analyst, my function up to the Baseline Review (BR) was to investigate the means to retain structural stiffness primarily for inflatable concepts. Hereafter structural analysis of the five concepts prior to the MTR was difficult by their structural diversity. Since structural performance is reflected by mass, based on the premise that mass can be added until sufficient structural integrity is achieved, I implemented mass estimation tools for stacked toroid, tension cone and trailing ballute concepts and analysed their outputs. 

I have taken measures to limit this analysis to what is needed to make a well-informed decision in the trade-off. On one hand this proved necessary to maintain a good overview and ensure proper reports, but on the other hand I feel dissatisfied that structural analysis has remained limited to parametric models, albeit extensive ones. After the MTR, however, structural analysis of the selected concept will proceed in more depth.

Moreover, my technical function has extended to work-supporting activities, such as the work definition, design interfaces and market analysis. Most importantly, I have been in charge of retaining the overview, structuring and documentation of the trade-off process.

As editor, a significant part of my time has been put into the written deliverables. I have set up a set of guidelines to which team members shall adhere, actively controlled that the guidelines are adhered to, proof-read all parts for contents-wise and textual correctness, implemented corrections and set up report and chapter structure. My performance in this function is reflected by the reports delivered, both in the chapters of other group members as well as in those chapters not corresponding to a technical department, such as the trade-off.

I feel satisfied with my performance as editor, but it has required me to commit a significant amount of time otherwise spent on structural analysis. Overall I would say that the time spent on completing parts aside from the structural analysis has been out of proportion with respect to other team members and their organisational functions, with the exception being the secretary and planner. Other team members have been so taken up in their tools at times, that other contributions are often lacking and intervention, by myself and Alexander mostly, is necessary to ensure that all content is covered.

\begin{table}[h]
\caption{Written contributions of B.R. van Dongen}
\centering
\begin{tabular}{|p{0.1\textwidth}|p{0.565\textwidth}|}
\hline
 \textbf{Report}   & \textbf{Written}                                                                                             \\ \hline
BR  & Section 2.3, Ch 4, Section 5.4, Ch 7, Ch 10                                                           \\ \hline
MTR & Introduction, Summary, Ch 2, Ch 3, Ch 5, Ch 6 (partially), Ch 7 (partially), Ch 10, Ch 13, Conclusion \\ \hline
\end{tabular}
\end{table}

I highly enjoy and appreciate the dedication and company of all group members. Work output is high for all group members and I find the work experience enjoyable. The only issue of note is that deadlines have not always been strictly adhered to, primarily the deadline of tool development.


