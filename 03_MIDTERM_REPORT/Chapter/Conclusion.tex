\section{Conclusion}\label{cha:conclusion}
Four trade-off criteria were used to reflect vehicle capability of meeting customer demands: decelerator mass, development risk, vehicle stability and deceleration time. To the end of analysing concepts for their performance in these aspects, four tools have been succesfully implemented and validated: a structural mass estimation tool, a thermal analysis and sizing tool, an astrodynamic tool and an aerodynamic tool. On the basis of these tools, the following can be concluded.
\newline
\newline
Parametric structural mass modelling yields relative structural masses of the tension cone, trailing ballute and isotensoid of 168, 221 and 110 \% of stacked toroid structural mass; \acrfull{tps} mass modelling on the basis of heat flux incurred yields relative \gls{tps} masses of 100, 84 and 76 \% of stacked toroid thermal mass; control system mass estimation yields relative masses of 100, 34 and 122 \% of stacked toroid control system mass. The rigid concept has an estimated combined thermal and structural mass of 2945 [kg], well in excess of the imposed 1000 [kg] limit. Combining the decelerator mass components by assigning 20, 50 and 15 \% of total decelerator weight (thus excluding connection mass) to structural, thermal and control subsystems respectively yields tension cone, stacked toroid, isotensoid and rigid concept decelerator masses of 116, 107 and 92 \% of stacked toroid mass. A low decelerator mass is preferred, since it allows a higher payload mass for full use of launcher capability.
%The stacked toroid is therefore deemed to be the lightest concept, while tension cone and trailing ballute concepts perform similarly. Isotensoid performance is %notably worse, while rigid concept performance is unacceptable.
\newline
\newline
Development risk is lowest for the conventionally applied rigid configuration, corresponding to \acrfull{trl} 9 by extensive application and testing in relevant environments. Research investigations by NASA have brought the stacked toroid configuration to \gls{trl} 7; the other three inflatable concepts have only undergone wind tunnel testing and analysis to a limited extent such that these have been designated \gls{trl} 4. The development risk of the trailing ballute is further amplified by a lack of thrusters and a \gls{cg} offset as viable control options, thus requiring the use of relatively underdeveloped and untested (for this application) morphing. This is reflected by \gls{trl} 2 for the trailing ballute. A high \gls{trl} is preferable, since it incurs less development risk.
\newline
\newline
Aerodynamic analysis of concept stability and deceleration time yielded the following. The stacked toroid and tension cone, based on the same aerodynamic model by similar shape, as well as the trailing ballute were found both statically stable; the rigid concept was found neutrally stable and the isotensoid unstable. Static stability is preferred since it alleviates control system requirements and improves adherence to the intended trajectory. In terms of deceleration time, the rigid concept performed best. This was followed by the stacked toroid, tension cone and ballute concept all performing average. Finally relatively poor performance is achieved by the isotensoid concept.
\newline
\newline
Future design work follows after design trade-off in the \acrfull{mtr} in conversation with the customer. Recommendations for future work entail preliminary design and analysis of the selected concept. This phase is aided by enhanced versions of the tools developed and described in this Baseline Report. 



