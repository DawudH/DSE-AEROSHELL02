\section{Control system mass estimation}
\label{sec:controlmass}
Using the results produced by the aerodynamic tool shown in Section \ref{subsec:appaeroanal} the relative masses of the control systems for each proposed structural concept can be determined. Section \ref{subsec:controlmoments} will derive the magnitudes of the control system moments, after which these will be used in Section \ref{subsec:controlmassest} to estimate the relative masses of the control systems required by each respective system-level concept.

\subsection{Required control system moments}
\label{subsec:controlmoments}
In Section \ref{subsec:appaeroanal} figure \ref{fig:cm} the \glspl{sym:cma} required for each concept are shown for a range of angles of attack (\gls{sym:alpha}). As can be seen in figure \ref{fig:cl} the \glspl{sym:CL} delivered by the various concepts are different at all angles of attack. However, the astrodynamics tool described in Chapter \ref{ch:astrocontrol} uses \glspl{sym:CL} to control the orbit of the spacecraft. Since \gls{sym:CM} and \gls{sym:CL} both vary with angle of attack \gls{sym:alpha} the angles of attack required for the correct orbits will vary for each concept. To adjust for these variations the ratio $\frac{\gls{sym:CM}}{\gls{sym:CL}}$ is used to estimate the control system masses for each concept relative to one another. Figure \ref{fig:cmcl} shows this ratio for all considered concepts. It can be seen from this figure that the ratio between the lift and moment coefficients can be considered constant for $0\deg\leq\gls{sym:alpha}\leq 25\deg$.

Note furthermore that the rigid concept has not been taken into account. This is the case because, as covered in Chapter \ref{ch:strucmass}, the structural mass of the rigid design option does not meet the requirements. As such the control system mass of this concept will not be taken into account.

\subsection{Relative mass estimation}
\label{subsec:controlmassest}
To determine the relative control system masses the absolute value of the ratio $\frac{\gls{sym:CM}}{\gls{sym:CL}}$ at $\gls{sym:alpha}=1\deg$ is determined for all concepts.
\begin{table}[h]
	\centering
	\caption{A caption}
	\begin{tabular}{|c|c|c|}
		\hline
		\textbf{Concept} & $\mathbf{\frac{\gls{sym:CM}}{\gls{sym:CL}}}$ & \textbf{Fraction of stacked toroid} \\ \hline \hline
		Stacked toroid & $5.81$ & $1.00$\\
		Isotensoid & $5.72$ & $0.99$\\
		Trailing ballute & $5.00$ & $0.86$\\
		\hline
	\end{tabular}
	\label{tab:controlmass}
\end{table}
