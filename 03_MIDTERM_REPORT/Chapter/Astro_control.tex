\section{Astrodynamics \& control tool development}
\label{ch:astrocontrol}
***Intro***\\

\subsection{Purpose of tool development}
\label{sec:astropurpose}
***intro***\\
***No other tool publicly available***\\
***Determine the required Aerodynamic characteristics to create an acceptable window of entry***\\
***Investigate the need for \& effect of control***\\

\subsection{Governing equations}
***intro***\\
***Kepler for hyperbolic: Vis-Viva, polar expression of hyperbola, tangent to hyperbola)***\\
***Forces give accelerations: like in baseline***\\
***Kepler for elliptic: symmetry***\\

\subsection{Working principles of the tool}
***intro***\\
***Flowchart + very short explanation***\\

\subsection{Verification \& validation}
***intro***\\
***sensitivity analysis: very sensitive to initial location, thus for now assumed initial location closer***\\
***discretisation error: error for different dt***\\
***Compare numerical results with Kepler, should be the same for rho=0***\\
***Validation through checking if output values are approximately the same as reference cases (peak dynamic pressure)***\\

\subsection{Performance of control systems}
***intro***\\
***Moment or dalpha/dt that control systems (cg offset, thrusters, control surfaces) can create***\\
***Weight estimate of each control system (per concept if needed)***\\

\subsection{Results \& Conclusions}
***intro***\\
***Plot of a trajectory***\\
***Plots of output variables for that particular trajectory***\\
***Results of needed CLmax, dalpha/dt, dCL/dalpha***\\
***Conclusion on which control system (implemented in which concept) can perform as such***\\
***Conclusion on weight the control systems will bring along***\\
***State values for in trade-off matrix***\\