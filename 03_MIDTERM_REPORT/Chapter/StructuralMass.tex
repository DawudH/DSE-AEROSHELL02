\section{Concept structural mass estimation} \label{ch:strucmass}
It is essential that decelerator concepts are analyzed in terms of mass to fully appreciate the mass benefits of a certain decelerator concept. A reduced decelerator mass allows for a larger payload mass to be taken on board, since launcher mass capability is intended to be fully fulfilled. It is therefore essential that the concept selection takes into account decelerator mass. 

Concepts differ in the decelerator configuration and resulting mass differences between concepts are mostly due to structural and thermal mass. For a payload capsule and centerbody containing subsystems (e.g. telemetry, on-board computer) indifferent from the proposed five concepts, the decelerator structure, the \acrfull{tps} and the control system contribute to concept mass differences. To this end a structural mass estimation tool is constructed to provide an estimate of concept structural mass for an informed decision in the concept trade-off. 

Detailed structural analysis and design is held off until after concept selection, based on the premise that the structure can be sized such that all concepts are structurally feasible and any resulting differences in structural performance are reflected by total structural mass. Hereafter all references to mass are with respect to structural mass.

This chapter is structured in the following manner. Firstly, the purpose of tool development is outlined in section \ref{sec:strucpurp} to prevent a clear view of the required tool output. Based hereupon, a tool is constructed using a number of methods as described in section \ref{sec:strucmeth}. Verification and validation activities are described in section \ref{sec:strucvv}. The tools are then applied firstly to compare the selected decelerator concepts in terms of mass, as described in section \ref{sec:struccc}, secondly to provide an overview of the effect of materials choice on decelerator mass, as described in section \ref{sec:strucmat}, and lastly to perform a sensitivity analysis, as described in section \ref{sec:strucsens}.

\subsection{Purpose of tool development}\label{sec:strucpurp}
The purpose of the tool is twofold:
\begin{itemize}
\item To obtain a decelerator mass estimate for each of the five concepts in order to evaluate their mass performance;
\item To investigate the effect of concept design variables on decelerator mass;
\end{itemize}
The former is investigated primarily in section \ref{sec:struccc}, where the decelerator concepts are evaluated for their mass performance, while the latter is investigated in sections \ref{sec:strucmat} and \ref{sec:strucsens}. The mass estimate is primarily intended for relative comparison in this design phase, closed by the \acrfull{mtr}. It remains, however, a valuable tool in analysis and design of the selected concept by providing an actual mass estimate for a selection of design variables. This selection of design variables can thus be made using the mass estimate provided by the tool, by a minimization of decelerator mass.

\subsection{Methods for mass estimation}\label{sec:strucmeth}
The structural mass estimation is done using two separate methods: mass estimates based on reference missions and mass estimates based on structural models. The latter will be used for the analysis of the structural mass for the four inflatable concepts: the stacked toroid, isotensoid, tension cone and trailing configurations. For these concepts few reference missions are available for an adequate mass estimation and the availability of parametric mass modeling \cite{Anderson1969, Samareh2011} make structures-based models the preferred option. Historically, rigid re-entry vehicles have been applied numerous times and a mass estimation can be done on the base of reference missions. These mass estimation methods are presented hereafter.

\subsubsection{Isotensoid mass estimation}
%los maar op
For the mass estimation based on the structural models two different methods are employed. First off, for the isotensoid and trailing concepts Anderson \cite{Anderson1969} hsa developed a structural merit model. Modern implementations, as presented by for example Miller \cite{Miller2014}, use this mass estimation for current investigations in the feasibility of aerodynamic decelerators. The mass estimation presented by Anderson for the isotensoid concept considers a model inflated by ram-air, an inflation method commonly considered for isotensoid structures \cite{Smith2011}. For the trailing ballute configuration Anderson considers a ram-air inflated structure as well, similar to the isotensoid configuration. For the trailing configuration different models may be considered and a ballute in the from of a "donut" is also commonly analysed. 
Andersons model provide the structural mass estimation as a single variable for the whole isotensoid or ballute configuration. The mass estimation is presented in the from of the \gls{bc} of the structure. This value can consequently be used for a mass estimation of the isotensoid or ballute configurations.

Andersons model is provided in two forms: First of a basic model requiring as input the peak dynamic pressure (\gls{sym:q}) and the frontal surface area (\gls{sym:A}) which can also be implicitly considered using the outer diameter. Moreover decelerator material properties are considered using the materials aerial density (\gls{sym:df}). Andersons model assumes the use of a single material for the whole decelerator design. For the computation of the aerial density a minimum gage thickness is considered.

 It must be noted that this simple model implicitly includes as set value for the drag coefficient (\gls{sym:CD}). The full model requires a complete set of parameters which is provided by \cite{Anderson1969} and allows control of the drag coefficient as well which can be considered useful for a sensitivity analysis. More detailed analysis of other parameters are not considered at this in the design. A more detailed estimation is not desired as it requires a detailed configuration of the concept
%tot hier
\subsubsection{Stacked toroid, tension cone and trailing IAD mass estimation}
Mass estimation for stacked toroid, tension cone and trailing \gls{iad} configurations is performed by the method outlined in Ref.\cite{Samareh2011}. Similar to the model by Anderson, it introduces a figure of merit by defining a dimensionless mass efficiency parameter \gls{sym:mbar}, defined as:
\begin{equation}
\gls{sym:mbar} = \frac{\gls{sym:m}}{\gls{sym:mfactor}}
\label{eq:dimmass}
\end{equation}
where \gls{sym:mfactor} is defined by the following equation
\begin{equation}
\gls{sym:mfactor} = \frac{\gls{sym:A_iad} \gls{sym:CD} \gls{sym:q}}{\gls{sym:ge}}
\label{eq:dimmass}
\end{equation}

\subsubsection{Rigid structure mass estimation}

\subsection{Tool verification and validation}\label{sec:strucvv}
Verification and validation of the mass estimation tools is performed as follows. These activites can be divided primarily into the verification and validation of the mass estimation tool adapted from Samareh \cite{Samareh2011} for stacked toroid, tension cone and trailing \gls{iad} devices on one hand and verification and validation of the mass estimation tool adapted from Anderson \cite{Anderson1969} for isotensoid and trailing \gls{iad} devices on the other hand. 

\subsubsection{Anderson inflatable mass estimation method}
In order to verify the implementation of the mass estimation tool presented by Anderson in Ref. \cite{Anderson1969} the mass estimation tools results were compared to the results presented in Anderson paper. The full estimation model was implemented and consequently checked with the coefficients of the simplified model. Finally the results full model were compared with the results of the merit function for the ballute and the attached isotensoid on page 30 of Ref. \cite{Anderson1969}. No errors were observed in the significant digits. The mass estimates were further verified using the results presented by Clark in Ref. \cite{Clark2009}. The results as presented by Clark were found to be of a similar order. 

Validation of the mass estimation model was presented by Anderson on the basis of reference inflatable structures and is presented together with model. The model within the context is merely a implementation and no additional validation on the model is performed.


\subsubsection{Samareh inflatable mass estimation method}
In order to verify that the mass estimation method described in Ref.\cite{Samareh2011} has been correctly implemented, results for the nine sample cases presented on page 16 of Ref.\cite{Samareh2011} have been checked. These nine sample cases were implemented by choosing the input parameters as given in tabular form (Tables 4 and 5) on page 16 of Ref.\cite{Samareh2011} and the output parameters, primarily component masses and geometric quantities, were compared. A maximum error of 3 $\%$ in terms of total mass was obtained; a maximum error of 2 $\%$ in component masses. These errors are deemed sufficiently small to verify succesful implementation of the mass estimation method.

Validation is performed indirectly: the method \cite{Samareh2011} has been applied in the \gls{edlsa} project \cite{Cianciolo2010}, where it was shown to yield results conforming well to the outcomes of high-fidelity \gls{fea}. The used \gls{fea} is a validated tool \cite{Cianciolo2010} and thereby the method outlined in Ref.\cite{Samareh2011} has been validated through comparison with a high-fidelity validated model. Moreover, the expression for minimum inflation pressure obtained by Samareh has been found to be in correspondence with Yamada et al \cite{Yamada2009}, Clark \cite{Clark2009} and Brown \cite{Brown2009}.

A qualitative check on the results obtained by this mass estimation method is performed by assessing the relations found in section \ref{sec:strucmat} and \ref{sec:strucsens} for their conformance to expectations and other literature on inflatable aerodynamic decelerators. The outcomes hereof are reflected by the discussions in these respective sections.




\subsection{Rigid concept mass analysis}\label{sec:rigid}   %Moet denk ik nog anders
For the estimation of the rigid concepts the \acrfull{tps} mass and the structural mass estimate are combine to yield a single value. This estimation is based on reference mission. Steinfeldt \cite{Steinfeldt2009} provides one such mass estimate for high mass Mars entry concept. Two concept are considered a slender and a blunt body of which the latter is our primary interest. The reference missions used in this analysis are all of the blunt body type, similar to the rigid concept considered in this report. The estimation provided below are all meant for such a blunt body.

Some remarks must be made with regards to the use of these reference missions. Al tough rigid re-entry mission are plenty their relevance must be considered. Within this report a re-entry within the Martian atmosphere with a approach velocity of 7 [$km/s$]. Most relevant reference missions are either for a planet different from Mars. For the re-entry vehicles entering the Martian atmosphere the approach velocity may still be of a completely different order. Previously executed Martian entries were typically unmanned featuring Hohmann transfers to Mars. The mission at hand features however more direct transfer considering human constraints yielding different approach velocities. As such the estimation should incorporate in some extend the thermal and structural loading of the mission trajectory.

The estimation provided by Steinfeldt takes this into account, dividing the estimation into three different parts: The forebody thermal protection system mass, the forebody structural mass and the back shell structural and thermal protection system mass estimate. The estimation takes into account the size of the concept by using the vehicles total mass at the start of the re-entry (\gls{sym:m0}). The heat load is taken into account by using the heat load \gls{sym:Q} the structural loading is taken into account by using the peak dynamic pressure.

The mass estimate for the forebody thermal protection system is based in the heatload \gls{sym:Q} and is provided by \cite{Laub2004} in equation \ref{eq:tps_heat}:

\begin{equation}
\gls{sym:mheat} = (0.000091Q^{0.51575})\gls{sym:m0}
\label{eq:tps_heat}
\end{equation}

The mass estimate for the forebody  structure is conseqently given by equation \ref{eq:m_structure}:

\begin{equation}
\gls{sym:mstrcuture} = (0.0232 \gls{sym:q} ^{-0.1708} ) \gls{sym:m0}
\label{eq:m_structure}
\end{equation}

\begin{figure}[h]
	\centering
	\begin{subfigure}[b]{0.49\textwidth}
	\centering
	\includegraphics[width=1.0\textwidth]{Figure/rigidheat.eps}
	\caption{Empirical relation for the forebody heat shield mass} 
	\label{rigidheat}
	\end{subfigure}
	\begin{subfigure}[b]{0.49\textwidth}
	\centering
	\includegraphics[width=1.0\textwidth]{Figure/rigidstruct.eps}
	\caption{Empirical relation for the forebody heat shield mass} 
	\label{fig:rigidstruct}
	\end{subfigure}
	\caption{Empirical relations for the forebody mass of a (blunt) rigid concept}
	\label{fig:rigid}
\end{figure}

Different from the inflatable concepts a rigid design features a additional back shell structure. 

m backshell 14 \%
\subsection{Inflatable concept comparison}\label{sec:struccc}
In terms of comparing the decelerator concepts in terms of mass, the total decelerator mass as yielded by the methods outlined in section \ref{sec:strucmeth} is computed for each of the five concepts. Due to the limited applicability of each of the methods used, a necessity is the use of multiple methods: the method by Samareh \cite{Samareh2011} for the mass estimation of stacked toroid, tension cone and trailing \gls{iad} concepts; the method by Anderson \cite{Anderson1969} for the mass estimation of the isotensoid and the reference method for the mass estimation of the rigid concept. 

\begin{figure}[H]
\centering
\includegraphics[width = 1.0\textwidth]{Figure/ISO_comp.eps}
\caption{\acrlong{dot} for mission duration}
\label{fig:ISO_comp}
\end{figure}





Figure \ref{fig:mass_dia} display the performance of all the inflatable concepts in one figure for a set value shape parameters. Figure \ref{fig:mass_dia} was made with a peak dynamic pressure of $\gls{sym:q}=3000$ [$Pa$], a drag coefficient $\gls{sym:CD}=1.5$ [$-$]. The Vectron material was used for.  .... en nu weet ik  niet wat jij hebt ingevuld...

Figure \ref{fig:mass_dia} serves to show how each different configurations mass scales with varying deployed diameter. It can be noted that the isotensoid configuration has a relatively, compared to the other inflatable concepts, high mass at the base diameter of five meters. This can be attributed to the fact that the isotensoid configuration features no deployed diameter. Different from the other inflatable concepts described by the model of Samareh in which the undeployed diameter has direct links the mass of for example the gas system. The isotensoid configuration, among featuring no pressurised gas inflation system but rather ram-air, has no such diameter defined. The isotensoid is a single all compassing inflatable covering the whole payload.   

\begin{figure}[H]
\includegraphics[width = 1.0\textwidth]{Figure/mass_dia.eps}
\caption{Decelerator mass versus deployed diameter for all the inflatable configurations}
\label{fig:mass_dia}
\end{figure}

\subsection{Inflatable structural materials comparison}\label{sec:strucmat}
Materials for inflatable structures are typically high-performance high-temperature-resistant woven fabrics, supplemented by coatings to reduce porosity \cite{Jenkins2001}. These materials are required to be lightweight, strong, heat-resistant and flexible for application in \glspl{hiad} \cite{Samareh2011}. On one hand there are heritage materials, such as Kapton, but for current \gls{hiad} missions these have been replaced by up-and-coming materials, such as Vectran and PBO Zylon \cite{Dillman2012,  Smith2010}. Key driver for the use of these materials is a high specific strength compared to metals (e.g. aluminium) \cite{Samareh2011}. 

An overview of materials, and their key properties, suitable for \gls{hiad} application is given in Table \ref{table:strucmatoverview}. Since the areal density is not computable, but instead measured, it is not accessible for all materials considered in Table \ref{table:strucmatoverview}. Hence, the method by Anderson \cite{Anderson1969} is only feasible for the material choices with a known areal density, as this is an essential property for this method. As of this stage, performance at high temperatures other than the process of selection leading up to the statement of these materials in references \cite{Dillman2012, Smith2010} is not investigated. The current efforts are centered around decelerator structural mass, hence the primary use for these material properties is for input in the models constructed for mass estimation. Structural and thermal performance are evaluated after concept selection, given the limited timeframe and the fact that structural concept design will not distinguish concepts in terms of the trade-off criteria other than in terms of mass.

\begin{table}[H]
\caption{Overview of candidate materials for \gls{hiad} application. Material properties from references \cite{Samareh2011,Miller2014}. A hyphen denotes unknown quantities.}
\vspace{41mm}
\hspace{-15mm}
\begin{tabular}{p{0.235\textwidth}|p{0.17\textwidth}|p{0.05\textwidth}|p{0.05\textwidth}|p{0.05\textwidth}|p{0.05\textwidth}|p{0.05\textwidth}|p{0.05\textwidth}|p{0.05\textwidth}|p{0.05\textwidth}|p{0.05\textwidth}|}
\begin{rotate}{60} ~~~~~Material \end{rotate}  &  \begin{rotate}{60} ~~~~~Type of material {[}-{]}  \end{rotate} & \begin{rotate}{60} ~~~~~Density [$\frac{kg}{m^3}$] \end{rotate}& \begin{rotate}{60} ~~~~~Areal density [$\frac{kg}{m^2}$] \end{rotate} & \begin{rotate}{60} ~~~~~Elongation at break [$\%$] \end{rotate} & \begin{rotate}{60} ~~~~~Specific strength [$\frac{kN km}{kg}$]\end{rotate} &  \begin{rotate}{60} ~~~~~Breaking strength [km]\end{rotate} & \begin{rotate}{60} ~~~~~Breaking tenacity [$\frac{g}{denier}$]\end{rotate}&  \begin{rotate}{60} ~~~~~Tensile strength [GPa] \end{rotate} & \begin{rotate}{60} ~~~~~Young's Modulus [GPa] \end{rotate} & \begin{rotate}{60} ~~~~~Poisson's Ratio [-] \end{rotate} \\
Kapton Type 100 HN           & Polyimide film               & 1420                                 & -                                          & 72                           & 163                              & 16.6                       & 1.84                             & 0.231                      & 2.5                       & 0.34                 \\ \hline
Kevlar 29     & Aramid fiber                 & 1440                                 & 0.208                                     & 3.6                          & 2031                             & 207.0                      & 23.00                            & 2.92                       & 70.5                      & 0.36                      \\ \hline
Kevlar 49   & Aramid fiber                 & 1440                                 & 0.181                                     & 2.4                          & 2084                             & 212.4                      & 23.60                            & 3.00                       & 112.4                     & 0.36                   \\ \hline
Nomex Type 430 & Aramid fiber                 & 1380                                 & 0.400                                     & 30.5                         & 441                              & 45.0                       & 5.00                             & 0.61                       & 11.45                     & -                         \\ \hline
PBO Zylon AS                 & Polybenzoxazole fiber        & 1540                                 & -                                          & 3.5                          & 3766                             & 383.9                      & 42.66                            & 5.80                       & 180.0                     & -                            \\ \hline
M5 & Synthetic fiber & 1700 & - & 1.4 & 2329 & 237.5 & 26.38 & 3.96 & 271 & - \\ \hline
Spectra 2000 & Polyethylene fiber           & 970                                  & -                                          & 3.0                            & 3443                             & 351.0                      & 39.00                            & 3.34                       & 124.0                     & -                        \\ \hline
Technora                     & Aramid fiber                 & 1390                                 & -                                          & 4.4                          & 2158                             & 220.0                      & 24.45                            & 3.00                       & 70.0                      & -                          \\ \hline
Upilex-25S                   & Polyimide film               & 1470                                 & 0.378                                     & 42                           & 354                              & 36.1                       & 4.01                             & 0.52                       & 9.1                       & -                           \\ \hline
Vectran HT                   & Liquid crystal polymer fiber & 1410                                 & 0.092                                     & 4.3                          & 2270                             & 229                        & 25.44                            & 3.20                       & 75.0                      & -                      \\ \hline
Nextel 610                   & Ceramic fiber                & 3900                                 & 0.278                                     & -                            & -                                & -                          & -                                & 3.10                       & 373.0                     & -                       
\label{table:strucmatoverview}
\end{tabular}
\end{table}
[M5?]

The mass performance of these materials is evaluated by plotting the \acrfull{bc} of the aerodynamic decelerator concepts versus the maximum dynamic pressure. A number of quantities could have been chosen as output (y-axis) and input (x-axis) parameters. The \gls{bc} has been chosen by its ability to reflect the mass-effectiveness of the design and its common application in this context within general literature; the dynamic pressure for its ability to reflect the environmental conditions in which the aerodynamic decelerator operates. Figure \ref{fig:all_mat} display the \gls{bc} of stacked toroid, tension cone and trailing ballute concepts respectively as a function of maximum dynamic pressure for the materials stated in Table \ref{table:strucmatoverview}, as obtained by the model implemented from Ref.\cite{Samareh2011}. Fig.\ref{fig:ISO_mat} does so for an isotensoid configuration, for those materials in Table \ref{table:strucmatoverview} with a known areal density. It should be noted that the ballistic coefficient is that of solely the aerodynamic decelerator, hence with the total structural mass calculated by the mass estimation tools.

\begin{figure}[H]
\centering
\includegraphics[width = 0.55\textwidth]{Figure/ISO_mat.eps}
\caption{Decelerator ballistic coefficient versus peak dynamic pressure for various materials of the isotensoid configuration}
\label{fig:ISO_mat}
\end{figure}

\begin{figure}[H]
\hspace{-35mm}
\includegraphics[width = 1.35\textwidth]{Figure/all_mat.eps}
\caption{Decelerator ballistic coefficient versus peak dynamic pressure for various materials of the stacked toroid, tension cone and trailing ballute configurations}
\label{fig:all_mat}
\end{figure}

From Fig.\ref{fig:ISO_mat}, it may be observed that the isotensoid configuration has the lowest ballistic coefficient (as per the tool implemented from Ref.\cite{Anderson1969}), and thereby the lowest mass, in case Vectran is used. Kevlar 49 and Kevlar 29 follow up, with ballistic coefficients approximately 30 $\%$ respectively 40 $\%$ higher than that obtained using Vectran. Use of Nextel 610, Upilex-25S and Nomex results in significantly higher ballistic coefficients. In case of Nomex, the ballistic coefficient is over 250 $\%$ higher than that in case of Vectran. Over the range of peak dynamic pressure considered, these mass performance differences are maintained. 

The ballistic coefficients of stacked toroid, tension cone and trailing \gls{iad}/ballute configurations, as produced by the tool implemented from Ref.\cite{Samareh2011}, are displayed in Fig.\ref{fig:all_mat}. The sub-plots in this figure confirm that Nextel 610, Upilex-25S and Nomex perform worst in terms of effecting a low mass and thereby \gls{bc}. Kapton, however, is shown to perform worse still. The other materials, Kevlar 29, Kevlar 49, M5, PBO Zylon, Vectran, Spectra 2000 and Technora perform best and yield similar ballistic coefficients. While there are still differences between the mass performance, differences remain limited (in the order of 20-30 $\%$). 

A combination of figures \ref{fig:ISO_mat} and \ref{fig:all_mat} yields that the best performance in terms of structural mass is achieved by using Technora, Spectra 2000 and PBO Zylon, followed closely upon by Kevlar 29, Kevlar 49 and Vectran. The former three are relatively new concepts, PBO Zylon the only one introduced in an \gls{iad} in IRVE-III \cite{Dillman2012}, in which it replaced the Kevlar (type unspecified) used in previous IRVE missions \cite{Lindell2006}.



\subsection{Sensitivity analysis}\label{sec:strucsens}
This section provides an indication of the effect of changing design parameters on one hand and external parameters on the other hand on the total structural decelerator mass and mass efficiency. Design parameters are primarily half-cone angle \gls{sym:theta}, drag coefficient \gls{sym:CD} and outer diameter \gls{sym:Do}. Primary external factor is the peak dynamic pressure. These parameters are evaluated with respect to their effect on decelerator structural mass, either in terms of ballistic coefficient or total mass. In addition, inflation gas molar mass and temperature are investigated for their effect on inflation gas mass and the number of toroids is investigated for the effect on the stacked toroid configuration. This indication is primarily useful to provide a good starting point for concept design in terms of the investigated design parameters and to quantify benefits associated with certain design concepts. 

Due to the limitations imposed by the mass estimation model for the isotensoid \cite{Anderson1969}, the sensitivity of isotensoid structural mass to these parameters can only be investigated in a limited manner. The more extensive model for the other three inflatable concepts \cite{Samareh2011} allows for a more involved sensitivity analysis.

\subsubsection{Inflation gas mass}
Stacked toroid, tension cone and trailing ballute configurations rely on the use of internal pressure to provide the structural rigidity required from the inflatable structure. The isotensoid uses ram-air to inflate itself and hence does not require inflation gas to be taken on-board for inflation purposes. In the mass estimation model, adapted from Samareh \cite{Samareh2011}, the inflation gas is characterized primarily by its operating temperature (in Kelvin or degrees Celsius) and its molar mass (in kilogram per mole). Inflation gas mass as a function of these two variables is displayed in Fig.\ref{fig:inflmass}

\begin{figure}[H]
\hspace{-5mm}
\includegraphics[width = 1.0\textwidth]{Figure/gas_temp_mass.eps}
\caption{Inflation gas mass as a function of gas temperature and molar mass of the stacked toroid, tension cone and trailing ballute configurations}
\label{fig:inflmass}
\end{figure}

Fig.\ref{fig:inflmass} confirms that a denser gas, characterized by a higher molar mass, yields a larger inflation gas mass. An overview of gas generator types and their output is given on page 7 by Brown \cite{Brown2003}. A typical molar mass is that of nitrogen, 22 [$\frac{kg}{mole}$], used in the IRVE satellites \cite{Dillman2012}. It should be noted that inflation system mass is typically estimated as a percentage of inflation gas mass. Total inflation system mass, thus consisting of inflation system and inflation gas mass, will therefore not be as low as suggested by Fig.\ref{fig:inflmass} for low molar mass inflation gases, but it will typically be higher by requiring a heavier inflation system \cite{Brown2003}. Inflation gas mass decreases for an increasing inflation gas temperature, again conforming to expectations. Via the ideal gas law the mass occupied by a certain number of moles of inflation gas will increase for a decreasing temperature through an increased gas density \cite{AndersonJr.2007}.

\subsubsection{Number of toroids for the stacked toroid configuration}
The stacked toroid features a number of toroids that are inflated. To investigate the effect of the number of toroids on total structural decelerator mass, the two are plotted against one another in Fig.\ref{fig:toro}. 

\begin{figure}[H]
\centering
\includegraphics[width = 1.0\textwidth]{Figure/mass_toroids.eps}
\caption{Decelerator structural mass as a function of the number of toroids for the stacked toroid configuration, for a peak dynamic pressure of 3000 [Pa], \gls{sym:CD}= 1.5 [-] and a deployed diameter of 12 [m]}
\label{fig:toro}
\end{figure}

From Fig.\ref{fig:toro}, it may be observed that stacked toroid structural mass decreases for an increasing number of toroids. The decrease is, however, so slight as to be insignificant. A mass decrease of approximately 4 $\%$ is effected by an increase from one to five toroids; a decrease of approximately 5 $\%$ for an increase from one to ten toroids. As the number of toroids is increased beyond ten, mass is more or less constant. Total structural decelerator mass is therefore deemed indifferent to the number of toroids. Care should still be taken to attach too much value to the results yielded by the model, given its limited fidelity.

\subsubsection{Inflatable decelerator diameter}
Figures \ref{fig:mass_dia} and \ref{fig:bc_dia} illustrate the effect of a change in inflatable decelerator diameter on its structural mass. Fig.\ref{fig:mass_dia} illustrates that an increase in deployed diameter, given a certain peak dynamic pressure and drag coefficient, effects an exponential increase in decelerator mass. This increase is significant, adding as much as 33 $\%$ of structural mass for a tension cone for an increase in diameter from 13 to 14 [m], for a peak dynamic pressure of 3000 [Pa] and a drag coefficient of 1.5 [-]; the other concepts exhibit similar increases. A better estimate of the mass-effectiveness for an increasing diameter is given by Fig.\ref{fig:bc_dia}, which displays the ballistic coefficient of the decelerator, taking into account only its structural mass. This figure shows that the ballistic coefficient increases with increasing diameter for stacked toroid, tension cone and trailing ballute, denoting that decelerator structural mass increases more than its decelerating capability (the product of \gls{sym:CD} and \gls{sym:A_iad}) and thus that it becomes less mass-effective for an increasing deployed diameter.

 For the isotensoid configuration a small decrease in ballistic coefficient with increasing deployed diameter may be noted. This follows from the minimum gage thickness requirements incorporated in \gls{sym:df}. As discussed in section \ref{sec:strucmeth} the isotensoid analysis is based on the parameter \gls{sym:df}. This is given as a maximum of the areal density or \gls{sym:df} as computed based on strength requirements of the material using the strength to mass ratio of the material ( \gls{sym:kf}). In the case of the isotensoid design it must be noted that in figure … the mass is based on minimum gage thickness requirements using the areal density of the material. This can be considered as suboptimal performance as rather the materials strength to mass ratio is considered for structural sizing. This suboptimal performance is the reasons that the ballistic coefficient increases rather than decreases for the isotensoid configuration



\begin{figure}[H]
\centering
\includegraphics[width = 1.0\textwidth]{Figure/bc_dia.eps}
\caption{Decelerator structural ballistic coefficient as a function of deployed diameter of inflatable concepts for a peak dynamic pressure of 3000 [Pa] and \gls{sym:CD}= 1.5 [-]}
\label{fig:bc_dia}
\end{figure}

\subsubsection{Half-cone angle}
Figures \ref{fig:mass_theta_cd} and \ref{fig:bc_theta_cd} display how the structural ballistic coefficient of the decelerator changes with half-cone angle and drag coefficient. Similar figures result if the peak dynamic pressure is taken as a variable rather than the drag coefficient, which conforms to expectations given that peak dynamic pressure and drag coefficient features almost exclusively as a product. From Fig.\ref{fig:mass_theta_cd}, it can be observed that mass increases for an increasing half-cone angle (and an increasing drag coefficient). From  Fig.\ref{fig:bc_theta_cd}, it can be observed that mass-effectiveness (reflected by the ballistic coefficient of the decelerator structural mass) has an optimum with respect to the half-cone angle. This optimum is well established for stacked toroid and tension cone configurations, but disappears as the drag coefficient is increases for the trailing ballute configuration.

For a drag coefficient of 1.5 [-], the optimum is located at approximately 60 [deg] for the stacked toroid configuration; at approximately 55 [deg] for the tension cone configuration. Except for the trailing ballute configuration, this optimum angle is more or less maintained. A study by Brown et al \cite{Brown2003}, using a parametric mass model based on the sizing of the structure to prevent buckling (different from the model used for the production of Fig.\ref{fig:bc_theta_cd}), finds a mass-optimum cone angle of 55 [deg] for a hypercone (comparable to the tension cone concept) \cite[p.6]{Brown2003}. Brown et al, however, establish this angle as having minimum hypercone mass, rather than maximum mass-efficiency. 
\begin{figure}[H]
\centering
\includegraphics[width = 0.9\textwidth]{Figure/mass_theta_cd.eps}
\caption{Decelerator structural mass as a function of half-cone angle and drag coefficient of inflatable concepts, for a peak dynamic pressure of 3000 [Pa] and a deployed diameter of 12 [m]}
\label{fig:mass_theta_cd}
\end{figure}

\begin{figure}[H]
\centering
\includegraphics[width = 0.9\textwidth]{Figure/bc_theta_cd.eps}
\caption{Decelerator structural ballistic coefficient as a function of half-cone angle and drag coefficient of inflatable concepts, for a peak dynamic pressure of 3000 [Pa] and a deployed diameter of 12 [m]}
\label{fig:bc_theta_cd}
\end{figure}
%\subsection{Conclusion}\label{sec:strucconc}

