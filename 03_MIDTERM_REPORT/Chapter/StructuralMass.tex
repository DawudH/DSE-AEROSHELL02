\section{Concept mass estimation} \label{ch:strucmass}
It is essential that concepts are analyzed in terms of mass to fully appreciate the mass benefits of a certain decelerator concept. A reduced decelerator mass allows for a larger payload mass to be taken on board, since launcher mass capability is intended to be fully fulfilled. It is therefore essential that the concept selection takes into account decelerator mass. To this end a mass estimation tool is constructed to provide a first estimate of concept mass for an informed decision in the concept trade-off.

This chapter is structured in the following manner. Firstly, the purpose of tool development is outlined in section \ref{sec:strucpurp} to prevent a clear view of the required tool output. Based hereupon, a tool is constructed using a number of methods as described in section \ref{sec:strucmeth}. Verification and validation activities are described in section \ref{sec:strucvv}. The tools are then applied firstly to compare the selected decelerator concepts in terms of mass, as described in section \ref{sec:struccc}, secondly to provide an overview of the effect of materials choice on decelerator mass, as described in section \ref{sec:strucmat}, and lastly to perform a sensitivity analysis, as described in section \ref{sec:strucsens}. Conclusions are formulated in section \ref{sec:strucconc}.

\subsection{Purpose of tool development}\label{sec:strucpurp}



\subsection{Methods for mass estimation}\label{sec:strucmeth}

\subsection{Tool verification and validation}\label{sec:strucvv}

\subsection{Concept comparison}\label{sec:struccc}

\subsection{Structural materials comparison}\label{sec:strucmat}

%\subsection{Inflation system mass}\label{sec:strucinfl}

\subsection{Sensitivity analysis}\label{sec:strucsens}

\subsection{Conclusion}\label{sec:strucconc}

