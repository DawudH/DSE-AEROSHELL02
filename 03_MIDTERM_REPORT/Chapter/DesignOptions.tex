\section{Conceptual design options} \label{ch:options}
This chapter will discuss the five design concepts that will be considered in this report. First the selection of concepts based on the shape, mission duration and controls system design option trees is made. Finally in Section \ref{sec:conf} the five selected configurations are sketched. 

\subsection{Concept selection}
 From the \acrfull{br} three \glspl{dot} were obtained for the design of the hypersonic decelerator. In these designs concept shape, mission duration and control system are considered. This yielded a set of all the individual design options. These options are shown in Figure \ref{fig:dotshape} to \ref{fig:dotcontrol}. From these \glspl{dot} a feasible set of design options can be obtained. In Figure \ref{fig:dotshape} six concept shapes can be seen. This can consequently be combined with three possible control systems from Figure \ref{fig:dotcontrol}. The mission duration varies for all concepts and is therefore considered separately for every design. 

\begin{figure}[H]
%\centering
\hspace{-23mm}
\includegraphics[width = 1.25\textwidth]{Figure/DOT_configuration.pdf}
\vspace{-5mm}
\caption{\acrlong{dot} for entry vehicle configurations}
\label{fig:dotshape}
\end{figure}

\begin{figure}[H]
\centering
\includegraphics[width = 1.0\textwidth]{Figure/DOT_missionduration.pdf}
\vspace{-5mm}
\caption{\acrlong{dot} for mission duration}
\label{fig:dotduration}
\end{figure}

\begin{figure}[H]
\centering
\includegraphics[width = 0.93\textwidth]{Figure/DOT_control.pdf}
\vspace{-5mm}
\caption{\acrlong{dot} for control systems}
\label{fig:dotcontrol}
\end{figure}


From the 18 options of shapes and control systems shown above some may be disregarded immediately since they cannot be combined. From this point onward the concept shape will be considered leading. The control systems from Figure \ref{fig:dotcontrol} are considered separately for each possible state for each defined shape. Design options were only considered for the concept shape and control surfaces, since these require individual matching as not any shape and control surface system may be combined. Analysis on the performance in the fields of aerodynamics, thermodynamics or structures follow directly from the shape. The control surface is however considered separately and exotic concepts such as the, for now, deemed infeasible \gls{mhd} control have their own strict requirements on the structure. Since these exotic options are at this point deemed infeasible the shape of the structure can be considered leading in the design. 

Table \ref{tab:designconcepts} shows the design option that will be considered within this report. The check-marked control systems are further investigated for their feasibility and performance within ?? the chapter discussing the control systems. For these control systems their performance need to be considered separately and their feasibility must be properly supported by either references or computations. The use of a \gls{cg} offset as a control mechanism may for example be doubtful since the demonstration on its performance is only done in the \gls{irve} in which the \gls{hiad} has a relatively high mass fraction compared to the payload \cite{Dillman2012}. For this design from the requirement CIA-Op-A02-02 (Appendix \ref{app:req}) this mass fraction is however specified to be only 10\% doubting the effectiveness of a \gls{cg} offset as control system. For control surfaces only specific designs may be possible resulting in only the use of for example body flaps or morphing the inflatable as control mechanism. All these considerations are again discussed in Chapter ... . These results are finally used as one of the selection criteria in the trade off.

It must again be noted that the control configurations of Table \ref{tab:designconcepts} feature the primary control mechanism. These may always be appended later on in the design process, after the trade-off, if additional control systems increase the overall performance. 

\begin{table}[H]
	\caption{Generation of design concepts}
	\label{tab:designconcepts}
		\begin{tabular}{|p{0.3\textwidth}|p{0.22\textwidth}|p{0.22\textwidth}|p{0.22\textwidth}|} \hline 
			\textbf{Concept} & \textbf{Thruster}	& \textbf{\gls{cg} offset} &  \textbf{Control surfaces} \\ \hline \hline
			Stacked toroid   & \cmark	& \cmark &  \cmark \\ \hline
			Iso-tensoid		 & \cmark	& \cmark &  \xmark\\ \hline
			Tension cone	 & \cmark	& \cmark &  \cmark \\ \hline
			Trailing 		 & \xmark	& \cmark &  \cmark \\ \hline
			Combined 		 & \xmark	& \cmark &  \cmark \\ \hline
			Rigid  		   	 & \cmark	& \cmark &  \cmark \\ \hline
		\end{tabular}
\end{table}

At this point it may be noted that at this point in Table \ref{tab:designconcepts} for each shape a possible control mechanism can be considered. Rather just three combinations are deemed infeasible. Thrusters are not considered for the Trailing an Combined configuration. Due to the aft elements of these designs thruster cannot be considered the primary control mechanism. Due to the large moments of inertia (due to aft elements) and very high stability of the aft systems (e.g. compare to a parachute) thrusters cannot be considered due their inefficiency. 

For the Iso tensoid configuration controls surface are not deemed possible since the whole outer surface is covered by the inflatable. Controls systems such as for example body flaps cannot therefore not be placed. Using morphing of the structure as control surface is also considered unfeasable since again the whole outer surface is covered. 

Finally, although not deemed infeasible from Table \ref{tab:designconcepts}, the Combined concept is disregarded. It is considered to be too similar to a Trailing concept. A trailing concept will still require a heat shield at the front of the payload and can therefore be considered, is some aspects, a Combined configuration as well. It is therefore considered that a deployable inflatable at the front will have no additional advantage. A small deployable will have a similar performance as the Trailing configuration, but features the additional complexity of the front inflation system. A large frontal inflation systems will however place the aft inflatable in the wake making it effectively useless. Removing the aft declarator will however yield a "simple" Stacked toroid, Iso tensoid or tension cone configuration. For these reasons the Trailing concept will not be considered. 


...Control department moet het hier mee eens zijn. Hier kunnen nog combinaties weg gehaald worden en naar het control chapter verplaatst worden (of andersom)... 

..Check of alle argumentatie zoals besporeken er in zit... Uiteindelijk te weing argumentatie --> Naar control chapter

\subsection{Concept configurations} \label{sec:conf}

In this section a the global configuration of each of the final five shapes is considered.

\textbf{Stacked toroid}

Figure ... shows the stacked toroid concept. A stacked toroid configuration features multiple inflatables which are stacked together to form the aeroshell. These inflatables are consequently covered with a thermal protection layer. In this design the payload is placed aft of the aeroshell.\

\textbf{Iso tensoid}

A Iso tensoid configuration as displayed in Figure ... features a single inflatable. This inflatable covers the whole of the payload. This inflatable is relatively large and is typically inflated using ram-air \cite{Smith2011}. 

\textbf{Tension cone}

A Tension cone, as shown in Figure ... configuration again consist only of a single inflatable. In this case the inflatable is ring formed, using the ring to provide stiffness to a web spanned within. This configuration the aeroshell is place front of the payload, warping around it in some extend.

\textbf{Trailing}

Figure ... shows a trailing configuration. A Trailing configuration consist of two parts. A aft placed inflatable, typically referred to as the the trailing, and a front placed rigid heatshield. Since the inflatable is aft placed the payload is directly exposed to the atmosphere requiring the addition of the rigid heatshield. The shock waves induced by the front of the payload consequently create a wake aft of the payload. A typical trailing device is therefore ring formed to stay out of this wake is is also displayed by Figure...

\textbf{Rigid}

The Rigid configuration is most typical configuration and frequently used by returns to the earth atmosphere such in the Apollo, Soyuz or planned Orion mission. Figure ... shows the Rigid configuration. This design features a rigid heatshield in front of the payload and has no inflatable parts.



