\section{Thermal mass estimation}
\label{ch:thermtool}
Decelerating an entry vehicle requires a thermal protection system. This system will contribute to a large extend to the mass of the entire vehicle. Therefore, a thermal protection system (\gls{tps}) has a critical impact on the mission an should be selected properly. This section will analyse the \gls{tps} of the five trade-off concepts with the use of a thermal protection tool. First the tool will be described in more detail. Afterwards it is described how the tool is verified and validated. In the third section, this tool will be used to analyse different lay-ups for the concepts under anlysis. As a result,the \gls{tps} masses for differnt concepts can be determined. Lastly, concluions and recomendations.

\subsection{Method of thermal analysis}
A mass estimations of the \gls{tps} requires analysis due to heat transfer in the heat shield. Enable to analyse different lay-ups in a short amount of time, it will be usefull to make use of a \gls{tps} tool. The working principles of this tool are describes in detail in this section.

\subsubsection{Assumptions}
In this section the assumptions that are used for the \gls{tps} mass estimation and tool developement are stated. Also, an elaboration is given of the impact of the assumption on the results.

\begin{itemize}
\item \textbf{1D solution instead of 3D}: 
\item \textbf{Sizing is done with under stagnation conditions}: 
\item \textbf{Model temperature and heat flux}: 
\item \textbf{Constant material properties}:
\item \textbf{Applicance to multiple layers}:
\item \textbf{•}:
\end{itemize}

\subsubsection{Leading equations}
Suthes
\subsubsection{Input}
Lucas
\subsubsection{Output}
Suthes


\subsection{Verification \& Validation}

\subsubsection{Verification}
Suthes
\subsubsection{Validation}



\subsection{Results per concept}



\subsection{Conclusion \& recomendations}
