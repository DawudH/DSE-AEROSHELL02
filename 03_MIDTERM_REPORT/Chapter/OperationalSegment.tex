\section{Operational segment}\label{cha:opseg}
This chapter describes the mission operational segment, giving an overview of mission phases and related activities. It serves to identify the actions to be performed by the system over the mission duration. It is essential that these are taken into account in mission design for a complete and feasible mission. To this end, the mission is divided into four phases: 
\begin{itemize}
\item[I]{Interplanetary flight up to the sphere of influence}
\item[II]{Re-entry from the sphere of influence to the point of atmospheric entry}
\item[III]{Re-entry from the point of atmospheric entry to the terminal point at 10 kilometers altitude}
\item[IV]{Final descent from the terminal point at 10 kilometers altitude to ground level}
\end{itemize}
These four phases are discussed hereafter in sections \ref{sec:p1}-\ref{sec:p4}.

\subsection{Phase I}\label{sec:p1}
The first phase imposes the following requirements on the vehicle:
\begin{itemize}
\item The payload shall provide sufficient volume and adequate living capability for human payload. Since the mission strictly only concerns the re-entry part, the relevance of this mission phase is mainly to impose human payload requirements. 
\item  The mission requires use of a launcher. The launcher limits the maximum diameter of the re-entry vehicle as well as its mass, since it is more economically feasible to adhere to current launcher availabilities rather than to develop a new launcher.
\end{itemize}
Moreover, transfer time should be within the limits of human payload endurance. As such, a Hohmann transfer is not a feasible option due to its large transfer time and a higher energy transfer orbit is more realistic. This affects how Mars is approached.

\subsection{Phase II}\label{sec:p2}
The start of the second mission phase signifies the initiation of control system activities in trajectory control.

\subsection{Phase III}\label{sec:p3}

\subsection{Phase IV}\label{sec:p4}
The final descent phase employs deployment mechanisms other than the inflatable aeroshell and demands a high precision in landing at a designated landing site. To this end, the terminal conditions with which the third phase is exited affect the final descent. To avoid requirements on aerodynamic braking in the final descent phase from being overly demanding, from this phase the requirement flows down that at the terminal altitude of 10 [km] the terminal Mach number shall not exceed 5 [-]. 

