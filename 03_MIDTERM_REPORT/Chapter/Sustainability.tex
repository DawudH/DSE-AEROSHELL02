\section{Approach with respect to sustainable development}
\label{ch:sustain}

With a increasing awareness with respect to sustainable development it is important to consider mission sustainability. This chapter discusses the approach with respect to sustainable development of the \gls{hiad} design concepts at hand. Masud et al. define development as being sustainable "by ensuring the needs of the present demands without compromising any power or ability of future generations to meet their own needs" \cite{Masud2011}.

Since the production series length of the controllable inflatable aeroshell is limited to very low numbers less emphasis is given to sustainable development when compared to (for example) a commercial passenger jet. As such the aeroshell's environmental impact is negligible and the sustainability of the concepts discussed in this report is not taken into account as a strong design driver. That is not to say sustainability is completely disregarded during product development. If for a certain design manufacturing methods are required that are very polluting these will be avoided and exchanged for less environmentally unfriendly, 'greener' methods. 
Not only sustainability on Earth is taken into account, but also the impact of a space mission to another orbital body on the environment of said body is considered. Special care will be given to prevent accidental contamination of other orbital bodies with organic lifeforms and other contaminants. This is in line with article IX of the Outer Space Treaty of 1967 \cite{UnitedNations2008}, enforced by the Committee on Space Research (COSPAR). In addition to preventing forward interplanetary contamination care will be taken to minimise the amount of space debris left behind in an orbit around Earth and Mars. 

The development of a controllable inflatable aeroshell can however contribute to making both planetary and interplanetary spaceflight more sustainable. Since implementing an inflatable aeroshell can contribute to reducing the total mass of a spacecraft this can reduce the required launcher mass and thereby the launch emissions. The differences with respect to sustainability between the different concepts to be analysed is very small and is thus not used as a separate element in the trade-off process.

Going back to the definition of sustainable development presented at the beginning of this chapter one can see that the measures taken during the design phase are indeed in line with sustainable development.