Materials for inflatable structures are typically high-performance high-temperature-resistant woven fabrics, supplemented by coatings to reduce porosity \cite{Jenkins2001}. These materials are required to be lightweight, strong, heat-resistant and flexible for application in \glspl{hiad} \cite{Samareh2011}. On one hand there are heritage materials, such as Kapton, but for current \gls{hiad} missions these have been replaced by up-and-coming materials, such as Vectran and PBO Zylon \cite{Dillman2012,  Smith2010}. Key driver for the use of these materials is a high specific strength compared to metals (e.g. aluminium) \cite{Samareh2011}. 

An overview of materials, and their key properties, suitable for \gls{hiad} application is given in Table \ref{table:strucmatoverview}. Since the areal density is not computable, but instead measured, it is not accessible for all materials considered in Table \ref{table:strucmatoverview}. Hence, the method by Anderson \cite{Anderson1969} is only feasible for the material choices with a known areal density, as this is an essential property for this method. As of this stage, performance at high temperatures other than the process of selection leading up to the statement of these materials in references \cite{Dillman2012, Smith2010} is not investigated. The current efforts are centered around decelerator structural mass, hence the primary use for these material properties is for input in the models constructed for mass estimation. Structural and thermal performance are evaluated after concept selection, given the limited timeframe and the fact that structural concept design will not distinguish concepts in terms of the trade-off criteria other than in terms of mass.

\begin{table}[H]
\caption{Overview of candidate materials for \gls{hiad} application. Material properties from references \cite{Samareh2011,Miller2014}. A hyphen denotes unknown quantities.}
\vspace{41mm}
\hspace{-15mm}
\begin{tabular}{p{0.235\textwidth}|p{0.17\textwidth}|p{0.05\textwidth}|p{0.065\textwidth}|p{0.05\textwidth}|p{0.05\textwidth}|p{0.05\textwidth}|p{0.05\textwidth}|p{0.05\textwidth}|p{0.05\textwidth}|p{0.05\textwidth}|}
\begin{rotate}{60} ~~~~~Material \end{rotate}  &  \begin{rotate}{60} ~~~~~Type of material {[}-{]}  \end{rotate} & \begin{rotate}{60} ~~~~~Density [$\frac{kg}{m^3}$] \end{rotate}& \begin{rotate}{60} ~~~~~Areal density [$\frac{kg}{m^2}$] \end{rotate} & \begin{rotate}{60} ~~~~~Elongation at break [$\%$] \end{rotate} & \begin{rotate}{60} ~~~~~Specific strength [$\frac{kN km}{kg}$]\end{rotate} &  \begin{rotate}{60} ~~~~~Breaking strength [km]\end{rotate} & \begin{rotate}{60} ~~~~~Breaking tenacity [$\frac{g}{denier}$]\end{rotate}&  \begin{rotate}{60} ~~~~~Tensile strength [GPa] \end{rotate} & \begin{rotate}{60} ~~~~~Young's Modulus [GPa] \end{rotate} & \begin{rotate}{60} ~~~~~Poisson's Ratio [-] \end{rotate} \\
Kapton Type 100 HN           & Polyimide film               & 1420                                 & -                                          & 72                           & 163                              & 16.6                       & 1.84                             & 0.231                      & 2.5                       & 0.34                 \\ \hline
Kevlar 29     & Aramid fiber                 & 1440                                 & 0.208                                     & 3.6                          & 2031                             & 207.0                      & 23.00                            & 2.92                       & 70.5                      & 0.36                      \\ \hline
Kevlar 49   & Aramid fiber                 & 1440                                 & 0.181                                     & 2.4                          & 2084                             & 212.4                      & 23.60                            & 3.00                       & 112.4                     & 0.36                   \\ \hline
Nomex Type 430 & Aramid fiber                 & 1380                                 & 0.400                                     & 30.5                         & 441                              & 45.0                       & 5.00                             & 0.61                       & 11.45                     & -                         \\ \hline
PBO Zylon AS                 & Polybenzoxazole fiber        & 1540                                 & -                                          & 3.5                          & 3766                             & 383.9                      & 42.66                            & 5.80                       & 180.0                     & -                            \\ \hline
Spectra 2000 & Polyethylene fiber           & 970                                  & -                                          & 3.0                            & 3443                             & 351.0                      & 39.00                            & 3.34                       & 124.0                     & -                        \\ \hline
Technora                     & Aramid fiber                 & 1390                                 & -                                          & 4.4                          & 2158                             & 220.0                      & 24.45                            & 3.00                       & 70.0                      & -                          \\ \hline
Upilex-25S                   & Polyimide film               & 1470                                 & 0.378                                     & 42                           & 354                              & 36.1                       & 4.01                             & 0.52                       & 9.1                       & -                           \\ \hline
Vectran HT                   & Liquid crystal polymer fiber & 1410                                 & 0.092                                     & 4.3                          & 2270                             & 229                        & 25.44                            & 3.20                       & 75.0                      & -                      \\ \hline
Nextel 610                   & Ceramic fiber                & 3900                                 & 0.278                                     & -                            & -                                & -                          & -                                & 3.10                       & 373.0                     & -                       
\label{table:strucmatoverview}
\end{tabular}
\end{table}

The mass performance of these materials is evaluated by plotting the \acrfull{bc} of the aerodynamic decelerator concepts versus the maximum dynamic pressure. A number of quantities could have been chosen as output (y-axis) and input (x-axis) parameters. The \gls{bc} has been chosen by its ability to reflect the mass-effectiveness of the design and its common application in this context within general literature; the dynamic pressure for its ability to reflect the environmental conditions in which the aerodynamic decelerator operates. Figures \ref{fig:bc_mat_stacked}, \ref{fig:bc_mat_tension} and \ref{fig:bc_mat_trailing} display the \gls{bc} of stacked toroid, tension cone and trailing ballute concepts respectively as a function of maximum dynamic pressure for the materials stated in Table \ref{table:strucmatoverview}, as obtained by the model implemented from Ref.\cite{Samareh2011}. Figure \ref{fig:bc_mat_isotensoid} does so for an isotensoid configuration, for those materials in Table \ref{table:strucmatoverview} with a known areal density. 

[INSERT FIGURES]



