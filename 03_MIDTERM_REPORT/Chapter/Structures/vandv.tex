Verification and validation of the mass estimation tools is performed as follows. These activites can be divided primarily into the verification and validation of the mass estimation tool adapted from Samareh \cite{Samareh2011} for stacked toroid, tension cone and trailing \gls{iad} devices on one hand and verification and validation of the mass estimation tool adapted from Anderson \cite{Anderson1969} for isotensoid and trailing \gls{iad} devices on the other hand. 

\subsubsection{Anderson inflatable mass estimation method}
In order to verify the implementation of the mass estimation tool presented by Anderson in Ref. \cite{Anderson1969} the tools results were compared to the results presented in the paper by Anderson. The full estimation model was implemented and consequently checked with the coefficients of the simplified model coefficients presented on page 16 and 20 of Ref. \cite{Anderson1969}. Finally the results of the full model were compared with the results of the merit function for the attached isotensoid on page 30 of Ref. \cite{Anderson1969}. No errors were observed in the significant digits of both estimates. The mass estimates were further verified using the results presented by Clark on page 10 in Ref. \cite{Clark2009}, similarly yielding no errors detectable within the range of significant digits.

No additional validation of the mass estimation model by Anderson \cite{Anderson1969} was performed, by the absence of experimental testing of attached isotensoid concepts. This absence is reflected to some extent by the figure on page 3 of Ref.\cite{Smith2010}, from which it may be deduced that isotensoids have been flown very seldom from a historical perspective.

\subsubsection{Samareh inflatable mass estimation method}
In order to verify that the mass estimation method described in Ref.\cite{Samareh2011} has been correctly implemented, results for the nine sample cases presented on page 16 of Ref.\cite{Samareh2011} have been checked. These nine sample cases were implemented by choosing the input parameters as given in tabular form (Tables 4 and 5) on page 16 of Ref.\cite{Samareh2011} and the output parameters, primarily component masses and geometric quantities, were compared. A maximum error of 3 $\%$ in terms of total mass was obtained; a maximum error of 2 $\%$ in component masses. These errors are deemed sufficiently small to verify succesful implementation of the mass estimation method.

Validation is performed indirectly: the method \cite{Samareh2011} has been applied in the \gls{edlsa} project \cite{Cianciolo2010}, where it was shown to yield results conforming well to the outcomes of high-fidelity \gls{fea}. The used \gls{fea} is a validated tool \cite{Cianciolo2010} and thereby the method outlined in Ref.\cite{Samareh2011} has been validated through comparison with a high-fidelity validated model. Moreover, the expression for minimum inflation pressure obtained by Samareh has been found to be in correspondence with Yamada et al \cite{Yamada2009}, Clark \cite{Clark2009} and Brown \cite{Brown2009}.

A qualitative check on the results obtained by this mass estimation method is performed by assessing the relations found in section \ref{sec:strucmat} and \ref{sec:strucsens} for their conformance to expectations and other literature on inflatable aerodynamic decelerators. The outcomes hereof are reflected by the discussions in these respective sections.



