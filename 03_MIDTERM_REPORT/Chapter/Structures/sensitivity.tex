This section provides an indication of the effect of changing design parameters on one hand and external parameters on the other hand on the total structural decelerator mass and mass efficiency. Design parameters are primarily half-cone angle \gls{sym:theta}, drag coefficient \gls{sym:CD} and outer diameter \gls{sym:Do}. Primary external factor is the peak dynamic pressure. These parameters are evaluated with respect to their effect on decelerator structural mass, either in terms of ballistic coefficient or total mass. In addition, inflation gas molar mass and temperature are investigated for their effect on inflation gas mass and the number of toroids is investigated for the effect on the stacked toroid configuration. This indication is primarily useful to provide a good starting point for concept design in terms of the investigated design parameters and to quantify benefits associated with certain design concepts. 

Due to the limitations imposed by the mass estimation model for the isotensoid \cite{Anderson1969}, the sensitivity of isotensoid structural mass to these parameters can only be investigated in a limited manner. The more extensive model for the other three inflatable concepts \cite{Samareh2011} allows for a more involved sensitivity analysis.

\subsubsection{Inflation gas mass}
Stacked toroid, tension cone and trailing ballute configurations rely on the use of internal pressure to provide the structural rigidity required from the inflatable structure. The isotensoid uses ram-air to inflate itself and hence does not require inflation gas to be taken on-board for inflation purposes. In the mass estimation model, adapted from Samareh \cite{Samareh2011}, the inflation gas is characterized primarily by its operating temperature (in Kelvin or degrees Celsius) and its molar mass (in kilogram per mole). Inflation gas mass as a function of these two variables is displayed in Fig.\ref{fig:inflmass}

\begin{figure}[H]
\hspace{-5mm}
\includegraphics[width = 1.1\textwidth]{Figure/gas_temp_mass.eps}
\caption{Inflation gas mass as a function of gas temperature and molar mass of the stacked toroid, tension cone and trailing ballute configurations}
\label{fig:inflmass}
\end{figure}

Fig.\ref{fig:inflmass} confirms that a denser gas, characterized by a higher molar mass, yields a larger inflation gas mass. An overview of gas generator types and their output is given on page 7 by Brown \cite{Brown2003}. A typical molar mass is that of nitrogen, 22 [$\frac{kg}{mole}$], used in the IRVE satellites \cite{Dillman2012}. It should be noted that inflation system mass is typically estimated as a percentage of inflation gas mass. Total inflation system mass, thus consisting of inflation system and inflation gas mass, will therefore not be as low as suggested by Fig.\ref{fig:inflmass} for low molar mass inflation gases, but it will typically be higher by requiring a heavier inflation system \cite{Brown2003}. Inflation gas mass decreases for an increasing inflation gas temperature, again conforming to expectations. Via the ideal gas law the mass occupied by a certain number of moles of inflation gas will increase for a decreasing temperature through an increased gas density \cite{AndersonJr.2007}.

\subsubsection{Number of toroids for the stacked toroid configuration}
The stacked toroid features a number of toroids that are inflated. To investigate the effect of the number of toroids on total structural decelerator mass, the two are plotted against one another in Fig.\ref{fig:toro}. 

\begin{figure}[H]
\centering
\includegraphics[width = 1.0\textwidth]{Figure/mass_toroids.eps}
\caption{Decelerator structural mass as a function of the number of toroids for the stacked toroid configuration}
\label{fig:toro}
\end{figure}

From Fig.\ref{fig:toro}, it may be observed that stacked toroid structural mass decreases for an increasing number of toroids. The decrease is, however, so slight as to be insignificant. A mass decrease of approximately 4 $\%$ is effected by an increase from one to five toroids; a decrease of approximately 5 $\%$ for an increase from one to ten toroids. As the number of toroids is increased beyond ten, mass is more or less constant. Total structural decelerator mass is therefore deemed indifferent to the number of toroids. Care should still be taken to attach too much value to the results yielded by the model, given its limited fidelity.

\subsubsection{Inflatable aeroshell diameter}
Fig.\ref{fig: