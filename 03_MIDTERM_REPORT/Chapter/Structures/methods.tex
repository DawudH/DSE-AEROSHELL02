The structural mass estimation is done using two separate methods. Mass estimates based on reference missions and mass estimates based on structural models. The latter will be used for the analysis of the structural mass for the, inflatable, Stacked toroid, Isotensoid, Tension cone and trailing concept. For these concepts no, or very little, relevant reference missions are available for an adequate mass estimation. The rigid concept is however frequently applied and for this reason a mass estimation can be done on the base of reference missions.

For the mass estimation based on the structural models two different methods are deployed. First of, for the isotensoid and trailing concept Anderson \cite{Anderson1969} developed a structural merit model. Modern implementations as presented by fro example Miller \cite{Miller2014} use this mass estimation for modern materials as well. The mass estimation presented by Anderson for the isotensoid concept considers a model inflated by ram-air. A inflation method commonly considered for isotensoid structures\cite{Smith2011}. For the trailing ballute configuration Anderson considers a ram-air inflated structure as well, similar to the isotensoid configuration. For the trailing configuration different models may be considered and a ballute in the from of a "donut" is also commonly analysed. 

The structural model presented by Samareh\cite{Samareh2011} is one of the configurations which features "donut" like shape for the trailing decelerator. Moreover this paper present a mass estimation method for the Stacked toroid and tension cone design.

\subsubsection{Rigid structure mass estimation}

...

\subsubsection{Andersons model}
Andersons model provide the structural mass estimation as a single variable for the whole isotensoid or ballute configuration. The mass estimation is presented in the from of the \gls{bc} of the structure. This value can consequently be used for a mass estimation of the isotensoid or ballute configurations.

Andersons model is provided in two forms: First of a basic model requiring as input the peak dynamic pressure (\gls{sym:q}) and the frontal surface area (\gls{sym:A}) which can also be implicitly considered using the outer diameter. Moreover decelerator material properties are considered using the materials aerial density (\gls{sym:df}). Andersons model assumes the use of a single material for the whole decelerator design. It must be noted that this simple model implicitly includes as set value for the drag coefficient (\gls{sym:CD}). The full model requires a complete set of parameters which is provided by \cite{Anderson1969} and allows control of the drag coefficient as well which can be considered useful for a sensitivity analysis. More detailed analysis of other parameters are not considered at this in the design. A more detailed estimation is not desired as it requires a detailed configuration of the concept

\subsubsection{Samarehs model}

The structural mass estimation model presented by Samareh 