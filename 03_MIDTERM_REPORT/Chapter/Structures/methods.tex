The structural mass estimation is done using two separate methods: mass estimates based on reference missions and mass estimates based on structural models. The latter will be used for the analysis of the structural mass for the four inflatable concepts: the stacked toroid, isotensoid, tension cone and trailing configurations. For these concepts few reference missions are available for an adequate mass estimation and the availability of parametric mass modeling \cite{Anderson1969, Samareh2011} make structures-based models the preferred option. Historically, rigid re-entry vehicles have been applied numerous times and a mass estimation can be done on the base of reference missions. These mass estimation methods are presented hereafter.

\subsubsection{Isotensoid mass estimation}
%los maar op
For the mass estimation based on the structural models two different methods are employed. For the isotensoid configuration Anderson \cite{Anderson1969} has developed a structural merit model. Modern implementations, as presented by for example Miller \cite{Miller2014}, use this mass estimation for current investigations in the feasibility of aerodynamic decelerators. The mass estimation presented by Anderson for the isotensoid concept considers a model inflated by ram-air, an inflation method commonly considered for isotensoid structures \cite{Smith2011}. For the trailing ballute configuration Anderson considers a ram-air inflated structure as well, similar to the isotensoid configuration. For the trailing concept different configurations may be considered and the model presented by Anderson is for this reason not considered, given that the only trailing concept is a ballute ocncept. The mass estimation for the trailing concept, along with the remaining inflatable concepts is given by a second model provided by Samareh \cite{Samareh2011} discussed below using a pressurized gas as inflation mechanism. The model provided by Anderson is thus only employed for the isotensoid analysis. 

Andersons model \cite{anderson 1969} is a basic model requiring as input the peak dynamic pressure (\gls{sym:q}) and the frontal surface area (\gls{sym:A_iad}) (which can also be implicitly considered using the outer diameter), in which decelerator material properties are considered using the areal density (\gls{sym:df}) of the material used. Andersons model assumes the use of a single material for the whole decelerator design. For the computation of the areal density a minimum gage thickness is considered as well as strength requirements in the form of a strength to mass ratio (\gls{sym:kf}) of the material

 It must be noted that this simple model implicitly includes a set value for the drag coefficient (\gls{sym:CD}) of 1.6. The full model requires a complete set of parameters which is provided by \cite{Anderson1969} for the basic model, and allows control of the drag coefficient as well. The full model can be considered useful for sensitivity analysis purposes. More detailed analysis of other parameters is not considered at this stage in the design. A more detailed estimation is not desired as it requires a more detailed configuration of the concept which is not yet feasible at this stage of the design.

The inputs of this model can be summarized as follows for the simple model:
\begin{itemize}
\item Peak dynamic pressure (\gls{sym:q})
\item Frontal surface area (\gls{sym:A})
\item Areal density of the material (\gls{sym:df})
\item Strength-to-mass ratio of the material (\gls{sym:kf})
\end{itemize}
This can be extended for with the following inputs for the full model:

\begin{itemize}
\item Drag coefficient (\gls{sym:CD})
\item Detailed isotensoid shape parameters
\end{itemize}

Yielding as a final output the \acrfull{bc} of the model defined by equation \ref{eq:bc} which can consequently be transformed to the total mass of the isotensoid configuration. 
\begin{equation}
\gls{sym:bc} = \frac{\gls{sym:m}}{\gls{sym:CD} \gls{sym:A_iad}}
\label{eq:bc}
\end{equation}
The mass is supplied as a single value following the assumption that all isotensoid components have the same material. The final output of the model can thus be given as:
\begin{itemize}
\item The total mass of the isotensoid configuration.
\end{itemize}
Finally \texttt{MATLAB} is employed for the construction of a tool which allows for the investigation of changing input parameters of which the results are shown in consequent sections.


\subsubsection{Stacked toroid, tension cone and trailing ballute mass estimation}
Mass estimation for stacked toroid, tension cone and trailing \gls{iad} configurations is performed by the method outlined in Ref.\cite{Samareh2011}. Similar to the model by Anderson, it introduces a figure of merit by defining a dimensionless mass efficiency parameter \gls{sym:mbar}, defined as:
\begin{equation}
\gls{sym:mbar} = \frac{\gls{sym:m}}{\gls{sym:mfactor}}
\label{eq:dimmass}
\end{equation}
where \gls{sym:mfactor} is defined by the following equation
\begin{equation}
\gls{sym:mfactor} = \frac{\gls{sym:A_iad} \gls{sym:CD} \gls{sym:q}}{\gls{sym:ge}}
\label{eq:mfactor}
\end{equation}
A concept having a high dimensionless mass efficiency parameter may be interpreted as a concept having a high IAD areal density scaled by dynamic pressure. It is desirable to have this parameter as small as possible, to effect a larger area capable of withstanding greater dynamic pressure with the lowest possible mass.

As inputs, the mass estimation requires \cite{Samareh2011}:
\begin{itemize}
\item Decelerator configuration (stacked toroid/tension cone/trailing ballute);
\item The material properties used corresponding to the material used for the inflatable structure; 
\item A specification of the material used for the wall lay-up. Options are solely a film, thus one material that does not require an additional coating and gas barrier, a film complemented by a coating but does require a gas barrier and a fiber reinforced fabric requiring both coating and a gas barrier;
\item The diameter of centerbody and deployed aeroshell;
\item The half-cone angle \gls{sym:theta};
\item The number of toroids in case of a stacked toroid configuration;
\item The reigning static and dynamic pressure and vehicle drag coefficient;
\item The toroid diameter;
\item Inflation gas temperature and molecular weight;
\item The number of radial straps;
%\item Vehicle (characteristic) length;
\item Margins and knockdown factors accounting for seams and stress concentrations. These are taken as defined in tabular form on page 16 of Ref.\cite{Samareh2011}.
\end{itemize}
These inputs are then processed by defining a number of dimensionless geometry parameters and using these in a set of derived equations for the stresses working on components \cite[p.12-p.14]{Samareh2011}. These component stresses are then translated to component masses by considering the tensile yield and density of the material, sized to a thickness required to withstand these stresses. Based on these inputs, the mass estimation procedure yields the following outputs \cite{Samareh2011}:
\begin{itemize}
\item Mass of reinforcing elements (radial and axial straps);
\item Mass of flexible bladder material;
\item Mass of inflation gas;
\item Inflation gas pressure;
\item Mass of inflation system;
\item Total mass.
\end{itemize}
All outputs can be given either in dimensionless or dimensional forms, with conversion from one to the other performed through Eq.\ref{eq:dimmass}. A \texttt{MATLAB} tool has been constructed that allows varying these inputs and provides a parametric estimation of the decelerator mass for each of these three inflatable concepts. This allows the investigation of the effect of changing input parameters on the various output parameters. Key output parameter at this stage is the total mass, for the purpose of concept comparison in terms of mass performance. Moreover, inflation gas pressure and inflation gas mass provide a benchmark for investigating the effect of using different inflation gases.


\subsubsection{Rigid structure mass estimation}

The mass estimation of the rigid structure is one primary based on reference missions. No parametric mass estimation models exist whereas however ample of reference mission using a heat shield for re-entry exist. Structural mass estimations are however rarely provided for solely the decelerator mass. This is for a twofold of reasons:

\begin{itemize}
\item For rigid concepts the decelerator structure is for a large extent incorporated in the \acrfull{tps} mass.
\item No deployment structure as seen in the inflatable structures can be observed. This causes the structural mass to be generally seen as single variable for the whole payload en decelerator mechanism
\end{itemize}

For these reasons the mass estimation analysis of the rigid concept is combined with mass estimation of the \acrfull{tps}. The total mass estimation of both these systems combined will be given as a single value. This mass estimation based on reference missions is given ... and is not discussed ...



