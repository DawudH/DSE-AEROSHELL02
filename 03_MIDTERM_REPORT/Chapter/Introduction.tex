\section{Introduction}
\label{cha:introduction}
Before human interplanetary spaceflight can be achieved technological gains have to be made in several fields. One of these fields comprises hypersonic deceleration systems. Significant weight gains are expected to be possible by using inflatable aeroshells. However, the development of a controllable inflatable aeroshell is very complex and consists of many different disciplines. To reduce the complexity of designing such a system first a \acrfull{pp} was made. Following that a \acrfull{br} was produced, in order to survey the current technology state and knowledge on this subject. Now a \acrfull{mtr} report is made in order to perform a concept trade-off. 

The purpose of this report is to present several concepts for a controllable inflatable aeroshell and to determine which concept is best suited for performing the design mission. First the group organisation for the period between the \acrlong{mtr} and \acrlong{fr} is discussed in chapter \ref{ch:organisation}, including individual and group tasks and work packages. A \acrfull{wbs} is made, together with a \acrfull{wfd} and Gantt chart. Secondly the approach with respect to sustainable development is presented in chapter \ref{ch:sustain}. Thirdly the \glspl{dot} are used in chapter \ref{ch:options} to generate several system concepts. These concepts will be analysed with tools from several different disciplines. These consist of an astrodynamics \& control tool, as well as tools for concept mass estimation, aerodynamical characteristics and thermodynamic behaviour. The development, verification, validation of these tools is discussed in chapters \ref{ch:astrocontrol}, \ref{ch:strucmass}, \ref{ch:aero_analysis} and \ref{ch:thermtool} respectively. These chapters also show the results obtained from analysing the proposed system concepts. After the tool development and concept analysis the risk inherent to each system concept is considered in chapter \ref{ch:riskestimation}. This will be done by making a risk map for each concept. Finally the concept trade-off will be performed in chapter \ref{ch:tradeoff}, based on results of the analyses conducted in the previous chapters. The result of this is a complete trade-off matrix, after which the customer will be able to select the preferred concept based on the weights they attach to each trade-off criterion.