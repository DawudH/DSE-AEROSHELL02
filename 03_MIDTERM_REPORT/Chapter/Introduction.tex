\section{Introduction}
\label{cha:introduction}
A guidable inflatable hypersonic decelerator has high potential to deliver sufficient aerodynamic deceleration to bring human payload to the surface of extraterrestrial locations, such as Mars, at a significantly higher mass efficiency than conventional solutions. Payload-carrying capability can therefore be increased to such an extent that the economic feasibility of extraterrestrial exploration and habitation missions is significantly increased. Human interest in space exploration persists and a large number of planned human spaceflight missions make such a lightweight solution high in demand. 

To this end the study focuses on designing a \acrfull{cia} that brings a payload of at least 9000 [kg] to the surface of Mars, one of the most challenging environments for aerobraking due to its thin atmosphere, with a decelerator mass less than a mere ten percent of the total vehicle mass. This Mid-Term Report follows in the wake of a Project Plan and Baseline Report that defined the steps to be taken in the design and outlined the requirements and functionalities to be met by the design respectively. The purpose of this report is to lay the groundwork for the trade-off performed in the \acrfull{mtr} by conceptual analysis of one rigid and four inflatable concepts through use of a set of developed tools.

Concepts follow from a structured \acrfull{dot} and are characterised by their shape. This step of concept generation has yielded one rigid concept and four inflatable concepts, namely the stacked toroid, tension cone, trailing ballute and isotensoid. Concepts are evaluated for the following trade-off criteria: decelerator mass, development risk, deceleration time and stability. To the end of providing estimations that allow comparison for the five concepts in these aspects the following tools have been developed: a parametric structural mass model, a basic thermal analysis tool, an astrodynamic tool with implemented control system and an aerodynamic tool. The mass of the concepts is distinguished by three mass components, namely control system, structures and \acrfull{tps}. On the basis of the structural, astrodynamic and thermal models masses for the three mass components are estimated and compared. Development risk of concepts is estimated on the basis of past and current research into the concepts and reflected by the \acrfullpl{trl}. Deceleration time and stability are reflected by the lift-to-drag ratio and static stability coefficients respectively which are output of the aerodynamic analysis tool. On the basis of this evaluation, a trade-off is made in the \gls{mtr} in cooperation with the customer.

An overview of the time-sequenced mission operations is given in Chapter \ref{cha:opseg}. Chapter \ref{ch:wdd} gives a definition of past, current and future work and discusses resource allocation along with the appointment of managerial and technical functions. Chapter \ref{ch:sustain} presents an approach with respect to sustainable development.  Concept trade-off criteria are presented in Chapter \ref{ch:tradeoff} and the concepts are evaluated in terms of these criteria in Chapter \ref{ch:options}. For an efficient design and the identification of iteration loops, Chapter \ref{ch:di} defines subsystem interactions. Chapters \ref{ch:astrocontrol} up to and including Chapter \ref{ch:thermtool} discuss tool development and concept evaluation in terms of trajectory, aerodynamic performance and structural and thermal mass. A risk assessment is performed in Chapter \ref{ch:riskestimation}. A summary of findings on concept performance in the trade-off criteria is given in Chapter \ref{ch:tfsum}. The report is concluded by Chapter \ref{cha:conclusion}.

%Before human interplanetary spaceflight can be achieved technological gains have to be made in several fields. One of these fields comprises hypersonic deceleration systems. Significant weight gains are expected to be possible by using inflatable aeroshells.

%However, the development of a controllable inflatable aeroshell is very complex and involves many different disciplines. 

%To reduce the complexity of designing such a system first a \acrfull{pp} was made. Following that a \acrfull{br} was produced, in order to survey the current technology state and knowledge on this subject. Now a \acrfull{mtr} report is made in order to perform a concept trade-off. 



%The purpose of this report is to present several concepts for a controllable inflatable aeroshell and to determine which concept is best suited for performing the design mission. First the group organisation for the period between the \acrlong{mtr} and \acrlong{fr} is discussed in chapter \ref{ch:wdd}, including individual and group tasks and work packages. A \acrfull{wbs} is made, together with a \acrfull{wfd} and Gantt chart. Secondly the approach with respect to sustainable development is presented in chapter \ref{ch:sustain}. Thirdly the \glspl{dot} are used in chapter \ref{ch:options} to generate several system concepts. These concepts will be analysed with tools from several different disciplines. These consist of an astrodynamics \& control tool, as well as tools for concept mass estimation, aerodynamical characteristics and thermodynamic behaviour. The development, verification, validation of these tools is discussed in chapters \ref{ch:astrocontrol}, \ref{ch:strucmass}, \ref{ch:aero_analysis} and \ref{ch:thermtool} respectively. These chapters also show the results obtained from analysing the proposed system concepts. After the tool development and concept analysis the risk inherent to each system concept is considered in chapter \ref{ch:riskestimation}. This will be done by making a risk map for each concept. Finally the concept trade-off will be performed in chapter \ref{ch:tradeoff}, based on results of the analyses conducted in the previous chapters. The result of this is a complete trade-off matrix, after which the customer will be able to select the preferred concept based on the weights they attach to each trade-off criterion.