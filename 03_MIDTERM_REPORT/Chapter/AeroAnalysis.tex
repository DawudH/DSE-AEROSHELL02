\section{Aerodynamic concept analysis}
\label{ch:aero_analysis}
In order to analyse the aerodynamic characteristics of the proposed concepts a software tool was developed. This chapter will deal with the development and implementation of this program. First section \ref{subsec:aerotool} will discuss the tool development process. Secondly section \ref{subsec:appaeroanal} 

\subsection{Development of aerodynamic analysis tool}
\label{subsec:aerotool}
For analysing the aerodynamic characteristics of each concept the modified Newtonian method will be used. This method relates the inclination angle of a flat plate with respect to an incoming flow to the magnitude of its coefficient of pressure, as shown in equation \ref{eq:modnewtonian} \cite{AndersonJr.2006}.
\begin{multicols}{2}
\begin{equation}
C_{p}=C_{p_{max}}sin^{2}(\theta)
\label{eq:modnewtonian}
\end{equation} \break
\begin{equation}
C_{p_{max}}=\frac{p_{O_{2}}-p_{\infty}}{\frac{1}{2}\rho_{\infty}V_{\infty}^{2}}
\label{eq:cpmax}
\end{equation}
\end{multicols}
In equation \ref{eq:modnewtonian} $C_{p_{max}}$ is the value of $C_{p}$ in the stagnation point of an arbitrary body. Since the stagnation point is per definition located behind a normal shock its value can be found from the normal shock relations. The result of doing this is shown in equation \ref{eq:cpmax}. Here $p_{O_{2}}$ denotes the total pressure in the stagnation point and can be found using equation \ref{eq:po2} \cite{AndersonJr.2007}.

\begin{equation}
\frac{p_{O_{2}}}{p_{\infty}}=
\label{eq:po2}
\end{equation}


\subsection{Application of analysis tool on system concepts}
\label{subsec:appaeroanal}