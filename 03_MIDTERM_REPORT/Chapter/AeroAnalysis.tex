\section{Aerodynamic concept analysis}
\label{ch:aero_analysis}
In order to analyse the aerodynamic characteristics of the proposed concepts a software tool was developed. This chapter will deal with the development and implementation of this program. First section \ref{subsec:aerotool} will discuss the tool development process. Secondly section \ref{subsec:appaeroanal} 

\subsection{Development of aerodynamic analysis tool}
\label{subsec:aerotool}
For analysing the aerodynamic characteristics of each concept the modified Newtonian method will be used. This method relates the inclination angle $\theta$ of a flat plate with respect to an incoming flow to the magnitude of its coefficient of pressure, as shown in equation \ref{eq:modnewtonian} \cite{AndersonJr.2006}.
\begin{multicols}{2}
\begin{equation}
C_{p}=C_{p_{max}}sin^{2}(\theta)
\label{eq:modnewtonian}
\end{equation} \break
\begin{equation}
C_{p_{max}}=\frac{p_{O_{2}}-p_{\infty}}{\frac{1}{2}\rho_{\infty}V_{\infty}^{2}}
\label{eq:cpmax}
\end{equation}
\end{multicols}
In equation \ref{eq:modnewtonian} $C_{p_{max}}$ is the value of $C_{p}$ in the stagnation point of an arbitrary body. Since the stagnation point is per definition located behind a normal shock its value can be found from the normal shock relations. The result of doing this is shown in equation \ref{eq:cpmax}. Here $p_{O_{2}}$ denotes the total pressure in the stagnation point and can be found using equation \ref{eq:po2} \cite{AndersonJr.2007}.

\begin{equation}
\frac{p_{O_{2}}}{p_{\infty}}=\left(\frac{(\gamma+1)^{2}M_{\infty}^{2}}{4\gamma M_{\infty}^{2}-2(\gamma-1)}\right)^{\frac{\gamma}{\gamma-1}}\left(\frac{1-\gamma+2\gamma M_{\infty}^{2}}{\gamma+1}\right)
\label{eq:po2}
\end{equation}

Furthermore it can be noted that $\frac{1}{2}\rho_{\infty}V_{\infty}^{2}=\frac{\gamma}{2}p_{\infty}M_{\infty}^{2}$ \cite{AndersonJr.2007}. Combining this with equation \ref{eq:cpmax} produces equation \ref{eq:cpmaxfinal}, where the ratio $\frac{p_{O_{2}}}{p_{\infty}}$ can be calculated using equation \ref{eq:po2}.

\begin{equation}
C_{p_{max}}=\frac{2}{\gamma M_{\infty}^{2}}\left(\frac{p_{O_{2}}}{p_{\infty}}-1\right)
\label{eq:cpmaxfinal}
\end{equation}

By dividing the surface of the body to be analysed into many triangular elements the pressure coefficient distribution of said body can be determined numerically. A velocity magnitude is given as input, together with the angle of attack $\alpha$ and sideslip angle $\beta$. Following this the outward surface normal vector is computed in Cartesian coordinates for every element, after which the inclination angle $\theta$ is determined by taking [TBD, add in future] 

Using $C_{p_{max}}$ and $\theta$ the $C_{p}$ for every surface element is calculated, after which it is multiplied with the element area and the element surface normal vector. This results in an elemental pressure force in three dimensions from which the lift and drag forces(, as well as the aerodynamic moments (how?)) can be determined. 

In addition to determining the aerodynamic forces and moments acting on the body, the heat flux in the stagnation point is also computed. A generalized equation to predict the heat flux on a body can be found in \cite{AndersonJr.2006,Tauber1986}. This equation is shown in \ref{eq:heatflux}.
\begin{equation}
q_{w}=\rho_{\infty}^{N}V_{\infty}^{M}C
\label{eq:heatflux}
\end{equation}

In the stagnation point it is furthermore known that: 
\begin{multicols}{2}
\begin{equation}
\label{eq:stagdens}
N=0.5
\end{equation} \break
\begin{equation}
\label{eq:stagspeed}
M=3.0
\end{equation}
\end{multicols}
\begin{equation}
\label{eq:stagcoefficient}
C=1.83 \times 10^{-8} R^{-\frac{1}{2}}\left(1-\frac{h_{w}}{h_{0}}\right)
\end{equation}
Where in equation \ref{eq:stagcoefficient} $R$ denotes the local body radius in the stagnation point and $h_{w}$ and $h_{0}$ comprise of the wall and total enthalpies respectively. An additional assumption that is made here is that $\frac{h_{w}}{h_{0}}\ll 1$. Justification for this statement can be found in the fact that the wall temperature must be smaller than the melting or decomposition temperature during the entire flight. Thus, although the temperature can become very high, the resulting wall enthalpy $h_{w}$ will still be much smaller than the total enthalpy $h_{0}$ \cite[p.347]{AndersonJr.2006}. %In addition to this the computed heat flux will increase as a result of neglecting this factor. One can see that if in later design phases the enthalpy ratio is included into the calculations this will relax the design constraints.
Combining equations \ref{eq:heatflux}, \ref{eq:stagdens}, \ref{eq:stagspeed}, \ref{eq:stagcoefficient} into one single equation produces:
\begin{equation}
q_{w_{stagnation}}=1.83 \times 10^{-8}\rho_{\infty}^{0.5} V_{\infty}^{3.0} R^{-\frac{1}{2}}
\label{eq:qstag}
\end{equation}
Where $q_{w_{stagnation}}$ denotes the heat flux into the body at the stagnation point. This will be used as input for the thermodynamic model in order to compute the required thicknesses of the \acrfull{tps} lay-up.

\subsection{Model verification \& validation}
\label{subsec:aeroverval}
After the model construction verification was carried out to determine whether the model correctly implemented the calculations of the modified Newtonian method. This was done by placing two triangular surface elements in a flow at an angle and comparing the model outputs with results obtained by manual computations. 

Following the verification the model was validated against \cite[p. 783]{AndersonJr.2007}

\subsection{Application of analysis tool to system concepts}
\label{subsec:appaeroanal}
After the model development 