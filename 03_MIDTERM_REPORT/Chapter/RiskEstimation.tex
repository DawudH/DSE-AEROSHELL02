\section{Risk assessment}
\label{ch:riskestimation}
Development risk will be used as one of the trade-off criteria. 

\begin{table}[h]
	\centering
	\caption{Risk map elements}
	\label{tab:riskmapelements}
	\begin{tabular}{|c|c|}
		\hline 
		\textbf{Number} & \textbf{Element} \\ \hline \hline
		1 & \acrlong{tps} \\
		2 & Deployment mechanism \\
		3 & Aerodynamic characteristics \\
		4 & \\
		\hline
	\end{tabular}
\end{table}

\begin{table}
	\caption[\acrlong{nasa} \acrfullpl{trl}]{\gls{nasa} \glspl{trl} \cite{NASA2007}}
	\begin{tabular}{|p{0.2\textwidth}|p{0.75\textwidth}|}
		\hline
		\textbf{\acrfull{trl}} & \textbf{Description} \\ \hline \hline
		\gls{trl} 9& Actual system "flight proven" through successful mission operations\\
		\gls{trl} 8& Actual system completed and "flight qualified" through test and demonstration (ground or space)\\
		\gls{trl} 7& System prototype demonstration in a space environment\\
		\gls{trl} 6& System/subsystem model or prototype demonstration in a relevant environment (ground or space)\\
		\gls{trl} 5& Component and/or breadboard validation in relevant environment\\
		\gls{trl} 4& Component and/or breadboard validation in laboratory environment\\
		\gls{trl} 3& Analytical \& experimental critical function and/or characteristic proof-of-concept\\
		\gls{trl} 2& Technology concept and/or application formulated\\
		\gls{trl} 1& Basic principles observed \& reported \\
		\hline
	\end{tabular}
\end{table}

\begin{table}[H]
	\centering
	\caption{Risk map}
	\label{tab:riskmap}
	\begin{tabular}{|c|c|c|c|c|} % MAKE SURE THAT THE TOTAL WIDTH IS 0.95\textwidth!! (that way its exactly the textwidth.... haha) 
		\hline
		\textbf{\gls{trl} 1} & \cellcolor{green!70} & \cellcolor{yellow!75} & \cellcolor{red!60} & \cellcolor{red!60} \\ \hline
		\textbf{\gls{trl} 2} & \cellcolor{green!70} & \cellcolor{yellow!75} & \cellcolor{red!60} & \cellcolor{red!60} \\ \hline
		\textbf{\gls{trl} 3} & \cellcolor{green!70} & \cellcolor{yellow!75} & \cellcolor{red!60} & \cellcolor{red!60} \\ \hline
		\textbf{\gls{trl} 4} & \cellcolor{green!70} & \cellcolor{yellow!75} & \cellcolor{yellow!75} & \cellcolor{yellow!75} \\ \hline
		\textbf{\gls{trl} 5} & \cellcolor{green!70} & \cellcolor{green!70} & \cellcolor{yellow!75} & \cellcolor{yellow!75} \\ \hline
		\textbf{\gls{trl} 6} & \cellcolor{green!70} & \cellcolor{green!70} & \cellcolor{green!70} & \cellcolor{green!70} \\ \hline
		\textbf{\gls{trl} 7} & \cellcolor{green!70} & \cellcolor{green!70} & \cellcolor{green!70} & \cellcolor{green!70} \\ \hline
		\textbf{\gls{trl} 8} & \cellcolor{green!70} & \cellcolor{green!70} & \cellcolor{green!70} & \cellcolor{green!70} \\ \hline
		\textbf{\gls{trl} 9} & \cellcolor{green!70} & \cellcolor{green!70} & \cellcolor{green!70} & \cellcolor{green!70} \\ \hline
		 & \textbf{Negligible} & \textbf{Marginal} & \textbf{Critical} & \textbf{Catastrophical} \\ \hline
	\end{tabular}
\end{table}

\subsection{Concept development risk}

The classification of selected concepts in terms of technology readiness serves to aid in the selection of a final concept that features minimal development risk. Since the selection is a relative process, technology readiness levels used hereafter are primarily intended for use with respect to each other rather than by themselves. For one, none of past inflatable missions have been flown with the dimensions considered nor were they suitable for human spaceflight.

Rigid concepts have long been the standard for manned (re-)entry missions and have been applied numerous times, as investigated in the Baseline Report \cite[p.2-3]{Balasooriyan2015a}. An overview of rigid (re-)entry vehicles is given by Laub et al. \cite{Laub2004} and Steinfeldt \cite{Steinfeldt2009} and were already used in the rigid concept mass estimation of chapter \ref{sec:rigid}. In addition, recent application in Mars missions, for example the Mars Science Laboratory \cite{Schoenenberger2009}, make the rigid concept a low development risk option. In terms of \gls{trl}, the rigid concept is therefore classified as \gls{trl} 9.

Inflatable concepts are still at a low technology readiness level. Whereas research efforts have been instigated as early as in the 1960s, a lull in research efforts up to recent years make the field of inflatable aerodynamic decelerators relatively underveloped \cite{Smith2010}. Hypersonic aerodynamic decelerator technology in particular has seen little application. Past and ongoing research by \gls{nasa} in hypersonic aerodynamic decelerators has focused on the stacked toroid configuration. A research programme consisting of a series of experiments, called \acrfull{irve}, was initiated in 2003 and the IRVE-II flight has brought the development risk of the stacked toroid configuration (as applied in the IRVE vehicles) to \gls{trl} 7 \cite{Player2005}.

Isotensoid, tension cone and ballute configurations have been tested primarily for supersonic entry (at Mach numbers lower than 5). Research efforts concentrated thereon have taken place primarily in the 1960s \cite{Smith2010}. A lack of research programmes on these configurations in recent years and the series of laboratory tests conducted in the 1960s have led to designating these concepts with a \gls{trl} of 4.

An overview of development risk for the five concepts is given in Table \ref{tab:concrisk}.

\begin{table}[h]
\caption{Concept development risk comparison}
\begin{tabular}{|l|l|l|l|l|l|}
\hline
\textbf{Concept {[}-{]}} & Stacked toroid & Tension cone & Trailing ballute & Isotensoid & Rigid \\ \hline
\textbf{TRL {[}-{]}}     & 7              & 4            & 4                & 4          & 9     \\ \hline
\end{tabular}
\label{tab:concrisk}
\end{table}

