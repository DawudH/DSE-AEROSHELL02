\section{Concept trade-off}
\label{ch:tradeoff}
This chapter gives an overview of the trade-off process and the elements involved. It is essential that concepts are evaluated for their qualities important to the customer. While all concepts shall meet the requirement, freedom in the design space remains to allow for the best possible solution to maximize customer satisfaction. This freedom is expressed by the differing design options, customer satisfaction by a selected set of trade-off criteria. A conceptual analysis, described in previous chapters, has been performed to establish concept performance in terms of these criteria. Dialogue with the customer  during the \acrfull{mtr} yields a decision for a final concept for preliminary analysis and design.

The chapter commences with a section on the trade-off criteria used, proceeds with a discussion on the performance of each concept for each of the trade-off criteria and concludes with a brief summary of concept performance for each of the concepts.

\subsection{Definition of trade-off criteria}
The following trade-off criteria are used in the concept selection phase. A brief discussion follows for each one of the selected criteria.

\subsubsection{Decelerator mass}\label{subsub:decelmass}
The (re-)entry vehicle consists of a decelerator, either in the form of a rigid heat shield and supporting structure or in the form of an inflatable heat shield and supporting structure and inflation system, and a payload module. Since total entry mass is constrained by launcher considerations, a reduction in decelerator mass will allow for a larger payload mass to be taken on-board. To fully exploit launcher mass-carrying capabilities, total entry mass is held at its maximum and increments in the two components making up total entry mass directly transfer from one to the other. It is highly desirable to maximize payload mass, since it is synonymous to maximizing the useful load-carrying capability of the (re-)entry vehicle. The latter allows either taking more payload in a fixed series of missions to Mars or taking a fixed total payload to Mars in a decreased amount of missions, thereby decreasing the total launches required. This translates to higher sustainability, by a decreased use of resources, and increased cost-effectiveness, by a decreased number of total launches and resulting total launch costs.

\subsubsection{Development risk}
Concepts with a lower development risk are preferable, since technology readiness directly affects schedule risk, cost risk and technical performance risk. Underdeveloped concepts inherently feature a larger amount of uncertainty. Eliminating this uncertainty requires additional resource expenditure, in the form of test runs for example, thereby posing a greater burden on available schedule time and monetary resources. In case the uncertainty would be left unexplored, technical performance would suffer by a decreased knowledge about product behavior and capability, increasing technical performance risk. To preserve mission reliability, concepts with a higher development risk require additional investigation and thereby carry additional resource and schedule loads. Moreover, in case these concepts are further explored and found to be ill-suited to the mission, a re-run of concept selection, analysis and design is required. It is therefore essential that development risk is as low as possible for economical reasons.

\subsubsection{Deceleration capability}
Deceleration capability is reflected by the time required for entry. A decreased entry time alleviates costs incurred by ground operations and minimizes physical taxation of human payload on-board. As such, it is desirable that entry time is as small as possible (and a high deceleration capability is preferable), within the requirements imposed by the customer and regulations. 

\subsubsection{Orbit controllability}
An increased orbit controllability brings about greater freedom in trajectory selection and optimization on one hand and increases vehicle capability of counteracting perturbations on the other hand. While the latter capability is to be safeguarded in all designs in order to maintain minimum mission reliability, the former distinguishes concepts by a greater design efficiency. Increased controllability allows fine-tuning trajectories to a larger extent, such that trajectories can be optimized to induce smaller thermal and aerodynamic loads. Lowering peak loads allows design of a structure and \acrfull{tps} with lower design loads, thereby sized smaller than an equivalent structure and \gls{tps} sized for higher peak loads. This effects a decrease im decelerator mass, thereby increasing payload-carrying capability (see subsection \ref{subsub:decelmass}).

\subsection{Concept performance in trade-off criteria}

\subsection{Concept overall performance}
