\section{Concept trade-off}
\label{ch:tradeoff}
This chapter gives an overview of the trade-off process and the elements involved. It is essential that concepts are evaluated for their qualities important to the customer. While all concepts shall meet the requirement, freedom in the design space remains to allow for the best possible solution to maximize customer satisfaction. This freedom is expressed by the differing design options, customer satisfaction by a selected set of trade-off criteria. A conceptual analysis, described in previous chapters, has been performed to establish concept performance in terms of these criteria. Dialogue with the customer  during the \acrfull{mtr} yields a decision for a final concept for preliminary analysis and design.

The chapter commences with a section on the trade-off criteria used, proceeds with a discussion on the performance of each concept for each of the trade-off criteria and concludes with a brief summary of concept performance for each of the concepts.

\subsection{Definition of trade-off criteria}
The following trade-off criteria are used in the concept selection phase. 

\subsubsection{Decelerator mass}
The (re-)entry vehicle consists of a decelerator, either in the form of a rigid heat shield and supporting structure or in the form of an inflatable heat shield and supporting structure and inflation system, and a payload module. Since total entry mass is constrained by launcher considerations, a reduction in decelerator mass will allow for a larger payload mass to be taken on-board. It is highly desirable to maximize payload mass. Increasing the amount of payload on board [XXXXXXXXX]

\subsubsection{Development risk}

\subsubsection{Deceleration capability}

\subsubsection{Orbit controllability}

\subsection{Concept performance in trade-off criteria}

\subsection{Concept overall performance}
