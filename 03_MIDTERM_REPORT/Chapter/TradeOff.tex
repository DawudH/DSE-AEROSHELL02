\section{Concept Trade-Off Criteria}\label{ch:tradeoff}
This chapter gives an overview of the trade-off process and the elements involved. It is essential that concepts are evaluated for the qualities that are important to the customer. While all concepts shall meet the requirements, freedom in the design space remains to allow for the best possible solution to maximize customer satisfaction. This freedom is expressed by the differing design options. The customer satisfaction is expressed by a selected set of trade-off criteria. Each trade-off criterion is stated and thereafter briefly discussed in the sections below.

\subsection{Decelerator mass}\label{subsub:decelmass}
The (re-)entry vehicle consists of a decelerator, either in the form of a rigid heat shield and supporting structure or in the form of an inflatable heat shield and supporting structure and inflation system, and a payload module. Since total entry mass is constrained by current launchers, a reduction in decelerator mass will allow for a larger payload mass to be taken on-board. To fully exploit launcher mass-carrying capabilities, total entry mass is held constant at its maximum and increments in the two components making up total entry mass directly transfer from one to the other. It is highly desirable to maximize payload mass, since it is synonymous to maximizing the useful load-carrying capability of the (re-)entry vehicle. The latter allows either taking more payload in a fixed number of missions to Mars or taking a fixed total payload to Mars in less missions, thereby decreasing the total launches required. This translates to higher sustainability, by a decreased use of resources, and increased cost-effectiveness, by a decreased number of total launches and resulting total launch costs. For the decelerator mass; the structural, \gls{tps} and control system mass are evaluated.

\subsection{Development risk}
Concepts with a lower development risk are preferable, since technology readiness directly affects schedule risk, cost risk and technical performance risk. Underdeveloped concepts inherently feature a larger amount of uncertainty. Eliminating this uncertainty requires additional resource expenditure, in the form of for example an increased number of test runs, thereby imposing a greater burden on available schedule time and monetary resources. In case the uncertainty would be left unexplored, technical performance would suffer by a decreased knowledge about product behavior and capability, increasing technical performance risk. To preserve mission reliability, concepts with a higher development risk require additional investigation and thereby carry additional resource and schedule loads. Moreover, in case these concepts are further explored and found to be ill-suited to the mission, a re-run of concept selection, analysis and design is required. It is therefore essential that development risk is as low as possible primarily for economical reasons.

\subsection{Deceleration time}
A decreased time to touchdown alleviates costs incurred by ground operations and mitigates physical taxation of human payload on-board. Ground operations are required to be fully active during entry operations and a decrease in entry duration will reduce the working hours and associated costs. Physical taxation is reduced by a shorter duration of high g-loads on human payload. A reduction in physical taxation will require less measures to counteract the negative effects thereof on the human body. As such, it is desirable that entry time is as small as possible, within the requirements imposed by the customer.

%Deceleration is maximized when the maximum tolerable acceleration is sustained throughout entry, as discussed in Chapter \ref{ch:astrocontrol}. Control is performed by a control system that alters the lift by changing the angle of attack of the vehicle, the performance of which increases for an increasing lift gradient. An increased performance of the control system leads to longer adherence to the 3-g limit imposed and thereby to a decreased deceleration time.

\subsection{Stability}
An increased vehicle stability is preferable, since a stable vehicle requires less control forces to counteract perturbations. A stable vehicle will react to a perturbation by an oppositely directed moment to revert itself to its equilibrium condition, where an increase in stability will bring about a larger reaction moment. This counters the perturbation faster than a less stable vehicle, while an unstable vehicle would increase the effects of the perturbation and bring it further from its equilibrium condition. It is desirable that the effect of perturbations is mitigated as much as possible, since these are unpredictable and cause vehicle deviations from its intended trajectory. Stability is the preferable method of mitigation, since counteraction of perturbations by active control requires additional expenditure of power or propellant mass. 

It should be noted that stability is in conflict with controllability, since stability and controllability are negatively correlated. Controllability is reflected by decelerator mass, in which the control system mass is incorporated.

%In the context of the two-dimensional orbit model, perturbances are considered in pitch direction and hence causing a change in angle of attack. A first indication for stability is decelerator static stability, measured in the moment coefficient derivative with respect to angle of attack. As discussed in Chapter \ref{ch:aero_analysis}, this parameter is obtainable using the developed aerodynamic analysis tool for each of the five concepts.
%An increased orbit controllability brings about greater freedom in trajectory selection and optimization on one hand and increases vehicle capability of counteracting perturbations on the other hand. While the latter capability is to be safeguarded in all designs in order to maintain minimum mission reliability, the former distinguishes concepts by a greater design efficiency. Increased controllability allows fine-tuning trajectories to a larger extent, such that trajectories can be optimized to induce smaller thermal and aerodynamic loads. Lowering peak loads allows design of a structure and \acrfull{tps} with lower design loads, thereby sized smaller than an equivalent structure and \gls{tps} sized for higher peak loads. This effects a decrease in decelerator mass, thereby increasing payload-carrying capability (see subsection \ref{subsub:decelmass}).



