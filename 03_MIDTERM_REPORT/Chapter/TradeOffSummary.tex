\section{Concept performance in trade-off criteria}\label{ch:tfsum}

 A conceptual analysis, described in previous chapters, has been performed to establish concept performance in terms of these criteria. Dialogue with the customer during the \acrfull{mtr} yields a decision for a final concept for preliminary analysis and design. This chapter treats the concept performance for all five selected concepts in terms of the trade-off criteria formulated in Chapter \ref{ch:tradeoff}.

\subsection{Decelerator mass}
Concepts have been evaluated for the three primary components that make up and distinguish concepts: structure, \acrfull{tps} and control system. The structural mass has been estimated using parametric models in Chapter \ref{ch:strucmass}; the \gls{tps} mass via a basic thermal layer analysis in Chapter \ref{ch:thermtool}; the control system mass via the required moments to be generated in Chapter \ref{ch:astrocontrol}.  These masses are given in Table \ref{tab:cmass}. It should be noted that the structural mass excludes connection mass, not taken into account in the mass estimation and deemed equal for all concepts. Hence the summed mass consists of only 85 $\%$ of the decelerator mass. 

While structural mass has been estimated directly in [kg], the thermal and control system mass, following from Chapters \ref{ch:thermotool} and \ref{ch:aero_analysis}, have been estimated on the basis of heat load and moment coefficients. Since the comparison requires only relative masses, all  masses have been expressed as a percentage of stacked toroid mass. Total mass is calculated as the weighted average of the three weight components.

Each of the components, structural mass, thermal mass and control system mass, has been given an importance factor, of respectively 20 $\%$,  50 $\%$ and  15 $\%$ corresponding to the weight fraction assigned in the budget breakdown in the Baseline Report \cite[p.28]{Balasooriyan2015a}. These weights have been used since masses are relative comparison and the subsystems contribute in a different magnitude to the total decelerator mass. 

\begin{table}[h]

\caption{Concept mass comparison (expressed as percent of stacked toroid mass)}\label{tab:cmass}
\hspace{-10mm}
\begin{tabular}{|p{0.2\textwidth}|p{0.2\textwidth}|p{0.2\textwidth}|p{0.2\textwidth}||p{0.08\textwidth}|}

\hline
                          & \textbf{Structural mass (20 \%)} & \textbf{Thermal mass (50 \%)} & \textbf{Control system mass (15 \%)} & \textbf{Total mass} \\ \hline
\textbf{Stacked toroid}   &  100                                 & 100                          & 100                                      &\cellcolor{green!70}  100                           \\ \hline
\textbf{Tension cone}     &  168                               & 100                               &                                       &\cellcolor{green!70}                                \\ \hline
\textbf{Trailing ballute} &  221                                 & 85                               &                                       &\cellcolor{green!70}                                \\ \hline
\textbf{Isotensoid}       &  516                                 & 79                               &                                       &\cellcolor{yellow!75}                             \\ \hline \hline
\textbf{Rigid}            &  \multicolumn{4}{|c|}{\cellcolor{red!60} Far in excess of 1000 [kg] limit (est. 2945 [kg] combined structural and thermal mass)}    \\ \hline
\end{tabular}
\end{table}

From the last column in Table \ref{tab:cmass} it follows that the lowest estimated decelerator mass is achieved using the stacked toroid configuration. Evidently the highest mass is achieved by the rigid concept, with an estimaed combined structural and thermal mass of 2945 [kg] (see section \ref{sec:rigid}) well in excess of the 1000 [kg] limit imposed. The latter is therefore deemed unacceptable; the former is deemed excellent performance. Mass performance of the other concepts is in between. 

As discussed previously, a lower decelerator mass (denoted by the total mass column in Table \ref{tab:cmass}) effectively means a larger payload mass given the objective of fully using launcher capability by a total entry vehicle mass of 10000 [kg]

\subsection{Development risk}
Development risk has been evaluated on the basis of concept technology readiness in Chapter \ref{ch:riskestimation}. It was concluded that frequent historic investigation, testing and application of rigid concepts for (re-)entry make this concept the most developed. Inflatable concepts are a relatively novel solution and investigation thereof has been limited: isotensoid, trailing ballute and tension cone concepts have been tested up to an estimated \gls{trl} of 4. The stacked toroid concept has undergone a more extensive research program (\acrfull{irve}) and is hence at a \gls{trl} of 7 while ongoing research by \gls{nasa} continues to further the technology readiness \cite{Dillman2014}. It should be noted that the trailing ballute is only capable of featuring morphing as a viable control option, as investigated in Chapter \ref{ch:astrocontrol}, an underdeveloped area of technology for the application at hand. To reflect the difficulty of control with a trailing ballute, its \gls{trl} is lowered to 2 since morphing has only been formulated, but not tested, for this concept.

The \glspl{trl} of the five selected concepts are stated in Table \ref{tab:gls_rev}.

\begin{table}[h]
\caption{Review of concept development risk}
\begin{tabular}{|l|l|l|l|l|l|}
\hline
\textbf{Concept {[}-{]}} & Stacked toroid & Tension cone & Trailing ballute & Isotensoid & Rigid \\ \hline
\textbf{TRL {[}-{]}}     &\cellcolor{green!70} 7  &\cellcolor{yellow!75}  4   &\cellcolor{red!60} 2 & \cellcolor{yellow!75}      4          &\cellcolor{green!70} 9     \\ \hline
\end{tabular}
\label{tab:gls_rev}
\end{table}

\subsection{Deceleration time}
The deceleration time is evaluated on the basis of vehicle lift gradient, taken as the average over a range of considered angles of attack. A higher ligt gradient is thereby deemed positive in achieving deceleration within a shorter time. This evaluation, discussed in Chapter \ref{ch:aero_analysis}, has yielded the results given in Table \ref{tab:decel_time}.

\begin{table}[h]
\caption{Review of concept lift gradient}
\hspace{-20mm}
\begin{tabular}{|c|c|c|c|c|c|}
\hline
\textbf{}                          & \textbf{Stacked toroid} & \textbf{Tension cone} & \textbf{Trailing ballute} & \textbf{Isotensoid} & \textbf{Rigid} \\ \hline
\textbf{Average lift gradient {[}1/rad{]}} &\cellcolor{green!70} 7  &\cellcolor{yellow!75}  4   &\cellcolor{red!60} 2 & \cellcolor{yellow!75}      4          &\cellcolor{green!70} 9                 \\ \hline
\end{tabular}
\end{table}

From Table \ref{tab:decel_time} it may be observed that the highest lift gradient is attained for the ... configuration. [ADD AERO EXPLANATION]

\subsection{Stability}
Stability of concepts is measured by the static stability, as explained in Chapter \ref{ch:aero_analysis}. A more stable concept is reflected by a more negative static stability coefficient. This derivative is given in Table \ref{tab:stab}.

\begin{table}[h]
\caption{Review of concept lift gradient}
\hspace{-20mm}
\begin{tabular}{|c|c|c|c|c|c|}
\hline
\textbf{}                          & \textbf{Stacked toroid} & \textbf{Tension cone} & \textbf{Trailing ballute} & \textbf{Isotensoid} & \textbf{Rigid} \\ \hline
\textbf{Static stability coefficient {[}1/rad{]}} &\cellcolor{green!70} 7  &\cellcolor{yellow!75}  4   &\cellcolor{red!60} 2 & \cellcolor{yellow!75}      4          &\cellcolor{green!70} 9                 \\ \hline
\end{tabular}
\end{table}

[ADD AERO EXPLANATION]

