\section*{Summary}\label{cha:summary}

\textbf{[ADD NUMBERS]}

Inflatable concepts hold the key to what is currently unattainable for interplanetary human spaceflight. In the wake of current \acrfull{nasa} investigations on the feasibility of inflatable decelerators for hypersonic guidable re-entry, this study focuses on Mars entry of a payload mass of 9000 [kg] using an inflatable aerodynamic decelerator of a mere 1000 [kg]. Such a solution provides a large economical advantage over conventional solutions by maximizing payload-carrying capability through a light-weight device less burdened by launcher size considerations. This Mid-term Report gives an overview of conceptual design activities, comprising aerodynamic, structural, thermal astrodynamic and control tool development and analysis, leading up to concept selection in a trade-off.
\newline
\newline
To aid in the conceptual analysis, design and selection process and to provide a basis for further design efforts after concept selection in the \acrfull{mtr} a number of tools have been developed with the following purposes:
\begin{itemize}
\item A tool for parametric structural mass modeling
\item A modified Newtonian flow aerodynamic tool for the characterisation of aerodynamic and aerothermal behavior
\item A thermal model for \acrfull{tps} sizing and analysis
\item An astrodynamic tool with an implemented control system for trajectory optimization 
\end{itemize}

Concepts were selected for trade-off as the outcome of a structured \acrfull{dot} on the basis of concept shape. Shape is chosen for it is a leading factor in the aerodynamic and trajectory performance, which directly flows down to thermal, structural and control performance. The initial concept selection yielded five concepts for trade-off: a rigid concept and four inflatable concepts, namely stacked toroid, tension cone, trailing ballute and isotensoid configurations. To reflect concept capability of meeting customer demand, the following concept aspects have been evaluated as trade-off criteria: decelerator mass, development risk, vehicle stability and deceleration time.
\newline
\newline
The first trade-off criterion, decelerator mass, is essential since it is highly desirable that payload-carrying capability is maximized. To make full use of the launcher carrying capability, a decrease in decelerator mass allows for an equal increase in payload mass. To this end, structural, thermal and control system mass were estimated as the distinguishing mass components between concepts. 

The lowest structural mass was achieved by the stacked toroid configuration, followed upon closely by tension cone and trailing ballute configurations (an estimated 168\% and 221\% of stacked toroid structural mass respectively) with the isotensoid trailing behind (with an estimated 516 \% of stacked toroid structural mass). \acrfull{tps} and control system mass analysis yielded estimates of 100, 84, and 76 \% and 100, 86, and 99 \% of stacked toroid thermal respectively control system mass for tension cone, trailing ballute and isotensoid configurations respectively. This yields total decelerator masses of 116, 117 and 184 \% of stacked toroid mass, based on weight contributions of 20, 50 and 15 \% by structure, \gls{tps} and control system mass. A mass analysis for the rigid concept showed a decelerator mass well in excess of the imposed maximum 1000 [kg], a markedly high 2945 [kg] for thermal and structural mass alone.
\newline
\newline
The second trade-off criterion, development risk, is essential since concepts with a low \acrfull{trl} incur higher schedule and cost risks. The tried-and-true rigid solution thereby has \gls{trl} 9, while inflatable concepts are notably less explored. A recent surge in interest in inflatable concepts by \gls{nasa} and a consistent development program has brought the stacked toroid configuration to \gls{trl} 7. The other three inflatable concepts are less explored, having only undergone a selected set of tests and research and hence designated \gls{trl} 4. The trailing ballute concept is deemed to have a higher development risk still, by its only feasible control option being a relatively underveloped one, namely morphing, reflected by a \gls{trl} of 2.
\newline
\newline
Concepts are evaluated in terms of the third trade-off criterion, deceleration time, for a shorter entry time is desirable. As ground operations are to be maintained at fully capacity during entry, a shortening of entry time will alleviate costs incurred. Moreover, physical taxation of human payload is lessened as deceleration time is decreased. To this end, concepts were evaluated for their performance in [ADD HERE]
\newline
\newline
It is preferable that concepts are stable, since a stable vehicle will counteract perturbations to revert back to its equilibrium condition and its intended trajectory. To this end, static stability was investigated by aerodynamic analysis. Stacked toroid, tension cone and ballutes were found stable; the rigid concept to be neutrally stable; the isotensoid to be unstable. The former three therefore perform well, the rigid concept adequately and performance of the isotensoid is deficient for the fourth and last trade-off criterion.
\newline
\newline
On the basis of the analysis presented in this Mid-term Report, design options and their prospective advantages and disadvantages are presented to the customer at the \gls{mtr}. Hereafter dialogue is entered to yield a final concept for preliminary design that satisfies customer demands. The next phase then commences with a more detailed analysis and design of the selected concepts, aided by further tool enhancement. This phase entails a more refined orbit optimization, aerodynamic shape determination and structural and \gls{tps} design and sizing. A structured approach to this next phase is facilitated by breaking down future work and resource allocation and aided by an interface definition given in this report to define the interaction of subsystems within the design process.